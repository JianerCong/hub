\documentclass[dvipsnames]{article}
% \documentclass[dvipsnames]{ctexart}

\title{number theory}
\usepackage{geometry}\geometry{
  a4paper,
  total={170mm,257mm},
  left=20mm,
  top=20mm,
}


\usepackage{svg}

\usepackage[skip=5pt plus1pt, indent=0pt]{parskip}
% Color
\newcommand{\mycola}{MidnightBlue}
\newcommand{\mycolb}{Mahogany}
\newcommand{\mycolc}{OliveGreen}

\newcommand{\cola}[1]{\textcolor{\mycola}{#1}}
\newcommand{\colb}[1]{\textcolor{\mycolb}{#1}}
\newcommand{\colc}[1]{\textcolor{\mycolc}{#1}}
\newcommand{\Cola}[1]{\textcolor{\mycola}{\textbf{#1}}}

% \let\emph\relax % there's no \RedeclareTextFontCommand
% \DeclareTextFontCommand{\emph}{\bfseries}
\renewcommand{\emph}[1]{\texbf{#1}}
\usepackage{amssymb}            %\mathbb

\usepackage{fontspec}

\setmonofont{Cascadia}[
Path=/usr/share/fonts/truetype/Cascadia_Code/,
Scale=0.85,
Extension = .ttf,
UprightFont=*Code,              %find CascadiaCode.ttf
BoldFont=*CodePL,               %find CascadiaCodePL.ttf ...
ItalicFont=*CodeItalic,
BoldItalicFont=*CodePLItalic
]
% --------------------------------------------------
% Windows
% \setmonofont{Cascadia}[
% Path=C:/Windows/Fonts/,
% Extension = .ttf,
% UprightFont=*Mono,              %find CascadiaMono.ttf
% BoldFont=*Code,               %find CascadiaCodePL.ttf ...
% ItalicFont=*Code,
% BoldItalicFont=*Code
% ]


\usepackage{minted}
\usepackage{tcolorbox}
\tcbuselibrary{skins}
\tcbuselibrary{minted}
\usepackage{tikz}
\usetikzlibrary{shapes} % ellipse node shape
\usetikzlibrary{shapes.multipart} % for line breaks in node text
\usetikzlibrary{arrows.meta}    %-o arrow head
\usetikzlibrary{arrows}
\usetikzlibrary{matrix}
\usetikzlibrary{snakes}

\usepackage{amsmath}
% ??? still xelatex?
% \usepackage{xeCJK}
\usepackage{emoji}
% \setemojifont{NotoColorEmoji.ttf}[Path=C:/Users/congj/repo/myFonts/]
% \setemojifont{TwitterColorEmoji-SVGinOT.ttf}[Path=C:/Users/congj/repo/myFonts/]


\newtcolorbox[auto counter]{myBox}[2][]{
  fonttitle=\bfseries,title={共识~\thetcbcounter: #2},#1
}
\newtcolorbox[]{noteBox}[1][]{
  tile,left=1mm,nobeforeafter,fontupper=\small,#1
}

\tikzstyle{myNode}=[inner sep=2pt,circle,text=white]
\date{\today}
\author{作者}

\newtcblisting{simplec}{
  listing engine=minted,
  minted language=c++,
  minted style=vs,
  minted options={fontsize=\small,autogobble,
    % framesep=1cm
  },
  tile,
  listing only,
  % bottom=0cm,
  % nobeforeafter, 
  boxsep=0mm,
  left=1mm,
  opacityback=0.5,
  colback=gray!20
}
\tcbuselibrary{breakable}
\newtcblisting{simplepy}{
  listing engine=minted,
  minted language=python,
  minted style=vs,
  minted options={fontsize=\small,autogobble,
    % framesep=1cm
  },
  tile,
  listing only,
  % bottom=0cm,
  % nobeforeafter,
  boxsep=0mm,
  left=1mm,
  opacityback=0.5,
  colback=gray!20,
  breakable
}
\newtcolorbox{blackbox}{tile,colback=black,colupper=white,nobeforeafter,halign=flush center}

\tikzstyle{myMatrix}=[matrix of nodes,below right,
nodes={above,text centered},                  %apply to all nodes
row sep=1cm,column sep=2cm]
\tikzstyle{every node}=[inner sep=0pt]

\newcommand\uptoleft[3][-o]{\draw[very thick,#1](#2.south) |- (#3.west);}
\newcommand\uptodown[3][-o]{\draw[very thick,#1](#2.south) to [out=270,in=90] (#3.north);}
\newcommand\downtoup[3][-latex]{\draw[very thick,#1](#2.north) to [out=90,in=270] (#3.south);}

\newcommand\lefttoright[3][-latex]{\draw[very thick,#1](#2.east) to[out=0,in=180] (#3.west);}
\newcommand\lefttodown[3][-latex]{\draw[very thick,#1](#2.east) to[out=0,in=90] (#3.north);}


\newtcolorbox{leftDialogBox}{
  tile, nobeforeafter, boxsep=0pt,
  % show bounding box,
  colback=\mycola!10,
  overlay={
    \begin{scope}
      % \fill[gray!10] (frame.east) circle (2pt);
      \fill[\mycola!10] (frame.east) --
      +(0,2mm) --
      +(3mm,0) --
      +(0,-2mm)
      ;
    \end{scope}
  }}


\newtcolorbox{rightDialogBox}{
  tile, nobeforeafter, boxsep=0pt,
  % show bounding box,
  colback=\mycola!10,
  overlay={
    \begin{scope}
      % \fill[gray!10] (frame.east) circle (2pt);
      \fill[\mycola!10] (frame.west) --
      +(0,2mm) --
      +(-3mm,0) --
      +(0,-2mm);
    \end{scope}
  }}

\newcommand{\mycolaa}{\mycola!20}


\usepackage{changepage}   % for the adjustwidth environment
\newenvironment{myIndent}[1][7mm]{\begin{adjustwidth}{#1}{}}{\end{adjustwidth}}

\newcounter{myDefCounter}
\newcounter{myTheoCounter}

\tcbuselibrary{theorems}
\newtcbtheorem[use counter=myDefCounter,number within=section]{myDef}{Definition}%
{
  % colback=green!5,colframe=green!35!black,
  fonttitle=\bfseries}{def}

\newtcbtheorem[use counter=myTheoCounter,number within=section]{myTheo}{Theorem}%
{
  % colback=green!5,
  colframe=\mycola,
  fonttitle=\bfseries,
  breakable,
  before lower={
    \texttt{Proof: }\par
    \begin{myIndent}
  },
  after lower={
    \end{myIndent}
    \qed
  }
}{thm}

\usepackage{cleveref}
\crefname{myDefCounter}{definition}{definitions}
                                   % ^^^ plural
\Crefname{myDefCounter}{Definition}{Definitions}
         % ^^^^^^ type = counter name

\crefname{myTheoCounter}{theorem}{theorems}
\Crefname{myTheoCounter}{Theorem}{Theorems}

\usepackage{amsthm}             %for {proof}


% --------------------------------------------------
\begin{document}
\maketitle

\section{The series of primes}

\subsection{Divisibility of intergers}

The numbers
% dots for commas
\[ \dotsc,-3,-2,-1,0,1,2, \dotsc\]
are the \Cola{intergers}, denoted by $\mathbb{Z}$. The number
\[ 0,1,2,3,\dotsc\] are the \Cola{non-negative integers} or \Cola{natural
  numbers}, denoted by $\mathbb{N}$, and the numbers \[1,2,3,\dotsc\] are the
\Cola{positive integers}, denoted by $\mathbb{N^{\times}}$. We also denote
\Cola{non-zero intergers} by $\mathbb{Z^{\times}}$

Let $a,\in \mathbb{Z}, b \in \mathbb{Z^{\times}}$, $a$ is saied to be
\Cola{devisible} by $b$ if there exists $c \in \mathbb{Z}$ such that
\[ a = bc.\] If $a,c, > 0$, so does $c$. If $a$ is divisable by $b$, we say $b$
is a \Cola{divisor} of $a$ and write \[ b \mid a.\] Thus
\[ 1 \mid n, \quad n \mid n, \quad \forall n \in \mathbb{Z}.\] If $b$ doesn't
divide $a$, we write $b \nmid a$.

It's plain that
\begin{align*}
  b \mid a \;,\; c \mid b &\Rightarrow   c \mid a\\
  b \mid a  &\Rightarrow   bc \mid ac\\
  c \mid a \;,\; c \mid b &\Rightarrow   c \mid ma + nb \quad \forall a,b,m,n
                            \in \mathbb{Z} \quad c \in \mathbb{Z^{\times}}
\end{align*}

\subsection{Prime numbers}

\begin{myDef}{Prime number}{prime-number}
  A number $p \in \mathbb{N}$ is said to be \Cola{prime}, if
  \begin{itemize}
  \item $p > 1$
  \item $p$ has no positive divisors except 1 and $p$
  \end{itemize}
\end{myDef}

% The \cref{def:prime-number}.
A number greater than 1 and not prime is called \Cola{composite}.
Our first theorem is
\begin{myTheo}{Product of primes}{product-of-primes}
  Every positive integer $n$ > 1, is a product of primes.
  \tcblower
    \cola{If} $n$ \cola{is prime}:
    \begin{myIndent}
      There's nothing to prove.
    \end{myIndent}
    \cola{Else:}
    \begin{myIndent}
      $n$ has \colc{divisors} between 1 and $n$.\\
      \emoji{pinching-hand} $m := \cola{\min{}} \{\colc{\text{these divisor of }n}\}$
      \begin{myIndent}
        \cola{If} $m$ \cola{is not prime}:
            \begin{align*}
              \exists l \in (1,m) \subset \mathbb{N}, l \mid m &\Rightarrow l \mid n \\
              & \Rightarrow m \text{ is not the smallest as defined \emoji{cross-mark}}
            \end{align*}
        So, $m$ \cola{is prime}, say $m = p_1$, then we can draw a prime out of n:
        \[
          n = p_1n_1, \quad n_1 \in [p_1,n)
        \]
        \cola{If} $n_1$ is a prime:
        \begin{myIndent}
          Done.
        \end{myIndent}
        \cola{Else}:
        \begin{myIndent}
          We use the same process again to draw out another prime divisor of $n_1$
          \begin{align*}
            \text{\emoji{pinching-hand} } p_2 \in (1,n_1), p_2 \mid n_1 & \Rightarrow n_1 = p_2n_2 \\
            & \Rightarrow n = p_1p_2n_2
          \end{align*}
          Sooner, or later we must meet a prime $n_i$, because \cola{we can't
            keep factoring out numbers greater than one from a finite number.}
        \end{myIndent}
      \end{myIndent}
    \end{myIndent}
\end{myTheo}

\end{document}


% Local Variables:
% TeX-engine: luatex
% TeX-command-extra-options: "-shell-escape"
% TeX-master: "m.tex"
% End:
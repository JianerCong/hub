\documentclass[dvipsnames]{ctexart}
\title{中期检查答题}
\usepackage{tabularx}
\usepackage{booktabs}
\usepackage{geometry}\geometry{
  a4paper,
  total={170mm,257mm},
  left=20mm,
  top=20mm,
}


\usepackage{svg}

\usepackage[skip=5pt plus1pt, indent=0pt]{parskip}
% Color
\newcommand{\mycola}{MidnightBlue}
\newcommand{\mycolb}{Mahogany}
\newcommand{\mycolc}{OliveGreen}

\newcommand{\cola}[1]{\textcolor{\mycola}{#1}}
\newcommand{\colb}[1]{\textcolor{\mycolb}{#1}}
\newcommand{\colc}[1]{\textcolor{\mycolc}{#1}}
\newcommand{\Cola}[1]{\textcolor{\mycola}{\textbf{#1}}}

% \let\emph\relax % there's no \RedeclareTextFontCommand
% \DeclareTextFontCommand{\emph}{\bfseries}
\renewcommand{\emph}[1]{\texbf{#1}}
\usepackage{amssymb}            %\mathbb

\usepackage{fontspec}

\setmonofont{Cascadia}[
Path=/usr/share/fonts/truetype/Cascadia_Code/,
Scale=0.85,
Extension = .ttf,
UprightFont=*Code,              %find CascadiaCode.ttf
BoldFont=*CodePL,               %find CascadiaCodePL.ttf ...
ItalicFont=*CodeItalic,
BoldItalicFont=*CodePLItalic
]
% --------------------------------------------------
% Windows
% \setmonofont{Cascadia}[
% Path=C:/Windows/Fonts/,
% Extension = .ttf,
% UprightFont=*Mono,              %find CascadiaMono.ttf
% BoldFont=*Code,               %find CascadiaCodePL.ttf ...
% ItalicFont=*Code,
% BoldItalicFont=*Code
% ]


\usepackage{minted}
\usepackage{tcolorbox}
\tcbuselibrary{skins}
\tcbuselibrary{minted}
\usepackage{tikz}
\usetikzlibrary{shapes} % ellipse node shape
\usetikzlibrary{shapes.multipart} % for line breaks in node text
\usetikzlibrary{arrows.meta}    %-o arrow head
\usetikzlibrary{arrows}
\usetikzlibrary{matrix}
\usetikzlibrary{snakes}

\usepackage{amsmath}
% ??? still xelatex?
% \usepackage{xeCJK}
\usepackage{emoji}
% \setemojifont{NotoColorEmoji.ttf}[Path=C:/Users/congj/repo/myFonts/]
% \setemojifont{TwitterColorEmoji-SVGinOT.ttf}[Path=C:/Users/congj/repo/myFonts/]


\newtcolorbox[auto counter]{myBox}[2][]{
  fonttitle=\bfseries,title={共识~\thetcbcounter: #2},#1
}
\newtcolorbox[]{noteBox}[1][]{
  tile,left=1mm,nobeforeafter,fontupper=\small,#1
}

\tikzstyle{myNode}=[inner sep=2pt,circle,text=white]
\date{\today}
\author{丛坚而 技术部 C++算法工程师}

\newtcblisting{simplec}{
  listing engine=minted,
  minted language=c++,
  minted style=vs,
  minted options={fontsize=\small,autogobble,
    % framesep=1cm
  },
  tile,
  listing only,
  % bottom=0cm,
  % nobeforeafter, 
  boxsep=0mm,
  left=1mm,
  opacityback=0.5,
  colback=gray!20
}
\tcbuselibrary{breakable}
\newtcblisting{simplepy}{
  listing engine=minted,
  minted language=python,
  minted style=vs,
  minted options={fontsize=\small,autogobble,
    % framesep=1cm
  },
  tile,
  listing only,
  % bottom=0cm,
  % nobeforeafter,
  boxsep=0mm,
  left=1mm,
  opacityback=0.5,
  colback=gray!20,
  breakable
}
\newtcolorbox{blackbox}{tile,colback=black,colupper=white,nobeforeafter,halign=flush center}

\tikzstyle{myMatrix}=[matrix of nodes,below right,
nodes={above,text centered},                  %apply to all nodes
row sep=1cm,column sep=2cm]
\tikzstyle{every node}=[inner sep=0pt]

\newcommand\uptoleft[3][-o]{\draw[very thick,#1](#2.south) |- (#3.west);}
\newcommand\uptodown[3][-o]{\draw[very thick,#1](#2.south) to [out=270,in=90] (#3.north);}
\newcommand\downtoup[3][-latex]{\draw[very thick,#1](#2.north) to [out=90,in=270] (#3.south);}

\newcommand\lefttoright[3][-latex]{\draw[very thick,#1](#2.east) to[out=0,in=180] (#3.west);}
\newcommand\lefttodown[3][-latex]{\draw[very thick,#1](#2.east) to[out=0,in=90] (#3.north);}


\newtcolorbox{leftDialogBox}{
  tile, nobeforeafter, boxsep=0pt,
  % show bounding box,
  colback=\mycola!10,
  overlay={
    \begin{scope}
      % \fill[gray!10] (frame.east) circle (2pt);
      \fill[\mycola!10] (frame.east) --
      +(0,2mm) --
      +(3mm,0) --
      +(0,-2mm)
      ;
    \end{scope}
  }}


\newtcolorbox{rightDialogBox}{
  tile, nobeforeafter, boxsep=0pt,
  % show bounding box,
  colback=\mycola!10,
  overlay={
    \begin{scope}
      % \fill[gray!10] (frame.east) circle (2pt);
      \fill[\mycola!10] (frame.west) --
      +(0,2mm) --
      +(-3mm,0) --
      +(0,-2mm);
    \end{scope}
  }}

\newcommand{\mycolaa}{\mycola!20}


\usepackage{changepage}   % for the adjustwidth environment
\newenvironment{myIndent}[1][7mm]{\begin{adjustwidth}{#1}{}}{\end{adjustwidth}}

% --------------------------------------------------

\usepackage{enumitem}
\setlist[description]{leftmargin=\parindent,labelindent=\parindent}

\begin{document}

\maketitle{}

\section*{共同题目}

\subsection*{1. 请阐述浙大,趣链,聚义岩,新华夏之间的关系}

\begin{itemize}
\item 趣链(杭州趣链科技有限公司)属于浙大AIF产研中心孵化企业。趣链核心团队均毕业于
  浙江大学,同时也是浙大AIF互联网金融技术研究中心的学术团队成员。
\item 聚义岩(北京聚义岩科技有限公司)是趣链的全资生态子公司
\item 新华夏(北京新华夏信息技术有限公司)是聚义岩的全资子公司
\end{itemize}

\subsection*{2. 请阐述公司当前战略}
公司进入生存模式。
优胜劣汰,认人为贤。
效绩考核,未达标者予以调岗,降薪或解聘。


\subsection*{3. 请简要阐述公司当下重大事项}
以当前手中十四五项目交付为重中之重。最重要的是4501,要求交付C++版本自主研发联盟链。

\subsection*{4. 清列出全部军委,战区,军种名称}
\begin{itemize}
\item 中华人民共和国中央军事委员会
  \begin{itemize}
  \item 三大委员会
    \begin{enumerate}
    \item \cola{中央军委}纪律检查委员会
    \item \cola{中央军委}政法委员会
    \item \cola{中央军委}科学技术委员会
    \end{enumerate}
  \item 七大部
    \begin{enumerate}
    \item \cola{中央军委}办公室
    \item \cola{中央军委}联合参谋部
    \item \cola{中央军委}政治工作部
    \item \cola{中央军委}后勤保障部
    \item \cola{中央军委}装备发展部
    \item \cola{中央军委}训练管理部
    \item \cola{中央军委}国防动员部
    \end{enumerate}
  \item 五大办公厅 (署/局)
    \begin{enumerate}
    \item \cola{中央军委}战略规划办公室
    \item \cola{中央军委}改革和编制办公室
    \item \cola{中央军委}国际军事合作办公室
    \item \cola{中央军委}审计部
    \item \cola{中央军委}机关事务管理总局
    \end{enumerate}
  \end{itemize}
\item 五大战区: 东,西,南,北,中
\item 六大军种: 陆军,海军,空军,火箭军,战略支援部队,联勤保障部队。
\end{itemize}

\subsection*{请阐述BM十不准}
\begin{enumerate}
\item 不准涉密人员随意向任何组织和个人泄露秘密;
\item 不准非涉密人员以任何理由、途径非法知悉秘密;
\item 不准用手机或普通电话谈论国家秘密;
\item 不准将涉密计算机接入互联网;
\item 不准将涉密 U 盘、移动硬盘、光盘接入与互联网连接的计算机;
\item 不准在与互联网连接的计算机上处理涉密文件;
\item 不准使用具有无线上网功能的计算机处理涉密文件;
\item 不准使用未加保密装置的传真机传递涉密文件资料与信息;
\item 不准擅自复制、摘抄、销毁和私自留存涉密文件资料;
\item 不准将涉密文件资料、涉密计算机带出办公室或家中处理涉密文件。
\end{enumerate}

\section*{专业科目}
\subsection*{阐述下公链和联盟链的区别}
\begin{center}
  \begin{tabularx}{0.8\textwidth} { 
       >{\raggedright\arraybackslash}X 
       >{\centering\arraybackslash}X 
       >{\raggedleft\arraybackslash}X  }
     \toprule
     \textbf{类型} & \textbf{共链} & \textbf{联盟链} \\
     \midrule
     网络  & 公网  & 一般为内部网络  \\
     \hline
     谁能进入集群成为节点?  & 谁都能  & 在固定CA获得证书的节点才能  \\
     \hline
     共识类型 & 弱共识(Weak Consensus, Eventual Consensus)比如 PoW,Pos & 状态
     同步共识 (State Duplication Consensus) 比如 Raft,PBFT\\
     \hline
     谁来打包? & 矿工 (miner) & 主节点 (primary)\\
     \hline
     世界状态在每一瞬间都是是确凿的(100\%)? & 不是 & 是\\
     \hline
     每个区块只有一个子区块吗? &不是&是\\
     \bottomrule
\end{tabularx}
\end{center}

\subsection*{阐述下联盟链常用的共识算法}

\begin{description}
\item[Raft] 每个节点有一个内置随机秒表(Timer)。秒表超时会触发\cola{选举}。第一
  个发送\cola{选举}到大多数节点的节点成为主节点。主节点确认后发
  送\cola{心跳}以及\cola{执行}到其他节点。特点是Partition Tolerance
\item[PBFT] 
\end{description}

\end{document}


% Local Variables:
% TeX-engine: luatex
% TeX-command-extra-options: "-shell-escape"
% TeX-master: "m.tex"
% End:
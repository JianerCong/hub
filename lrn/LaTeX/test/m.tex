% \documentclass{ctexart}


\documentclass[dvipsnames]{article}

\title{test}
\usepackage{geometry}\geometry{
  a4paper,
  total={170mm,257mm},
  left=20mm,
  top=20mm,
}

\usepackage[skip=10pt plus1pt, indent=0pt]{parskip}
% Color
\newcommand{\mycola}{MidnightBlue}
\newcommand{\mycolb}{Mahogany}
\newcommand{\mycolc}{OliveGreen}

\newcommand{\cola}[1]{\textcolor{\mycola}{#1}}
\newcommand{\colb}[1]{\textcolor{\mycolb}{#1}}
\newcommand{\colc}[1]{\textcolor{\mycolc}{#1}}
\newcommand{\Cola}[1]{\textcolor{\mycola}{\textbf{#1}}}

% \let\emph\relax % there's no \RedeclareTextFontCommand
% \DeclareTextFontCommand{\emph}{\bfseries}
\renewcommand{\emph}[1]{\texbf{#1}}


\usepackage{fontspec}
\setmonofont{Cascadia}[
Path=/usr/share/fonts/truetype/Cascadia_Code/,
Scale=0.85,
Extension = .ttf,
UprightFont=*Code,              %find CascadiaCode.ttf
BoldFont=*CodePL,               %find CascadiaCodePL.ttf ...
ItalicFont=*CodeItalic,
BoldItalicFont=*CodePLItalic
]
\usepackage{minted}
\usepackage{tcolorbox}
\tcbuselibrary{skins}
\tcbuselibrary{minted}
\usepackage{tikz}
\usetikzlibrary{shapes} % ellipse node shape
\usetikzlibrary{shapes.multipart} % for line breaks in node text
\usetikzlibrary{arrows.meta}    %-o arrow head
\usetikzlibrary{arrows}
\usetikzlibrary{matrix}


% Redefine em
% latex.sty just do: \DeclareTextFontCommand{\emph}{\em}

\let\emph\relax % there's no \RedeclareTextFontCommand
\DeclareTextFontCommand{\emph}{\bfseries\em}
\usepackage{amsmath}

% ??? still xelatex?
% \usepackage{xeCJK}
\usepackage{emoji}

\date{\today}
\author{me}

\newtcblisting{simplec}{
  listing engine=minted,
  minted language=c++,
  minted style=vs,
  minted options={fontsize=\small,autogobble,
    % framesep=1cm
  },
  tile,
  listing only,
  % bottom=0cm,
  % nobeforeafter, 
  boxsep=0mm,
  left=1mm,
  opacityback=0.5,
  colback=gray!20
}

\begin{document}
\maketitle

% black and white
% {\fontspec{Symbola}\symbol{"1F343}}
% sudo apt -y install fonts-symbola

% The luatex emoji
Usually you need to initialize a \Cola{sink} in order to use \texttt{boost.log}.

\begin{tcolorbox}
  \emoji{parrot}  : But we still used trivial log all the time no? \\
  \emoji{turtle} : Yeah, that's because the library contains a `default` sink.
  (like the defualt constructor in C++).
  Once you defined a sink yourself, the default one is gone. Although you will
  still be able to use trivial logging macro.
\end{tcolorbox}

\section{File logging}

Use the following to \Cola{initialize a logging sink} that stores log records into a text log.
\begin{simplec}
  void init(){
    logging::add_file_log("sample.log");

    logging::core::get()->set_filter(
    logging::trivial::severity >= logging::trivial::info
    );
}
\end{simplec}

\section{More sinks}

You can register sink manually. Previously we created a file sink which is almost like
\begin{simplec}
  void init()
  {
    // Construct the sink
    typedef sinks::synchronous_sink< sinks::text_ostream_backend > text_sink;
    boost::shared_ptr< text_sink > sink = boost::make_shared< text_sink >();

    // Add a stream to write log to
    sink->locked_backend()->add_stream(
    boost::make_shared< std::ofstream >("sample.log"));

    // Register the sink in the logging core
    logging::core::get()->add_sink(sink);
}
\end{simplec}

Okay, the 

\end{document}


% Local Variables:
% TeX-engine: luatex
% TeX-command-extra-options: "-shell-escape"
% End:

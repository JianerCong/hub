% \documentclass[dvipsnames]{article}
\documentclass[dvipsnames]{ctexart}

\title{FollowMe 共识}
\usepackage{geometry}\geometry{
  a4paper,
  total={170mm,257mm},
  left=20mm,
  top=20mm,
}


\usepackage{svg}

\usepackage[skip=5pt plus1pt, indent=0pt]{parskip}
% Color
\newcommand{\mycola}{MidnightBlue}
\newcommand{\mycolb}{Mahogany}
\newcommand{\mycolc}{OliveGreen}

\newcommand{\cola}[1]{\textcolor{\mycola}{#1}}
\newcommand{\colb}[1]{\textcolor{\mycolb}{#1}}
\newcommand{\colc}[1]{\textcolor{\mycolc}{#1}}
\newcommand{\Cola}[1]{\textcolor{\mycola}{\textbf{#1}}}

% \let\emph\relax % there's no \RedeclareTextFontCommand
% \DeclareTextFontCommand{\emph}{\bfseries}
\renewcommand{\emph}[1]{\texbf{#1}}

\usepackage{fontspec}

% \setmonofont{Cascadia}[
% Path=/usr/share/fonts/truetype/Cascadia_Code/,
% Scale=0.85,
% Extension = .ttf,
% UprightFont=*Code,              %find CascadiaCode.ttf
% BoldFont=*CodePL,               %find CascadiaCodePL.ttf ...
% ItalicFont=*CodeItalic,
% BoldItalicFont=*CodePLItalic
% ]

% --------------------------------------------------
% Windows
\setmonofont{Cascadia}[
Path=C:/Windows/Fonts/,
% Path=C:\\Windows\\Fonts\\,
Extension = .ttf,
UprightFont=*Mono,              %find CascadiaMono.ttf
BoldFont=*Code,               %find CascadiaCodePL.ttf ...
ItalicFont=*Code,
BoldItalicFont=*Code
]


\usepackage{minted}
\usepackage{tcolorbox}
\tcbuselibrary{skins}
\tcbuselibrary{minted}
\usepackage{tikz}
\usetikzlibrary{shapes} % ellipse node shape
\usetikzlibrary{shapes.multipart} % for line breaks in node text
\usetikzlibrary{arrows.meta}    %-o arrow head
\usetikzlibrary{arrows}
\usetikzlibrary{matrix}


\usepackage{amsmath}
% ??? still xelatex?
% \usepackage{xeCJK}
\usepackage{emoji}
\setemojifont{NotoColorEmoji.ttf}[Path=C:/Users/congj/repo/myFonts/]
% \setemojifont{TwitterColorEmoji-SVGinOT.ttf}[Path=C:/Users/congj/repo/myFonts/]


\newtcolorbox[auto counter]{myBox}[2][]{
  fonttitle=\bfseries,title={共识~\thetcbcounter: #2},#1
}
\newtcolorbox[]{noteBox}[1][]{
  tile,left=1mm,nobeforeafter,fontupper=\small,#1
}

\tikzstyle{myNode}=[inner sep=2pt,circle,text=white]
\date{\today}
\author{作者}

\newtcblisting{simplec}{
  listing engine=minted,
  minted language=c++,
  minted style=vs,
  minted options={fontsize=\small,autogobble,
    % framesep=1cm
  },
  tile,
  listing only,
  % bottom=0cm,
  % nobeforeafter, 
  boxsep=0mm,
  left=1mm,
  opacityback=0.5,
  colback=gray!20
}
\tcbuselibrary{breakable}
\newtcblisting{simplepy}{
  listing engine=minted,
  minted language=python,
  minted style=vs,
  minted options={fontsize=\small,autogobble,
    % framesep=1cm
  },
  tile,
  listing only,
  % bottom=0cm,
  % nobeforeafter,
  boxsep=0mm,
  left=1mm,
  opacityback=0.5,
  colback=gray!20,
  breakable
}
\newtcolorbox{blackbox}{tile,colback=black,colupper=white,nobeforeafter,halign=flush center}

\tikzstyle{myMatrix}=[matrix of nodes,below right,
nodes={above,text centered},                  %apply to all nodes
row sep=1cm,column sep=2cm]
\tikzstyle{every node}=[inner sep=0pt]

\newcommand\uptoleft[3][-o]{\draw[very thick,#1](#2.south) |- (#3.west);}
\newcommand\uptodown[3][-o]{\draw[very thick,#1](#2.south) to [out=270,in=90] (#3.north);}
\newcommand\downtoup[3][-latex]{\draw[very thick,#1](#2.north) to [out=90,in=270] (#3.south);}

\newcommand\lefttoright[3][-latex]{\draw[very thick,#1](#2.east) to[out=0,in=180] (#3.west);}
\newcommand\lefttodown[3][-latex]{\draw[very thick,#1](#2.east) to[out=0,in=90] (#3.north);}


\newtcolorbox{leftDialogBox}{
  tile, nobeforeafter, boxsep=0pt,
  % show bounding box,
  colback=\mycola!10,
  overlay={
    \begin{scope}
      % \fill[gray!10] (frame.east) circle (2pt);
      \fill[\mycola!10] (frame.east) --
      +(0,2mm) --
      +(3mm,0) --
      +(0,-2mm)
      ;
    \end{scope}
  }}


\newtcolorbox{rightDialogBox}{
  tile, nobeforeafter, boxsep=0pt,
  % show bounding box,
  colback=\mycola!10,
  overlay={
    \begin{scope}
      % \fill[gray!10] (frame.east) circle (2pt);
      \fill[\mycola!10] (frame.west) --
      +(0,2mm) --
      +(-3mm,0) --
      +(0,-2mm);
    \end{scope}
  }}

\newcommand{\mycolaa}{\mycola!20}


% --------------------------------------------------
\begin{document}
\maketitle

\section{听一个共识}
让我们来想一个共识。优先要求这几点:
\begin{itemize}
\item 单节点可跑
\item 可(相对方便地)动态新增节点
\end{itemize}
\emoji{parrot} : 那么最简单的应该就是
\begin{myBox}{听一个共识 Listen-to-one Consensus}
  \label{cons-a}
  \begin{itemize}
  \item 第一个起的节点为Primary。
  \item 新增的节点往Primary上连接。
  \end{itemize}
\end{myBox}

\emoji{parrot} : 这个其实本质上就是在集群内部的一个 Server-Client 架构嘛。
也就是所有人都只听Primary的。
\begin{center}
  \begin{tikzpicture}
    % \tikzstyle{every node}=[inner sep=2pt]
    \node[myNode,fill=\mycola,minimum height=4em] (a1) {primary};
    \foreach \d in {0,60,...,300}{
      \draw[latex-, very thick] (a1) -- (\d:3cm) node[myNode,fill=gray]{新增节点};
      % \node[myNode,fill=gray]  at (\d:2cm) {};
    }
  \end{tikzpicture}
\end{center}
这个的缺点就是\cola{主机down了,集群就down了}。它的优点就是新增的节点只需要和主
节点连接就行,而且非primary彼此不需要知道彼此。

\begin{tikzpicture}
  \emoji{parrot} 那么代码该怎么写呢?\\
  \emoji{turtle} 在写代码之前我们先定义共识所需要的接口吧。
\end{tikzpicture}
\subsection{共识需要的接口}
抽象一点讲,共识最少需要两组外部接口。
\begin{enumerate}
\item 网络,这个包括
  \begin{itemize}
  \item 接受外部的请求(监听)
  \item P2P沟通 (监听 + 发送)
  \end{itemize}
\item 执行,把一个命令写到内部存储。
\end{enumerate}
网络的接口大概可以定义为:
\begin{simplepy}
from collections.abc import Callable
class IEndpointBasedNetworkable:
    def listen(self,
               target: str,     # The target, e.g. "/"
               handler: Callable[[str,str],Optional[str]]
               # The handler, accepts
               #  - endpoint, e.g. "localhost:8080"
               #  - request data
               # returns the result
               ):
        pass
    def listened_endpoint() -> str:
        pass
    def send(self,endpoint: str, target: str, data: str) -> str:
        # endpoint: the target, e.g. "localhost:8080"
        # target: e.g. "/hi"
        pass
\end{simplepy}
而执行的接口,大概是这样的:
\begin{simplepy}
class IExecutable:
    def execute(self,command: str):
        pass
\end{simplepy}

\subsection{\textbf{听一个共识},伪代码实现}
那么现在我们可以开始码了,先从构建函数开始,除了执行和网络以外,构建函数还接受一
个可选的\texttt{nodeToConnect},如果没给,则说明这个是第一个起的节点(primary)。
\begin{simplepy}
class ListenToOneConsensus:
    def __init__(self,
                 n: IEndpointBasedNetworkable,
                 e: IExecutable,
                 nodeToConnect=''):
        self.net = n
        self.exe = e
        if nodeToConnect:
            self.primary = nodeToConnect
            self.is_primary = false
            self.ask_primary_for_entry()
            self.start_listening_as_sub()
        else:
            self.is_primary = true
            self.start_listening_as_primary()
\end{simplepy}
\emoji{parrot} : 那么具体primary和非primary节点都要监听哪些事件呢?

在介绍具体要监听哪些事件之前,我们最好先看一下在“时光静好”的情况下,集群是怎么
工作的:
\begin{center}
  \begin{tikzpicture}
    \node[myNode,fill=\mycola,minimum height=4em] (a1) {primary};
    \foreach \d in {300,20}{
      \draw[latex-, very thick] (a1) -- (\d:3cm) node[myNode,fill=gray,name=s-\d]{新增节点};
      % \node[myNode,fill=gray]  at (\d:2cm) {};
    }
    \node (a1n) [text width=7cm,above left] at ([shift={(-2cm,0cm)}]a1.west){
      \begin{tcolorbox}[tile,fontupper=\small,
        left=0mm,
        boxsep=0mm
        ]
        \begin{itemize}
        \item 每收到一个\colc{请求}:
          \begin{enumerate}
          \item 先执行一下,
          \item 然后添加到一个\cola{笔记本}里,
          \item 然后发给所有\colb{已知下级}。
          \end{enumerate}
        \item 每当有\colb{新节点}要求加入时:
          \begin{enumerate}
          \item 先把\colb{它}加入\colb{已知下级},
          \item 然后把\cola{笔记本}发给\colc{它}一份。
          \end{enumerate}
        \end{itemize}
      \end{tcolorbox}
    };
    \draw[help lines,-o] (a1n) -- (a1);

    \node (s2n) [text width=7cm,above] at ([shift={(0cm,2cm)}] s-20.north){
      \begin{tcolorbox}[tile,fontupper=\small,
        left=0.5mm,
        boxsep=0mm]
        在刚开始的时候请求\cola{primary}加入。
        \begin{itemize}
        \item 每收到来自\cola{primary}的\colc{请求}: 执行
        \item 每收到来自\colb{client}的\colc{请求}: 传给\cola{primary}
        \end{itemize}
      \end{tcolorbox}
    };

    \draw[help lines,o-] (s-20) -- (s2n);
  \end{tikzpicture}
\end{center}
Emm,那么看来,primary和非主节点所要监听的(和发送的)也就很明显了。首先我们来看
primary需要用到的方法。
\begin{simplepy}
    def start_listening_as_primary(self):
        self.known_subs = []
        self.command_history = []
        self.net.listen('/pleaseAddMe',
                        self.handle_add_new_node)
        self.net.listen('/pleaseExecuteThis',
                        self.handle_execute_for_primary)

    def handle_add_new_node(self,sub_endpoint: str,
                            data: str) -> str:
        """[For primary] to add new node"""
        self.known_subs.append(sub_endpoint)
        return f"""
        Dear {sub_endpoint}
            You are in. and here is what we have so far:
            {''.join(self.command_history)}
                   Sincerely
                   {self.net.listened_endpoint()}, The primary.
        """

    def handle_execute_for_primary(self, endpoint: str,
                                   data: str) -> str:
        cmd = data
        self.command_history.append(cmd)
        self.exe.execute(cmd)
        for sub in self.known_subs:
            self.net.send(sub,'/pleaseExecuteThis',cmd)
        return f"""
        Dear {endpoint}
             Your request has been carried out by our group.
             Members: {self.known_subs}
                 Sincerely
                 {self.net.listened_endpoint()}, The primary.
        """
    def start_listening_as_sub(self):
        self.net.listen('/pleaseExecuteThis',
                        self.handle_execute_for_sub)
\end{simplepy}
然后是其他节点用到的方法:
\begin{simplepy}
    def ask_primary_for_entry(self):
        r = self.net.send(self.primary,
                    '/pleaseAddMe',
                    self.n.listened_endpoint())
        # In response, the primary should send you the history .
        cmd = r.split('\n')[3]  # forth line is the command
        if cmd:
            self.exe.execute(cmd)
    def start_listening_as_sub(self):
        self.net.listen('/pleaseExecuteThis',
                        self.handle_execute_for_sub)
    def handle_execute_for_sub(self,endpoint: str,data: str) -> str:
        cmd = data
        if (endpoint == self.primary):
            self.exe.execute(cmd)
            return f"""
            Dear boss,
                 Mission accomplished.
                     Sincerely
                     {self.net.listened_endpoint()}
            """
        else:
            # forward
            return self.net.send(self.primary,
                                 '/pleaseExecuteThis',data)
\end{simplepy}
所以最后总结起来应该就是:
\begin{simplepy}
class ListenToOneConsensus:
    def __init__(self,
                 n: IEndpointBasedNetworkable,
                 e: IExecutable,
                 nodeToConnect=''):
        self.net = n
        self.exe = e
        if nodeToConnect:
            self.primary = nodeToConnect
            self.is_primary = false
            self.ask_primary_for_entry()
            self.start_listening_as_sub()
        else:
            self.is_primary = true
            self.start_listening_as_primary()

    def ask_primary_for_entry(self):
        r = self.net.send(self.primary,
                    '/pleaseAddMe',
                    self.n.listened_endpoint())
        # In response, the primary should send you the history .
        cmd = r.split('\n')[3]  # forth line is the command
        if cmd:
            self.exe.execute(cmd)

    def start_listening_as_primary(self):
        self.known_subs = []
        self.command_history = []
        self.net.listen('/pleaseAddMe',
                        self.handle_add_new_node)
        self.net.listen('/pleaseExecuteThis',
                        self.handle_execute_for_primary)

    def handle_add_new_node(self,sub_endpoint: str,
                            data: str) -> str:
        """[For primary] to add new node"""
        self.known_subs.append(sub_endpoint)
        return f"""
        Dear {sub_endpoint}
            You are in. and here is what we have so far:
            {''.join(self.command_history)}
                   Sincerely
                   {self.net.listened_endpoint()}, The primary.
        """

    def handle_execute_for_primary(self, endpoint: str,
                                   data: str) -> str:
        cmd = data
        self.command_history.append(cmd)
        self.exe.execute(cmd)
        for sub in self.known_subs:
            self.net.send(sub,'/pleaseExecuteThis',cmd)
        return f"""
        Dear {endpoint}
             Your request has been carried out by our group.
             Members: {self.known_subs}
                 Sincerely
                 {self.net.listened_endpoint()}, The primary.
        """
    def start_listening_as_sub(self):
        self.net.listen('/pleaseExecuteThis',
                        self.handle_execute_for_sub)
    def handle_execute_for_sub(self,endpoint: str,data: str) -> str:
        cmd = data
        if (endpoint == self.primary):
            self.exe.execute(cmd)
            return f"""
            Dear boss,
                 Mission accomplished.
                     Sincerely
                     {self.net.listened_endpoint()}
            """
        else:
            # forward
            return self.net.send(self.primary,
                                 '/pleaseExecuteThis',data)
\end{simplepy}

\section{听一个共识-留一手版}

让我们稍微再复杂复杂一点,看看能不能新增这两个功能:
\begin{enumerate}
\item 在\cola{primary}死的时候希望可以有接班的。
\item 我们希望新增节点的时候可以连接到任何一个节点。
\end{enumerate}
\emoji{parrot} : 既然\cola{primary}可以给命令编号,那么自然也可以给新加的节点编号
不是吗? 而我们也就可以用这个编号来决定\cola{谁是下一任primary}。

是的那么这样以来我们的集群就变成了一个有序号的了:
\begin{center}
  \begin{tikzpicture}
    \begin{scope}
      \node[myNode,fill=\mycola,minimum height=4em] (a0) {primary};
      \draw[latex-, very thick] (a0) -- (30:3cm) node[myNode,fill=gray,name=a2]{...};
      \draw[latex-, very thick] (a0) -- (0:3cm) node[myNode,fill=gray,name=a4]{...};

      \draw[latex-, very thick] (a0) -- (300:3cm) node[myNode,fill=\mycolc,name=a1]{1};
      \draw[latex-, very thick] (a1) -- (270:3cm) node[myNode,fill=gray,name=a3]{...};

      \node[text width=6cm,below] at ([shift={(0,-3cm)}]a0.south) {
        \begin{noteBox}
          1.新增的节点可以通过\textbf{任意现
            有集群节点}连接,之后会管\cola{primary}要一个序号。
        \end{noteBox}
      };
      \node[left,text width=2cm] at ([xshift=-0.5cm]a3.west) {
        \begin{leftDialogBox}
          求加入
        \end{leftDialogBox}
      };

      \foreach \n in {a2}{
        \node[right,text width=2cm] at ([xshift=0.5cm,yshift=0.5cm]\n.east) {
          \begin{rightDialogBox}
            求加入
          \end{rightDialogBox}
        };
      }

    \end{scope}


    \begin{scope}[shift={(9.5cm,0)}]
      \node[myNode,fill=\mycola,minimum height=4em] (a0) {primary};
      \draw[latex-, very thick] (a0) -- (30:3cm) node[myNode,fill=\mycolaa,name=a2]{2};
      \draw[latex-, very thick] (a0) -- (0:3cm) node[myNode,fill=\mycolaa,name=a4]{4};

      \draw[latex-, very thick] (a0) -- (300:3cm) node[myNode,fill=\mycolc,name=a1]{1};
      \draw[latex-, very thick] (a1) -- (270:3cm) node[myNode,fill=\mycolaa,name=a3]{3};

      \draw[-latex,very thick] (a4) -- (a1);
      \draw[-latex,very thick] (a2) -- (a1);
      \draw[-latex,very thick] (a3) -- (a0);

      \node[text width=6cm,below] at ([shift={(0,-3cm)}]a0.south) {
        \begin{noteBox}
          2.在完成以下两个步骤后,新增的节点就算加入集群了:
          \begin{enumerate}
          \item 记住\cola{primary}和下任\cola{primary}的地址
          \item 获得自己的序列号。
          \item 同步命令历史记录
          \end{enumerate}
        \end{noteBox}
      };

      \node[left,text width=4cm] at ([xshift=-0.5cm]a0.west) {
        \begin{leftDialogBox}
          \small
          来吧,给你们\textbf{过往我们执行的记录},如果我没了就找下一任\colc{节点1}吧。
        \end{leftDialogBox}
      };
    \end{scope}
      \end{tikzpicture}
\end{center}

\begin{center}
  \begin{tikzpicture}
    \begin{scope}
      \node[myNode,fill=red!40!black,minimum height=4em] (a0) {primary};
      \draw[latex-, very thick] (a0) -- (30:3cm) node[myNode,fill=\mycolaa,name=a2]{2};
      \draw[latex-, very thick] (a0) -- (0:3cm) node[myNode,fill=\mycolaa,name=a4]{4};

      \draw[latex-, very thick] (a0) -- (300:3cm) node[myNode,fill=\mycolc,name=a1]{1};
      \draw[latex-, very thick] (a1) -- (270:3cm) node[myNode,fill=\mycolaa,name=a3]{3};

      \draw[-latex,very thick] (a4) -- (a1);
      \draw[-latex,very thick] (a2) -- (a1);
      \draw[-latex,very thick, red] (a3) -- node[midway] {\emoji{cross-mark}}(a0);

      \node[left,text width=4cm] at ([xshift=-0.5cm]a3.west) {
        \begin{leftDialogBox}
          欸? \cola{primary}老不回我,它应该没了,\colc{你}上呗?
        \end{leftDialogBox}
      };

      \node[right,text width=4cm] at ([xshift=0.5cm,yshift=0.5cm]a1.east) {
        \begin{rightDialogBox}
          是吗? 让我确认一下....喂喂喂? \cola{primary}在吗?
        \end{rightDialogBox}
      };

      \node[text width=8cm,below] at ([shift={(0,-4cm)}]a0.south) {
        \begin{noteBox}
          3. 当有人往\cola{primary}发消息失败后就会和\colc{接班人}说,请求\textbf{view-change}。
          然后,\colc{接班人}会:
          \begin{enumerate}
          \item 确认一下\cola{primary}是不是真的没了
          \item 如果不是,那说明是这个小节点自己的问题
          \end{enumerate}
        \end{noteBox}
      };
    \end{scope}

    \begin{scope}[shift={(6cm,-6cm)}]
      \node[myNode,fill=red!40!black,minimum height=4em] (a0) {dead};
      \draw[latex-, very thick] (a0) -- (30:3cm) node[myNode,fill=\mycolc,name=a2]{2};
      \draw[latex-, very thick] (a0) -- (0:3cm) node[myNode,fill=\mycolaa,name=a4]{4};

      \draw[latex-, very thick] (a0) -- (300:3cm) node[myNode,fill=\mycola,
      name=a1,minimum height=4em]{primary};
      \draw[latex-, very thick] (a1) -- (270:3cm) node[myNode,fill=\mycolaa,name=a3]{3};

      \draw[-latex,very thick] (a4) -- (a1);
      \draw[-latex,very thick] (a2) -- (a1);
      \draw[-latex,very thick, red] (a3) -- node[midway] {\emoji{cross-mark}}(a0);

      % \node[left,text width=4cm] at ([xshift=-0.5cm]a3.west) {
      %   \begin{leftDialogBox}
      %     欸? \cola{primary}老不回我,它应该没了,\colc{你}上呗?
      %   \end{leftDialogBox}
      % };

      \node[right,text width=4cm] at ([xshift=0.5cm,yshift=0.5cm]a1.east) {
        \begin{rightDialogBox}
          诶呦,还真没了...那好吧。\textbf{大家听好了},\cola{primary}该换我
          了,\colc{下一个是节点\textbf{2}},你们都记住啊。
        \end{rightDialogBox}
      };

      \node[text width=6cm,below] at ([shift={(0,-3cm)}]a0.south) {
        \begin{noteBox}
          4.如果确实\colc{接班人自己}也没有连上\cola{主节点},则广播\textbf{view-change},并让大家了
          解下一任\colc{接班人}。
        \end{noteBox}
      };
    \end{scope}

  \end{tikzpicture}
\end{center}

\begin{myBox}{听一个共识-留一手版 (Follow-me Consensus)}

\end{myBox}

\end{document}


% Local Variables:
% TeX-engine: luatex
% TeX-command-extra-options: "-shell-escape"
% End:
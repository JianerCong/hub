\documentclass[dvipsnames]{article}

\title{RSA}
\usepackage{geometry}\geometry{
  a4paper,
  total={170mm,257mm},
  left=20mm,
  top=20mm,
}

\usepackage[skip=10pt plus1pt, indent=0pt]{parskip}
% Color
\newcommand{\mycola}{MidnightBlue}
\newcommand{\mycolb}{Mahogany}
\newcommand{\mycolc}{OliveGreen}

\newcommand{\cola}[1]{\textcolor{\mycola}{#1}}
\newcommand{\colb}[1]{\textcolor{\mycolb}{#1}}
\newcommand{\colc}[1]{\textcolor{\mycolc}{#1}}
\newcommand{\Cola}[1]{\textcolor{\mycola}{\textbf{#1}}}

% \let\emph\relax % there's no \RedeclareTextFontCommand
% \DeclareTextFontCommand{\emph}{\bfseries}
\renewcommand{\emph}[1]{\texbf{#1}}

\usepackage{fontspec}
\setmonofont{Cascadia}[
Path=/usr/share/fonts/truetype/Cascadia_Code/,
Scale=0.85,
Extension = .ttf,
UprightFont=*Code,              %find CascadiaCode.ttf
BoldFont=*CodePL,               %find CascadiaCodePL.ttf ...
ItalicFont=*CodeItalic,
BoldItalicFont=*CodePLItalic
]
\usepackage{minted}
\usepackage{tcolorbox}
\tcbuselibrary{skins}
\tcbuselibrary{minted}
\usepackage{tikz}
\usetikzlibrary{shapes} % ellipse node shape
\usetikzlibrary{shapes.multipart} % for line breaks in node text
\usetikzlibrary{arrows.meta}    %-o arrow head
\usetikzlibrary{arrows}
\usetikzlibrary{matrix}


% Redefine em
% latex.sty just do: \DeclareTextFontCommand{\emph}{\em}

\let\emph\relax % there's no \RedeclareTextFontCommand
\DeclareTextFontCommand{\emph}{\bfseries\em}
\usepackage{amsmath}

% ??? still xelatex?
% \usepackage{xeCJK}
\usepackage{emoji}

\date{\today}
\author{me}

\newtcblisting{simplec}{
  listing engine=minted,
  minted language=c++,
  minted style=vs,
  minted options={fontsize=\small,autogobble,
    % framesep=1cm
  },
  tile,
  listing only,
  % bottom=0cm,
  % nobeforeafter, 
  boxsep=0mm,
  left=1mm,
  opacityback=0.5,
  colback=gray!20
}

\begin{document}
\maketitle

\section{What is asymmetric encryption?}

The first step in knowing how to encrypt a message is understanding \Cola{
  asymmetric encryption} (also called \Cola{public key encryption} ).

\emoji{turtle}  : You probably know that all the emails you send are ``in the clear'' for anyone
sitting in between you and your recipients's email provider to read. That's not
great. How do you fix this? You could use a cryptographic primitive like
AES-GCM. To do that, you would need to set up a shared symmetric secret for each
person that wants to message you.\\
\emoji{parrot}  :But you don't known in advance who'll want to message you.\\
\emoji{turtle}  :This is where \Cola{ asymmetric encryption} helps by allowing anyone in
possesion of your public key to encrypt messages to you.\\
Furthurmore, you are the only one who can decrypt these messages using the
associated private key that only you own.

\tikzstyle{myMatrix}=[matrix of nodes,below right,
nodes={right},                  %apply to all nodes
row sep=1cm,column sep=2cm]
\tikzstyle{every node}=[inner sep=0pt]
\begin{center}
  \begin{tikzpicture}
    \matrix (M1) [myMatrix]{
      \emoji{crocodile} \\
      \emoji{lizard} & \emoji{t-rex}\\
      \emoji{snake} \\
    };
    \foreach \i in {1,2,3}{
      \draw[-latex] (M1-\i-1) -- (M1-2-2);
    }

    % The left speaking box
    \node[left,text width=5cm] at ([xshift=-0.5cm]M1-3-1.west) {
      \begin{tcolorbox}[tile,
        nobeforeafter,
        boxsep=0pt,
        % show bounding box,
        colback=gray!10,
        overlay={
          \begin{scope}
            % \fill[gray!10] (frame.east) circle (2pt);
            \fill[gray!10] (frame.east) --
            +(0,2mm) --
            +(3mm,0) --
            +(0,-2mm)
            ;
          \end{scope}
        }
        ]
        I can't decrypt other's messages.
      \end{tcolorbox}
    };

    \node[right,text width=5cm] at ([xshift=0.5cm]M1-2-2.east) {
      \begin{tcolorbox}[tile,
        nobeforeafter,
        boxsep=0pt,
        % show bounding box,
        colback=gray!10,
        overlay={
          \begin{scope}
            % \fill[gray!10] (frame.east) circle (2pt);
            \fill[gray!10] (frame.west) --
            +(0,2mm) --
            +(-3mm,0) --
            +(0,-2mm)
            ;
          \end{scope}
        }
        ]
        I will decrypt them all.
      \end{tcolorbox}
     };
  \end{tikzpicture}
\end{center}

\end{document}


% Local Variables:
% TeX-engine: luatex
% TeX-command-extra-options: "-shell-escape"
% End:
% \documentclass[dvipsnames]{article}
\documentclass[dvipsnames]{ctexart}

\title{FollowMe 共识}
\usepackage{geometry}\geometry{
  a4paper,
  total={170mm,257mm},
  left=20mm,
  top=20mm,
}


\usepackage{svg}

\usepackage[skip=5pt plus1pt, indent=0pt]{parskip}
% Color
\newcommand{\mycola}{MidnightBlue}
\newcommand{\mycolb}{Mahogany}
\newcommand{\mycolc}{OliveGreen}

\newcommand{\cola}[1]{\textcolor{\mycola}{#1}}
\newcommand{\colb}[1]{\textcolor{\mycolb}{#1}}
\newcommand{\colc}[1]{\textcolor{\mycolc}{#1}}
\newcommand{\Cola}[1]{\textcolor{\mycola}{\textbf{#1}}}

% \let\emph\relax % there's no \RedeclareTextFontCommand
% \DeclareTextFontCommand{\emph}{\bfseries}
\renewcommand{\emph}[1]{\texbf{#1}}

\usepackage{fontspec}

\setmonofont{Cascadia}[
Path=/usr/share/fonts/truetype/Cascadia_Code/,
Scale=0.85,
Extension = .ttf,
UprightFont=*Code,              %find CascadiaCode.ttf
BoldFont=*CodePL,               %find CascadiaCodePL.ttf ...
ItalicFont=*CodeItalic,
BoldItalicFont=*CodePLItalic
]
% --------------------------------------------------
% Windows
% \setmonofont{Cascadia}[
% Path=C:/Windows/Fonts/,
% Extension = .ttf,
% UprightFont=*Mono,              %find CascadiaMono.ttf
% BoldFont=*Code,               %find CascadiaCodePL.ttf ...
% ItalicFont=*Code,
% BoldItalicFont=*Code
% ]


\usepackage{minted}
\usepackage{tcolorbox}
\tcbuselibrary{skins}
\tcbuselibrary{minted}
\usepackage{tikz}
\usetikzlibrary{shapes} % ellipse node shape
\usetikzlibrary{shapes.multipart} % for line breaks in node text
\usetikzlibrary{arrows.meta}    %-o arrow head
\usetikzlibrary{arrows}
\usetikzlibrary{matrix}
\usetikzlibrary{snakes}

\usepackage{amsmath}
% ??? still xelatex?
% \usepackage{xeCJK}
\usepackage{emoji}
% \setemojifont{NotoColorEmoji.ttf}[Path=C:/Users/congj/repo/myFonts/]
% \setemojifont{TwitterColorEmoji-SVGinOT.ttf}[Path=C:/Users/congj/repo/myFonts/]


\newtcolorbox[auto counter]{myBox}[2][]{
  fonttitle=\bfseries,title={共识~\thetcbcounter: #2},#1
}
\newtcolorbox[]{noteBox}[1][]{
  tile,left=1mm,nobeforeafter,fontupper=\small,#1
}

\tikzstyle{myNode}=[inner sep=2pt,circle,text=white]
\date{\today}
\author{作者}

\newtcblisting{simplec}{
  listing engine=minted,
  minted language=c++,
  minted style=vs,
  minted options={fontsize=\small,autogobble,
    % framesep=1cm
  },
  tile,
  listing only,
  % bottom=0cm,
  % nobeforeafter, 
  boxsep=0mm,
  left=1mm,
  opacityback=0.5,
  colback=gray!20
}
\tcbuselibrary{breakable}
\newtcblisting{simplepy}{
  listing engine=minted,
  minted language=python,
  minted style=vs,
  minted options={fontsize=\small,autogobble,
    % framesep=1cm
  },
  tile,
  listing only,
  % bottom=0cm,
  % nobeforeafter,
  boxsep=0mm,
  left=1mm,
  opacityback=0.5,
  colback=gray!20,
  breakable
}
\newtcolorbox{blackbox}{tile,colback=black,colupper=white,nobeforeafter,halign=flush center}

\tikzstyle{myMatrix}=[matrix of nodes,below right,
nodes={above,text centered},                  %apply to all nodes
row sep=1cm,column sep=2cm]
\tikzstyle{every node}=[inner sep=0pt]

\newcommand\uptoleft[3][-o]{\draw[very thick,#1](#2.south) |- (#3.west);}
\newcommand\uptodown[3][-o]{\draw[very thick,#1](#2.south) to [out=270,in=90] (#3.north);}
\newcommand\downtoup[3][-latex]{\draw[very thick,#1](#2.north) to [out=90,in=270] (#3.south);}

\newcommand\lefttoright[3][-latex]{\draw[very thick,#1](#2.east) to[out=0,in=180] (#3.west);}
\newcommand\lefttodown[3][-latex]{\draw[very thick,#1](#2.east) to[out=0,in=90] (#3.north);}


\newtcolorbox{leftDialogBox}{
  tile, nobeforeafter, boxsep=0pt,
  % show bounding box,
  colback=\mycola!10,
  overlay={
    \begin{scope}
      % \fill[gray!10] (frame.east) circle (2pt);
      \fill[\mycola!10] (frame.east) --
      +(0,2mm) --
      +(3mm,0) --
      +(0,-2mm)
      ;
    \end{scope}
  }}


\newtcolorbox{rightDialogBox}{
  tile, nobeforeafter, boxsep=0pt,
  % show bounding box,
  colback=\mycola!10,
  overlay={
    \begin{scope}
      % \fill[gray!10] (frame.east) circle (2pt);
      \fill[\mycola!10] (frame.west) --
      +(0,2mm) --
      +(-3mm,0) --
      +(0,-2mm);
    \end{scope}
  }}

\newcommand{\mycolaa}{\mycola!20}


% --------------------------------------------------
\begin{document}
\maketitle
% \section{听一个共识}
让我们来想一个共识。优先要求这几点:
\begin{itemize}
\item 单节点可跑
\item 可(相对方便地)动态新增节点
\end{itemize}
\emoji{parrot} : 那么最简单的应该就是
\begin{myBox}{听一个共识 Listen-to-one Consensus}
  \label{cons-a}
  \begin{itemize}
  \item 第一个起的节点为Primary。
  \item 新增的节点往Primary上连接。
  \end{itemize}
\end{myBox}

\emoji{parrot} : 这个其实本质上就是在集群内部的一个 Server-Client 架构嘛。
也就是所有人都只听Primary的。
\begin{center}
  \begin{tikzpicture}
    % \tikzstyle{every node}=[inner sep=2pt]
    \node[myNode,fill=\mycola,minimum height=4em] (a1) {primary};
    \foreach \d in {0,60,...,300}{
      \draw[latex-, very thick] (a1) -- (\d:3cm) node[myNode,fill=gray]{新增节点};
      % \node[myNode,fill=gray]  at (\d:2cm) {};
    }
  \end{tikzpicture}
\end{center}
这个的缺点就是\cola{主机down了,集群就down了}。它的优点就是新增的节点只需要和主
节点连接就行,而且非primary彼此不需要知道彼此。

\begin{tikzpicture}
  \emoji{parrot} 那么代码该怎么写呢?\\
  \emoji{turtle} 在写代码之前我们先定义共识所需要的接口吧。
\end{tikzpicture}
\subsection{共识需要的接口}
抽象一点讲,共识最少需要两组外部接口。
\begin{enumerate}
\item 网络,这个包括
  \begin{itemize}
  \item 接受外部的请求(监听)
  \item P2P沟通 (监听 + 发送)
  \end{itemize}
\item 执行,把一个命令写到内部存储。
\end{enumerate}
网络的接口大概可以定义为:
\begin{simplepy}
class IEndpointBasedNetworkable(ABC):
    def listen(self,
               target: str,     # The target, e.g. "/"
               handler: Callable[[str,str],Optional[str]]
               # The handler, accepts
               #  - endpoint, e.g. "localhost:8080"
               #  - request data
               # returns the result
               ):
        pass
    def listened_endpoint(self) -> str:
        pass
    def send(self,endpoint: str, target: str, data: str) -> Optional[str]:
        # endpoint: the target, e.g. "localhost:8080"
        # target: e.g. "/hi"
        pass
    def clear(self):                # clears up all the listeners
        pass
\end{simplepy}
而执行的接口,大概是这样的:
\begin{simplepy}
class IExecutable(ABC):
    def execute(self,command: str) -> str:
        pass
\end{simplepy}

\subsection{\textbf{听一个共识},代码实现}
那么现在我们可以开始码了,先从构建函数开始,除了执行和网络以外,构建函数还接受一
个可选的\texttt{nodeToConnect},如果没给,则说明这个是第一个起的节点(primary)。
\begin{simplepy}
class ListenToOneConsensus:
    def __init__(self,
                 n: IEndpointBasedNetworkable,
                 e: IExecutable,
                 nodeToConnect=''):
        self.net = n
        self.exe = e
        if nodeToConnect:
            self.primary = nodeToConnect
            self.is_primary = False
            self.ask_primary_for_entry()
            self.start_listening_as_sub()
            print_mt(f'{self.net.listened_endpoint()} started as sub 🐸')
        else:
            print_mt(f'{self.net.listened_endpoint()} started as primary 🐸')
            self.is_primary = True
            self.start_listening_as_primary()

\end{simplepy}
\emoji{parrot} : 那么具体primary和非primary节点都要监听哪些事件呢?

在介绍具体要监听哪些事件之前,我们最好先看一下在“时光静好”的情况下,集群是怎么
工作的:
\begin{center}
  \begin{tikzpicture}
    \node[myNode,fill=\mycola,minimum height=4em] (a1) {primary};
    \foreach \d in {300,20}{
      \draw[latex-, very thick] (a1) -- (\d:3cm) node[myNode,fill=gray,name=s-\d]{新增节点};
      % \node[myNode,fill=gray]  at (\d:2cm) {};
    }
    \node (a1n) [text width=7cm,above left] at ([shift={(-2cm,0cm)}]a1.west){
      \begin{tcolorbox}[tile,fontupper=\small,
        left=0mm,
        boxsep=0mm
        ]
        \begin{itemize}
        \item 每收到一个\colc{请求}:
          \begin{enumerate}
          \item 先执行一下,
          \item 然后添加到一个\cola{笔记本}里,
          \item 然后发给所有\colb{已知下级}。
          \end{enumerate}
        \item 每当有\colb{新节点}要求加入时:
          \begin{enumerate}
          \item 先把\colb{它}加入\colb{已知下级},
          \item 然后把\cola{笔记本}发给\colc{它}一份。
          \end{enumerate}
        \end{itemize}
      \end{tcolorbox}
    };
    \draw[help lines,-o] (a1n) -- (a1);

    \node (s2n) [text width=7cm,above] at ([shift={(0cm,2cm)}] s-20.north){
      \begin{tcolorbox}[tile,fontupper=\small,
        left=0.5mm,
        boxsep=0mm]
        在刚开始的时候请求\cola{primary}加入。
        \begin{itemize}
        \item 每收到来自\cola{primary}的\colc{请求}: 执行
        \item 每收到来自\colb{client}的\colc{请求}: 传给\cola{primary}
        \end{itemize}
      \end{tcolorbox}
    };

    \draw[help lines,o-] (s-20) -- (s2n);
  \end{tikzpicture}
\end{center}
Emm,那么看来,primary和非主节点所要监听的(和发送的)也就很明显了。首先我们来看
primary需要用到的方法。
\begin{simplepy}
    def start_listening_as_primary(self):
        self.known_subs = []
        self.command_history = []
        self.net.listen('/pleaseAddMe',
                        self.handle_add_new_node)
        self.net.listen('/pleaseExecuteThis',
                        self.handle_execute_for_primary)

    def handle_add_new_node(self,sub_endpoint: str,
                            data: str) -> str:
        """[For primary] to add new node"""
        self.known_subs.append(sub_endpoint)
        return f"""
        Dear {sub_endpoint}
            You are in. and here is what we have so far:
            {','.join(self.command_history)}
                   Sincerely
                   {self.net.listened_endpoint()}, The primary.
        """

    def handle_execute_for_primary(self, endpoint: str,
                                   data: str) -> str:
        cmd = data
        self.command_history.append(cmd)
        self.exe.execute(cmd)
        for sub in self.known_subs:
            r = self.net.send(sub,'/pleaseExecuteThis',cmd)
            if r == None:
                print(f'❌️ Node-{sub} is down,kick it off the group.')
                self.known_subs.remove(sub)

        return f"""
        Dear {endpoint}
             Your request has been carried out by our group.
             Members: {self.known_subs}
                 Sincerely
                 {self.net.listened_endpoint()}, The primary.
        """
\end{simplepy}
然后是其他节点用到的方法:
\begin{simplepy}
    def ask_primary_for_entry(self):
        r = self.net.send(self.primary,
                    '/pleaseAddMe',f"""
                    Hi primary {self.primary},
                       please add me in the group
                          Regards {self.net.listened_endpoint()}
                    """
                          )
        if r == None:
            raise Exception('Failed to join the group')
        # In response, the primary should send you the history .
        cmd = r.split('\n')[3]  # forth line is the command
        if cmd:
            self.exe.execute(cmd)

    def start_listening_as_sub(self):
        self.net.listen('/pleaseExecuteThis',
                        self.handle_execute_for_sub)
    def handle_execute_for_sub(self,endpoint: str,data: str) -> str:
        cmd = data
        if (endpoint == self.primary):
            self.exe.execute(cmd)
            return f"""
            Dear boss {self.primary},
                 Mission [{data}] is accomplished.
                     Sincerely
                     {self.net.listened_endpoint()}
            """
        else:
            # forward
            r = self.net.send(self.primary,
                                 '/pleaseExecuteThis',data)
            if r == None:
                raise Exception('failed to forward message to primary')
            return r
\end{simplepy}
所以最后总结起来应该就是:
\begin{simplepy}
class ListenToOneConsensus:
    def __init__(self,
                 n: IEndpointBasedNetworkable,
                 e: IExecutable,
                 nodeToConnect=''):
        self.net = n
        self.exe = e
        if nodeToConnect:
            self.primary = nodeToConnect
            self.is_primary = False
            self.ask_primary_for_entry()
            self.start_listening_as_sub()
            print_mt(f'{self.net.listened_endpoint()} started as sub 🐸')
        else:
            print_mt(f'{self.net.listened_endpoint()} started as primary 🐸')
            self.is_primary = True
            self.start_listening_as_primary()
    def start_listening_as_primary(self):
        self.known_subs = []
        self.command_history = []
        self.net.listen('/pleaseAddMe',
                        self.handle_add_new_node)
        self.net.listen('/pleaseExecuteThis',
                        self.handle_execute_for_primary)

    def handle_add_new_node(self,sub_endpoint: str,
                            data: str) -> str:
        """[For primary] to add new node"""
        self.known_subs.append(sub_endpoint)
        return f"""
        Dear {sub_endpoint}
            You are in. and here is what we have so far:
            {','.join(self.command_history)}
                   Sincerely
                   {self.net.listened_endpoint()}, The primary.
        """

    def handle_execute_for_primary(self, endpoint: str,
                                   data: str) -> str:
        cmd = data
        self.command_history.append(cmd)
        self.exe.execute(cmd)
        for sub in self.known_subs:
            r = self.net.send(sub,'/pleaseExecuteThis',cmd)
            if r == None:
                print(f'❌️ Node-{sub} is down,kick it off the group.')
                self.known_subs.remove(sub)

        return f"""
        Dear {endpoint}
             Your request has been carried out by our group.
             Members: {self.known_subs}
                 Sincerely
                 {self.net.listened_endpoint()}, The primary.
        """

    # --------------------------------------------------
    def ask_primary_for_entry(self):
        r = self.net.send(self.primary,
                    '/pleaseAddMe',f"""
                    Hi primary {self.primary},
                       please add me in the group
                          Regards {self.net.listened_endpoint()}
                    """
                          )
        if r == None:
            raise Exception('Failed to join the group')
        # In response, the primary should send you the history .
        cmd = r.split('\n')[3]  # forth line is the command
        if cmd:
            self.exe.execute(cmd)

    def start_listening_as_sub(self):
        self.net.listen('/pleaseExecuteThis',
                        self.handle_execute_for_sub)
    def handle_execute_for_sub(self,endpoint: str,data: str) -> str:
        cmd = data
        if (endpoint == self.primary):
            self.exe.execute(cmd)
            return f"""
            Dear boss {self.primary},
                 Mission [{data}] is accomplished.
                     Sincerely
                     {self.net.listened_endpoint()}
            """
        else:
            # forward
            r = self.net.send(self.primary,
                                 '/pleaseExecuteThis',data)
            if r == None:
                raise Exception('failed to forward message to primary')
            return r

\end{simplepy}
那么让我写一下mocked方法然后测试一下吧,mocked网络和存储如下:
\begin{simplepy}
  
\end{simplepy}
而起集群的话则是要这样:
\begin{simplepy}
  
\end{simplepy}

这里我们可以同过命令行动态的向集群发送:
\begin{itemize}
\item \texttt{append} : 新增一个节点
\item \texttt{kick} \cola{\texttt{<节点名>}} : 删除一个节点
\item \cola{\texttt{<节点名>}} \texttt{<命令>} : 向某节点发起命令
\end{itemize}

示例:
\begin{tcolorbox}[breakable]
  
\begin{verbatim}
mark started as primary 🐸 节点【mark】加入了集群----------------------------
Adding handler: mark-/pleaseAddMe
Adding handler: mark-/pleaseExecuteThis
 Calling handler: mark-/pleaseAddMe  with data:
  
                    Hi primary mark,
                       please add me in the group
                          Regards moai
                     
Got result: 
        Dear moai
            You are in. and here is what we have so far:
            
                   Sincerely
                   mark, The primary.
         
🦜  [moai] Exec:              
Adding handler: moai-/pleaseExecuteThis
moai started as sub 🐸 节点【moai】加入了集群------------------------
 Calling handler: mark-/pleaseAddMe  with data:
  
                    Hi primary mark,
                       please add me in the group
                          Regards lay
                     
Got result: 
        Dear lay
            You are in. and here is what we have so far:
            
                   Sincerely
                   mark, The primary.
         
🦜  [lay] Exec:              
Adding handler: lay-/pleaseExecuteThis
lay started as sub 🐸 节点【lay】加入了集群-----------------------
Enter: moai aaa 往【moai】发送命令“aaa”---------------------------
 Sending aaa to moai 
 Calling handler: moai-/pleaseExecuteThis  with data:
  aaa 
 Calling handler: mark-/pleaseExecuteThis  with data:
  aaa 
🦜  [mark] Exec: aaa 
 Calling handler: moai-/pleaseExecuteThis  with data:
  aaa 
🦜  [moai] Exec: aaa 
Got result: 
            Dear boss mark,
                 Mission [aaa] is accomplished.
                     Sincerely
                     moai
             
 Calling handler: lay-/pleaseExecuteThis  with data:
  aaa 
🦜  [lay] Exec: aaa 
Got result: 
            Dear boss mark,
                 Mission [aaa] is accomplished.
                     Sincerely
                     lay
             
Got result: 
        Dear moai
             Your request has been carried out by our group.
             Members: ['moai', 'lay']
                 Sincerely
                 mark, The primary.
         
Got result: 
        Dear moai
             Your request has been carried out by our group.
             Members: ['moai', 'lay']
                 Sincerely
                 mark, The primary.
         
Enter: lay bbb 往【lay】发送命令“bbb”----------------------------
 Sending bbb to lay 
 Calling handler: lay-/pleaseExecuteThis  with data:
  bbb 
 Calling handler: mark-/pleaseExecuteThis  with data:
  bbb 
🦜  [mark] Exec: bbb 
 Calling handler: moai-/pleaseExecuteThis  with data:
  bbb 
🦜  [moai] Exec: bbb 
Got result: 
            Dear boss mark,
                 Mission [bbb] is accomplished.
                     Sincerely
                     moai
             
 Calling handler: lay-/pleaseExecuteThis  with data:
  bbb 
🦜  [lay] Exec: bbb 
Got result: 
            Dear boss mark,
                 Mission [bbb] is accomplished.
                     Sincerely
                     lay
             
Got result: 
        Dear lay
             Your request has been carried out by our group.
             Members: ['moai', 'lay']
                 Sincerely
                 mark, The primary.
         
Got result: 
        Dear lay
             Your request has been carried out by our group.
             Members: ['moai', 'lay']
                 Sincerely
                 mark, The primary.
         
Enter: kick lay  把【lay】节点踢了-------------------------
 Kicking node lay 
Removing handler: lay-/pleaseExecuteThis
Enter: moai ccc 往【moai】发送命令”ccc“--------------------------- 
 Sending ccc to moai 
 Calling handler: moai-/pleaseExecuteThis  with data:
  ccc 
 Calling handler: mark-/pleaseExecuteThis  with data:
  ccc 
🦜  [mark] Exec: ccc 
 Calling handler: moai-/pleaseExecuteThis  with data:
  ccc 
🦜  [moai] Exec: ccc 
Got result: 
            Dear boss mark,
                 Mission [ccc] is accomplished.
                     Sincerely
                     moai
             
 Calling handler: lay-/pleaseExecuteThis  with data:
  ccc 
Handler lay-/pleaseExecuteThis not found
Got result: None 
❌️ Node-lay is down,kick it off the group.
Got result: 
        Dear moai
             Your request has been carried out by our group.
             Members: ['moai']
                 Sincerely
                 mark, The primary.
         
Got result: 
        Dear moai
             Your request has been carried out by our group.
             Members: ['moai']
                 Sincerely
                 mark, The primary.
         
Enter: append 新增节点【vest】----------------------------------
 Appending node 
 Calling handler: mark-/pleaseAddMe  with data:
  
                    Hi primary mark,
                       please add me in the group
                          Regards vest
                     
Got result: 
        Dear vest
            You are in. and here is what we have so far:
            aaa,bbb,ccc
                   Sincerely
                   mark, The primary.
         
🦜  [vest] Exec:             aaa,bbb,ccc 
Adding handler: vest-/pleaseExecuteThis
vest started as sub 🐸
Enter: vest ddd 往【vest】节点发送命令"ddd"------------------------
 Sending ddd to vest 往vest发送命令ddd---------------------------
 Calling handler: vest-/pleaseExecuteThis  with data:
  ddd 
 Calling handler: mark-/pleaseExecuteThis  with data:
  ddd 
🦜  [mark] Exec: ddd 
 Calling handler: moai-/pleaseExecuteThis  with data:
  ddd 
🦜  [moai] Exec: ddd 
Got result: 
            Dear boss mark,
                 Mission [ddd] is accomplished.
                     Sincerely
                     moai
             
 Calling handler: vest-/pleaseExecuteThis  with data:
  ddd 
🦜  [vest] Exec: ddd 
Got result: 
            Dear boss mark,
                 Mission [ddd] is accomplished.
                     Sincerely
                     vest
             
Got result: 
        Dear vest
             Your request has been carried out by our group.
             Members: ['moai', 'vest']
                 Sincerely
                 mark, The primary.
         
Got result: 
        Dear vest
             Your request has been carried out by our group.
             Members: ['moai', 'vest']
                 Sincerely
                 mark, The primary.

\end{verbatim}

\end{tcolorbox}
\section{听一个共识-留一手版}

让我们稍微再复杂复杂一点,看看能不能新增这两个功能:
\begin{enumerate}
\item 在\cola{primary}死的时候希望可以有接班的。
\item 我们希望新增节点的时候可以连接到任何一个节点。
\end{enumerate}
\emoji{parrot} : 既然\cola{primary}可以给命令编号,那么自然也可以给新加的节点编号
不是吗? 而我们也就可以用这个编号来决定\cola{谁是下一任primary}。

是的那么这样以来我们的集群就变成了一个有序号的了:
\begin{center}
  \begin{tikzpicture}
    \begin{scope}
      \node[myNode,fill=\mycola,minimum height=4em] (a0) {primary};
      \draw[latex-, very thick] (a0) -- (30:3cm) node[myNode,fill=gray,name=a2]{...};
      \draw[latex-, very thick] (a0) -- (0:3cm) node[myNode,fill=gray,name=a4]{...};

      \draw[latex-, very thick] (a0) -- (300:3cm) node[myNode,fill=\mycolc,name=a1]{1};
      \draw[latex-, very thick] (a1) -- (270:3cm) node[myNode,fill=gray,name=a3]{...};

      \node[text width=6cm,below] at ([shift={(0,-3cm)}]a0.south) {
        \begin{noteBox}
          1.新增的节点可以通过\textbf{任意现
            有集群节点}连接,之后会管\cola{primary}要一个序号。
        \end{noteBox}
      };
      \node[left,text width=2cm] at ([xshift=-0.5cm]a3.west) {
        \begin{leftDialogBox}
          求加入
        \end{leftDialogBox}
      };

      \foreach \n in {a2}{
        \node[right,text width=2cm] at ([xshift=0.5cm,yshift=0.5cm]\n.east) {
          \begin{rightDialogBox}
            求加入
          \end{rightDialogBox}
        };
      }

    \end{scope}


    \begin{scope}[shift={(9.5cm,0)}]
      \node[myNode,fill=\mycola,minimum height=4em] (a0) {primary};
      \draw[latex-, very thick] (a0) -- (30:3cm) node[myNode,fill=\mycolaa,name=a2]{2};
      \draw[latex-, very thick] (a0) -- (0:3cm) node[myNode,fill=\mycolaa,name=a4]{4};

      \draw[latex-, very thick] (a0) -- (300:3cm) node[myNode,fill=\mycolc,name=a1]{1};
      \draw[latex-, very thick] (a1) -- (270:3cm) node[myNode,fill=\mycolaa,name=a3]{3};

      \draw[-latex,very thick] (a4) -- (a1);
      \draw[-latex,very thick] (a2) -- (a1);
      \draw[-latex,very thick] (a3) -- (a0);

      \node[text width=6cm,below] at ([shift={(0,-3cm)}]a0.south) {
        \begin{noteBox}
          2.在完成以下两个步骤后,新增的节点就算加入集群了:
          \begin{enumerate}
          \item 记住\cola{primary}和下任\cola{primary}的地址
          \item 获得自己的序列号。
          \item 同步命令历史记录
          \end{enumerate}
        \end{noteBox}
      };

      \node[left,text width=4cm] at ([xshift=-0.5cm]a0.west) {
        \begin{leftDialogBox}
          \small
          来吧,给你们\textbf{过往我们执行的记录},如果我没了就找下一任\colc{节点1}吧。
        \end{leftDialogBox}
      };
    \end{scope}
      \end{tikzpicture}
\end{center}

\begin{center}
  \begin{tikzpicture}
    \begin{scope}
      \node[myNode,fill=red!40!black,minimum height=4em] (a0) {primary};
      \draw[latex-, very thick] (a0) -- (30:3cm) node[myNode,fill=\mycolaa,name=a2]{2};
      \draw[latex-, very thick] (a0) -- (0:3cm) node[myNode,fill=\mycolaa,name=a4]{4};

      \draw[latex-, very thick] (a0) -- (300:3cm) node[myNode,fill=\mycolc,name=a1]{1};
      \draw[latex-, very thick] (a1) -- (270:3cm) node[myNode,fill=\mycolaa,name=a3]{3};

      \draw[-latex,very thick] (a4) -- (a1);
      \draw[-latex,very thick] (a2) -- (a1);
      \draw[-latex,very thick, red] (a3) -- node[midway] {\emoji{cross-mark}}(a0);

      \node[left,text width=4cm] at ([xshift=-0.5cm]a3.west) {
        \begin{leftDialogBox}
          ? \cola{primary}老不回我,它应该没了,\colc{你}上呗?
        \end{leftDialogBox}
      };

      \node[right,text width=4cm] at ([xshift=0.5cm,yshift=0.5cm]a1.east) {
        \begin{rightDialogBox}
          是吗? 让我确认一下....喂喂喂? \cola{primary}在吗?
        \end{rightDialogBox}
      };

      \node[text width=8cm,below] at ([shift={(0,-4cm)}]a0.south) {
        \begin{noteBox}
          3. 当有人往\cola{primary}发消息失败后就会和\colc{接班人}说,请求\textbf{view-change}。
          然后,\colc{接班人}会:
          \begin{enumerate}
          \item 确认一下\cola{primary}是不是真的没了
          \item 如果不是,那说明是这个小节点自己的问题
          \end{enumerate}
        \end{noteBox}
      };
    \end{scope}

    \begin{scope}[shift={(6cm,-6cm)}]
      \node[myNode,fill=red!40!black,minimum height=4em] (a0) {dead};
      \draw[latex-, very thick] (a0) -- (30:3cm) node[myNode,fill=\mycolc,name=a2]{2};
      \draw[latex-, very thick] (a0) -- (0:3cm) node[myNode,fill=\mycolaa,name=a4]{4};

      \draw[latex-, very thick,red] (a0) -- node[midway] {\emoji{cross-mark}}
      (300:3cm) node[myNode,fill=\mycola,
      name=a1,minimum height=4em]{primary};
      \draw[latex-, very thick] (a1) -- (270:3cm) node[myNode,fill=\mycolaa,name=a3]{3};

      \draw[-latex,very thick] (a4) -- (a1);
      \draw[-latex,very thick] (a2) -- (a1);
      \draw[-latex,very thick, red] (a3) -- node[midway,sloped] {\emoji{cross-mark}}(a0);

      % \node[left,text width=4cm] at ([xshift=-0.5cm]a3.west) {
      %   \begin{leftDialogBox}
      %     欸? \cola{primary}老不回我,它应该没了,\colc{你}上呗?
      %   \end{leftDialogBox}
      % };

      \node[right,text width=4cm] at ([xshift=0.5cm,yshift=0.5cm]a1.east) {
        \begin{rightDialogBox}
          诶呦,还真没了...那好吧。\textbf{大家听好了},\cola{primary}该换我
          了,\colc{下一个是节点\textbf{2}},你们都记住啊。
        \end{rightDialogBox}
      };

      \node[text width=6cm,below] at ([shift={(0,-3cm)}]a0.south) {
        \begin{noteBox}
          4.如果确实\colc{接班人自己}也没有连上\cola{主节点},则广播\textbf{view-change},并让大家了
          解下一任\colc{接班人}。
        \end{noteBox}
      };
    \end{scope}

  \end{tikzpicture}
\end{center}

\begin{myBox}{听一个共识-留一手版 (Follow-me Consensus)}
  \begin{itemize}
  \item 每个新加的的节点会管主节点要一个序号
  \item 之后所有的节点都会记住两个地址: \cola{主节点}和\colb{下一任主节点}。(存
    在一个列表里)
  \item 当节点们受到来自\colb{下一任主节点}的\textbf{上任通知时},切换主节点。
  \end{itemize}
\end{myBox}

\emoji{parrot} : 那如果一个节点同时收到两个\cola{primary}的消息怎么办呢?
那么就和其他的共识算法一样(Raft,PBFT), 节点只需要接受最新的命令就行了。
% \section{静态节点集群共识}

很多教科书里的集群都是基于\Cola{静态集群}的,也就是说集群节点的数量从头到尾都不变,
而且是已知的。这些共识包括:Raft,PBFT 等等。它们多多少少都要依赖一个``投票''的机制。

\emoji{parrot} : 确实,你得知道有一共有多少人才能知道\cola{“大多数”}是多少人。

\begin{center}
  \begin{tikzpicture}
    \matrix[column sep=1cm,row sep=1cm,nodes={myNode,fill=gray}]{
      \node[name=a1]{1}; & 
      \node[name=a2]{2}; \\
      \node[name=a3]{3}; & 
      \node[name=a4]{4}; \\
    };

    \foreach \i / \j in {1/2,1/3,1/4,2/3,2/4,3/4}{
      \draw[thick,latex-latex] (a\j) -- (a\i);
    }

  \end{tikzpicture}
\end{center}

那么对于这样的共识他们的网络应该是这样的:
\begin{simplepy}
from abc import ABC
class IStaticIdBasedNetworkable(ABC):
    def listen(self,
               target: str,
               handler: Callable[[int,str],Optional[str]]
               ):
        pass
    def my_id(self) -> int:
        pass
    def all_ids(self) -> list[int]:
        pass
    def send(self,i: int, target: str, data: str) -> Optional[str]:
        # i: the target, e.g. 2
        # target: e.g. "/hi"
        pass
    def clear(self):                # clears up all the listeners
        pass
\end{simplepy}
那么我们看到大部分和之前\texttt{IEndpointBasedNetworkable}是一样的,有才华的同学
可能想把这两个写成一个\texttt{template<T>},不过请暂时先别。

\section{例子: Raft 共识实现}
那么基于上面的这个接口,我们的Raft共识应该就是这样的:
\begin{simplepy}
class RaftConsensus:
    def __init__(self,
                 n: IStaticIdBasedNetworkable,
                 e: IExecutable):
        self.net = n
        self.exe = e
        self.lock_for_patience = Lock()
        self.primary = None
        self.dynasty = 0
        self.voted_dynasty = dict()
        self.lock_for_voted_dynasty = Lock()

        self.start_listening_as_follower()

    def start_listening_as_follower(self):

        self.say('Started as follower')
        self.net.clear()
        self.net.listen('/pleaseAppendEntry', self.handle_append_entry)
        self.net.listen('/pleaseExecute', self.handle_execute)
        self.net.listen('/pleaseVoteMe', self.handle_ask_for_vote)
        self.net.listen('/IamThePrimary', self.handle_primary)

        Thread(target=self.start_internal_clock).start()  # here the ctor ends

    def start_internal_clock(self):
        self.comfort()

        p = -1
        with self.lock_for_patience:
            p = self.patience

        while p > 0:
            sleep(2)
            with self.lock_for_patience:
                self.patience -= 1
                p = self.patience
            self.say(f' patience >> {self.patience}, 🐢PR: {self.primary}')

        self.have_enough()

    def have_enough(self):
        self.say(f' have had enough')
        self.ask_for_votes()

    def handle_append_entry(self, i: int, d: str) -> Optional[str]:
        if i == self.primary:
            self.comfort()
            self.exe.execute(d)     # only primary will append entry
            return f"""
            Dear {i}, the primary
                 I have appended your requested entry. ✅️ {d}
                        Yours {self.net.my_id()}
            """
        return f"""
        Sorry, {i} ❌️
             My primary is {self.primary}, and I will only listen to him/her.
                Regards {self.net.my_id()}
        """

    def comfort(self):          # reset timer
        with self.lock_for_patience:
            self.patience = random.randrange(start=10,stop=20,step=2)
            self.say(f'patience set to = {self.patience}')

    def say(self,s: str):
        print_mt(f'[{self.net.my_id()}]: ' + s)

    def ask_for_votes(self):
        self.dynasty += 1
        with self.lock_for_voted_dynasty:
            self.voted_dynasty[self.dynasty] = self.net.my_id()
        c : int = 1
        for i in [i for i in self.net.all_ids() if i != self.net.my_id()]:
            r: str = self.net.send(i,'/pleaseVoteMe',f"""
            Hi {i}, vote me in the
               {self.dynasty}
                     compaign
                       Yours, {self.net.my_id()}
            """)
            if (r and r.startswith('Yes')):
                c += 1

        if c >= len(self.net.all_ids()) / 2:
            self.start_listening_as_primary(c)
        else:
            self.start_listening_as_follower()

    def start_listening_as_primary(self, c: int):
        self.primary = self.net.my_id()

        self.say(f'Listening as primary')
        for i in [i for i in self.net.all_ids() if i != self.net.my_id()]:
            self.net.send(i,'/IamThePrimary',f"""
            Hi, I got {c} votes to be the new primary,{self.net.my_id()}
                the dynasty number is:
                   {self.dynasty}
                               Regards {self.net.my_id()}
            """)

        self.net.clear()
        self.net.listen('/pleaseExecute',self.handle_execute_for_primary)



        # Make the following a thread
        self.alive = True
        Thread(target=self.heart_beat).start()

    def heart_beat(self):
        while self.alive:
            sleep(6)
            self.say(f'[Primary 💙]')
            for i in [i for i in self.net.all_ids() if i != self.net.my_id()]:
                self.net.send(i,'/pleaseAppendEntry','Beep')


    def handle_ask_for_vote(self, i: int, d: str) -> Optional[str]:
        self.primary = None
        self.comfort()
        asked_dynasty = int(d.splitlines()[2].strip())
        with self.lock_for_voted_dynasty:
            if asked_dynasty not in self.voted_dynasty:
                self.voted_dynasty[asked_dynasty] = i
                return 'Yes'
        return f"""Sorry,
              I have already voted for this term for
                     {self.voted_dynasty[asked_dynasty]}
                               Regards {self.net.my_id()}
        """

    def handle_primary(self, i: int, d: str) -> Optional[str]:
        self.primary = i
        self.dynasty = int(d.splitlines()[3].strip())
        return f"""
        Dear primary {i},
             from now on I will listen to you in term {self.dynasty}.
                      Regards {self.net.my_id()}
        """

    def handle_execute(self, i: int, d: str) -> Optional[str]:
        if self.primary == None:
            return f"""
            Dear client {i},
                Sorry, our group is electing primary for the moment,
                please try again latter.
                    Regards {self.net.my_id()}
            """
        return self.net.send(self.primary,'/pleaseExecute',d)  # forward

    def handle_execute_for_primary(self, i: int, d: str) -> Optional[str]:
        self.say(f'Primary executing {d}')
        self.exe.execute(d)
        subs = [s for s in self.net.all_ids() if s != self.net.my_id()]
        self.say(f'Asking subs: {subs}')
        for sub in subs:
            self.say(f'Ask {sub} to execute {d}')
            self.net.send(sub,'/pleaseAppendEntry',d)
            # Thread(target=
            #        lambda : 
            #        ).start()
        return f"""
        Dear client,
             Your requests {d} have been executed by our group
                  Regards {self.net.my_id()}, the primary
        """
\end{simplepy}
然后,如果想要把集群跑起来,我们还需要一个假的网络和执行器:
\begin{simplepy}
class MockedExecutable(IExecutable):
    def __init__(self, i: int):
        self.id = i
    def execute(self,command: str):
        print_mt(f'🦜 {S.RED} [{self.id}] Exec: {command} {S.NOR}')


class MockedIdNetworkNode(IStaticIdBasedNetworkable):
    def __init__(self,i:int):
        self.id = i
    def my_id(self) -> int:
        return self.id
    def all_ids(self) -> list[int]:
        return network_nodes
    def send(self,i: int, target: str, data: str) -> Optional[str]:
        k = f'{i}-{target}'
        print_mt(f'{S.CYAN} Calling handler: {k} {S.NOR} with data:\n {S.CYAN} {data} {S.NOR}')
        if k in network_hub:
            r = network_hub[k](self.id,data)
        else:
            print_mt(f'Handler {k} not found')
            r = None
        print_mt(f'Got result:{S.GREEN} {r} {S.NOR}')
        return r

    def clear(self):                # clears up all the listeners
        with lock_for_netwok_hub:
            for k in list(network_hub.keys()):
                if k.startswith(f'{self.id}-'):
                    print_mt(f'Removing handler: {k}')
                    network_hub.pop(k)

    def listen(self,
               target: str,
               handler: Callable[[int,str],Optional[str]]
               ):
        k = f'{self.id}-{target}'
        print_mt(f'Adding handler: {k}')
        network_hub[k] = handler

\end{simplepy}
\begin{tcolorbox}
  \emoji{parrot} 这里我们还用了一些帮手函数:这些函数没啥重要的,就是用来方便打印
  日志的。
  \begin{simplepy}
import threading
from threading import Thread, Timer, Lock
import random
# from random import randrange
from time import sleep

lock_for_print = Lock()
def print_mt(*args,**kwargs):
    with lock_for_print:
        print(*args,**kwargs)

class S:
    HEADER = '\033[95m'
    BLUE = '\033[94m'
    CYAN = '\033[96m'
    GREEN = '\033[92m'
    MAG = '\033[93m'
    RED = '\033[91m'
    NOR = '\033[0m'
    BOLD = '\033[1m'
    UNDERLINE = '\033[4m'
  \end{simplepy}
\end{tcolorbox}
而使用的话就是这么用:
\begin{simplepy}
network_hub : dict[str, Callable[[int,str],Optional[str]]] = dict()
lock_for_netwok_hub = Lock()
network_nodes = list(range(3))

network_iface = [MockedIdNetworkNode(i) for i in network_nodes]
executor_iface = [MockedExecutable(i) for i in network_nodes]
consensus_nodes = [
    RaftConsensus(network_iface[i],executor_iface[i]) for i in network_nodes
]

nClient = MockedIdNetworkNode(1234)
while True:
    # reply = input('Enter cmd: <id> <cmd>')
    reply = input('Enter: ')
    if reply == 'stop': break

    l = reply.split(' ')

    # Simulate a dead node
    if l[0] == 'down':
        i = int(l[1])
        print_mt(f'{S.HEADER} Turning down node-{i} {S.NOR}')
        consensus_nodes[i].net.clear()
        consensus_nodes[i].alive = False
    else:
        # Execute a command
        i = int(l[0])
        print_mt(f'{S.HEADER} Sending {l[1]} to node-{i} {S.NOR}')
        nClient.send(i,'/pleaseExecute',l[1])
  
\end{simplepy}

然后在命令行里我们就可以部署集群并发送消息了。 如果想要往集群发送命令使用:
\begin{center}
  \cola{\texttt{<节点id>}} \colb{\texttt{<命令>}}
\end{center}
比如:
\begin{center}
  \cola{\texttt{1}} \colb{\texttt{aaa}} (往1发送\texttt{aaa}) \\
  \cola{\texttt{2}} \colb{\texttt{bbb}} (往2发送\texttt{bbb})\\
\end{center}

而如果想要模拟一个节点down了,那就用:
\begin{center}
  \cola{\texttt{down}} \colb{\texttt{<节点id>}}
\end{center}

 \begin{tcolorbox}[breakable,title=Raft共识模拟的命令行输出(部分没啥用的输出被省略了)]
  \begin{verbatim}
从这里开始 --------------------------------------------------
[0]: Started as follower
Adding handler: 0-/pleaseAppendEntry
Adding handler: 0-/pleaseExecute
Adding handler: 0-/pleaseVoteMe
Adding handler: 0-/IamThePrimary
[0]: patience set to = 18
[1]: Started as follower
Adding handler: 1-/pleaseAppendEntry
Adding handler: 1-/pleaseExecute
Adding handler: 1-/pleaseVoteMe
Adding handler: 1-/IamThePrimary
[1]: patience set to = 18
[2]: Started as follower
Adding handler: 2-/pleaseAppendEntry
Adding handler: 2-/pleaseExecute
Adding handler: 2-/pleaseVoteMe
Adding handler: 2-/IamThePrimary
[2]: patience set to = 16
Enter: [0]:  patience >> 17, 🐢PR: None
[1]:  patience >> 17, 🐢PR: None
... 略 ... 
[2]:  patience >> 0, 🐢PR: None
[2]:  have had enough
 Calling handler: 0-/pleaseVoteMe  with data:
  
            Hi 0, vote me in the
               1
                     compaign
                       Yours, 2
             
[0]: patience set to = 10
Got result: Yes 
 Calling handler: 1-/pleaseVoteMe  with data:
  
            Hi 1, vote me in the
               1
                     compaign
                       Yours, 2
             
[1]: patience set to = 14
Got result: Yes 
[2]: Listening as primary
 Calling handler: 0-/IamThePrimary  with data:
  
            Hi, I got 3 votes to be the new primary,2
                the dynasty number is:
                   1
                               Regards 2
             
Got result: 
        Dear primary 2,
             from now on I will listen to you in term 1.
                      Regards 0
         
 Calling handler: 1-/IamThePrimary  with data:
  
            Hi, I got 3 votes to be the new primary,2
                the dynasty number is:
                   1
                               Regards 2
             
Got result: 
        Dear primary 2,
             from now on I will listen to you in term 1.
                      Regards 1
         
Removing handler: 2-/pleaseAppendEntry
Removing handler: 2-/pleaseExecute
Removing handler: 2-/pleaseVoteMe
Removing handler: 2-/IamThePrimary
Adding handler: 2-/pleaseExecute
从这里开始 2 就是 primary--------------------------------------------------
[0]:  patience >> 9, 🐢PR: 2
[1]:  patience >> 13, 🐢PR: 2
[0]:  patience >> 8, 🐢PR: 2
[1]:  patience >> 12, 🐢PR: 2
[0]:  patience >> 7, 🐢PR: 2
[1]:  patience >> 11, 🐢PR: 2
[2]: [Primary 💙]
 Calling handler: 0-/pleaseAppendEntry  with data:
  Beep 
[0]: patience set to = 18
🦜  [0] Exec: Beep 
..... 略
这里往节点 1 发送命令 (a)--------------------------------------------------
 Sending a to node-1 
 Calling handler: 1-/pleaseExecute  with data:
  a 
 Calling handler: 2-/pleaseExecute  with data:
  a 
[2]: Primary executing a
🦜  [2] Exec: a  (这里primary执行了) --------------------------------------------------
[2]: Asking subs: [0, 1]
[2]: Ask 0 to execute a
 Calling handler: 0-/pleaseAppendEntry  with data:
  a 
[0]: patience set to = 16
🦜  [0] Exec: a  (0 执行了) --------------------------------------------------
Got result: 
            Dear 2, the primary
                 I have appended your requested entry. ✅️ a
                        Yours 0
             
[2]: Ask 1 to execute a
 Calling handler: 1-/pleaseAppendEntry  with data:
  a 
[1]: patience set to = 18
🦜  [1] Exec: a  (1 也执行了)
Got result: 
            Dear 2, the primary
                 I have appended your requested entry. ✅️ a
                        Yours 1
             
Got result: 
        Dear client,
             Your requests a have been executed by our group
                  Regards 2, the primary
         
... 略...
down 2
(这里关掉节点2 (primary)心跳不再发送) ------------------------------------------
 Turning down node-2 
Removing handler: 2-/pleaseExecute
Enter: [2]: [Primary 💙]
[1]:  patience >> 10, 🐢PR: 2
 Calling handler: 0-/pleaseAppendEntry  with data:
  Beep 
[0]: patience set to = 14
🦜  [0] Exec: Beep 
Got result: 
            Dear 2, the primary
                 I have appended your requested entry. ✅️ Beep
                        Yours 0
             
 Calling handler: 1-/pleaseAppendEntry  with data:
  Beep 
[1]: patience set to = 16
🦜  [1] Exec: Beep 
Got result: 
            Dear 2, the primary
                 I have appended your requested entry. ✅️ Beep
                        Yours 1
(还在等心跳的其他节点) --------------------------------------------------
[0]:  patience >> 13, 🐢PR: 2
... 略 ...
[0]:  patience >> 1, 🐢PR: 2
[1]:  patience >> 3, 🐢PR: 2
[0]:  patience >> 0, 🐢PR: 2
[0]:  have had enough
(没有等到心跳,节点0发起投票)----------------------------------------------
 Calling handler: 1-/pleaseVoteMe  with data:
  
            Hi 1, vote me in the
               2
                     compaign
                       Yours, 0
             
[1]: patience set to = 10
Got result: Yes 
 Calling handler: 2-/pleaseVoteMe  with data:
  
            Hi 2, vote me in the
               2
                     compaign
                       Yours, 0
             
Handler 2-/pleaseVoteMe not found
Got result: None 
--------------------------------------------------
(节点0 成为新的primary)
[0]: Listening as primary
 Calling handler: 1-/IamThePrimary  with data:
  
            Hi, I got 2 votes to be the new primary,0
                the dynasty number is:
                   2
                               Regards 0
             
Got result: 
        Dear primary 0,
             from now on I will listen to you in term 2.
                      Regards 1
         
 Calling handler: 2-/IamThePrimary  with data:
  
            Hi, I got 2 votes to be the new primary,0
                the dynasty number is:
                   2
                               Regards 0
             
Handler 2-/IamThePrimary not found
Got result: None 
Removing handler: 0-/pleaseAppendEntry
Removing handler: 0-/pleaseExecute
Removing handler: 0-/pleaseVoteMe
Removing handler: 0-/IamThePrimary
Adding handler: 0-/pleaseExecute
[1]:  patience >> 9, 🐢PR: 0
[1]:  patience >> 8, 🐢PR: 0
[1]:  patience >> 7, 🐢PR: 0
节点 0 发送心跳 --------------------------------------------------
[0]: [Primary 💙] 
 Calling handler: 1-/pleaseAppendEntry  with data:
  Beep 
[1]: patience set to = 10
🦜  [1] Exec: Beep 
Got result: 
            Dear 0, the primary
                 I have appended your requested entry. ✅️ Beep
                        Yours 1
             
 Calling handler: 2-/pleaseAppendEntry  with data:
  Beep 
Handler 2-/pleaseAppendEntry not found
Got result: None 
[1]:  patience >> 9, 🐢PR: 0
[1]:  patience >> 8, 🐢PR: 0
[0]: [Primary 💙]
..略...
\end{verbatim}
\end{tcolorbox}

% Local Variables:
% TeX-engine: luatex
% TeX-command-extra-options: "-shell-escape"
% TeX-master: "m.tex"
% End:


\end{document}


% Local Variables:
% TeX-engine: luatex
% TeX-command-extra-options: "-shell-escape"
% TeX-master: "m.tex"
% End:

\section{听一个共识}
让我们来想一个共识。优先要求这几点:
\begin{itemize}
\item 单节点可跑
\item 可(相对方便地)动态新增节点
\end{itemize}
\emoji{parrot} : 那么最简单的应该就是
\begin{myBox}{听一个共识 Listen-to-one Consensus}
  \label{cons-a}
  \begin{itemize}
  \item 第一个起的节点为Primary。
  \item 新增的节点往Primary上连接。
  \end{itemize}
\end{myBox}

\emoji{parrot} : 这个其实本质上就是在集群内部的一个 Server-Client 架构嘛。
也就是所有人都只听Primary的。
\begin{center}
  \begin{tikzpicture}
    % \tikzstyle{every node}=[inner sep=2pt]
    \node[myNode,fill=\mycola,minimum height=4em] (a1) {primary};
    \foreach \d in {0,60,...,300}{
      \draw[latex-, very thick] (a1) -- (\d:3cm) node[myNode,fill=gray]{新增节点};
      % \node[myNode,fill=gray]  at (\d:2cm) {};
    }
  \end{tikzpicture}
\end{center}
这个的缺点就是\cola{主机down了,集群就down了}。它的优点就是新增的节点只需要和主
节点连接就行,而且非primary彼此不需要知道彼此。

\begin{tikzpicture}
  \emoji{parrot} 那么代码该怎么写呢?\\
  \emoji{turtle} 在写代码之前我们先定义共识所需要的接口吧。
\end{tikzpicture}
\subsection{共识需要的接口}
抽象一点讲,共识最少需要两组外部接口。
\begin{enumerate}
\item 网络,这个包括
  \begin{itemize}
  \item 接受外部的请求(监听)
  \item P2P沟通 (监听 + 发送)
  \end{itemize}
\item 执行,把一个命令写到内部存储。
\end{enumerate}
网络的接口大概可以定义为:
\begin{simplepy}
class IEndpointBasedNetworkable(ABC):
    def listen(self,
               target: str,     # The target, e.g. "/"
               handler: Callable[[str,str],Optional[str]]
               # The handler, accepts
               #  - endpoint, e.g. "localhost:8080"
               #  - request data
               # returns the result
               ):
        pass
    def listened_endpoint(self) -> str:
        pass
    def send(self,endpoint: str, target: str, data: str) -> Optional[str]:
        # endpoint: the target, e.g. "localhost:8080"
        # target: e.g. "/hi"
        pass
    def clear(self):                # clears up all the listeners
        pass
\end{simplepy}
而执行的接口,大概是这样的:
\begin{simplepy}
class IExecutable(ABC):
    def execute(self,command: str) -> str:
        pass
\end{simplepy}

\subsection{\textbf{听一个共识},代码实现}
那么现在我们可以开始码了,先从构建函数开始,除了执行和网络以外,构建函数还接受一
个可选的\texttt{nodeToConnect},如果没给,则说明这个是第一个起的节点(primary)。
\begin{simplepy}
class ListenToOneConsensus:
    def __init__(self,
                 n: IEndpointBasedNetworkable,
                 e: IExecutable,
                 nodeToConnect=''):
        self.net = n
        self.exe = e
        if nodeToConnect:
            self.primary = nodeToConnect
            self.is_primary = False
            self.ask_primary_for_entry()
            self.start_listening_as_sub()
            print_mt(f'{self.net.listened_endpoint()} started as sub 🐸')
        else:
            print_mt(f'{self.net.listened_endpoint()} started as primary 🐸')
            self.is_primary = True
            self.start_listening_as_primary()

\end{simplepy}
\emoji{parrot} : 那么具体primary和非primary节点都要监听哪些事件呢?

在介绍具体要监听哪些事件之前,我们最好先看一下在“时光静好”的情况下,集群是怎么
工作的:
\begin{center}
  \begin{tikzpicture}
    \node[myNode,fill=\mycola,minimum height=4em] (a1) {primary};
    \foreach \d in {300,20}{
      \draw[latex-, very thick] (a1) -- (\d:3cm) node[myNode,fill=gray,name=s-\d]{新增节点};
      % \node[myNode,fill=gray]  at (\d:2cm) {};
    }
    \node (a1n) [text width=7cm,above left] at ([shift={(-2cm,0cm)}]a1.west){
      \begin{tcolorbox}[tile,fontupper=\small,
        left=0mm,
        boxsep=0mm
        ]
        \begin{itemize}
        \item 每收到一个\colc{请求}:
          \begin{enumerate}
          \item 先执行一下,
          \item 然后添加到一个\cola{笔记本}里,
          \item 然后发给所有\colb{已知下级}。
          \end{enumerate}
        \item 每当有\colb{新节点}要求加入时:
          \begin{enumerate}
          \item 先把\colb{它}加入\colb{已知下级},
          \item 然后把\cola{笔记本}发给\colc{它}一份。
          \end{enumerate}
        \end{itemize}
      \end{tcolorbox}
    };
    \draw[help lines,-o] (a1n) -- (a1);

    \node (s2n) [text width=7cm,above] at ([shift={(0cm,2cm)}] s-20.north){
      \begin{tcolorbox}[tile,fontupper=\small,
        left=0.5mm,
        boxsep=0mm]
        在刚开始的时候请求\cola{primary}加入。
        \begin{itemize}
        \item 每收到来自\cola{primary}的\colc{请求}: 执行
        \item 每收到来自\colb{client}的\colc{请求}: 传给\cola{primary}
        \end{itemize}
      \end{tcolorbox}
    };

    \draw[help lines,o-] (s-20) -- (s2n);
  \end{tikzpicture}
\end{center}
Emm,那么看来,primary和非主节点所要监听的(和发送的)也就很明显了。首先我们来看
primary需要用到的方法。
\begin{simplepy}
    def start_listening_as_primary(self):
        self.known_subs = []
        self.command_history = []
        self.net.listen('/pleaseAddMe',
                        self.handle_add_new_node)
        self.net.listen('/pleaseExecuteThis',
                        self.handle_execute_for_primary)

    def handle_add_new_node(self,sub_endpoint: str,
                            data: str) -> str:
        """[For primary] to add new node"""
        self.known_subs.append(sub_endpoint)
        return f"""
        Dear {sub_endpoint}
            You are in. and here is what we have so far:
            {','.join(self.command_history)}
                   Sincerely
                   {self.net.listened_endpoint()}, The primary.
        """

    def handle_execute_for_primary(self, endpoint: str,
                                   data: str) -> str:
        cmd = data
        self.command_history.append(cmd)
        self.exe.execute(cmd)
        for sub in self.known_subs:
            r = self.net.send(sub,'/pleaseExecuteThis',cmd)
            if r == None:
                print(f'❌️ Node-{sub} is down,kick it off the group.')
                self.known_subs.remove(sub)

        return f"""
        Dear {endpoint}
             Your request has been carried out by our group.
             Members: {self.known_subs}
                 Sincerely
                 {self.net.listened_endpoint()}, The primary.
        """
\end{simplepy}
然后是其他节点用到的方法:
\begin{simplepy}
    def ask_primary_for_entry(self):
        r = self.net.send(self.primary,
                    '/pleaseAddMe',f"""
                    Hi primary {self.primary},
                       please add me in the group
                          Regards {self.net.listened_endpoint()}
                    """
                          )
        if r == None:
            raise Exception('Failed to join the group')
        # In response, the primary should send you the history .
        cmd = r.split('\n')[3]  # forth line is the command
        if cmd:
            self.exe.execute(cmd)

    def start_listening_as_sub(self):
        self.net.listen('/pleaseExecuteThis',
                        self.handle_execute_for_sub)
    def handle_execute_for_sub(self,endpoint: str,data: str) -> str:
        cmd = data
        if (endpoint == self.primary):
            self.exe.execute(cmd)
            return f"""
            Dear boss {self.primary},
                 Mission [{data}] is accomplished.
                     Sincerely
                     {self.net.listened_endpoint()}
            """
        else:
            # forward
            r = self.net.send(self.primary,
                                 '/pleaseExecuteThis',data)
            if r == None:
                raise Exception('failed to forward message to primary')
            return r
\end{simplepy}
所以最后总结起来应该就是:
\begin{simplepy}
class ListenToOneConsensus:
    def __init__(self,
                 n: IEndpointBasedNetworkable,
                 e: IExecutable,
                 nodeToConnect=''):
        self.net = n
        self.exe = e
        if nodeToConnect:
            self.primary = nodeToConnect
            self.is_primary = False
            self.ask_primary_for_entry()
            self.start_listening_as_sub()
            print_mt(f'{self.net.listened_endpoint()} started as sub 🐸')
        else:
            print_mt(f'{self.net.listened_endpoint()} started as primary 🐸')
            self.is_primary = True
            self.start_listening_as_primary()
    def start_listening_as_primary(self):
        self.known_subs = []
        self.command_history = []
        self.net.listen('/pleaseAddMe',
                        self.handle_add_new_node)
        self.net.listen('/pleaseExecuteThis',
                        self.handle_execute_for_primary)

    def handle_add_new_node(self,sub_endpoint: str,
                            data: str) -> str:
        """[For primary] to add new node"""
        self.known_subs.append(sub_endpoint)
        return f"""
        Dear {sub_endpoint}
            You are in. and here is what we have so far:
            {','.join(self.command_history)}
                   Sincerely
                   {self.net.listened_endpoint()}, The primary.
        """

    def handle_execute_for_primary(self, endpoint: str,
                                   data: str) -> str:
        cmd = data
        self.command_history.append(cmd)
        self.exe.execute(cmd)
        for sub in self.known_subs:
            r = self.net.send(sub,'/pleaseExecuteThis',cmd)
            if r == None:
                print(f'❌️ Node-{sub} is down,kick it off the group.')
                self.known_subs.remove(sub)

        return f"""
        Dear {endpoint}
             Your request has been carried out by our group.
             Members: {self.known_subs}
                 Sincerely
                 {self.net.listened_endpoint()}, The primary.
        """

    # --------------------------------------------------
    def ask_primary_for_entry(self):
        r = self.net.send(self.primary,
                    '/pleaseAddMe',f"""
                    Hi primary {self.primary},
                       please add me in the group
                          Regards {self.net.listened_endpoint()}
                    """
                          )
        if r == None:
            raise Exception('Failed to join the group')
        # In response, the primary should send you the history .
        cmd = r.split('\n')[3]  # forth line is the command
        if cmd:
            self.exe.execute(cmd)

    def start_listening_as_sub(self):
        self.net.listen('/pleaseExecuteThis',
                        self.handle_execute_for_sub)
    def handle_execute_for_sub(self,endpoint: str,data: str) -> str:
        cmd = data
        if (endpoint == self.primary):
            self.exe.execute(cmd)
            return f"""
            Dear boss {self.primary},
                 Mission [{data}] is accomplished.
                     Sincerely
                     {self.net.listened_endpoint()}
            """
        else:
            # forward
            r = self.net.send(self.primary,
                                 '/pleaseExecuteThis',data)
            if r == None:
                raise Exception('failed to forward message to primary')
            return r

\end{simplepy}
那么让我写一下mocked方法然后测试一下吧,mocked网络和存储如下:
\begin{simplepy}
  
\end{simplepy}
而起集群的话则是要这样:
\begin{simplepy}
  
\end{simplepy}

这里我们可以同过命令行动态的向集群发送:
\begin{itemize}
\item \texttt{append} : 新增一个节点
\item \texttt{kick} \cola{\texttt{<节点名>}} : 删除一个节点
\item \cola{\texttt{<节点名>}} \texttt{<命令>} : 向某节点发起命令
\end{itemize}

示例:
\begin{tcolorbox}[breakable]
  
\begin{verbatim}
mark started as primary 🐸 节点【mark】加入了集群----------------------------
Adding handler: mark-/pleaseAddMe
Adding handler: mark-/pleaseExecuteThis
 Calling handler: mark-/pleaseAddMe  with data:
  
                    Hi primary mark,
                       please add me in the group
                          Regards moai
                     
Got result: 
        Dear moai
            You are in. and here is what we have so far:
            
                   Sincerely
                   mark, The primary.
         
🦜  [moai] Exec:              
Adding handler: moai-/pleaseExecuteThis
moai started as sub 🐸 节点【moai】加入了集群------------------------
 Calling handler: mark-/pleaseAddMe  with data:
  
                    Hi primary mark,
                       please add me in the group
                          Regards lay
                     
Got result: 
        Dear lay
            You are in. and here is what we have so far:
            
                   Sincerely
                   mark, The primary.
         
🦜  [lay] Exec:              
Adding handler: lay-/pleaseExecuteThis
lay started as sub 🐸 节点【lay】加入了集群-----------------------
Enter: moai aaa 往【moai】发送命令“aaa”---------------------------
 Sending aaa to moai 
 Calling handler: moai-/pleaseExecuteThis  with data:
  aaa 
 Calling handler: mark-/pleaseExecuteThis  with data:
  aaa 
🦜  [mark] Exec: aaa 
 Calling handler: moai-/pleaseExecuteThis  with data:
  aaa 
🦜  [moai] Exec: aaa 
Got result: 
            Dear boss mark,
                 Mission [aaa] is accomplished.
                     Sincerely
                     moai
             
 Calling handler: lay-/pleaseExecuteThis  with data:
  aaa 
🦜  [lay] Exec: aaa 
Got result: 
            Dear boss mark,
                 Mission [aaa] is accomplished.
                     Sincerely
                     lay
             
Got result: 
        Dear moai
             Your request has been carried out by our group.
             Members: ['moai', 'lay']
                 Sincerely
                 mark, The primary.
         
Got result: 
        Dear moai
             Your request has been carried out by our group.
             Members: ['moai', 'lay']
                 Sincerely
                 mark, The primary.
         
Enter: lay bbb 往【lay】发送命令“bbb”----------------------------
 Sending bbb to lay 
 Calling handler: lay-/pleaseExecuteThis  with data:
  bbb 
 Calling handler: mark-/pleaseExecuteThis  with data:
  bbb 
🦜  [mark] Exec: bbb 
 Calling handler: moai-/pleaseExecuteThis  with data:
  bbb 
🦜  [moai] Exec: bbb 
Got result: 
            Dear boss mark,
                 Mission [bbb] is accomplished.
                     Sincerely
                     moai
             
 Calling handler: lay-/pleaseExecuteThis  with data:
  bbb 
🦜  [lay] Exec: bbb 
Got result: 
            Dear boss mark,
                 Mission [bbb] is accomplished.
                     Sincerely
                     lay
             
Got result: 
        Dear lay
             Your request has been carried out by our group.
             Members: ['moai', 'lay']
                 Sincerely
                 mark, The primary.
         
Got result: 
        Dear lay
             Your request has been carried out by our group.
             Members: ['moai', 'lay']
                 Sincerely
                 mark, The primary.
         
Enter: kick lay  把【lay】节点踢了-------------------------
 Kicking node lay 
Removing handler: lay-/pleaseExecuteThis
Enter: moai ccc 往【moai】发送命令”ccc“--------------------------- 
 Sending ccc to moai 
 Calling handler: moai-/pleaseExecuteThis  with data:
  ccc 
 Calling handler: mark-/pleaseExecuteThis  with data:
  ccc 
🦜  [mark] Exec: ccc 
 Calling handler: moai-/pleaseExecuteThis  with data:
  ccc 
🦜  [moai] Exec: ccc 
Got result: 
            Dear boss mark,
                 Mission [ccc] is accomplished.
                     Sincerely
                     moai
             
 Calling handler: lay-/pleaseExecuteThis  with data:
  ccc 
Handler lay-/pleaseExecuteThis not found
Got result: None 
❌️ Node-lay is down,kick it off the group.
Got result: 
        Dear moai
             Your request has been carried out by our group.
             Members: ['moai']
                 Sincerely
                 mark, The primary.
         
Got result: 
        Dear moai
             Your request has been carried out by our group.
             Members: ['moai']
                 Sincerely
                 mark, The primary.
         
Enter: append 新增节点【vest】----------------------------------
 Appending node 
 Calling handler: mark-/pleaseAddMe  with data:
  
                    Hi primary mark,
                       please add me in the group
                          Regards vest
                     
Got result: 
        Dear vest
            You are in. and here is what we have so far:
            aaa,bbb,ccc
                   Sincerely
                   mark, The primary.
         
🦜  [vest] Exec:             aaa,bbb,ccc 
Adding handler: vest-/pleaseExecuteThis
vest started as sub 🐸
Enter: vest ddd 往【vest】节点发送命令"ddd"------------------------
 Sending ddd to vest 往vest发送命令ddd---------------------------
 Calling handler: vest-/pleaseExecuteThis  with data:
  ddd 
 Calling handler: mark-/pleaseExecuteThis  with data:
  ddd 
🦜  [mark] Exec: ddd 
 Calling handler: moai-/pleaseExecuteThis  with data:
  ddd 
🦜  [moai] Exec: ddd 
Got result: 
            Dear boss mark,
                 Mission [ddd] is accomplished.
                     Sincerely
                     moai
             
 Calling handler: vest-/pleaseExecuteThis  with data:
  ddd 
🦜  [vest] Exec: ddd 
Got result: 
            Dear boss mark,
                 Mission [ddd] is accomplished.
                     Sincerely
                     vest
             
Got result: 
        Dear vest
             Your request has been carried out by our group.
             Members: ['moai', 'vest']
                 Sincerely
                 mark, The primary.
         
Got result: 
        Dear vest
             Your request has been carried out by our group.
             Members: ['moai', 'vest']
                 Sincerely
                 mark, The primary.

\end{verbatim}

\end{tcolorbox}
\section{听一个共识-留一手版}

让我们稍微再复杂复杂一点,看看能不能新增这两个功能:
\begin{enumerate}
\item 在\cola{primary}死的时候希望可以有接班的。
\item 我们希望新增节点的时候可以连接到任何一个节点。
\end{enumerate}
\emoji{parrot} : 既然\cola{primary}可以给命令编号,那么自然也可以给新加的节点编号
不是吗? 而我们也就可以用这个编号来决定\cola{谁是下一任primary}。

是的那么这样以来我们的集群就变成了一个有序号的了:
\begin{center}
  \begin{tikzpicture}
    \begin{scope}
      \node[myNode,fill=\mycola,minimum height=4em] (a0) {primary};
      \draw[latex-, very thick] (a0) -- (30:3cm) node[myNode,fill=gray,name=a2]{...};
      \draw[latex-, very thick] (a0) -- (0:3cm) node[myNode,fill=gray,name=a4]{...};

      \draw[latex-, very thick] (a0) -- (300:3cm) node[myNode,fill=\mycolc,name=a1]{1};
      \draw[latex-, very thick] (a1) -- (270:3cm) node[myNode,fill=gray,name=a3]{...};

      \node[text width=6cm,below] at ([shift={(0,-3cm)}]a0.south) {
        \begin{noteBox}
          1.新增的节点可以通过\textbf{任意现
            有集群节点}连接,之后会管\cola{primary}要一个序号。
        \end{noteBox}
      };
      \node[left,text width=2cm] at ([xshift=-0.5cm]a3.west) {
        \begin{leftDialogBox}
          求加入
        \end{leftDialogBox}
      };

      \foreach \n in {a2}{
        \node[right,text width=2cm] at ([xshift=0.5cm,yshift=0.5cm]\n.east) {
          \begin{rightDialogBox}
            求加入
          \end{rightDialogBox}
        };
      }

    \end{scope}


    \begin{scope}[shift={(9.5cm,0)}]
      \node[myNode,fill=\mycola,minimum height=4em] (a0) {primary};
      \draw[latex-, very thick] (a0) -- (30:3cm) node[myNode,fill=\mycolaa,name=a2]{2};
      \draw[latex-, very thick] (a0) -- (0:3cm) node[myNode,fill=\mycolaa,name=a4]{4};

      \draw[latex-, very thick] (a0) -- (300:3cm) node[myNode,fill=\mycolc,name=a1]{1};
      \draw[latex-, very thick] (a1) -- (270:3cm) node[myNode,fill=\mycolaa,name=a3]{3};

      \draw[-latex,very thick] (a4) -- (a1);
      \draw[-latex,very thick] (a2) -- (a1);
      \draw[-latex,very thick] (a3) -- (a0);

      \node[text width=6cm,below] at ([shift={(0,-3cm)}]a0.south) {
        \begin{noteBox}
          2.在完成以下两个步骤后,新增的节点就算加入集群了:
          \begin{enumerate}
          \item 记住\cola{primary}和下任\cola{primary}的地址
          \item 获得自己的序列号。
          \item 同步命令历史记录
          \end{enumerate}
        \end{noteBox}
      };

      \node[left,text width=4cm] at ([xshift=-0.5cm]a0.west) {
        \begin{leftDialogBox}
          \small
          来吧,给你们\textbf{过往我们执行的记录},如果我没了就找下一任\colc{节点1}吧。
        \end{leftDialogBox}
      };
    \end{scope}
      \end{tikzpicture}
\end{center}

\begin{center}
  \begin{tikzpicture}
    \begin{scope}
      \node[myNode,fill=red!40!black,minimum height=4em] (a0) {primary};
      \draw[latex-, very thick] (a0) -- (30:3cm) node[myNode,fill=\mycolaa,name=a2]{2};
      \draw[latex-, very thick] (a0) -- (0:3cm) node[myNode,fill=\mycolaa,name=a4]{4};

      \draw[latex-, very thick] (a0) -- (300:3cm) node[myNode,fill=\mycolc,name=a1]{1};
      \draw[latex-, very thick] (a1) -- (270:3cm) node[myNode,fill=\mycolaa,name=a3]{3};

      \draw[-latex,very thick] (a4) -- (a1);
      \draw[-latex,very thick] (a2) -- (a1);
      \draw[-latex,very thick, red] (a3) -- node[midway] {\emoji{cross-mark}}(a0);

      \node[left,text width=4cm] at ([xshift=-0.5cm]a3.west) {
        \begin{leftDialogBox}
          ? \cola{primary}老不回我,它应该没了,\colc{你}上呗?
        \end{leftDialogBox}
      };

      \node[right,text width=4cm] at ([xshift=0.5cm,yshift=0.5cm]a1.east) {
        \begin{rightDialogBox}
          是吗? 让我确认一下....喂喂喂? \cola{primary}在吗?
        \end{rightDialogBox}
      };

      \node[text width=8cm,below] at ([shift={(0,-4cm)}]a0.south) {
        \begin{noteBox}
          3. 当有人往\cola{primary}发消息失败后就会和\colc{接班人}说,请求\textbf{view-change}。
          然后,\colc{接班人}会:
          \begin{enumerate}
          \item 确认一下\cola{primary}是不是真的没了
          \item 如果不是,那说明是这个小节点自己的问题
          \end{enumerate}
        \end{noteBox}
      };
    \end{scope}

    \begin{scope}[shift={(6cm,-6cm)}]
      \node[myNode,fill=red!40!black,minimum height=4em] (a0) {dead};
      \draw[latex-, very thick] (a0) -- (30:3cm) node[myNode,fill=\mycolc,name=a2]{2};
      \draw[latex-, very thick] (a0) -- (0:3cm) node[myNode,fill=\mycolaa,name=a4]{4};

      \draw[latex-, very thick,red] (a0) -- node[midway] {\emoji{cross-mark}}
      (300:3cm) node[myNode,fill=\mycola,
      name=a1,minimum height=4em]{primary};
      \draw[latex-, very thick] (a1) -- (270:3cm) node[myNode,fill=\mycolaa,name=a3]{3};

      \draw[-latex,very thick] (a4) -- (a1);
      \draw[-latex,very thick] (a2) -- (a1);
      \draw[-latex,very thick, red] (a3) -- node[midway,sloped] {\emoji{cross-mark}}(a0);

      % \node[left,text width=4cm] at ([xshift=-0.5cm]a3.west) {
      %   \begin{leftDialogBox}
      %     欸? \cola{primary}老不回我,它应该没了,\colc{你}上呗?
      %   \end{leftDialogBox}
      % };

      \node[right,text width=4cm] at ([xshift=0.5cm,yshift=0.5cm]a1.east) {
        \begin{rightDialogBox}
          诶呦,还真没了...那好吧。\textbf{大家听好了},\cola{primary}该换我
          了,\colc{下一个是节点\textbf{2}},你们都记住啊。
        \end{rightDialogBox}
      };

      \node[text width=6cm,below] at ([shift={(0,-3cm)}]a0.south) {
        \begin{noteBox}
          4.如果确实\colc{接班人自己}也没有连上\cola{主节点},则广播\textbf{view-change},并让大家了
          解下一任\colc{接班人}。
        \end{noteBox}
      };
    \end{scope}
  \end{tikzpicture}
\end{center}

\begin{myBox}{听一个共识-留一手版 (Follow-me Consensus)}
  \begin{itemize}
  \item 每个新加的的节点会管主节点要一个序号
  \item 之后所有的节点都会记住两个地址: \cola{主节点}和\colb{下一任主节点}。(存
    在一个列表里)
  \item 当节点们受到来自\colb{下一任主节点}的\textbf{上任通知时},切换主节点。
  \end{itemize}
\end{myBox}

\begin{tcolorbox}
  \emoji{parrot} : 那如果一个节点同时收到两个\cola{primary}的消息怎么办呢?\\
  \emoji{turtle} : 那么就和其他的共识算法一样(Raft,PBFT), 节点只需要接受最新的命令就行了。
\end{tcolorbox}

% Local Variables:
% TeX-engine: luatex
% TeX-command-extra-options: "-shell-escape"
% TeX-master: "m.tex"
% End:
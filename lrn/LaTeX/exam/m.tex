\documentclass[addpoints]{exam}

\begin{document}

\begin{center}
  {\Huge Exam}

  \vskip 1pc
  \fbox{\fbox{\parbox{5.5in}{\centering
        Answer the questions in the spaces provided. If you run out of room
        for an answer, continue on the back of the page.}}}
\end{center}

\vspace{5mm}
\makebox[0.75\textwidth]{Name and section:\enspace\hrulefill}

\vspace{5mm}

\makebox[0.75\textwidth]{Instructor’s name:\enspace\hrulefill}

\begin{questions}
\question Given the equation \(x^n + y^n = z^n\) for \(x,y,z\) and \(n\) positive
  integers. 
  \begin{parts}
  \part For what values of \(n\) is the statement in the previous question true?
    \vspace{\stretch{1}}

  \part For \(n=2\) there's a theorem with a special name. What's that name?
    \vspace{\stretch{1}}

  \part What famous mathematician had an elegant proof for this theorem but
    there was not enough space in the margin to write it down?
    \vspace{\stretch{1}}

    \begin{subparts}
    \subpart Who actually proved the theorem?
      \vspace{\stretch{1}}
    \subpart How long did actually take to solve this problem?
      \vspace{\stretch{1}}
    \end{subparts}

  \end{parts}

\question Prove that the real part of all non-trivial zeros of the function 
  \(\zeta(z)\) is \(\frac{1}{2}\)
  ...

\question Which of these famous physicists invented time?

  \begin{oneparchoices}
  \choice Stephen Hawking 
  \choice Albert Einstein
  \choice Emmy Noether
  \choice This makes no sense
  \end{oneparchoices}
  
\question Which of these famous physicists published a paper on Brownian Motion?

  \begin{checkboxes}
  \choice Stephen Hawking 
  \choice Albert Einstein
  \choice Emmy Noether
  \choice I don't know
  \end{checkboxes}


\question Given the equation \(x^n + y^n = z^n\) for \(x,y,z\) and \(n\) positive
  integers. 
  \begin{parts}
  \part[10] For what values of \(n\) is the statement in the previous question true?
    \vspace{\stretch{1}}

  \part[10] For \(n=2\) there's a theorem with a special name. What's that name?
    \vspace{\stretch{1}}

  \part[10] What famous mathematician had an elegant proof for this theorem but there was
    not enough space in the margin to write it down?
    \vspace{\stretch{1}}

  \end{parts}

\question[20] Compute \[\int_{0}^{\infty} \frac{\sin(x)}{x}\]

  \vspace{\stretch{1}}

Here we'll use \verb|\pointsinmargin|
\pointsinmargin
\question Given the equation \(x^n + y^n = z^n\) for \(x,y,z\) and \(n\) positive
integers. 
\begin{parts}
\part[5] For what values of \(n\) is the statement in the previous question true?
\vspace{40pt}

\part[2 \half] For \(n=2\) there's a theorem with a special name. What's that name?
\vspace{40pt}

\part[2 \half] What famous mathematician had an elegant proof for this theorem but there was
not enough space in the margin to write it down?

\vspace{40pt}
\end{parts}

Now we'll use \verb|\pointsinrightmargin|

\pointsinrightmargin
\question Given the equation \(x^n + y^n = z^n\) for \(x,y,z\) and \(n\) positive
integers. 
\begin{parts}
\part[5] For what values of \(n\) is the statement in the previous question true?
\vspace{40pt}

\part[2 \half] For \(n=2\) there's a theorem with a special name. What's that name?
\vspace{40pt}

\part[2 \half] What famous mathematician had an elegant proof for this theorem but there was
not enough space in the margin to write it down?
integers. 
\end{parts}

\question Given the equation \(x^n + y^n = z^n\) for \(x,y,z\) and \(n\) positive
  integers. 
  \begin{parts}
  \part[5] For what values of \(n\) is the statement in the previous question true?
    \vspace{\stretch{1}}

  \part[2 \half] For \(n=2\) there's a theorem with a special name. What's that name?
    \vspace{\stretch{1}}

    \bonuspart[2 \half] What famous mathematician had an elegant proof for this theorem but there was
    not enough space in the margin to write it down?
    \vspace{\stretch{1}}

  \end{parts}
  \droptotalpoints%
\question[20] Compute \[\int_{0}^{\infty} \frac{\sin(x)}{x}\]
  \vspace{\stretch{1}}

\bonusquestion[30] Prove that the real part of all non-trivial zeros of the function 
\(\zeta(z)\) is \(\frac{1}{2}\)
\vspace{\stretch{1}}

\begin{center}
  \combinedgradetable[h][questions]
\end{center}
\end{questions}

\end{document}
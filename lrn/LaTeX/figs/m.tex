% Use xelatex
\documentclass{article}
\usepackage[margin=1in]{geometry}
\title{My First Photo doc}
\author{Jianer Cong}
\date{\today}
% for Japanese
\usepackage{xeCJK}
\usepackage{graphicx}
% setup path
\graphicspath{ {images/} }
% for inline figures
\usepackage{wrapfig}
% package for rightcaption
\usepackage[rightcaption]{sidecap}

\begin{document}
\maketitle
\listoffigures
\newpage{}

PSYDUCK uses a mysterious power. When it does so, this POKéMON generates brain
waves that are supposedly only seen in sleepers. This discovery spurred
controversy among scholars.

If it uses its mysterious power, PSYDUCK can’t remember having done so. It
apparently can’t form a memory of such an event because it goes into an altered
state that is much like deep sleep. When headaches stimulate its brain cells,
which are usually inactive, it can use a mysterious power. This Pokémon is
troubled by constant headaches. The more pain it’s in, the more powerful its
psychokinesis becomes.It has been found that its brain cells are 10 times more
active when Psyduck is experiencing a headache. See figure\ref{fig:psy}.

\begin{SCfigure}[0.5][h]
  \includegraphics[width=0.6\textwidth]{psyduck.jpeg}
  \caption{Always tormented by headaches. It uses psychic powers, but it is not
    known if it intends to do so.}\label{fig:psy}
\end{SCfigure}

See my figure in figure \ref{fig:n} on page \pageref{fig:n}, this is the first
file I included from \verb|METAPOST|.

\begin{figure}[ht]
  \centering
  \includegraphics[width=0.8\textwidth]{n-latex-labs.eps}
  \caption{My First mpost input}\label{fig:n}
\end{figure}

\newpage{}
Slowpoke (Japanese: ヤドン Yadon) is a dual-type Water/Psychic Pokémon
introduced in Generation I.

It evolves into Slowbro starting at level 37 or Slowking when traded while
holding a King's Rock.

In Galar, Slowpoke has a pure Psychic-type regional form, introduced in Pokémon
Sword and Shield's 1.1.0 patch. It evolves into Galarian Slowbro when exposed to
a Galarica Cuff or Galarian Slowking when exposed to a Galarica Wreath.

\begin{wrapfigure}{r}{0.25\textwidth} %this figure will be at the right
  \centering
  \includegraphics[width=0.25\textwidth]{slowpoke.png}
\end{wrapfigure}

Slowpoke is a pink Pokémon that resembles a cross between a salamander and a
hippopotamus. It has vacant eyes that never seem focused, curled ears, and a
rounded, tan muzzle. It has four legs, each of which ends in a single white
claw. Its long, tapering tail has a white tip. This tail drips a sweet, sappy
substance that is attractive to many species of fish. Slowpoke uses the tail as
a fishing lure. The tail often breaks off, but it will grow back. In Alola, its
tail is often dried and used in home-cooked stews.

\begin{wrapfigure}{l}{0.25\textwidth}
  \centering
  \includegraphics[width=0.25\textwidth]{slowpoke.png}
\end{wrapfigure}
Slowpoke has a notoriously dim intellect and often forgets what it was doing. It
takes a long time to respond to outside stimuli. For example, it can take up to
five seconds to process pain and can take an entire day to notice when its tail
has been bitten. Slowpoke is commonly found at the water's edge. In some places,
it is believed that Slowpoke's yawn causes rain. This Pokémon is worshiped in
those areas.

In Galar, Slowpoke's appearance is a result of particles being built from eating
Galarica seeds. The main difference is that it has gained yellow coloration on
its forehead and tail. Unlike its counterpart, its tail is always down and said
to have a spicy taste.

Galarian Slowpoke are known to relax on seashores and riverbanks without a care
in the world. Occasionally, it will get a very sharp look in its eyes as if it
is about to think of something tremendous. However, Slowpoke will quickly forget
and return to its normal expression. It is believed that its behavior is a
result of Galarica particles affecting its brain.
\end{document}

% LaTeX units and legths
% Abbreviation 	Definition
% pt 	A point, is the default length unit. About 0.3515mm
% mm 	a millimetre
% cm 	a centimetre
% in 	an inch
% ex 	the height of an x in the current font
% em 	the width of an m in the current font
% \columnsep 	distance between columns
% \columnwidth 	width of the column
% \linewidth 	width of the line in the current environment
% \paperwidth 	width of the page
% \paperheight 	height of the page
% \textwidth 	width of the text
% \textheight 	height of the text
% \unitlength 	units of length in the picture environment. 
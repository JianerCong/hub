\begin{tikzpicture}
  \draw[step=1cm,style=help lines] (0,0) grid (10cm,10cm);
  \begin{scope}
    \draw (0,0) circle (1);
  \end{scope}

  \begin{scope}[x=.5cm]         %squeeze the x axis
    \draw (0,0) circle (1);
  \end{scope}

  
  \begin{scope}[xshift=2cm]     %shift the x axis
    \draw (0,0) circle (1);
  \end{scope}

  
  \begin{scope}[xshift=4cm]     %shift the x axis
    \fill[fill=red]
    (0,0) node{n1}
    -- (1,1) node[ellipse,draw] {n---------2}
    -- (0,2) node[circle,fill=red!20] {n3};
  \end{scope}

  
  \begin{scope}[yshift=4cm,shape=ellipse]     %all nodes are ellipse
    \tikzstyle{every node}=[draw]             %Equivalent to add [draw] to all nodes
    \fill[fill=yellow!20]
    (0,0) node{n1}
    -- (1,1) node{n---------2}
    -- (0,2) node{n3};
  \end{scope}

  
\end{tikzpicture}

A flow diagram

\begin{tikzpicture}
  \draw[style=help lines] (0,-5) grid (20,5);
  \begin{scope}[node distance=2cm]

    \tikzstyle{every node}=[draw, rectangle,text centered]

    \node (0,0) (pick_ep)[text width=2cm] {Pick $\epsilon$ from 0 - 0.0035};

    \node (ep) [circle,right of=pick_ep] {$\epsilon$};

    \node at ([yshift=-2.5cm] ep.west) (find_dc) [%
    right,                                        % anchor=west
    text width=4cm,
    text justified,
    node distance=2.5cm] {%
      Find $d_c$
      
      {\small 1. Assume a $d_c$;

        2. Calculate the net force $f$ from $d_c$ and
        $\epsilon$;

        3. If $f$ is (approximately) zero: done; Else: Assume a new
        $d_c$ and go to step 2;}
    };

    \node (dc) [circle, below of find_dc] {$d_c$};


    % {{{ Draw lines

    \begin{scope}[thick, ->]
      \draw (ep) -- (ep |- find_dc.north);
    \end{scope}

    % }}}
    
  \end{scope}
\end{tikzpicture}
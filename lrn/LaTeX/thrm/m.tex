\documentclass{article}
\usepackage[english]{babel}

\usepackage{amssymb}
\usepackage{amsthm}
\begin{document}

\newtheorem{theorem}{Theorem}[section]
\newtheorem{corollary}{Corollary}[theorem]
\newtheorem{lemma}[theorem]{Lemma}
% Unnumbered therom-like evironment

\theoremstyle{definition}
\newtheorem{definition}{Definition}[section]
\newtheorem*{remark}{Remark}

\section{Introduction}
Theorems can easily be defined:

\begin{theorem}
  Let \(f\) be a function whose derivative exists in every point, then \(f\) 
  is a continuous function.
\end{theorem}


\begin{theorem}[Pythagorean theorem]
  \label{pythagorean}
  This is a theorem about right triangles and can be summarised in the next 
  equation 
  \[ x^2 + y^2 = z^2 \]
\end{theorem}

And a consequence of theorem \ref{pythagorean} is the statement in the next 
corollary.

\begin{corollary}
  There's no right rectangle whose sides measure 3cm, 4cm, and 6cm.
\end{corollary}

You can reference theorems such as \ref{pythagorean} when a label is assigned.

\begin{lemma}
  Given two line segments whose lengths are \(a\) and \(b\) respectively there is a 
  real number \(r\) such that \(b=ra\).
\end{lemma}

\begin{remark}
  This statement is true, I guess.
\end{remark}

\begin{definition}[Fibration]
  A fibration is a mapping between two topological spaces that has the homotopy lifting property for every space \(X\).
\end{definition}

% Use blacksquare
\renewcommand\qedsymbol{$\blacksquare$}
\section{Proofs}
\begin{lemma}
  Given two line segments whose lengths are \(a\) and \(b\) respectively there 
  is a real number \(r\) such that \(b=ra\).
\end{lemma}

\begin{proof}
  To prove it by contradiction try and assume that the statement is false,
  proceed from there and at some point you will arrive to a contradiction.
\end{proof}
\end{document}
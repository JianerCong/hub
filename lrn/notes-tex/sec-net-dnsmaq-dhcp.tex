\subsection{How it does DHCP}

\colz{
  The tag system works as follows:

  \begin{enumerate}
  \item For each DHCP request, \dmq{} collects a set of valid tags from active
    configuration lines which include \texttt{set:<tag>}, including
    \begin{enumerate}
    \item one from the \colZ{\texttt{--dhcp-range}} used to allocate the
      address,
    \item one from any matching \colZt{--dhcp-host}
    \item and ``known'' or ``known-ethernet'' if a \texttt{--dhcp-host} matches.
    \end{enumerate}
  \item The tag ``bootp'' is set for BOOTP requests, 
  \item a tag whose name is the name of the interface on which the request
    arrived is also set.
  \end{enumerate}

  \subsubsection{dhcp-range}
  \cSay{
    Here are some details, first,

    \texttt{-F,--dhcp-range=
      \cola{tag1:<tag1>,tag2:<tag2>,...}\colb{, set:<tag>}
      \colc{<start-addr>,[<end-addr>],}
      [<many options>]
    } Enable the DHCP server.

    Addresses will be given out from the range \colZt{<start-addr>} to
    \colZt{<end-addr>} and from statically defined addresses given in
    \texttt{--dhcp-host} options.

    \colz{

      A lease time can be given. By default, it's one hour for IPv4 and
      one day for IPv6.

      For IPv6, there's an prefix length which must be equal to or larger then
      the prefix length on the local interface. This defaults to 64. Unlike
      the IPv4 case, the prefix length is not automatically derived from the
      interface. confiugration. (minimum prefix length = 64.)

    }
  }

  \cSay{
    \colz{
      IPv6 supports another type of range, In this, the start address and
      optional end address \colZ{contain only the network part (ie
        \texttt{::1})} and they are followed by
      \texttt{constructor:<interface>}, for example,
      \colZt{--dhcp-range=::1,::400,constructor:eth0}, will look for addresses
      on \texttt{eth0} , and then create a range from \texttt{<network>::1} to
      \texttt{<network>::400}.  Note that not just any address on
      \texttt{eth0} will not do: it must not be an autoconfigured or privacy
      address, or be deprecated.

      If a \texttt{--dhcp-range} is only being used for stateless DHCP, then
      it can simply be:
      \colZt{--dhcp-range=::,constructor:eth0}
    }
  }

  \dSay{
    I remember for IPv6, there're different ways to do automatic address
    configuration...Right? What's said about this in the manual?
  }

  \cSay{
    For IPv6, the mode may be some combination of
    \begin{enumerate}
    \item \texttt{slaac} \colz{set the \texttt{A} bit in the RA (addresses
        available via DHCP).When used with a DHCP range or static DHCP address
        this results in the client having \cola{both a SLAAC and a
          DHCP-assigned} address.},
    \item \texttt{ra-only} \colz{router advertisement only, no DHCP},
    \item \texttt{ra-names} \colz{router advertisement plus names from DNS}.
    \item \texttt{ra-stateless} \colz{set \texttt{A,O} bits in RA. address from SLAAC and others from DHCP}.
    \item \texttt{ra-advrouter} \colz{enables a mode where router address(es)rather
        than prefix(es) are included in the advertisements.}
    \item \texttt{off-link} \colz{
        tells \dmq{} to advertise the prefix without the on-link bit set.
      }
    \end{enumerate}
  }

  \subsubsection{dhcp-host}

  \dSay{\colz{What about \texttt{--dhcp-host} ?}}

  \cSay{
    {

      \ttfamily
      -G,--dhcp-host= \cola{[<hwaddr>]}
      \colz{
        [,id:<client\_id>|*]
        [,set:<tag>] [,tag:<tag>] \colb{[,<ipaddr>]} \cola{[,<hostname>]} \colZ{[,<lease\_time>]} [,ignore]
      }  
    }
    

    Specify per host parameters for the DHCP server. This allows a machine with
    a particular hardware address to be always allocated the same
    \begin{enumerate}
    \item hostname,
    \item IP address and
    \item lease time.
    \end{enumerate}
  }

  \dSay{
    Oh, this one is pretty important for those servers that need a static IP.
  }

  \cSay{
    \colz{
      A hostname specified like this overrides any supplied by the \colZ{DHCP
        client on the host}.

      It is also allowable to omit the hardware address and include the
      hostname. In this case, the IP address and lease times will \colZ{applied
        to any machine claiming that name}. For example
      \colZt{--dhcp-host=00:11:22:33:44:55,myhost,infinite} tells \dmq{} to give
      the machine with hardware address \texttt{00:11:22:33:44:55} the name
      ``\texttt{myhost}'' and an infinite lease. For another example,
      \colZt{--dhcp-host=myhost2,10.0.0.2} tells \dmq{} to give the machine
      ``\texttt{myhost2}'' the IP address \texttt{10.0.0.2}
    }
  }

  \dSay{
    What ? Hosts can have names before they are assigned an IP address ?
  }

  \cSay{It just said so in the manual. It looks like in the DHCP client message,
    the client can specify its hostname.

  }

  \dSay{
    Okay...
  }

  \cSay{
    \colz{
      Addresses allocated like this are not constrained to be in the range
      given in \texttt{--dhcp-range} options. For subnets which don't need a pool
      of dynamically allocated addresses, you can use a ``\texttt{static}''
      keyword in the \texttt{--dhcp-range} declaration.
    }
  }

  \dSay{ Oh, so DHCP can also be used to assign static IP addresses?}

  \cSay{Seem so.}


  \dSay{What is \texttt{client\_id} ?}

  \cSay{It's a unique identifier for the client. It's called \texttt{DUID} in
    IPv6. \colz{(IPv6 abondoned the MAC address).} It's an alternative
    to the MAC address.

    \colz{
      Thus:

      \colZt{--dhcp-host=id:01:02:03:04,...} refers to the host with client
      identifier \texttt{01:02:03:04}.
    }
  }

  \cSay{
    \colz{
      A single \texttt{--dhcp-host} option may contain
      \begin{enumerate}
      \item an IPv4 address, or 
      \item one or more IPv6 addresses, or
      \item both
      \end{enumerate}.

      IPv6 addresses must be bracketed by square brackets thus:

      \colZt{--dhcp-host=laptop,[2001:db8::1],[2001:db8::2]}

      
      IPv6 addresses may contain only the
      host-identifier part: \colZt{--dhcp-host=aaa,[:56]} in which case they act
      as \cola{wildcards} in constructed DHCP ranges, with the appropriate
      network part inserted. For IPv6, an address may include a prefix length:
      \texttt{--dhcp-host=laptop,[1234:50/126]} which (in this case) specifies
      four addresses, \texttt{1234:50} to \texttt{1234:53}. 
    }
  }

  \dSay{
    What about the host file \texttt{/etc/hosts} ? Does \dmq{} care about it
    when serving DHCP?
  }

  \cSay{
    \colz{
      Yes. If a name appears in \texttt{/etc/hosts}, the associated address can
      be allocated to a DHCP lease, but \colZ{only if} a \texttt{--dhcp-host} option
      specifying the name also exists.

      Only one hostname can be given in a \texttt{--dhcp-host} option, but
      aliases are possible by using \texttt{--cname}. \cola{\texttt{/etc/hosts}
        is not used when the DNS side of \dmq{} is disabled.
      }
    }
  } 

  \subsubsection{host-file}

  \dSay{
    It feels like \texttt{--dhcp-host} is such an important option that it might
    need a file to store all the information ?
  }

  \cSay{
    \colz{
      Yes. \texttt{--dhcp-hostfile = <file>} does it. Each line is a \texttt{--dhcp-host=<line>}. 
      }
  }

  \subsubsection{dhcp-option}

  \dSay{
    What about \texttt{--dhcp-option} ?
  }

  \cSay{
    {
      \ttfamily
      -O,--dhcp-option=
      \colz{[tag:<tag>,]}
      [encap:<opt>] \colz{[vi-encap:<enterprise>,]}
      \cola{[vendor:<vendor-class>,]} <many more options>
    }
  }
  
  Any configuration lines which include one or more \texttt{tag:<tag>}
  constructs will only be valid if all that tags are matched in the set derived
  above. Typically this is \colZt{--dhcp-option}, tagged version of
  \texttt{--dhcp-options} is prefered, provided that all the tags match
  somewhere in the set collected as described above. The prefix \texttt{!} on a
  tag means ``not'' so \colZt{--dhcp-option=tag:!purple,3,1.2.2.3.4} sends the
  option when the taf \texttt{purple} is not in the set of valid tags.
  (\emoji{turtle} : If using
  this in a command line rather than a config file, be sure to escape
  \texttt{!}, which is a shell metachar...)

  When selecting \colZt{--dhcp-options}, a tag from \colZt{--dhcp-range} is
  second class relative to other tags, to make it easy to override options for
  individual hosts, so
  \colZt{
    \newline
    --dhcp-range=set:interface1,....\newline
    --dhcp-option=tag:interface1,\cola{option:nis-domain,domain1} \newline
    --dhcp-option=tag:myhost,\cola{option:nis-domain,domain2 }
  }

  will set the NIS-domain to \colZt{domain1} for hosts in range, but override
  that to \colZt{domain2} for \colZt{myhost}.

  \colZ{
    Note that for \texttt{--dhcp-range} both \texttt{tag:<tag>} and
    \texttt{set:<tag>} are allowed, to both select the range in use based on
    (eg) \texttt{--dhcp-host},and to affect the options sent, based on the range
    selected.
  }
}
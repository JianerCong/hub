\documentclass[dvipsnames]{article}
% \documentclass[dvipsnames]{ctexart}


\title{Linux Network}

\usepackage{geometry}\geometry{
a4paper,
total={170mm,257mm},
left=20mm,
top=20mm,
}


\usepackage[inkscapearea=page]{svg}
% 🐢 : I guess what we need is to just include the page... not the whole
% drawing... So set inkscapearea=page should work... If not, delete the
% .\svg-inkscape and try again. (this pkg does some magical caching..)
\usepackage[skip=5pt plus1pt, indent=0pt]{parskip}
% Color
\newcommand{\mycola}{MidnightBlue}
\newcommand{\mycolb}{Mahogany}
\newcommand{\mycolc}{OliveGreen}

\newcommand{\mycoli}{\mycolb}
\newcommand{\mycolii}{\mycola!30!\mycolb}
\newcommand{\mycoliii}{\mycola!70!\mycolb}
\newcommand{\mycoliv}{\mycola!90!\mycolb}
\newcommand{\mycolv}{\mycola}

\newcommand{\cola}[1]{\textcolor{\mycola}{#1}}
\newcommand{\colb}[1]{\textcolor{\mycolb}{#1}}
\newcommand{\colc}[1]{\textcolor{\mycolc}{#1}}
\newcommand{\Cola}[1]{\textcolor{\mycola}{\textbf{#1}}}

\newcommand{\coli}[1]{\textcolor{\mycoli}{#1}}
\newcommand{\colii}[1]{\textcolor{\mycolii}{#1}}
\newcommand{\coliii}[1]{\textcolor{\mycoliii}{#1}}
\newcommand{\coliv}[1]{\textcolor{\mycoliv}{#1}}
\newcommand{\colv}[1]{\textcolor{\mycolv}{#1}}

% 🦜 : \textcolor doesn't allow multiple paragraphs in it, so we used {\color{...}}
\newcommand{\colZ}[1]{
{\color{black}#1}
} %go back
\newcommand{\colz}[1]{
{\color{gray}#1}
}

% \let\emph\relax % there's no \RedeclareTextFontCommand
% \DeclareTextFontCommand{\emph}{\bfseries}
\renewcommand{\emph}[1]{\texbf{#1}}
\usepackage{amssymb}            %\mathbb

\usepackage{fontspec}

\setmonofont{Cascadia}[
Path=/usr/share/fonts/truetype/Cascadia_Code/,
Scale=0.85,
Extension = .ttf,
UprightFont=*Code,              %find CascadiaCode.ttf
BoldFont=*CodePL,               %find CascadiaCodePL.ttf ...
ItalicFont=*CodeItalic,
BoldItalicFont=*CodePLItalic
]
% --------------------------------------------------
% Windows
% \setmonofont{Cascadia}[
% Path=C:/Windows/Fonts/,
% Extension = .ttf,
% UprightFont=*Mono,              %find CascadiaMono.ttf
% BoldFont=*Code,               %find CascadiaCodePL.ttf ...
% ItalicFont=*Code,
% BoldItalicFont=*Code
% ]


\usepackage{minted}
\usepackage{tcolorbox}
\tcbuselibrary{skins}
\tcbuselibrary{minted}
\usepackage{tikz}
\usetikzlibrary{shapes} % ellipse node shape
\usetikzlibrary{shapes.multipart} % for line breaks in node text
\usetikzlibrary{arrows.meta}    %-o arrow head
\usetikzlibrary{arrows}
\usetikzlibrary{matrix}
\usetikzlibrary{snakes}

\usepackage{amsmath}
% ??? still xelatex?
% \usepackage{xeCJK}
\usepackage{emoji}
% \setemojifont{NotoColorEmoji.ttf}[Path=C:/Users/congj/repo/myFonts/]
% \setemojifont{TwitterColorEmoji-SVGinOT.ttf}[Path=C:/Users/congj/repo/myFonts/]


\newtcolorbox[auto counter]{myBox}[2][]{
fonttitle=\bfseries,title={共识~\thetcbcounter: #2},#1
}
\newtcolorbox[]{noteBox}[1][]{
tile,left=1mm,nobeforeafter,fontupper=\small,#1
}

\tikzstyle{myNode}=[inner sep=2pt,circle,text=white]
\date{\today}
\author{作者}

\newtcblisting{simplec}{
listing engine=minted,
minted language=c++,
minted style=vs,
minted options={fontsize=\small,autogobble,
% framesep=1cm
},
tile,
listing only,
% bottom=0cm,
% nobeforeafter, 
boxsep=0mm,
left=1mm,
opacityback=0.5,
colback=gray!20
}
\tcbuselibrary{breakable}
\newtcblisting{simplepy}{
listing engine=minted,
minted language=python,
minted style=vs,
minted options={fontsize=\small,autogobble,
% framesep=1cm
},
tile,
listing only,
% bottom=0cm,
% nobeforeafter,
boxsep=0mm,
left=1mm,
opacityback=0.5,
colback=gray!20,
breakable
}
\newtcolorbox{blackbox}{tile,colback=black,colupper=white,nobeforeafter,halign=flush center}

% \tikzstyle{myMatrix}=[matrix of nodes,below right,
% nodes={above,text centered},                  %apply to all nodes
% row sep=1cm,column sep=2cm]
% \tikzstyle{every node}=[inner sep=0pt]

\newcommand\uptoleft[3][-o]{\draw[very thick,#1](#2.south) |- (#3.west);}
\newcommand\uptodown[3][-o]{\draw[very thick,#1](#2.south) to [out=270,in=90] (#3.north);}
\newcommand\downtoup[3][-latex]{\draw[very thick,#1](#2.north) to [out=90,in=270] (#3.south);}

\newcommand\lefttoright[3][-latex]{\draw[very thick,#1](#2.east) to[out=0,in=180] (#3.west);}
\newcommand\lefttodown[3][-latex]{\draw[very thick,#1](#2.east) to[out=0,in=90] (#3.north);}

\newcommand{\topDialogBoxColored}[4][0.8\linewidth]{
  \node[above,text width=#1] at ([yshift=0.5cm]#2.north) {
    \begin{tcolorbox}[tile,
      nobeforeafter,
      boxsep=0pt,
      % show bounding box,
      colback=#4!10,
      overlay={
        \begin{scope}
          \fill[#4!10] (frame.south) --
          +(2mm,0) --
          +(0,-3mm) --
          +(-2mm,0)
          ;
        \end{scope}
      } ] #3
    \end{tcolorbox}
  };
}

\newcommand{\rightDialogBoxColored}[4][0.8\linewidth]{
  \node[right,text width=#1] at ([xshift=0.5cm]#2.east) {
    \begin{tcolorbox}[tile,
      nobeforeafter,
      boxsep=0pt,
      % show bounding box,
      colback=#4!10,
      overlay={
        \begin{scope}
          % \fill[gray!10] (frame.east) circle (2pt);
          \fill[#4!10] (frame.west) --
          +(0,2mm) --
          +(-3mm,0) --
          +(0,-2mm)
          ;
        \end{scope}
      } ] #3
    \end{tcolorbox}
  };
}

\newcommand{\rightDialogBox}[3][0.8\linewidth]{
% \node[right,text width=#1] at ([xshift=0.5cm]#2.east) {
% \begin{tcolorbox}[tile,
% nobeforeafter,
% boxsep=0pt,
% % show bounding box,
% colback=\mycola!10,
% overlay={
% \begin{scope}
% % \fill[gray!10] (frame.east) circle (2pt);
% \fill[\mycola!10] (frame.west) --
% +(0,2mm) --
% +(-3mm,0) --
% +(0,-2mm)
% ;
% \end{scope}
% } ] #3
% \end{tcolorbox}
% };
  \rightDialogBoxColored[#1]{#2}{#3}{\mycola}
}


\newcommand{\leftDialogBoxColored}[4][0.8\linewidth]{
  \node[left,text width=#1] at ([xshift=-0.5cm]#2.west) {
    \begin{tcolorbox}[tile,
      nobeforeafter,
      boxsep=0pt,
      % show bounding box,
      colback=#4!10,
      overlay={
        \begin{scope}
          % \fill[gray!10] (frame.east) circle (2pt);
          \fill[#4!10] (frame.east) --
          +(0,2mm) --
          +(3mm,0) --
          +(0,-2mm)
          ;
        \end{scope}
      } ] #3
    \end{tcolorbox}
  };
}

\newcommand{\leftDialogBox}[3][0.8\linewidth]{
% \node[left,text width=#1] at ([xshift=-0.5cm]#2.west) {
% \begin{tcolorbox}[tile,
% nobeforeafter,
% boxsep=0pt,
% % show bounding box,
% colback=gray!10,
% overlay={
% \begin{scope}
% % \fill[gray!10] (frame.east) circle (2pt);
% \fill[gray!10] (frame.east) --
% +(0,2mm) --
% +(3mm,0) --
% +(0,-2mm)
% ;
% \end{scope}
% } ] #3
% \end{tcolorbox}
% };
  \leftDialogBoxColored[#1]{#2}{#3}{gray}
}




\newcommand{\mycolaa}{\mycola!20}

\newcommand{\colZt}[1]{
  \colZ{\texttt{#1}}
}

\usepackage{changepage}   % for the adjustwidth environment
\newenvironment{myIndent}[1][7mm]{\begin{adjustwidth}{#1}{}}{\end{adjustwidth}}

\newcounter{myDefCounter}
\newcounter{myTheoCounter}

\tcbuselibrary{theorems}
\newtcbtheorem[use counter=myDefCounter,number within=section]{myDef}{Definition}%
{
% colback=green!5,colframe=green!35!black,
fonttitle=\bfseries}{def}

\newcommand{\myThmBeforeProof}{
\texttt{Proof: }\par
\begin{myIndent}
}

\newcommand{\myThmAfterProof}{
\end{myIndent}
\qed
}

\tcbuselibrary{skins}
\tcbsubskin{myThmSkin}{enhanced}{
parbox=false,
% colback=green!5,
colframe=\mycola,
fonttitle=\bfseries,
breakable,
before lower={\myThmBeforeProof{}},
after lower={\myThmAfterProof{}}
}

\newtcbtheorem[
use counter=myTheoCounter,number within=section
]{myTheo}{Theorem}%
{
skin=myThmSkin
}{thm}


\newtcbtheorem[use counter=myTheoCounter,number within=section]{myCor}{Corollary}%
{
% colback=green!5,
skin=myThmSkin
}{cor}

\newtcbtheorem[use counter=myTheoCounter,number within=section]{myLem}{Lemma}%
{% colback=green!5,
skin=myThmSkin
}{lem}

\newcommand{\refCorollary}[1]{Corollary~\ref{cor:#1}}
\newcommand{\refLemma}[1]{Lemma~\ref{lem:#1}}

\usepackage{hyperref}

\usepackage{cleveref}           %🦜 : must be loaded after hyperref
\crefname{myDefCounter}{definition}{definitions}
% ^^^ plural
\Crefname{myDefCounter}{Definition}{Definitions}
% ^^^^^^ type = counter name

\crefname{myTheoCounter}{theorem}{theorems}
\Crefname{myTheoCounter}{Theorem}{Theorems}

\usepackage{amsthm}             %for {proof}

\usepackage{tabularx}



\usepackage{enumitem}
\setlist[description]{leftmargin=0.1\linewidth,labelindent=0.1\linewidth}
\usepackage{placeins}           %for \FloatBarrier
% 🦜 : Remember to specify the mode, because without a \documentclass{}, emacs
% will consider this as a plain TeX document.

% --------------------------------------------------
\newcommand{\ans}{
  % \newline{}
  \par
  \hspace{0.1\linewidth}
  % \phantom{Ans: }
  % \colz{Ans: }
  \tikz[baseline=.5em]{\draw[thick,\mycola] (0,0) -- (0.8\linewidth,0pt);}
  % \rule[-1em]{0.8\linewidth}{1pt}
}
\newcommand{\tf}{\Cola{\hfill[T/F]}}

\usepackage{tabularray}
\UseTblrLibrary{booktabs,siunitx}
% Local Variables:
% mode: LaTeX
% TeX-engine: luatex
% TeX-command-extra-options: "-shell-escape"
% TeX-master: "m.tex"
% TeX-parse-self: t
% TeX-auto-save: t
% End:

% svgs
\newcommand{\svgOs}[2][0.3\linewidth]{\includesvg[width=#1]{../../mysvgs/#2.svg}}
% \newcommand{\svgNeutron}[1][1.5cm]{\includesvg[width=#1]{/home/me/Pictures/opstk/neutron.svg}}
\newcommand{\svgD}[1][8mm]{\includesvg[width=#1]{/home/me/Pictures/d1.svg}}
\newcommand{\svgC}[1][8mm]{\includesvg[width=#1]{/home/me/Pictures/c1.svg}}

% params:
% 1: width of dialog box
% 2: the speaker
% 3: the text
\newcommand{\leftSay}[3][0.7\linewidth]{
  \begin{tikzpicture}
    \node (N1) {
      #2
    };
    \leftDialogBox[#1]{N1}{
      #3
    }
  \end{tikzpicture}
}

\newcommand{\rightSay}[3][0.7\linewidth]{
  \begin{tikzpicture}
    \node (N1) {
      #2
    };
    \rightDialogBox[#1]{N1}{
      #3
    }
  \end{tikzpicture}
}

% 🦜 : Let d1 and c1 say and ask .
\newcommand{\dSay}[1]{
  \begin{tikzpicture}
    \node (D1) {
      \svgD
    };
    \rightDialogBox[0.7\linewidth]{D1}{
      #1
    }
  \end{tikzpicture}
}


\newcommand{\cSay}[1]{
  \begin{flushright}
    \begin{tikzpicture}
      \node (C1) {
        \svgC
      };
      \leftDialogBox[0.7\linewidth]{C1}{
        #1
      }
    \end{tikzpicture}
  \end{flushright}
}

\hypersetup{
  colorlinks=true,
  linkcolor=blue,
  filecolor=magenta,      
  urlcolor=cyan,
  pdftitle={Linux Network},
  pdfpagemode=FullScreen,
}
\urlstyle{same}

% --------------------------------------------------
\begin{document}
\maketitle

% \section{Media Access Control (MAC) Address}

\subsection{What does it look like}

MAC address has format something like: \texttt{00-0c-29-3b-73-cb}

\cSay{
  In practice, these addresses are used to communicate between hosts in the
  same VLAN or subnet. If you look at a packet capture, at the start of a TCP
  conversation you'll see the sending host send a broadcast \Cola{ARP request} saying
  \begin{center}
    ``Who has IP address x.x.x.x ?''
  \end{center}
  The \Cola{ARP reply} from some host will be
  \begin{center}
    ``That's me, and my MAC address is aaaa.bbbb.cccc ?''
  \end{center}
}

\dSay{
  What if the target IP is on a different subnet ?
}

\cSay{
  Then the sender will \Cola{ARP for} the gateway for that subnet.

  Switch is much faster than router, because hosts can use MAC to communicate.
}

\dSay{So why router ?}

\cSay{
  Because switches only work in LAN.
}

\subsection{Organizationally Unique Identifier (OUI)}

\cSay{The leading (usually 3 ) bytes of a MAC address usually points to its
  manufacturer, like \texttt{OOtronics Ltd. Kibbutz Yizrael, IL} or \texttt{OOel Corporate Kedah,
    MY} see \url{https://standards-oui.ieee.org/oui/oui.txt}
}

\subsection{Address Resolution Protocol (ARP)}

\Cola{ARP} is used to find the MAC address of a network neighbour for a given IPv4 Address.

\cSay{
  When a host wants to \cola{talk} to another host \cola{in the same subnet} using its IP address, it will
  \begin{enumerate}
  \item check its \Cola{ARP cache} to see whether there's a MAC address that matches that IP.
  \item If there isn't, it will send an ARP request to the local broadcast address.
  \item When the corresponding MAC is found, it will \colb{establish port-to-port communications (Next)}.
  \end{enumerate}
}
% 
\section{Port to Port : TCP and UDP}

\cSay{There're two types of ports:
  \begin{description}
  \item[fixed server ports:] $0 <= p <= 1023$ usually for \cola{server}
  \item[ephemeral (short-lived) ports:] $1024 <= p <= 65535$ usually for
    \cola{client}. Among them, we have
    \begin{itemize}
    \item $1024 <= p <= 49151$ user ports
    \item $49152  <= p <= 65535$ dynamic or private ports
    \end{itemize}
  \end{description}
}
% \input{sec-net-dns}
% 
\newcommand{\dmq}{\texttt{dnsmasq}}
\section{\dmq{}}

\subsection{Intro}


\Cola{\dmq{}} is a lightweight DNS, TFTP, PXE, router advertisement and
DHCP server. It is intended to provide coupled DNS and DHCP service to a LAN.

\colz{
  \dmq{} \colZ{accepts DNS queries} and either answers them from a small, local, cache
  or forwards them to a real, recursive, DNS server. It loads the contents of
  \colZ{\texttt{/etc/hosts}} so that \colZ{local hostnames which do not appear in the global DNS can be resolved and also
    answers DNS queries for DHCP  configured hosts.}

  It can also act as the authoritative DNS server for one or more domains,
  allowing local names to appear in the global DNS. 

  \colZ{The \dmq{} DHCP server} supports static address assignments and multiple
  networks. It automatically sends a sensible default set of DHCP options
  (configurable).
}

\dSay{
  Oh.. I just realize that, for a local DNS name resolve, you just need to edit
  the \texttt{/etc/hosts}...
}

\subsection{Configure \dmq{}}

\colz{
  Configuring \dmq{} to act as an authoritative DNS server is complicated by the
  fact that it involves configuration of \colZ{external DNS servers} to provide
  delegation. 

  At startup, \dmq{} reads \colZ{\texttt{/etc/dnsmasq.conf}} (configurable
  through \texttt{--conf-file, --conf-dir}) The format of this file consists of
  one option per line, exactly as the long options detailed in the OPTIONS
  section but without the leading \texttt{--}. Line starting with \texttt{\#}
  are comments and ignored. (\emoji{parrot}: latest style)  For options which may only be
  specified once, \colZ{the configuration file overrides the commandline.}

  \colZ{Quoting} is allowed in a config file: between \texttt{"} quotes the
  special meanings of ``\colZ{\texttt{,:;\#}}''. The following escapes are
  allowed: ``\colZ{
    \texttt{
      \textbackslash{}\textbackslash{}
      \textbackslash{}"
      \textbackslash{}t
      \textbackslash{}e
      \textbackslash{}b
      \textbackslash{}r
      \textbackslash{}n
    }
  }'' }

\subsection{Notes for \texttt{SIGHUP}}
\colz{
  When it recieves a \colZ{\texttt{SIGHUP}}, \dmq{} \colZ{clears its cache and
    then re-loads
    \texttt{/etc/hosts, /etc/ethers}
    and any file given by \texttt{--dhcp-hostsfile, --dhcp--hostsdir,
      --dhcp-optsfile, --dhcp-optsdir, --addn-hosts, --hostsdir}.
  }  

  The DHCP lease change script is called for all existing DHCP leases. if
  \colZ{\texttt{--no-poll}} is set, SIGHUP also re-reads
  \texttt{\colZ{/etc/resolve.conf}}.
}

\texttt{SIGHUP} does \Cola{not} re-read the configuration file.

\dSay{
  So it's like a ``refresh''.
}

\cSay{Yeah.}

\subsection{dump log}

\colz{
  
  When it receives a \colZ{\texttt{SIGUSR1}}, \dmq{} writes stastics to the
  system log.

  When it receives a \colZ{\texttt{SIGUSR2}}, \dmq{} is logging directly to a
  file. (configured by \texttt{--log-facility}).
}

\subsection{some options}

\begin{itemize}
\item \texttt{-h,--no-hosts}: \colz{Don't read the hostnames in
    \texttt{/etc/hosts}.}
\item \texttt{-H,--addn-hosts=<file>}: \colz{Read the hostnames in
    \texttt{<file>} in addition to \texttt{/etc/hosts}.

    If \texttt{--no-hists} is given, read only \texttt{<file>}.
  }
\item \texttt{-d,--no-daemon}: \colz{Debug mode: \colZ{don't fork to the background}.
    don't write a pid file, don't change user id, generate a complete cache dump
    on receipt on \colZt{SIGUSR1}, log to stderr as well as syslog, don't fork new
    processes to handle TCP queries.

    \emoji{turtle} : Use \texttt{-k,--keep-in-foreground} in production.
  }
\item \texttt{-q,--log-queries}: \colz{Log the results of DNS queries handled by
    dnsmasq.}
\item \texttt{-8,--log-facility=<facility>}: \colz{
    If \texttt{<facility>} contains \texttt{/}, then it's considered a filename,
    and \dmq{} won't log to syslog, but will log to the specified file.
  }
\item \texttt{-p,--port=<port>}: \colz{Listen on \texttt{<port>} instead of the
    standard DNS port (53). Setting this to zero completely disables DNS,
    leaving only DHCP and/or TFTP.
  }
\end{itemize}

\subsection{How it does DNS}

\dmq{} is a  query forwarder: it is not capable of recursively answering
arbitrary queries starting from the root servers but forwards such queries to a
\cola{fully recursive DNS updtream server}
\colz{ which is typically provided by an \colZ{ISP}. By default, \dmq{} reads
  \colZ{\texttt{/etc/resolve.conf}} (or equivalent if \colZt{--resolv-file} is
  used) and re-reads it if it changes.

  This allows the DNS servers to be \colZ{set dynamically by PPP or DHCP}
}

\dSay{Oh, so it will watch a \texttt{resolve.conf} file.}

\cSay{Yeah..
  \texttt{-r,--resolv-filr = <file>}\newline
  Read the IP addresses of the \cola{upstream nameservers} from \texttt{<file>}.
  \colz{default to \texttt{/etc/resolv.conf}.}
}

\dSay{
  Is it possible to use commandline to specify it ?
}

\cSay{Yes. Use:

  \begin{enumerate}
  \item \texttt{-R,--no-resolv} Don't read \texttt{/etc/resolv.conf}. \colz{ Get
      upstream servers only from the command line or the \dmq{} configuration
      file.}
  \item \texttt{--rev-server=...} This specify the \cola{upstream server(s)} directly.
    \colz{For example, --rev-server=1.2.3.0/24,192.168.0.1}
  \end{enumerate}


}


\dSay{Wait, what's PPP?}

\cSay{ Point-to-Point protocol. \colz{ I think it's a
    protocol that also touch the ip address, but to the best of our knowledge,
    it has nothing to do with DHCP. } }

\colz{Absense of \texttt{/etc/resolve.conf} is not an error since it may not
  have been created before a PPP connection exists. }

\cSay{Oh, so PPP creates the \texttt{resolve.conf} ?}

\dSay{Seem so.}

\colz{
  \dmq{} simply keep checking in case this file is created at any time. \dmq{}
  can be told to parse more than one \texttt{resolve.conf} file. This is useful
  on a laptop, where both PPP and DHCP may be used.


  \colZ{Upstream servers may also be specified.} These server specifications
  optionally take a domain name which tells \dmq{} to use that server only to
  find names in that particular domain.

  In order to configure \dmq{} to act as chache on the host on which it is
  running, put \colZt{"name server 127.0.0.1"} in \colZt{/etc/resolve.conf} to
  force local processes to send queries to \dmq{}.

  Then either

  \begin{enumerate}
  \item specifying the upstream servers directly to \dmq{} using
    \colZt{--server} options, or
  \item putting their addresses real in another file, say
    \texttt{/home/me/my-resolve.conf} and use \texttt{--resolv-file
      /home/me/my-resolve.conf}.
  \end{enumerate}

  The second method allows for dynamic update of server addresses by PPP or DHCP.

  \colZ{Addresses in \texttt{/etc/hosts} will ``shadow'' different addresses for
    the same names in the upstream DNS, so ``\texttt{aaa.com 1.2.3.4}'' in
    \texttt{/etc/hosts} will ensure that queries for ``\texttt{aaa.com}'' always
    return \texttt{1.2.3.4} even if queries in the upstream DNS would say
    something else.}There's one exception to this: it's something to do with an
  upstream \texttt{CNAME} entry, but let's stop here.
}

\subsection{How it does DHCP}

\colz{
  The tag system works as follows:

  \begin{enumerate}
  \item For each DHCP request, \dmq{} collects a set of valid tags from active
    configuration lines which include \texttt{set:<tag>}, including
    \begin{enumerate}
    \item one from the \colZ{\texttt{--dhcp-range}} used to allocate the
      address,
    \item one from any matching \colZt{--dhcp-host}
    \item and ``known'' or ``known-ethernet'' if a \texttt{--dhcp-host} matches.
    \end{enumerate}
  \item The tag ``bootp'' is set for BOOTP requests, 
  \item a tag whose name is the name of the interface on which the request
    arrived is also set.
  \end{enumerate}

  \cSay{
    Here are some details, first,

    \texttt{-F,--dhcp-range=
        \cola{tag1:<tag1>,tag2:<tag2>,...}\colb{, set:<tag>}
        \colc{<start-addr>,[<end-addr>],}
        [<many options>]
      } Enable the DHCP server.

      Addresses will be given out from the range \colZt{<start-addr>} to
      \colZt{<end-addr>} and from statically defined addresses given in
      \texttt{--dhcp-host} options.

      \colz{

        A lease time can be given. By default, it's one hour for IPv4 and
        one day for IPv6.

        For IPv6, there's an prefix length which must be equal to or larger then
        the prefix length on the local interface. This defaults to 64. Unlike
        the IPv4 case, the prefix length is not automatically derived from the
        interface. confiugration. (minimum prefix length = 64.)

      }
  }

  \cSay{
    \colz{
      IPv6 supports another type of range, In this, the start address and
      optional end address \colZ{contain only the network part (ie
        \texttt{::1})} and they are followed by
      \texttt{constructor:<interface>}, for example,
      \colZt{--dhcp-range=::1,::400,constructor:eth0}, will look for addresses
      on \texttt{eth0} , and then create a range from \texttt{<network>::1} to
      \texttt{<network>::400}.  Note that not just any address on
      \texttt{eth0} will not do: it must not be an autoconfigured or privacy
      address, or be deprecated.

      If a \texttt{--dhcp-range} is only being used for stateless DHCP, then
      it can simply be:
      \colZt{--dhcp-range=::,constructor:eth0}
    }
  }

  \dSay{\colz{What about \texttt{--dhcp-host} ?}}

  \cSay{
{

  \ttfamily
  -G,--dhcp-host= \cola{[<hwaddr>]}
  \colz{
    [,id:<client\_id>|*]
    [,set:<tag>] [,tag:<tag>] \colb{[,<ipaddr>]} \cola{[,<hostname>]} \colZ{[,<lease\_time>]} [,ignore]
  }  
}
    

    Specify per host parameters for the DHCP server. This allows a machine with
    a particular hardware address to be always allocated the same
    \begin{enumerate}
    \item hostname,
    \item IP address and
    \item lease time.
    \end{enumerate}
  }

  \dSay{
    Oh, this one is pretty important for those servers that need a static IP.
  }

  \cSay{
    \colz{
      A hostname specified like this overrides any supplied by the \colZ{DHCP
        client on the host}.

      It is also allowable to omit the hardware address and include the
      hostname. In this case, the IP address and lease times will \colZ{applied
        to any machine claiming that name}. For example
      \colZt{--dhcp-host=00:11:22:33:44:55,myhost,infinite} tells \dmq{} to give
      the machine with hardware address \texttt{00:11:22:33:44:55} the name
      ``\texttt{myhost}'' and an infinite lease. For another example,
      \colZt{--dhcp-host=myhost2,10.0.0.2} tells \dmq{} to give the machine
      ``\texttt{myhost2}'' the IP address \texttt{10.0.0.2}
    }
  }

  \dSay{
    What ? Hosts can have names before they are assigned an IP address ?
  }

  \cSay{It just said so in the manual. Let's just believe it.}

  \dSay{
    Okay...
  }

  \cSay{
    \colz{
      Addresses allocated like this are not constrained to be in the range
      given in \texttt{--dhcp-range} options. For subnets which don't need a pool
      of dynamically allocated addresses, you can use a ``\texttt{static}''
      keyword in the \texttt{--dhcp-range} declaration.
    }
  }

  \dSay{ Oh, so DHCP can also be used to assign static IP addresses?}

  \cSay{Seem so.}

  \dSay{
    I remember for IPv6, there're different ways to do automatic address
    configuration...Right? What's said about this in the manual?
  }

  \cSay{
    For IPv6, the mode may be some combination of
    \begin{enumerate}
    \item \texttt{slaac} \colz{stateless address autoconfiguration},
    \item \texttt{ra-only} \colz{router advertisement only},
    \item \texttt{ra-names} \colz{router advertisement plus names from DNS}.
    \item \texttt{ra-stateless} \colz{address from SLAAC and others from DHCP}.
    \item \texttt{ra-advrouter} \colz{enables a mode where router address(es)rather
      than prefix(es) are included in the advertisements.}
  \item \texttt{off-link} \colz{
      tells \dmq{} to advertise the prefix without the on-link bit set.
    }
    \end{enumerate}
  }

  \dSay{What is \texttt{client\_id} ?}

  \cSay{It's a unique identifier for the client. It's called \texttt{DUID} in
    IPv6. It's an alternative to the hardware address.

    \colz{
      Thus:

      \colZt{--dhcp-host=id:01:02:03:04,...} refers to the host with client
      identifier \texttt{01:02:03:04}. It is also possible to use text as in
      \colZt{--dhcp-host=id:myhost,...}.
    }
  }

  \cSay{
    \colz{
      A single \texttt{--dhcp-host} option may contain
      \begin{enumerate}
      \item an IPv4 address, or 
      \item one or more IPv6 addresses, or
      \item both
      \end{enumerate}.

      IPv6 addresses must be bracketed by square brackets thus:

      \colZt{--dhcp-host=laptop,[2001:db8::1],[2001:db8::2]}
    }
  }

  

  Any configuration lines which include one or more \texttt{tag:<tag>}
  constructs will only be valid if all that tags are matched in the set derived
  above. Typically this is \colZt{--dhcp-option}, tagged version of
  \texttt{--dhcp-options} is prefered, provided that all the tags match
  somewhere in the set collected as described above. The prefix \texttt{!} on a
  tag means ``not'' so \colZt{--dhcp-option=tag:!purple,3,1.2.2.3.4} sends the
  option when the taf \texttt{purple} is not in the set of valid tags.
  (\emoji{turtle} : If using
  this in a command line rather than a config file, be sure to escape
  \texttt{!}, which is a shell metachar...)

  When selecting \colZt{--dhcp-options}, a tag from \colZt{--dhcp-range} is
  second class relative to other tags, to make it easy to override options for
  individual hosts, so
  \colZt{
    \newline
    --dhcp-range=set:interface1,....\newline
    --dhcp-option=tag:interface1,\cola{option:nis-domain,domain1} \newline
    --dhcp-option=tag:myhost,\cola{option:nis-domain,domain2 }
  }

  will set the NIS-domain to \colZt{domain1} for hosts in range, but override
  that to \colZt{domain2} for \colZt{myhost}.

  \colZ{
    Note that for \texttt{--dhcp-range} both \texttt{tag:<tag>} and
    \texttt{set:<tag>} are allowed, to both select the range in use based on
    (eg) \texttt{--dhcp-host},and to affect the options sent, based on the range
    selected.
  }
}

% 
\section{NDP}


\Cola{Neighbor Discovery Protocol (NDP) }is a protocol that solves a some problems
related to the interaction between nodes \cola{attached to the same link}. Some
of these problems inlcude:

\begin{itemize}
\item Router Discovery: \colz{how to find routers on the link.}
\item Prefix Discovery: \colz{how to find the addresses of other nodes on the
    link.(Nodes uses prefixes to distinguish destinations that reside on-link
    from those only reachable through a router (these are usually public IPs).)}
\item Parameter Discovery: \colz{how to find link parameters such as MTU.}
\item Address Autoconfiguration: \colz{how to assign addresses to nodes.}
\item ...and more
\end{itemize}

NDP defines five different ICMP packet types as shown in \cref{fig:dnp-5kinds}.

\begin{figure}
  \centering
  \begin{tikzpicture}
    \draw[step=1cm,help lines,] (-7.5cm,-5cm) grid +(15cm,10cm);
    \node at (0,0) {
    \includesvg[width=8cm]{../../mysvgs/ndp.svg}
    };

    \draw (-4cm,0) node {RS}
          (-2.5cm,3cm) node {RA}
          (1cm,3.5cm) node {NS}
          (3cm,3.5cm) node {NA}
          (3.5cm,0cm) node {Redirect}
    ;
    
  \end{tikzpicture}
  \caption{Five kinds of ICMP defined by NDP}
  \label{fig:dnp-5kinds}
\end{figure}

The 5 steps are shown in \cref{fig:dnp-step12,fig:dnp-step34,fig:dnp-step5}.

\dSay{
  What's contained in RA ?
}

\cSay{
  \colz{

    RA contains a lot of information. For example, it contains a \cola{list of
      prefixes} used for \colZ{on-link determination} and/or \colZ{autonomous
      address configuration};flags associated with the prefixes specify the
    \cola{intended uses of a particular prefix}.

    Hosts use the advertised on-link prefixes to build and maintain a list that
    is used in deciding when a packet's destination is on-link or beyond a
    router.

  }
}

\dSay{
  Do all traffic on the same link go through the router ?
}

\cSay{ \colz{ No. If the destination is on-link, then it is not covered by any
    advertised on- link prefix.

    In such cases, a router can send a \colZt{Redirect} informing
    the sender that the destination is a neighbor.   

    As a result, the traffic will be sent directly to the
    destination.
  }
}

\dSay{
  So when to use DHCPv6 and when to use autonomous (stateless) address configuration ?
}

\cSay{
  \colz{
    There's a flag in RA that indicates whether or not to use DHCPv6.
  }
}

\begin{figure}
  \centering
 \begin{tikzpicture}
  \draw[step=1cm,help lines,] (-7.5cm,-5cm) grid +(15cm,10cm);
  \tikzstyle{ndp-dialog}= [text width=7cm, fill=gray!10]
  \begin{scope}

    \node[below right, ndp-dialog] at (-7cm,5cm){
      \coli{Step 1: \texttt{Router Solicitation (RS)}:}
      {\color{gray}\small
        When an interface becomes enabled,
        hosts may send a \texttt{Router Solicitation} that request routers to
        generate \texttt{Router Advertisement} messages immediately.
        
      }
    };
    
    \node[above right, ndp-dialog] at (-4cm,-4cm){
      \colii{Step 2: \texttt{Router Advertisement(RA)}:}
      \colz{\small
        Routers advertise their presence and many parameters periodically and in
        response to a \texttt{Router Solicitation} message. (\emoji{parrot}: Oh,
        so it's like heartbeat. \emoji{turtle} : Yeah, kinda.) 
      }
    };
    
    \fill (3cm,0) circle (0pt)
    node {
      \includesvg[width=8cm]{../../mysvgs/ndp5.svg}
    };

    \coordinate (newcomer) at (-1cm,0);
    \coordinate (router) at (4cm,0);

    \leftDialogBoxColored[3cm]{newcomer}{
        1. Who's the router?
    }{\mycoli}

    \rightDialogBoxColored[2cm]{router}{
        2. Me
    }{\mycolii}
  \end{scope}
\end{tikzpicture}
  \caption{Step 1 and 2 of NDP}
  \label{fig:dnp-step12}
\end{figure}

\begin{figure}
  \centering
 \begin{tikzpicture}
  \draw[step=1cm,help lines,] (-7.5cm,-5cm) grid +(15cm,10cm);
  \tikzstyle{ndp-dialog}= [text width=7cm, fill=gray!10]
  \begin{scope}

    \node[below right, ndp-dialog] at (-7cm,5cm){
      \coliii{Step 3: \texttt{Neighbor Solicitation (NS)}:}
      {\color{gray}\small
        Sent by a node to
        \begin{enumerate}
        \item verify that a neighbor is still up.
        \item determine the link-layer address
        \end{enumerate}
        This is also used for \colZ{Duplicate Address Detection}
      }
    };

    \node[above right, ndp-dialog] at (-4cm,-4cm){
      \coliv{Step 4: \texttt{Neighbor Advertisement (NA)}:}
      \colz{\small
        A response to a \texttt{Neighbor Solicitation} message.
      }
    };
    
    \fill (3cm,0) circle (0pt)
    node {
      \includesvg[width=8cm]{../../mysvgs/ndp5.svg}
    };

    \coordinate (newcomer) at (-1cm,0);
    \coordinate (router) at (4cm,0);

    \coordinate (n1) at (1.6cm,3.6cm);
    \coordinate (n2) at (6cm,3.4cm);
    \coordinate (n3) at (6cm,-2.3cm);

    \leftDialogBoxColored[4cm]{newcomer}{
      3. Who's in ? \newline
      \small
      \colz{Okay..then I can't use these addresses.
      }
    }{\mycoliii}

    \foreach \i in {1,2,3}{
      \topDialogBoxColored[2cm]{n\i}{
        4. Me \texttt{fe80::\i}
      }{\mycoliv}
    }

  \end{scope}
\end{tikzpicture}
  \caption{Step 3 and 4 of NDP}
  \label{fig:dnp-step34}
\end{figure}

\begin{figure}
  \centering
 \begin{tikzpicture}
  \draw[step=1cm,help lines,] (-7.5cm,-5cm) grid +(15cm,10cm);
  \tikzstyle{ndp-dialog}= [text width=7cm, fill=gray!10]
  \begin{scope}

    \node[below right, ndp-dialog] at (-7cm,5.5cm){
      \colv{Step 5: \texttt{Redirect}:}
      {\color{gray}\small
        Used by routers to inform hosts of a better first hop for a destination.
      }
    };

    \fill (0cm,0) circle (0pt)
    node {
      \includesvg[width=8cm]{../../mysvgs/ndp5-after.svg}
    };

    \coordinate (router) at (-.5cm,0);

    \leftDialogBoxColored[3cm]{router}{
      5. Hey, guys, here's a better first hop for \texttt{2001:db8::1},....
    }{\mycolv}
  \end{scope}
\end{tikzpicture}
  \caption{Step 5 of NDP}
  \label{fig:dnp-step5}
\end{figure}

% flush all the floats
\clearpage

\section{DHCP}
\label{sec:dhcp}

DHCP uses UDP. The client uses a link-local address (obtained from whatever
means) to send a DHCP message to a \cola{reserved multicast address}. Then the server will answer it.

\subsection{Terminology}
\label{sec:dhcp-term}

\begin{enumerate}
\item appropriate to the link: 
  \colz{
    An address is \cola{appropriate to the link} if
    the address is consistent with the DHCP's server's knowledge of the
    \begin{enumerate}
    \item network topology
    \item prefix assignment
    \item address assignment policies
    \end{enumerate}
  }
\item Identity association (IA) \colz{
    A collection of addresses assigned to a client.

    Usually, it's one IA per interface.

    Each IA has an associated \texttt{IAID}, which is chosen by the client.
    IAIDs should be unique within the client.
  }
\item IA for temporary addresses (IA-TA) \colz{
    A type of IA, self-explanatory.
  }
\item IA for non-temporary addresses (IA-NA) \colz{
    Another type of IA.
  }
\item \texttt{DUID} \colz{
    DHCP Unique Identifier. It's unique for both DHCP client and server.

\colZ{So it's not one per interface, but one per client.}
  }
\item binding \colz{

    A \cola{binding} is a set of some server data the server knows about the
    addresses in
    \begin{enumerate}
    \item an IA
    \item configuration explicitly assigned to the client
    \end{enumerate} .
    \colZ{
      So you can bind configuration to an IA, or explicitly to a client.
    }

    In the first case, binding is indexed by the tuple:
    \begin{center}
      \ttfamily
      (\cola{DUID}, \colb{IA\_Type}, \colc{IAID})
    \end{center}
    In the second case, binding is indexed just by \cola{\texttt{DUID}}.

    Some congifuration info do not need a binding. They are sent through a
    policy, such as info boardcast to all clients on the link.
  }
\item configuration parameter \colz{data to kept by server and to be
    sent to client.}
\item DHCP client \colz{a host using DHCP to obtain \colZ{configuration
      parameters}.}
\item DHCP server \colz{a node that responds to requests from DHCP clients.}
\item DHCP relay agent \colz{
    a node that acts as an intermediary to deliver DHCP messages between clients
    and servers. It's on the same link as the client, but not necessarily same
    as the server.
  }
\item DHCP domain \colz{
    a set of links managed by DHCP and operated by a single administrative entity.
  }
\item DHCP realm \colz{a name used to identify a DHCP domain.}
\item Reconfigure key \colz{
    A key supplied to a client by a server used to provide security for
    Reconfigure messages.
  }
\item transaction ID \colz{
    An opaque value used to match responses with requests.
  }
\end{enumerate}

\subsection{DHCP Constant}
\label{sec:dhcp-const}

Some constants used in DHCP are shown in \cref{tab:dhcp-const}.

\begin{table}[h]
  \centering
  \begin{tabularx}{\linewidth}{XsX}
    Name & Value & Description \\[1ex]
    \hline
    \verb|All_DHCP_Relay_Agents_and_Servers| & \verb|ff02::1:2| & A link-scoped
                                                                  multicast
                                                                  address. \\
    \verb|All_DHCP_Servers| & \verb|ff05::1:3| & A site-scoped multicast\\
    client UDP port & 546 & \\
    server UDP port & 547 & \\
  \end{tabularx}
  \caption{DHCP Constant}
  \label{tab:dhcp-const}
\end{table}

DHCP also defines some message types, shown in \cref{tab:dhcp-msg-type}.


\begin{longtblr}[caption = {DHCP Message Type.},
  label = {tab:dhcp-msg-type}
  ]%
  {colspec={ssX},
    rowhead = 1, % first row is shown on every page
  }
  \toprule
  % A & B & C \\
  \mbox{Name} & \mbox{Value} & \mbox{Description} \\
  \midrule
    \texttt{Solicit} & 1 & \colz{Sent by client to locate servers.} \\
    \texttt{Advertise} & 2 & \colz{Sent by server to indicate its presence.Usually as a
                           response to \texttt{Solicit}.} \\

    \texttt{Request} & 3 & \colz{Sent by client to request configuration parameters
                         (including addresses) from a server.} \\
    \texttt{Confirm} & 4 & \colz{Sent by client to say whether the once assigned
                         addresses are still \cola{appropriate to the link}.} \\
    \texttt{Renew} & 5 & \colz{
                       Sent by client to \colZ{the origin server} to extend the lifetimes of addresses and
                       to update other configuration parameters. 
                       }\\
    \texttt{Rebind} & 6 & \colz{
                        Similar to \texttt{Renew}, but sent to \colZ{any server}.
                        This is sent when no response to \texttt{Renew} is received.
                        } \\
    \texttt{Reply} & 7 & Sent by server to respond to \cola{all kinds of message} sent
                       by client.
                       \colz{
                       This more or less dumps the current configuration parameters that
                       server knows to the client.
                       } \\
    \texttt{Release} & 8 & \colz{
                         Sent by client to release some addresses assigned to it.
                         } \\
    \texttt{Decline} & 9 & \colz{
                         Sent by client to complain that the addresses assigned
                         to it are already in use.
                         } \\
    \texttt{Reconfigure} & 10 & \colz{
                              Sent by server to inform clients that the server
                              has new configuration parameters, so clients
                              should request them.(i.e. start a new
                              \texttt{Renew/Reply} transaction).
                              }\\
                              \bottomrule
\end{longtblr}



% We will want to consider what we need for special purpose devices such as
% iPhones, \Cola{Wireless Access Points (WAP)}, or even \Cola{Pre Execution
% Environment (PXE)} devices that can load their entire OS from DHCP
% information.




\section{IPv6 Jargen}

\begin{itemize}
\item IP : \colz{Internet Protocol Version 6.}
\item ICMP : \colz{Internet Control Message Protocol for IPv6. (e.g. ping)}
\item node : \colz{a device that implement IP.}
\item router: \colz{a node that forwards IP packets not explicitly addressed to
    itself.}
\item host: \colz{a node that is not a router.}
\item upper layer: \colz{any layer above IP. Examples are TCP , UDP, control
    protocols (ICMP)}
\item link : \colz{a communication facility or medium over which nodes can
    communicate at the link layer, i.e. the layer immediately below IP. Examples
    are } Ethernet, 802.11 (Wifi).
\item interface : \colz{a node's attachment to a link}
\item neighbors : \colz{nodes attached to the same link.}
\item address : \colz{an IP-layer identifier for an interface or a set of
    interfaces.}
\item anycast address : \colz{an identifier for a set of interfaces (typically
    belonging to different nodes). \colZ{A packet sent to an anycast address is
      delivered to one of the interfaces identified by that address.
    }

    An anycast address is syntactically indistinguishable from a unicast
    address. So nodes sending packets to anycast addresses don't generally known
    that it is anycast.
  }
\item prefix : \colz{a bit string that consists of some number of intial bits
    of an address.}
\item link-layer address : \colz{a link-layer identifier for an interface. For
    example, IEEE 802 addresses (such as MAC addresses)}
\item on-link, off-link : \colz{an address is \colZ{on-link}, if it is assigned to an
    interface on a specified link.

    A node considers an address to be \colZ{on-link} if:

    \begin{enumerate}
    \item it is covered by one of the \colZ{link's prefixes},(e.g., as indicated
      by the on-link flag in the Prefix Information option) or
    \item a neighboring router specifies the address as the target of a \cola{Redirect
        message}, or
    \item a \cola{Neighbor Advertisement message} is received for the address, or
    \item any \cola{Neibor Discovery} message is received from the address.
    \end{enumerate}

    \colZ{If an address is not \cola{on-link}, it is \cola{off-link}.} It means
    that it is not assigned to any interfaces on the specified link.
  }
\end{itemize}

\dSay{
  This first one is a bit confusing. What's \cola{Prefix Information option} ?
}

\cSay{
  I am not sure either. Let's just move on.
}

\begin{itemize}
\item longest prefix match : \colz{  the process of determining which prefix (if any) in
  a set of prefixes covers a target address. }
\item reachablility : whether or not the one-way ``forward'' path to a neighbor
  is functioning properly. \colz{ It's kinda like ``whether IP is working for a
    node''.}
\item packet : \colz{an IP header plus payload.}
\item link MTU : \colz{the maximum transmission unit, i.e. the maximum packet
    size in octets that can be transmitted on a link.}
\item target : \colz{an address about which address resolution information is
    sought, or an address that is the new first hop when being redirected.
  }
\item proxy : a node that responds to Neighbor Discovery query messages on
  behalf of another node. 
\end{itemize}

\section{Addresses}

% 2.4 Address Type Representation

%    The specific type of an IPv6 address is indicated by the leading bits
%    in the address.  The variable-length field comprising these leading
%    bits is called the Format Prefix (FP).  The initial allocation of
%    these prefixes is as follows:

%     Allocation                            Prefix         Fraction of
%                                           (binary)       Address Space
%     -----------------------------------   --------       -------------
%     Reserved                              0000 0000      1/256
%     Unassigned                            0000 0001      1/256

%     Reserved for NSAP Allocation          0000 001       1/128
%     Reserved for IPX Allocation           0000 010       1/128

%     Unassigned                            0000 011       1/128
%     Unassigned                            0000 1         1/32
%     Unassigned                            0001           1/16

%     Aggregatable Global Unicast Addresses 001            1/8
%     Unassigned                            010            1/8
%     Unassigned                            011            1/8
%     Unassigned                            100            1/8
%     Unassigned                            101            1/8
%     Unassigned                            110            1/8

%     Unassigned                            1110           1/16
%     Unassigned                            1111 0         1/32
%     Unassigned                            1111 10        1/64
%     Unassigned                            1111 110       1/128
%     Unassigned                            1111 1110 0    1/512

%     Link-Local Unicast Addresses          1111 1110 10   1/1024
%     Site-Local Unicast Addresses          1111 1110 11   1/1024

%     Multicast Addresses                   1111 1111      1/256

\subsection{Address type representation}

The initial bits of an address form the \Cola{Format Prefix (FP)}. This tells us
the type of the address. The initial allocation of these prefixes is as follows
in \cref{tab:fp}:

\begin{table}[h]
  \centering
  \begin{tabularx}{1.0\linewidth}{XXX}
    Allocation& Prefix & Fraction of Address Space \\
    \hline
    Reserved & \texttt{0000 0000 (00)} & 1/256\\
    Unassigned & \texttt{0000 0001 (01)} & 1/256\\
    Reserved for NSAP Allocation & \texttt{0000 001 (02)} & 1/128\\
    Reserved for IPX Allocation & \texttt{0000 010 (04)} & 1/128\\
    Link-Local Unicast Addresses & \texttt{1111 1110 10 (FE80)} & 1/1024\\
    Site-Local Unicast Addresses & \texttt{1111 1110 11 (FEC0)} & 1/1024\\
    Multicast Addresses & \texttt{1111 1111 (FF)} & 1/256\\
  \end{tabularx}
  \caption{The Format Prefix}
  \label{tab:fp}
\end{table}

In addition to the above, the prefixes \texttt{
  010, 011, 100, 101, 110, 1110, 1111 0, 1111 10, 1111 110, 1111 1110 0} are
\cola{unassigned} and reserved for future use.




\clearpage{}

\begin{itemize}
\item link-local address : \colz{a unicast address having link-only scope that
    can be used to reach neighbors.}
\item unspecified address : \colz{the address with all zero bits:
    \texttt{0:0:0:0:0:0:0:0},
    which kinda means ``I don't known my address''. It can be the source address.
  }
\item all-nodes multicast address : \colz{the link-local address to reach all
    nodes, \texttt{FF02::1}.}
\item all-routers multicast address : \colz{the link-local address to reach all
    routers, \texttt{FF02::2}.}
\end{itemize}

\end{document}

% Local Variables:
% TeX-engine: luatex
% TeX-command-extra-options: "-shell-escape"
% TeX-master: "m.tex"
% TeX-parse-self: t
% TeX-auto-save: t
% End:
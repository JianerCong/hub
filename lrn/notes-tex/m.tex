\documentclass[dvipsnames]{article}
% \documentclass[dvipsnames]{ctexart}

\title{Linux Network}

\usepackage{geometry}\geometry{
a4paper,
total={170mm,257mm},
left=20mm,
top=20mm,
}


\usepackage[inkscapearea=page]{svg}
% 🐢 : I guess what we need is to just include the page... not the whole
% drawing... So set inkscapearea=page should work... If not, delete the
% .\svg-inkscape and try again. (this pkg does some magical caching..)
\usepackage[skip=5pt plus1pt, indent=0pt]{parskip}
% Color
\newcommand{\mycola}{MidnightBlue}
\newcommand{\mycolb}{Mahogany}
\newcommand{\mycolc}{OliveGreen}

\newcommand{\mycoli}{\mycolb}
\newcommand{\mycolii}{\mycola!30!\mycolb}
\newcommand{\mycoliii}{\mycola!70!\mycolb}
\newcommand{\mycoliv}{\mycola!90!\mycolb}
\newcommand{\mycolv}{\mycola}

\newcommand{\cola}[1]{\textcolor{\mycola}{#1}}
\newcommand{\colb}[1]{\textcolor{\mycolb}{#1}}
\newcommand{\colc}[1]{\textcolor{\mycolc}{#1}}
\newcommand{\Cola}[1]{\textcolor{\mycola}{\textbf{#1}}}

\newcommand{\coli}[1]{\textcolor{\mycoli}{#1}}
\newcommand{\colii}[1]{\textcolor{\mycolii}{#1}}
\newcommand{\coliii}[1]{\textcolor{\mycoliii}{#1}}
\newcommand{\coliv}[1]{\textcolor{\mycoliv}{#1}}
\newcommand{\colv}[1]{\textcolor{\mycolv}{#1}}

% 🦜 : \textcolor doesn't allow multiple paragraphs in it, so we used {\color{...}}
\newcommand{\colZ}[1]{
{\color{black}#1}
} %go back
\newcommand{\colz}[1]{
{\color{gray}#1}
}

% \let\emph\relax % there's no \RedeclareTextFontCommand
% \DeclareTextFontCommand{\emph}{\bfseries}
\renewcommand{\emph}[1]{\texbf{#1}}
\usepackage{amssymb}            %\mathbb

\usepackage{fontspec}

\setmonofont{Cascadia}[
Path=/usr/share/fonts/truetype/Cascadia_Code/,
Scale=0.85,
Extension = .ttf,
UprightFont=*Code,              %find CascadiaCode.ttf
BoldFont=*CodePL,               %find CascadiaCodePL.ttf ...
ItalicFont=*CodeItalic,
BoldItalicFont=*CodePLItalic
]
% --------------------------------------------------
% Windows
% \setmonofont{Cascadia}[
% Path=C:/Windows/Fonts/,
% Extension = .ttf,
% UprightFont=*Mono,              %find CascadiaMono.ttf
% BoldFont=*Code,               %find CascadiaCodePL.ttf ...
% ItalicFont=*Code,
% BoldItalicFont=*Code
% ]


\usepackage{minted}
\usepackage{tcolorbox}
\tcbuselibrary{skins}
\tcbuselibrary{minted}
\usepackage{tikz}
\usetikzlibrary{shapes} % ellipse node shape
\usetikzlibrary{shapes.multipart} % for line breaks in node text
\usetikzlibrary{arrows.meta}    %-o arrow head
\usetikzlibrary{arrows}
\usetikzlibrary{matrix}
\usetikzlibrary{snakes}

\usepackage{amsmath}
% ??? still xelatex?
% \usepackage{xeCJK}
\usepackage{emoji}
% \setemojifont{NotoColorEmoji.ttf}[Path=C:/Users/congj/repo/myFonts/]
% \setemojifont{TwitterColorEmoji-SVGinOT.ttf}[Path=C:/Users/congj/repo/myFonts/]


\newtcolorbox[auto counter]{myBox}[2][]{
fonttitle=\bfseries,title={共识~\thetcbcounter: #2},#1
}
\newtcolorbox[]{noteBox}[1][]{
tile,left=1mm,nobeforeafter,fontupper=\small,#1
}

\tikzstyle{myNode}=[inner sep=2pt,circle,text=white]
\date{\today}
\author{作者}

\newtcblisting{simplec}{
listing engine=minted,
minted language=c++,
minted style=vs,
minted options={fontsize=\small,autogobble,
% framesep=1cm
},
tile,
listing only,
% bottom=0cm,
% nobeforeafter, 
boxsep=0mm,
left=1mm,
opacityback=0.5,
colback=gray!20
}
\tcbuselibrary{breakable}
\newtcblisting{simplepy}{
listing engine=minted,
minted language=python,
minted style=vs,
minted options={fontsize=\small,autogobble,
% framesep=1cm
},
tile,
listing only,
% bottom=0cm,
% nobeforeafter,
boxsep=0mm,
left=1mm,
opacityback=0.5,
colback=gray!20,
breakable
}
\newtcolorbox{blackbox}{tile,colback=black,colupper=white,nobeforeafter,halign=flush center}

% \tikzstyle{myMatrix}=[matrix of nodes,below right,
% nodes={above,text centered},                  %apply to all nodes
% row sep=1cm,column sep=2cm]
% \tikzstyle{every node}=[inner sep=0pt]

\newcommand\uptoleft[3][-o]{\draw[very thick,#1](#2.south) |- (#3.west);}
\newcommand\uptodown[3][-o]{\draw[very thick,#1](#2.south) to [out=270,in=90] (#3.north);}
\newcommand\downtoup[3][-latex]{\draw[very thick,#1](#2.north) to [out=90,in=270] (#3.south);}

\newcommand\lefttoright[3][-latex]{\draw[very thick,#1](#2.east) to[out=0,in=180] (#3.west);}
\newcommand\lefttodown[3][-latex]{\draw[very thick,#1](#2.east) to[out=0,in=90] (#3.north);}

\newcommand{\topDialogBoxColored}[4][0.8\linewidth]{
  \node[above,text width=#1] at ([yshift=0.5cm]#2.north) {
    \begin{tcolorbox}[tile,
      nobeforeafter,
      boxsep=0pt,
      % show bounding box,
      colback=#4!10,
      overlay={
        \begin{scope}
          \fill[#4!10] (frame.south) --
          +(2mm,0) --
          +(0,-3mm) --
          +(-2mm,0)
          ;
        \end{scope}
      } ] #3
    \end{tcolorbox}
  };
}

\newcommand{\rightDialogBoxColored}[4][0.8\linewidth]{
  \node[right,text width=#1] at ([xshift=0.5cm]#2.east) {
    \begin{tcolorbox}[tile,
      nobeforeafter,
      boxsep=0pt,
      % show bounding box,
      colback=#4!10,
      overlay={
        \begin{scope}
          % \fill[gray!10] (frame.east) circle (2pt);
          \fill[#4!10] (frame.west) --
          +(0,2mm) --
          +(-3mm,0) --
          +(0,-2mm)
          ;
        \end{scope}
      } ] #3
    \end{tcolorbox}
  };
}

\newcommand{\rightDialogBox}[3][0.8\linewidth]{
% \node[right,text width=#1] at ([xshift=0.5cm]#2.east) {
% \begin{tcolorbox}[tile,
% nobeforeafter,
% boxsep=0pt,
% % show bounding box,
% colback=\mycola!10,
% overlay={
% \begin{scope}
% % \fill[gray!10] (frame.east) circle (2pt);
% \fill[\mycola!10] (frame.west) --
% +(0,2mm) --
% +(-3mm,0) --
% +(0,-2mm)
% ;
% \end{scope}
% } ] #3
% \end{tcolorbox}
% };
  \rightDialogBoxColored[#1]{#2}{#3}{\mycola}
}


\newcommand{\leftDialogBoxColored}[4][0.8\linewidth]{
  \node[left,text width=#1] at ([xshift=-0.5cm]#2.west) {
    \begin{tcolorbox}[tile,
      nobeforeafter,
      boxsep=0pt,
      % show bounding box,
      colback=#4!10,
      overlay={
        \begin{scope}
          % \fill[gray!10] (frame.east) circle (2pt);
          \fill[#4!10] (frame.east) --
          +(0,2mm) --
          +(3mm,0) --
          +(0,-2mm)
          ;
        \end{scope}
      } ] #3
    \end{tcolorbox}
  };
}

\newcommand{\leftDialogBox}[3][0.8\linewidth]{
% \node[left,text width=#1] at ([xshift=-0.5cm]#2.west) {
% \begin{tcolorbox}[tile,
% nobeforeafter,
% boxsep=0pt,
% % show bounding box,
% colback=gray!10,
% overlay={
% \begin{scope}
% % \fill[gray!10] (frame.east) circle (2pt);
% \fill[gray!10] (frame.east) --
% +(0,2mm) --
% +(3mm,0) --
% +(0,-2mm)
% ;
% \end{scope}
% } ] #3
% \end{tcolorbox}
% };
  \leftDialogBoxColored[#1]{#2}{#3}{gray}
}




\newcommand{\mycolaa}{\mycola!20}

\newcommand{\colZt}[1]{
  \colZ{\texttt{#1}}
}

\usepackage{changepage}   % for the adjustwidth environment
\newenvironment{myIndent}[1][7mm]{\begin{adjustwidth}{#1}{}}{\end{adjustwidth}}

\newcounter{myDefCounter}
\newcounter{myTheoCounter}

\tcbuselibrary{theorems}
\newtcbtheorem[use counter=myDefCounter,number within=section]{myDef}{Definition}%
{
% colback=green!5,colframe=green!35!black,
fonttitle=\bfseries}{def}

\newcommand{\myThmBeforeProof}{
\texttt{Proof: }\par
\begin{myIndent}
}

\newcommand{\myThmAfterProof}{
\end{myIndent}
\qed
}

\tcbuselibrary{skins}
\tcbsubskin{myThmSkin}{enhanced}{
parbox=false,
% colback=green!5,
colframe=\mycola,
fonttitle=\bfseries,
breakable,
before lower={\myThmBeforeProof{}},
after lower={\myThmAfterProof{}}
}

\newtcbtheorem[
use counter=myTheoCounter,number within=section
]{myTheo}{Theorem}%
{
skin=myThmSkin
}{thm}


\newtcbtheorem[use counter=myTheoCounter,number within=section]{myCor}{Corollary}%
{
% colback=green!5,
skin=myThmSkin
}{cor}

\newtcbtheorem[use counter=myTheoCounter,number within=section]{myLem}{Lemma}%
{% colback=green!5,
skin=myThmSkin
}{lem}

\newcommand{\refCorollary}[1]{Corollary~\ref{cor:#1}}
\newcommand{\refLemma}[1]{Lemma~\ref{lem:#1}}

\usepackage{hyperref}

\usepackage{cleveref}           %🦜 : must be loaded after hyperref
\crefname{myDefCounter}{definition}{definitions}
% ^^^ plural
\Crefname{myDefCounter}{Definition}{Definitions}
% ^^^^^^ type = counter name

\crefname{myTheoCounter}{theorem}{theorems}
\Crefname{myTheoCounter}{Theorem}{Theorems}

\usepackage{amsthm}             %for {proof}

\usepackage{tabularx}



\usepackage{enumitem}
\setlist[description]{leftmargin=0.1\linewidth,labelindent=0.1\linewidth}
\usepackage{placeins}           %for \FloatBarrier
% 🦜 : Remember to specify the mode, because without a \documentclass{}, emacs
% will consider this as a plain TeX document.

% --------------------------------------------------
\newcommand{\ans}{
  % \newline{}
  \par
  \hspace{0.1\linewidth}
  % \phantom{Ans: }
  % \colz{Ans: }
  \tikz[baseline=.5em]{\draw[thick,\mycola] (0,0) -- (0.8\linewidth,0pt);}
  % \rule[-1em]{0.8\linewidth}{1pt}
}
\newcommand{\tf}{\Cola{\hfill[T/F]}}

\usepackage{tabularray}
\UseTblrLibrary{booktabs,siunitx}
% Local Variables:
% mode: LaTeX
% TeX-engine: luatex
% TeX-command-extra-options: "-shell-escape"
% TeX-master: "m.tex"
% TeX-parse-self: t
% TeX-auto-save: t
% End:

% svgs
\newcommand{\svgOs}[2][0.3\linewidth]{\includesvg[width=#1]{/home/me/Pictures/opstk/#2.svg}}
% \newcommand{\svgNeutron}[1][1.5cm]{\includesvg[width=#1]{/home/me/Pictures/opstk/neutron.svg}}
\newcommand{\svgD}[1][8mm]{\includesvg[width=#1]{/home/me/Pictures/d1.svg}}
\newcommand{\svgC}[1][8mm]{\includesvg[width=#1]{/home/me/Pictures/c1.svg}}

% params:
% 1: width of dialog box
% 2: the speaker
% 3: the text
\newcommand{\leftSay}[3][0.7\linewidth]{
  \begin{tikzpicture}
    \node (N1) {
      #2
    };
    \leftDialogBox[#1]{N1}{
      #3
    }
  \end{tikzpicture}
}

\newcommand{\rightSay}[3][0.7\linewidth]{
  \begin{tikzpicture}
    \node (N1) {
      #2
    };
    \rightDialogBox[#1]{N1}{
      #3
    }
  \end{tikzpicture}
}

% 🦜 : Let d1 and c1 say and ask .
\newcommand{\dSay}[1]{
  \begin{tikzpicture}
    \node (D1) {
      \svgD
    };
    \rightDialogBox[0.7\linewidth]{D1}{
      #1
    }
  \end{tikzpicture}
}


\newcommand{\cSay}[1]{
  \begin{flushright}
    \begin{tikzpicture}
      \node (C1) {
        \svgC
      };
      \leftDialogBox[0.7\linewidth]{C1}{
        #1
      }
    \end{tikzpicture}
  \end{flushright}
}

\hypersetup{
  colorlinks=true,
  linkcolor=blue,
  filecolor=magenta,      
  urlcolor=cyan,
  pdftitle={Linux Network},
  pdfpagemode=FullScreen,
}
\urlstyle{same}

% --------------------------------------------------
\begin{document}
\maketitle

\begin{comment}
\section{Media Access Control (MAC) Address}

\subsection{What does it look like}

MAC address has format something like: \texttt{00-0c-29-3b-73-cb}

\cSay{
  In practice, these addresses are used to communicate between hosts in the
  same VLAN or subnet. If you look at a packet capture, at the start of a TCP
  conversation you'll see the sending host send a broadcast \Cola{ARP request} saying
  \begin{center}
    ``Who has IP address x.x.x.x ?''
  \end{center}
  The \Cola{ARP reply} from some host will be
  \begin{center}
    ``That's me, and my MAC address is aaaa.bbbb.cccc ?''
  \end{center}
}

\dSay{
  What if the target IP is on a different subnet ?
}

\cSay{
  Then the sender will \Cola{ARP for} the gateway for that subnet.

  Switch is much faster than router, because hosts can use MAC to communicate.
}

\dSay{So why router ?}

\cSay{
  Because switches only work in LAN.
}

\subsection{Organizationally Unique Identifier (OUI)}

\cSay{The leading (usually 3 ) bytes of a MAC address usually points to its
  manufacturer, like \texttt{OOtronics Ltd. Kibbutz Yizrael, IL} or \texttt{OOel Corporate Kedah,
    MY} see \url{https://standards-oui.ieee.org/oui/oui.txt}
}

\subsection{Address Resolution Protocol (ARP)}

\Cola{ARP} is used to find the MAC address of a network neighbour for a given IPv4 Address.

\cSay{
  When a host wants to \cola{talk} to another host \cola{in the same subnet} using its IP address, it will
  \begin{enumerate}
  \item check its \Cola{ARP cache} to see whether there's a MAC address that matches that IP.
  \item If there isn't, it will send an ARP request to the local broadcast address.
  \item When the corresponding MAC is found, it will \colb{establish port-to-port communications (Next)}.
  \end{enumerate}
}

\section{Port to Port : TCP and UDP}

\cSay{There're two types of ports:
  \begin{description}
  \item[fixed server ports:] $0 <= p <= 1023$ usually for \cola{server}
  \item[ephemeral (short-lived) ports:] $1024 <= p <= 65535$ usually for
    \cola{client}. Among them, we have
    \begin{itemize}
    \item $1024 <= p <= 49151$ user ports
    \item $49152  <= p <= 65535$ dynamic or private ports
    \end{itemize}
  \end{description}
}


\section{DNS}

\subsection{Public DNS}

DNS has a large and complex infrastructure on the internet. This is made up of
\begin{itemize}
\item 13 \cola{root name servers}(which are each a reliable cluster
  of servers)
\item a group of commonly used name servers (for instance, the servers at Google or Cloudflare)
\item a series of registrars who will, for a fee, register a DNS domain name for
  you -- for instance, your organization's domain name.
\end{itemize}

\dSay{So Route 53 service in AWS is one of the third kind}

\cSay{I guess so.}

\dSay{Who are the 13 clusters?}

\cSay{
  They answer you questions liks : ``Where is the \texttt{.com} server?''
  \colz{(These servers are called \Cola{top-level domain (TLD)} authoritative name
    server.)
    In face, all 13 root (clusters of) servers  are same for redundancy.
    They are listed in \url{https://www.iana.org/domains/root/servers}. For example,
    Verisign,Inc, Department of Defence, NASA,...
  }
}

\dSay{
  Oh, do they need a consensus mechanism then ?
}

\cSay{
  I think say do. But let's stop.
}

\subsection{Internal DNS}
\cSay{

  \colz{
    However, for the most part, most administrators are working with the needs of
    their organization -- working with their \Cola{internal DNS} name servers that
    face their internal folks, or with their \colb{external DNS} name servers that
    face the internet. 
  }

}

The internal DNS server has a \Cola{zone file} populated with DNS records for internal
DNS resolution. This file can either be populated
\begin{itemize}
\item manually by editing it
\item automatically by the clients
\item automatically by \Cola{Dynamic Host Configuration Protocol (DHCP) leases}.
\end{itemize}

\cSay{Often, all three methods are combined.}

\subsubsection{basic request flow}


The basic request flow is simple. A client makes a DNS request. \cola{If that
  request is for a host that's internal to the organization and the request is
  to an internal DNS server,} the DNS response is supplied immediately since
it's on that local DNS server.

If it's for an external host, then things are bit more complex -- for instance,
let's query for \texttt{www.example.com}.

\begin{tikzpicture}
  % \draw[step=1cm,help lines,] (0,0) grid +(15cm,10cm);
  
  \tikzstyle{dnsStyle}=[inner sep=7pt,fill=\mycola!20,rounded corners,
  text width=2.5cm,text centered]

  % the wall
  \draw[fill=gray]
  (6cm , 0) node[text centered] {Organization's Firewall}
  ++(-0.5cm,1cm) rectangle ++(1cm,9cm);

  \newcounter{req}
  \setcounter{req}{1}
  \newcommand{\dns}{\textbf{DNS}}

  % internal dns
  \node at (9cm, 3cm)[dnsStyle, name=internal-dns, text width=2cm] {Internal \dns};

  % us
  \node at ([xshift=3cm]internal-dns) [name=we,dnsStyle,fill=\mycolb!20, text width=1cm] {we};

  \newcommand{\refPlusPlus}{
    \arabic{req}
    \stepcounter{req}
  }

  

  % \uptodown[-latex]{we}{internal-dns}
  \draw[-latex] (we) --
  node [midway,above] {\refPlusPlus{}}
  (internal-dns);
  
  \matrix (M1) [matrix of nodes, nodes={style=dnsStyle},
  row sep=1.5cm,matrix anchor=south]{
    Public \dns{} Server (forwarder) \\
    \dns{} Root Name Servers \\
    \dns{} Authoritative server for \texttt{.com}\\
    \dns{} Authoritative server for \texttt{example.com}\\};

  % \draw[->] (M1-1-1.west) .. controls +(left:5mm) and +(left:10mm) ..  (M1-2-1.west);
  % 🦜 : another way 

  \newcommand{\comeAndGo}[2]{
    \draw[-latex] (#1) to[out = 180, in=180] node[auto,swap] {
      \refPlusPlus{}
    } (#2);

    \draw[-latex] (#2) to[out = 0, in=0] node[auto,swap] {
      \refPlusPlus{}
    } (#1);
  }

  
  % over the wall
  \draw[-latex] (internal-dns) to[out=180,in=0]
  node [midway, left] {\refPlusPlus{}}
  (M1-1-1);


  % pass around
  \comeAndGo{M1-1-1}{M1-2-1}{2}{3}
  \comeAndGo{M1-1-1}{M1-3-1}{4}{5}
  \comeAndGo{M1-1-1}{M1-4-1}{6}{7}

  % go back  in wall
  \draw[-latex] (M1-1-1) to[out=90,in=90]
  node [midway, left] {\refPlusPlus{}}
  (internal-dns);

  % back to us
  \draw[-latex] (internal-dns) to[out=90,in=90]
  node [midway, left] {\refPlusPlus{}}
  (we);
  
\end{tikzpicture}

The worst-case process is as follows:

\newcommand{\goTo}[1]{$\Rightarrow $ Step #1}
\begin{enumerate}
\item If the entry is in the \cola{DNS cache} of the \cola{internal DNS server},
  and the \Cola{time to live (TTL)} of that entry has not expired, then the
  response is supplied \cola{immediately }to the client.

  Similarly, if the client is requesting an entry that's hosted on the server in
  a \Cola{zone file}, the answer is supplied immediately to the client.
\colz{  (\emoji{parrot} : So it feels like the zone file holds the ``defined DNS
  entries'', and the cache holds entries of ``somebody else's zone file's
  entry''?, \emoji{turtle} : Seem so....)}
\item If the entry is not in the cache of the internal DNS server, or if it's
  expired, then the server \cola{forwards the request} to its \Cola{upstream
    providers} (often called \Cola{forwarders}) to refresh the entry.

  If the query is in the \cola{cache} of the forwarder, it will simply return the
  answer. If not found, \goTo{3}.

  If this server has the \cola{authoritative name server} for the domain, it
  will simply query that host (\goTo{5}).
\item Forwarder request upstream. In this case, though, it will likely query the
  \Cola{root name servers}.
\colz{  The goal in this is to find the ``authoritative name
  server'' that has the actual entries (in a zone file) for that domain. In this
  case, the query is made to the root name servers for \texttt{.com}.}
  % \goTo{4}
\item The \Cola{root name server} will not return the actual answer, but will
  instead return the authoritative name server for the \Cola{top-level domain
    (TLD)} -- in this case, for \texttt{.com}. 
  \colz{(
    \emoji{parrot} : Oh, so this is
    the server that you can ask ``Who should I ask for \texttt{.com} name''.
    \emoji{turtle} : Yeah.)}
\item After the forwarder gets this response, it updates its \cola{cache} with
  that \cola{name server entry}, then makes the actual query against that
  server. \colz{(\emoji{parrot} : Oh, so there's a cache entry just for the \texttt{.com} name
  server. \emoji{turtle} : If you can cache \texttt{www.example.com}, why not
  \texttt{.com}? \emoji{parrot} : Oh...)}
\item The authoritative server for \texttt{.com} returns the authoritative DNS
  server for \texttt{example.com}.
\item
\colz{\colZ{The forwarder server} then makes a request against this \colZ{final authoritative name server}.}
\item
\colz{The \colZ{authoritative name server} for \colZ{\texttt{example.com}} returns the actual query
  ``answer'' to the \colZ{forwarder server}.}
\item \colz{The \colZ{forwarder name server} caches that answer, then sends a reply back to
  your \colZ{internal name server}.}
\item \colz{Your \colZ{internal DNS server }also caches that answer, then
    forwards it back to the client. The \colZ{client} caches the request in its
    local cache, then passes the requested information (the DNS response) to the
    application that requested it (perhaps your web browser)}. (\emoji{parrot} :
  This is like ``backward propagation?'' \emoji{turtle} : No, not at all. They
  just return the previously asked requests....)
\end{enumerate}

\cSay{ \colz{
    Again, this process shows the \colZ{worst-case}. In practice, once the
    servers have been up for even a short period of time, caching shortens this
    \colZ{considerably}. Once in a steady state, the \colZ{internal DNS server}
    for most organizations will have most requests cached.
  }
}

\dSay{So usually just two steps. Right?}

\cSay{Yeah.
  \colz{
    In addition, your \colZ{forwarding DNS server} will cache -- in particular,
    it will almost never query the root name servers; usually, it will have \colZ{the
    TLD servers (in this case, the server for \texttt{.com} cached as well.)}
  }
}



\subsubsection{Needed features}
\label[subsubsection]{sec:needed-features}
\dSay{
  \colz{
    Which key features fo we need to enable on our \colZ{internal DNS server} to make
    all this work?
  }
}

\cSay{We need:}

\begin{enumerate}
  \item DNS recursion: \colz{This model relies on DNS recursion -- the
      ability for each server in turn to make the client's DNS request ``up the
      line''. }
  \item Forwarder entries: \colz{If the requested DNS entry is not hosted on the
    internal server, \Cola{internal DNS service (iDNS)} requests are forwarded
    to these configured IP addresses -- these should be two or more reliable
    upstream DNS servers.

    These upstream servers will in turn cache DNS entries.

    In days past, people would use their \colZ{internet service provider's (ISP)
      (e.g. \texttt{OO Telecom})} DNS servers for forwarders. In more modern
    times, the larger DNS providers are both more reliable and provide more
    features. Some are shown in \cref{tab:dns-prov}
  }
\item Caching: \colz{Just cache what it knows.}
\item Dynamic registration: \colz{While server usually have static IP addresses
    and static DNS entries, it's common for client hosts to have addresses
    assigned by DHCP, and having those hosts in DNS would be nice.

    DNS is often configured to allow dynamic registration of these hosts, either
    by populating DNS from DHCP addresses as they are assigned or by permitting
    the hosts to register themselves in DNS.

    Microsoft implements an authentication mechanism into their dynamic update
    process, and this is where it is most commonly seen. It is, however, an
    option in Linux DNS \colZ{(Berkeley Internet Name Domain, or BIND)} as well.
  }
\item Host redundancy: \colz{Please get a backup.}
\end{enumerate}

\begin{table}
  \centering

  \begin{tabularx}{0.8\textwidth} {X|X}
    \hline
    \textbf{Provider} & \textbf{Addresses}\\
    \hline
    Google & \texttt{8.8.(8.8|4.4),\newline
      2001:4860:4860::(8888 | 8844)
    }\\
    \hline
    Cloudflare & \texttt{1.(1.1 | 0.0).1,\newline
      2606:4700:4700::(1111 | 1001)
    }\\
    \hline
    Quad9 & \texttt{9.9.9.9,\newline
      149.112.112.112,\newline
      2620:fe::(fe|9),
    }\\
    \hline
    OpenDNS (now Cisco Umbrella) &
    \texttt{
      208.67.(222 | 220).(220 | 222),
      2620:119:(35::35 | 53::53)
    }
  \end{tabularx}

  \caption{Common DNS providers}
  \label{tab:dns-prov}
\end{table}

\subsection{internet-facing DNS server}

\dSay{
  \colz{
    So how to make a DNS server to serve a public zone ? For example, if I already
    have \texttt{aaa.com} as our domain, how can we serve \texttt{bbb.aaa.com}?
  }
}

\cSay{
  \colz{
    For a public-facing DNS server, \colZ{security} matters more than
    \colZ{efficiency.}
  }
  In particular, we might implement:
}

\begin{enumerate}
\item Restrict recursion: \colz{This server should \colZ{answer the request
      directly}. It doesn't forward it. }
\item Cache is less important : \colz{Similar reason. This server is about
    \colZ{answering}, not \colZ{asking}.}
\item Host redundancy: \colz{Please get a back up...}
\item Restricting zone transfers:
\colz{  You might want to answer individual DNS
  queries as they arrive. There isn't a good reason for a DNS client on the
  internet to request all entries for an organization. (\emoji{parrot} : What
  does it mean..? \emoji{turtle} : I didn't get that either.)

  \colZ{Zone transfers} are meant to maintain your zone between redundant
  servers so that as a zone is edited, the changes are replicated to the other
  servers in the cluster. (\emoji{parrot} : Oh...so zone transfer is like a
  mirroring process? \emoji{turtle} : Yeah, more or less.)
}
\item Rate limiting : \colz{anti-DoS}
\item Restricting dynamic registration: \colz{
    This is usually disabled for most cases.}
\end{enumerate}

\subsection{common DNS implementations}

\Cola{BIND} (see \cref{sec:needed-features}). also called \Cola{\texttt{named} (name
  daemon)}, is the DNS tool most often implemented in Linux.

\colz{

  It's and is arguably both the most flexible and complete, as well as the most
  difficult to configure and troubleshoot. For better or worse, though, it's the
  service you are most likely to see and to implement in most organizations.
}

\cSay{
  \colz{
    \colZ{DNS masquerade (\texttt{dnsmasq})} is a competing DNS server
    implementation. It's commonly seen on network appliances because of its
    \colZ{small footprint}, but also makes a fine DNS server for a smaller
    organization. The key advantages to \texttt{dnsmasq} would be:
    \begin{enumerate}
    \item it has a \colZ{built-in GUI}
    \item it has \colZ{integration with DHCP}, allowing DNS registration
      directly from the DHCP database.
    \end{enumerate}

    In addition, \texttt{dnsmasq} implements a friendly way to implement
    \colZ{DNS blocklists}, which are very nicely packaged up in the Pi-hole
    application.  If your home network has a DNS server on its perimeter
    firewall or \colZ{Wireless Access Point (WAP)}, that DNS server is most
    likely \texttt{dnsmasq}.

    However, here, we will only talk about \colZ{BIND} in \cref{sec:install-bind}.
  }
}

\end{comment}

\subsection{Basic installation: BIND for internal use}
\label[subsection]{sec:install-bind}

Detailed installation process sees \texttt{linux-network-lower.org}.

\subsubsection{SOA record}

\colz{

  A \Cola{start of authority record (SOA record)} is a type of \colZ{resource
    record} in the Domain Name System (DNS) containing administrative
  information about the zone, espacially regarding \colZ{zone transfers}. The
  SOA record format is specified in RFC 1025. (\emoji{parrot} : Oh, so it's not
  a BIND thing. \emoji{turtle} : It's not.)

  Normally DNS name servers are set up in clusters. The database within each
  cluster is synchronized through \colZ{zone transder}. \colZ{The SOA record for
    a zone contains data to control the zone transfer.} This is the serial
  number and different timespans.

  It also contains the \colZ{email address} of the responsible person for this
  zone, as well as the name of the \colZ{primary master name server.} Usually
  the SOA record is located at the top of the zone. A zone without a SOA record
  does not conform to the standard required by RFC 1035. 
}

\dSay{
  So what's the structure of a record?
}

\cSay{
  It's usually
  \begin{enumerate}
  \item \texttt{NAME}: \texttt{@} \colz{is a shortcut to match previous record
      in BIND}.
  \item \texttt{CLASS}: \texttt{IN} \colz{for internet}
  \item \texttt{TYPE (= SOA)}
  \item \texttt{MNAME}: Primary master name for this zone.
  \item \texttt{RNAME}: Email address of the admin respinsible for this zone.
    \colz{(As usual (\emoji{turtle} : Is it?), the email address is encoded as a
      name. The part of the email address before the \texttt{@} becomes the
      \colZ{first label} of the name; the domain name after the \texttt{@}
      becomes the \colZ{rest of the name}.

      In zone-file format, dots in labels are escaped with blackslashes; thus
      the email address \texttt{aaa.doe@example.com} would be represented in a
      zone file as
      \texttt{aaa\textbackslash{}.doe.example.com}
      )
    }
  \item \texttt{SERIAL}: Serial number for this zone.
  \item \texttt{REFRESH}: \colz{Number of seconds after which secondary name
      servers should \colZ{query the master} for the SOA record, to detect zone
      changes.} \colz{(recommanded = 86 400 = 24 hrs)}

  \item \texttt{RETRY}: \colz{Number of seconds after which secondary name
      servers should \colZ{retry to request} the serial number from the master.
      recommanded = 7 200 = 2hr.}

  \item \texttt{EXPIRE}: = 1000hrs
  \item \texttt{MINIMUM}: = 2 days.

  \end{enumerate}
}

\subsubsection{Fully qualified domain name (FQDN)}

\colz{Fully qualified domain name is a \colZ{domain name} that specifies its
  exact location in the tree hierarchy of the DNS. It specifies all domain
  levels, including the \colZ{top level domain} and the \colZ{root zone}.

  Example : \texttt{en.wikipedia.org.} (note the last dot.)

  What's not : \texttt{localhost}
}

\subsubsection{The important zone file}
\label{sec:zone-file}

\colz{The \colZ{zone files} are where the DNS records are all stored. They are
  in the \colZ{\texttt{/var/cache/bind/}}
  folder. For example,
  \texttt{/var/cache/bind/aaa.net.zone}.
}

Usually you will need to do the following:

\colz{
  \begin{enumerate}
  \item Add records as needed.
  \item Update the \colZ{SOA} line with your zone and name server's \colZ{FQDN}.
  \item If needed, update the \colZ{TTL} calue in the last line in the
    \colZ{SOA} record -- the default is \colZ{24 hours} (86400 seconds). This is
    usually a good compromise as it favors caching of records across multiple
    servers

    If you are doing any DNS maintenance, though, you might want to
    edit the file the day before (that is, 24 hours or more prior to maintenance
   ) and shorten this to 5 or 10 minutes so that your changes aren't delayed due
   to caching.
 \item Update the \colZ{ns} record, which identidies the DNS server(s) for your
   domain.
 \item Add \Cola{A record} as needed -- these identify the IP addresses for each
   host. Note that for \Cola{A records,} we're only using the 
  \end{enumerate}
}

\newcommand{\dmq}{\texttt{dnsmasq}}
\subsection{\dmq{}}

\subsubsection{Intro}


\Cola{\dmq{}} is a lightweight DNS, TFTP, PXE, router advertisement and
DHCP server. It is intended to provide coupled DNS and DHCP service to a LAN.

\colz{
  \dmq{} \colZ{accepts DNS queries} and either answers them from a small, local, cache
  or forwards them to a real, recursive, DNS server. It loads the contents of
  \colZ{\texttt{/etc/hosts}} so that \colZ{local hostnames which do not appear in the global DNS can be resolved and also
    answers DNS queries for DHCP  configured hosts.}

  It can also act as the authoritative DNS server for one or more domains,
  allowing local names to appear in the global DNS. 

  \colZ{The \dmq{} DHCP server} supports static address assignments and multiple
  networks. It automatically sends a sensible default set of DHCP options
  (configurable).
}

\dSay{
  Oh.. I just realize that, for a local DNS name resolve, you just need to edit
  the \texttt{/etc/hosts}...
}

\subsubsection{Configure \dmq{}}

\colz{
  Configuring \dmq{} to act as an authoritative DNS server is complicated by the
  fact that it involves configuration of \colZ{external DNS servers} to provide
  delegation. 

  At startup, \dmq{} reads \colZ{\texttt{/etc/dnsmasq.conf}} (configurable
  through \texttt{--conf-file, --conf-dir}) The format of this file consists of
  one option per line, exactly as the long options detailed in the OPTIONS
  section but without the leading \texttt{--}. Line starting with \texttt{\#}
  are comments and ignored. (\emoji{parrot}: latest style)  For options which may only be
  specified once, \colZ{the configuration file overrides the commandline.}

  \colZ{Quoting} is allowed in a config file: between \texttt{"} quotes the
  special meanings of ``\colZ{\texttt{,:;\#}}''. The following escapes are
  allowed: ``\colZ{
    \texttt{
      \textbackslash{}\textbackslash{}
      \textbackslash{}"
      \textbackslash{}t
      \textbackslash{}e
      \textbackslash{}b
      \textbackslash{}r
      \textbackslash{}n
    }
  }'' }

\subsubsection{Notes for \texttt{SIGHUP}}
\colz{
  When it recieves a \colZ{\texttt{SIGHUP}}, \dmq{} \colZ{clears its cache and
    then re-loads
    \texttt{/etc/hosts, /etc/ethers}
    and any file given by \texttt{--dhcp-hostsfile, --dhcp--hostsdir,
      --dhcp-optsfile, --dhcp-optsdir, --addn-hosts, --hostsdir}.
  }  

  The DHCP lease change script is called for all existing DHCP leases. if
  \colZ{\texttt{--no-poll}} is set, SIGHUP also re-reads
  \texttt{\colZ{/etc/resolve.conf}}.
}

\texttt{SIGHUP} does \Cola{not} re-read the configuration file.

\dSay{
  So it's like a ``refresh''.
}

\cSay{Yeah.}

\subsubsection{dump log}

\colz{
  
  When it receives a \colZ{\texttt{SIGUSR1}}, \dmq{} writes stastics to the
  system log.

  When it receives a \colZ{\texttt{SIGUSR2}}, \dmq{} is logging directly to a
  file. (configured by \texttt{--log-facility}).
}

\subsubsection{some options}

\begin{itemize}
\item \texttt{-h,--no-hosts}: \colz{Don't read the hostnames in
    \texttt{/etc/hosts}.}
\item \texttt{-H,--addn-hosts=<file>}: \colz{Read the hostnames in
    \texttt{<file>} in addition to \texttt{/etc/hosts}.

    If \texttt{--no-hists} is given, read only \texttt{<file>}.
  }
\item \texttt{-d,--no-daemon}: \colz{Debug mode: \colZ{don't fork to the background}.
    don't write a pid file, don't change user id, generate a complete cache dump
    on receipt on \colZt{SIGUSR1}, log to stderr as well as syslog, don't fork new
    processes to handle TCP queries.

    \emoji{turtle} : Use \texttt{-k,--keep-in-foreground} in production.
  }
\item \texttt{-q,--log-queries}: \colz{Log the results of DNS queries handled by
    dnsmasq.}
\item \texttt{-8,--log-facility=<facility>}: \colz{
    If \texttt{<facility>} contains \texttt{/}, then it's considered a filename,
    and \dmq{} won't log to syslog, but will log to the specified file.
  }
\item \texttt{-p,--port=<port>}: \colz{Listen on \texttt{<port>} instead of the
    standard DNS port (53). Setting this to zero completely disables DNS,
    leaving only DHCP and/or TFTP.
  }
\end{itemize}

\subsubsection{How it does DNS}

\dmq{} is a  query forwarder: it is not capable of recursively answering
arbitrary queries starting from the root servers but forwards such queries to a
\cola{fully recursive DNS updtream server}
\colz{ which is typically provided by an \colZ{ISP}. By default, \dmq{} reads
  \colZ{\texttt{/etc/resolve.conf}} (or equivalent if \colZt{--resolv-file} is
  used) and re-reads it if it changes.

  This allows the DNS servers to be \colZ{set dynamically by PPP or DHCP}
}

\dSay{Oh, so it will watch a \texttt{resolve.conf} file.}

\cSay{Yeah..
  \texttt{-r,--resolv-filr = <file>}\newline
  Read the IP addresses of the \cola{upstream nameservers} from \texttt{<file>}.
  \colz{default to \texttt{/etc/resolv.conf}.}
}

\dSay{
  Is it possible to use commandline to specify it ?
}

\cSay{Yes. Use:

  \begin{enumerate}
  \item \texttt{-R,--no-resolv} Don't read \texttt{/etc/resolv.conf}. \colz{ Get
      upstream servers only from the command line or the \dmq{} configuration
      file.}
  \item \texttt{--rev-server=...} This specify the \cola{upstream server(s)} directly.
    \colz{For example, --rev-server=1.2.3.0/24,192.168.0.1}
  \end{enumerate}


}


\dSay{Wait, what's PPP?}

\cSay{ Point-to-Point protocol. \colz{ I think it's a
    protocol that also touch the ip address, but to the best of our knowledge,
    it has nothing to do with DHCP. } }

\colz{Absense of \texttt{/etc/resolve.conf} is not an error since it may not
  have been created before a PPP connection exists. }

\cSay{Oh, so PPP creates the \texttt{resolve.conf} ?}

\dSay{Seem so.}

\colz{
  \dmq{} simply keep checking in case this file is created at any time. \dmq{}
  can be told to parse more than one \texttt{resolve.conf} file. This is useful
  on a laptop, where both PPP and DHCP may be used.


  \colZ{Upstream servers may also be specified.} These server specifications
  optionally take a domain name which tells \dmq{} to use that server only to
  find names in that particular domain.

  In order to configure \dmq{} to act as chache on the host on which it is
  running, put \colZt{"name server 127.0.0.1"} in \colZt{/etc/resolve.conf} to
  force local processes to send queries to \dmq{}.

  Then either

  \begin{enumerate}
  \item specifying the upstream servers directly to \dmq{} using
    \colZt{--server} options, or
  \item putting their addresses real in another file, say
    \texttt{/home/me/my-resolve.conf} and use \texttt{--resolv-file
      /home/me/my-resolve.conf}.
  \end{enumerate}

  The second method allows for dynamic update of server addresses by PPP or DHCP.

  \colZ{Addresses in \texttt{/etc/hosts} will ``shadow'' different addresses for
    the same names in the upstream DNS, so ``\texttt{aaa.com 1.2.3.4}'' in
    \texttt{/etc/hosts} will ensure that queries for ``\texttt{aaa.com}'' always
    return \texttt{1.2.3.4} even if queries in the upstream DNS would say
    something else.}There's one exception to this: it's something to do with an
  upstream \texttt{CNAME} entry, but let's stop here.
}

\subsubsection{How it does DHCP}

\colz{
  The tag system works as follows:

  \begin{enumerate}
  \item For each DHCP request, \dmq{} collects a set of valid tags from active
    configuration lines which include \texttt{set:<tag>}, including
    \begin{enumerate}
    \item one from the \colZ{\texttt{--dhcp-range}} used to allocate the
      address,
    \item one from any matching \colZt{--dhcp-host}
    \item and ``known'' or ``known-ethernet'' if a \texttt{--dhcp-host} matches.
    \end{enumerate}
  \item The tag ``bootp'' is set for BOOTP requests, 
  \item a tag whose name is the name of the interface on which the request
    arrived is also set.
  \end{enumerate}

  \cSay{
    Here are some details, first,

    \texttt{-F,--dhcp-range=
        \cola{tag1:<tag1>,tag2:<tag2>,...}\colb{, set:<tag>}
        \colc{<start-addr>,[<end-addr>],}
        [<many options>]
      } Enable the DHCP server.

      Addresses will be given out from the range \colZt{<start-addr>} to
      \colZt{<end-addr>} and from statically defined addresses given in
      \texttt{--dhcp-host} options.

      \colz{

        A lease time can be given. By default, it's one hour for IPv4 and
        one day for IPv6.

        For IPv6, there's an prefix length which must be equal to or larger then
        the prefix length on the local interface. This defaults to 64. Unlike
        the IPv4 case, the prefix length is not automatically derived from the
        interface. confiugration. (minimum prefix length = 64.)

      }
  }

  \cSay{
    \colz{
      IPv6 supports another type of range, In this, the start address and
      optional end address \colZ{contain only the network part (ie
        \texttt{::1})} and they are followed by
      \texttt{constructor:<interface>}, for example,
      \colZt{--dhcp-range=::1,::400,constructor:eth0}, will look for addresses
      on \texttt{eth0} , and then create a range from \texttt{<network>::1} to
      \texttt{<network>::400}.  Note that not just any address on
      \texttt{eth0} will not do: it must not be an autoconfigured or privacy
      address, or be deprecated.

      If a \texttt{--dhcp-range} is only being used for stateless DHCP, then
      it can simply be:
      \colZt{--dhcp-range=::,constructor:eth0}
    }
  }

  \dSay{\colz{What about \texttt{--dhcp-host} ?}}

  \cSay{
{

  \ttfamily
  -G,--dhcp-host= \cola{[<hwaddr>]}
  \colz{
    [,id:<client\_id>|*]
    [,set:<tag>] [,tag:<tag>] \colb{[,<ipaddr>]} \cola{[,<hostname>]} \colZ{[,<lease\_time>]} [,ignore]
  }  
}
    

    Specify per host parameters for the DHCP server. This allows a machine with
    a particular hardware address to be always allocated the same
    \begin{enumerate}
    \item hostname,
    \item IP address and
    \item lease time.
    \end{enumerate}
  }

  \dSay{
    Oh, this one is pretty important for those servers that need a static IP.
  }

  \cSay{
    \colz{
      A hostname specified like this overrides any supplied by the \colZ{DHCP
        client on the host}.

      It is also allowable to omit the hardware address and include the
      hostname. In this case, the IP address and lease times will \colZ{applied
        to any machine claiming that name}. For example
      \colZt{--dhcp-host=00:11:22:33:44:55,myhost,infinite} tells \dmq{} to give
      the machine with hardware address \texttt{00:11:22:33:44:55} the name
      ``\texttt{myhost}'' and an infinite lease. For another example,
      \colZt{--dhcp-host=myhost2,10.0.0.2} tells \dmq{} to give the machine
      ``\texttt{myhost2}'' the IP address \texttt{10.0.0.2}
    }
  }

  \dSay{
    What ? Hosts can have names before they are assigned an IP address ?
  }

  \cSay{It just said so in the manual. Let's just believe it.}

  \dSay{
    Okay...
  }

  \cSay{
    \colz{
      Addresses allocated like this are not constrained to be in the range
      given in \texttt{--dhcp-range} options. For subnets which don't need a pool
      of dynamically allocated addresses, you can use a ``\texttt{static}''
      keyword in the \texttt{--dhcp-range} declaration.
    }
  }

  \dSay{ Oh, so DHCP can also be used to assign static IP addresses?}

  \cSay{Seem so.}

  \dSay{What is \texttt{client\_id} ?}

  \cSay{It's a unique identifier for the client. It's called \texttt{DUID} in
    IPv6. It's an alternative to the hardware address.

    \colz{
      Thus:

      \colZt{--dhcp-host=id:01:02:03:04,...} refers to the host with client
      identifier \texttt{01:02:03:04}. It is also possible to use text as in
      \colZt{--dhcp-host=id:myhost,...}.
    }
  }

  \cSay{
    \colz{
      A single \texttt{--dhcp-host} option may contain
      \begin{enumerate}
      \item an IPv4 address, or 
      \item one or more IPv6 addresses, or
      \item both
      \end{enumerate}.

      IPv6 addresses must be bracketed by square brackets thus:

      \colZt{--dhcp-host=laptop,[2001:db8::1],[2001:db8::2]}
    }
  }

  

  Any configuration lines which include one or more \texttt{tag:<tag>}
  constructs will only be valid if all that tags are matched in the set derived
  above. Typically this is \colZt{--dhcp-option}, tagged version of
  \texttt{--dhcp-options} is prefered, provided that all the tags match
  somewhere in the set collected as described above. The prefix \texttt{!} on a
  tag means ``not'' so \colZt{--dhcp-option=tag:!purple,3,1.2.2.3.4} sends the
  option when the taf \texttt{purple} is not in the set of valid tags.
  (\emoji{turtle} : If using
  this in a command line rather than a config file, be sure to escape
  \texttt{!}, which is a shell metachar...)

  When selecting \colZt{--dhcp-options}, a tag from \colZt{--dhcp-range} is
  second class relative to other tags, to make it easy to override options for
  individual hosts, so
  \colZt{
    \newline
    --dhcp-range=set:interface1,....\newline
    --dhcp-option=tag:interface1,\cola{option:nis-domain,domain1} \newline
    --dhcp-option=tag:myhost,\cola{option:nis-domain,domain2 }
  }

  will set the NIS-domain to \colZt{domain1} for hosts in range, but override
  that to \colZt{domain2} for \colZt{myhost}.

  \colZ{
    Note that for \texttt{--dhcp-range} both \texttt{tag:<tag>} and
    \texttt{set:<tag>} are allowed, to both select the range in use based on
    (eg) \texttt{--dhcp-host},and to affect the options sent, based on the range
    selected.
  }
}


\section{DHCP}
\label{sec:dhcp}

We will want to consider what we need for special purpose devices such as
iPhones, \Cola{Wireless Access Points (WAP)}, or even \Cola{Pre Execution
  Environment (PXE)} devices that can load their entire OS from DHCP
information.


\end{document}

% Local Variables:
% TeX-engine: luatex
% TeX-command-extra-options: "-shell-escape"
% TeX-master: "m.tex"
% TeX-parse-self: t
% TeX-auto-save: t
% End:
\documentclass[dvipsnames]{article}
% \documentclass[dvipsnames]{ctexart}

\title{number theory}
\usepackage{geometry}\geometry{
  a4paper,
  total={170mm,257mm},
  left=20mm,
  top=20mm,
}


\usepackage{svg}

\usepackage[skip=5pt plus1pt, indent=0pt]{parskip}
% Color
\newcommand{\mycola}{MidnightBlue}
\newcommand{\mycolb}{Mahogany}
\newcommand{\mycolc}{OliveGreen}

\newcommand{\cola}[1]{\textcolor{\mycola}{#1}}
\newcommand{\colb}[1]{\textcolor{\mycolb}{#1}}
\newcommand{\colc}[1]{\textcolor{\mycolc}{#1}}
\newcommand{\Cola}[1]{\textcolor{\mycola}{\textbf{#1}}}

% \let\emph\relax % there's no \RedeclareTextFontCommand
% \DeclareTextFontCommand{\emph}{\bfseries}
\renewcommand{\emph}[1]{\texbf{#1}}
\usepackage{amssymb}            %\mathbb

\usepackage{fontspec}

\setmonofont{Cascadia}[
Path=/usr/share/fonts/truetype/Cascadia_Code/,
Scale=0.85,
Extension = .ttf,
UprightFont=*Code,              %find CascadiaCode.ttf
BoldFont=*CodePL,               %find CascadiaCodePL.ttf ...
ItalicFont=*CodeItalic,
BoldItalicFont=*CodePLItalic
]
% --------------------------------------------------
% Windows
% \setmonofont{Cascadia}[
% Path=C:/Windows/Fonts/,
% Extension = .ttf,
% UprightFont=*Mono,              %find CascadiaMono.ttf
% BoldFont=*Code,               %find CascadiaCodePL.ttf ...
% ItalicFont=*Code,
% BoldItalicFont=*Code
% ]


\usepackage{minted}
\usepackage{tcolorbox}
\tcbuselibrary{skins}
\tcbuselibrary{minted}
\usepackage{tikz}
\usetikzlibrary{shapes} % ellipse node shape
\usetikzlibrary{shapes.multipart} % for line breaks in node text
\usetikzlibrary{arrows.meta}    %-o arrow head
\usetikzlibrary{arrows}
\usetikzlibrary{matrix}
\usetikzlibrary{snakes}

\usepackage{amsmath}
% ??? still xelatex?
% \usepackage{xeCJK}
\usepackage{emoji}
% \setemojifont{NotoColorEmoji.ttf}[Path=C:/Users/congj/repo/myFonts/]
% \setemojifont{TwitterColorEmoji-SVGinOT.ttf}[Path=C:/Users/congj/repo/myFonts/]


\newtcolorbox[auto counter]{myBox}[2][]{
  fonttitle=\bfseries,title={共识~\thetcbcounter: #2},#1
}
\newtcolorbox[]{noteBox}[1][]{
  tile,left=1mm,nobeforeafter,fontupper=\small,#1
}

\tikzstyle{myNode}=[inner sep=2pt,circle,text=white]
\date{\today}
\author{作者}

\newtcblisting{simplec}{
  listing engine=minted,
  minted language=c++,
  minted style=vs,
  minted options={fontsize=\small,autogobble,
    % framesep=1cm
  },
  tile,
  listing only,
  % bottom=0cm,
  % nobeforeafter, 
  boxsep=0mm,
  left=1mm,
  opacityback=0.5,
  colback=gray!20
}
\tcbuselibrary{breakable}
\newtcblisting{simplepy}{
  listing engine=minted,
  minted language=python,
  minted style=vs,
  minted options={fontsize=\small,autogobble,
    % framesep=1cm
  },
  tile,
  listing only,
  % bottom=0cm,
  % nobeforeafter,
  boxsep=0mm,
  left=1mm,
  opacityback=0.5,
  colback=gray!20,
  breakable
}
\newtcolorbox{blackbox}{tile,colback=black,colupper=white,nobeforeafter,halign=flush center}

\tikzstyle{myMatrix}=[matrix of nodes,below right,
nodes={above,text centered},                  %apply to all nodes
row sep=1cm,column sep=2cm]
\tikzstyle{every node}=[inner sep=0pt]

\newcommand\uptoleft[3][-o]{\draw[very thick,#1](#2.south) |- (#3.west);}
\newcommand\uptodown[3][-o]{\draw[very thick,#1](#2.south) to [out=270,in=90] (#3.north);}
\newcommand\downtoup[3][-latex]{\draw[very thick,#1](#2.north) to [out=90,in=270] (#3.south);}

\newcommand\lefttoright[3][-latex]{\draw[very thick,#1](#2.east) to[out=0,in=180] (#3.west);}
\newcommand\lefttodown[3][-latex]{\draw[very thick,#1](#2.east) to[out=0,in=90] (#3.north);}


\newtcolorbox{leftDialogBox}{
  tile, nobeforeafter, boxsep=0pt,
  % show bounding box,
  colback=\mycola!10,
  overlay={
    \begin{scope}
      % \fill[gray!10] (frame.east) circle (2pt);
      \fill[\mycola!10] (frame.east) --
      +(0,2mm) --
      +(3mm,0) --
      +(0,-2mm)
      ;
    \end{scope}
  }}


\newtcolorbox{rightDialogBox}{
  tile, nobeforeafter, boxsep=0pt,
  % show bounding box,
  colback=\mycola!10,
  overlay={
    \begin{scope}
      % \fill[gray!10] (frame.east) circle (2pt);
      \fill[\mycola!10] (frame.west) --
      +(0,2mm) --
      +(-3mm,0) --
      +(0,-2mm);
    \end{scope}
  }}

\newcommand{\mycolaa}{\mycola!20}


\usepackage{changepage}   % for the adjustwidth environment
\newenvironment{myIndent}[1][7mm]{\begin{adjustwidth}{#1}{}}{\end{adjustwidth}}

\newcounter{myDefCounter}
\newcounter{myTheoCounter}

\tcbuselibrary{theorems}
\newtcbtheorem[use counter=myDefCounter,number within=section]{myDef}{Definition}%
{
  % colback=green!5,colframe=green!35!black,
  fonttitle=\bfseries}{def}

\newcommand{\myThmBeforeProof}{
  \texttt{Proof: }\par
  \begin{myIndent}
}

\newcommand{\myThmAfterProof}{
\end{myIndent}
\qed
}

\tcbuselibrary{skins}
\tcbsubskin{myThmSkin}{enhanced}{
  parbox=false,
  % colback=green!5,
  colframe=\mycola,
  fonttitle=\bfseries,
  breakable,
  before lower={\myThmBeforeProof{}},
  after lower={\myThmAfterProof{}}
}

\newtcbtheorem[
use counter=myTheoCounter,number within=section
]{myTheo}{Theorem}%
{
  skin=myThmSkin
}{thm}


\newtcbtheorem[use counter=myTheoCounter,number within=section]{myCor}{Corollary}%
{
  % colback=green!5,
  skin=myThmSkin
}{cor}

\newtcbtheorem[use counter=myTheoCounter,number within=section]{myLem}{Lemma}%
{% colback=green!5,
  skin=myThmSkin
}{lem}

\newcommand{\refCorollary}[1]{Corollary~\ref{cor:#1}}
\newcommand{\refLemma}[1]{Lemma~\ref{lem:#1}}

\usepackage{cleveref}
\crefname{myDefCounter}{definition}{definitions}
                                   % ^^^ plural
\Crefname{myDefCounter}{Definition}{Definitions}
         % ^^^^^^ type = counter name

\crefname{myTheoCounter}{theorem}{theorems}
\Crefname{myTheoCounter}{Theorem}{Theorems}

\usepackage{amsthm}             %for {proof}


% --------------------------------------------------
\begin{document}
\maketitle

\section{The series of primes}

\subsection{Divisibility of intergers}

The numbers
% dots for commas
\[ \dotsc,-3,-2,-1,0,1,2, \dotsc\]
are the \Cola{intergers}, denoted by $\mathbb{Z}$. The number
\[ 0,1,2,3,\dotsc\] are the \Cola{non-negative integers} or \Cola{natural
  numbers}, denoted by $\mathbb{N}$, and the numbers \[1,2,3,\dotsc\] are the
\Cola{positive integers}, denoted by $\mathbb{N^{\times}}$. We also denote
\Cola{non-zero intergers} by $\mathbb{Z^{\times}}$

Let $a,\in \mathbb{Z}, b \in \mathbb{Z^{\times}}$, $a$ is saied to be
\Cola{devisible} by $b$ if there exists $c \in \mathbb{Z}$ such that
\[ a = bc.\] If $a,c, > 0$, so does $c$. If $a$ is divisable by $b$, we say $b$
is a \Cola{divisor} of $a$ and write \[ b \mid a.\] Thus
\[ 1 \mid n, \quad n \mid n, \quad \forall n \in \mathbb{Z}.\] If $b$ doesn't
divide $a$, we write $b \nmid a$.

It's plain that
\begin{align*}
  b \mid a \;,\; c \mid b &\Rightarrow   c \mid a\\
  b \mid a  &\Rightarrow   bc \mid ac\\
  c \mid a \;,\; c \mid b &\Rightarrow   c \mid ma + nb \quad \forall a,b,m,n
                            \in \mathbb{Z} \quad c \in \mathbb{Z^{\times}}
\end{align*}

\subsection{Prime numbers}

\begin{myDef}{Prime number}{prime-number}
  A number $p \in \mathbb{N}$ is said to be \Cola{prime}, if
  \begin{itemize}
  \item $p > 1$
  \item $p$ has no positive divisors except 1 and $p$
  \end{itemize}
\end{myDef}

% The \cref{def:prime-number}.
A number greater than 1 and not prime is called \Cola{composite}.
Our first theorem is
\begin{myTheo}{Product of primes}{product-of-primes}
  Every positive integer $n$ > 1, is a product of primes.
  \tcblower
    \cola{If} $n$ \cola{is prime}:
    \begin{myIndent}
      There's nothing to prove.
    \end{myIndent}
    \cola{Else:}
    \begin{myIndent}
      $n$ has \colc{divisors} between 1 and $n$.\\
      \emoji{pinching-hand} $m := \cola{\min{}} \{\colc{\text{these divisor of }n}\}$
      \begin{myIndent}
        \cola{If} $m$ \cola{is not prime}:
            \begin{align*}
              \exists l \in (1,m) \subset \mathbb{N}, l \mid m &\Rightarrow l \mid n \\
              & \Rightarrow m \text{ is not the smallest as defined \emoji{cross-mark}}
            \end{align*}
        So, $m$ \cola{is prime}, say $m = p_1$, then we can draw a prime out of n:
        \[
          n = p_1n_1, \quad n_1 \in [p_1,n)
        \]
        \cola{If} $n_1$ is a prime:
        \begin{myIndent}
          Done.
        \end{myIndent}
        \cola{Else}:
        \begin{myIndent}
          We use the same process again to draw out another prime divisor of $n_1$
          \begin{align*}
            \text{\emoji{pinching-hand} } p_2 \in (1,n_1), p_2 \mid n_1 & \Rightarrow n_1 = p_2n_2 \\
            & \Rightarrow n = p_1p_2n_2
          \end{align*}
          Sooner, or later we must meet a prime $n_i$, because \cola{we can't
            keep factoring out numbers greater than one from a finite number.}
        \end{myIndent}
      \end{myIndent}
    \end{myIndent}
  \end{myTheo}

  These prime factors are not necessarily distinct.

  If we arrange them in increasing order, we obtain
  \[
    n = p_1^{a_1}p_2^{a_2}\dotsc{}p_n^{a_n}, \quad a_i \in \mathbb{N^{\times}},
    p_1 < p_2 < \dotsm{} < p_n
  \]
  In this case, we say that $n$ is in \Cola{standard form}.
  
\subsection{The fundamental theorem of arithmetic}
\label{sec:fund-theo-arth}

We defer the proof for now but present a lemma:
\begin{myLem}{Euclid's first theorem}{p-cut-ab} 
  Let $p$ be prime, $a,b \in \mathbb{N^{\times}}$, then $
  p \mid ab \Rightarrow p \mid a \text{ or } p \mid b$
\end{myLem}

We take \refLemma{p-cut-ab} for granted for the moment and use that to prove
\cref{thm:fund-theo-arth}

\begin{myTheo}{Fundamental theorem of arithmetic}{fund-theo-arth}
  The standard form of $n$ is unique, $\forall n \in \mathbb{N}^{\times}$.
  \tcblower{}
  From \refLemma{p-cut-ab}, we have:
  \[ p\mid n_1n_2\dotsm{}n_k \Rightarrow p \mid n_i \text{ for some }i\] In
  particular, if $n_i$ are primes, then $p$ is one of $n_i$. In other words,
  \cola{if a prime divides a product of primes, it must be one of them}.

  Now suppose \cola{if} $n$ has two standard forms:
  \begin{myIndent}
    \newcommand{\myP}[3]{\ensuremath{{#1}_{#3}^{{#2}_{#3}}}}
    We have
    \[
      n =
      \colc{
        \myP{p}{a}{1}
        \myP{p}{a}{2}
        \dotsm
        \myP{p}{a}{k}
      }
      =
      \colb{
        \myP{q}{b}{1}
        \myP{q}{b}{2}
        \dotsm
        \myP{q}{b}{j}
      }
    \]
    So
    $\forall i, \colc{p_i} \mid \colb{ \myP{q}{b}{1} \myP{q}{b}{2} \dotsm
      \myP{q}{b}{j} } $. This means \cola{every \colc{$p$} is a \colb{$q$}}.
    Symmetrically \cola{every \colb{$q$} is a \colc{$p$}}. Hence $k=j$ and,
    since both sides are arranged in increasing order, $p_i = q_i$ for all $i$.

    So now only the exponents may be different.

    \cola{If} $a_i > b_i$ for some $i$:
    \begin{myIndent}
      Dividing both side by $\myP{p}{b}{i}$ gives:
      \[
        \colc{
          \myP{p}{a}{1}
          \dotsm{}
          p_i^{a_i - b_i}
          \dotsm{}
          \myP{p}{a}{k}
        }
        = 
        \colb{
          \myP{p}{b}{1}
          \dotsm{}
          \myP{p}{b}{i-1}
          \myP{p}{b}{i+1}
          \dotsm{}
          \myP{p}{a}{k}
        }
      \]
      The left hand side is divisable by $p_i$ while the right hand side is not.
      \emoji{cross-mark} a contradiction.
    \end{myIndent}
    Symmetrically $a_i$ can't be less than $b_i$.
  \end{myIndent}
  As as result, $n$ can't have two different normal forms.
\end{myTheo}

\emoji{parrot} : Oh, that's why 1 is not a prime. If it is, then
\cref{thm:fund-theo-arth} would no longer hold.

\subsection{The sequence of primes}

The first primes are:

\[2, 3, 5, 7, 11, 13, 17, 19, 23, \dotsc{}\]

It is easy to construct a table of primes, up to a moderate limit $N$ by a
procedure known as the \Cola{sieve of Eratosthenes}.


\end{document}


% Local Variables:
% TeX-engine: luatex
% TeX-command-extra-options: "-shell-escape"
% TeX-master: "m.tex"
% End:
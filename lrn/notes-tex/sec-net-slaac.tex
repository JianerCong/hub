
\section{SLAAC}
\label{sec:slaac}

\Cola{Stateless Address Autoconfiguration (SLAAC)} is designed with the
following goals:

\colz{
  \begin{enumerate}
  \item \colZ{Manual configuration} of individual machines before connecting them to the
    network should \colZ{not be required}.
  \item Small sites consisting of a set of machines attached to a single link
    should \colZ{not require }the presence of a \colZ{DHCPv6 server or router}.
    \colZ{Plug-and-play} should is achieved through the use of link-local
    addresses.
  \item global address -  an address with unlimited scope.


    % Link-local addresses have a \colZ{``well-known'' prefixes} that identifies
    % the (single) shared link to which a set of nodes attach.
  \item A large site with multiple networks and routers should not require the
    presence of a \colZ{DHCPv6 server} for address configuration. In order to
    generate \colZ{global addresses}, hosts must determine the prefixes that
    identify the the subnets to which they attach. \colZ{Routers generate
      periodic Router Advertisements
    } that include options listing the set of active prefixes on a link.
  \item Address configuration should facilitate the graceful renumbering of a
    site's machines.
  \end{enumerate}
}


\subsection{jargons}

\begin{enumerate}
\item link layer address \colz{

    a link-layer identifier for an interface. For example, IEEE 802 addresses
    for Ethernet links.
  }
\item tentative address \colz{

    an address whose uniqueness on a link is being verified. It's not
    \colZ{assigned to an interface yet}.

    An interface discards packets sent to a tentative address, but accepts
    Neighbor Discovery packets related to Duplicate Address Detection (DAD) for
    the tentative address.
  }
\item preferred address \colz{
    a "normal" address assigned to an interface.
  }
\item deprecated address \colz{
    an address that is assigned to an interface whose use is discouraged. It can
    be \colZ{source}, but \colZ{not destination}. However, sometimes, there are
    exceptions. For example, when there's an active TCP connection.
  }
\item valid address \colz{
    preferred or deprecated address.
  }
\item invalid address \colz{
    an address that is not assigned to any interface.
  }
\item interface identifier
  a link-dependent identifier for an interface that is unique per link.
  \cola{SLACC combines an interface identifier with a prefix to form an address.}
  \colz{
    From address autoconfiguration's perspective, an interface identifier is a
    bit string of known length. 
  }
\end{enumerate}

\subsection{Quiz}

\newcommand{\ans}{
  % \newline{}
  \par
  \hspace{0.1\linewidth}
  % \phantom{Ans: }
  % \colz{Ans: }
  \tikz[baseline=.5em]{\draw[thick,\mycola] (0,0) -- (0.8\linewidth,0pt);}
  % \rule[-1em]{0.8\linewidth}{1pt}
}
\newcommand{\tf}{\Cola{\hfill[T/F]}}

\begin{enumerate}
\item When to use SLAAC and when to use DHCPv6 ? Are they mutually exclusive? \ans{}
\item What are the two states associated to an address that enable graceful
  expiration? \ans{}
\item ``Deprecated''  addresses are not routable. \tf{}
\item The node check for duplication before it assigns an address to its
  interface. \tf{}
\item SLAAC duplicate address detection(DAD) also works for DHCPv6-assigned
  addresses. \tf{}
\item A router is not needed in order to use DHCPv6. \tf{}
\item What type of ICMP message is used when checking the uniqueness of a tentative
  address? \ans{}
\item What happens when a tentative address is found to be duplicated? \ans{}
\item SLAAC only works on multicast-enabled links. \tf{}
\item The tentative address randomly guessed. \tf{}
\item In what ICMP message is a host be able to obtain information necessary to generate
  its global addresses? \ans{}
\item DAD can be disabled. \tf{}
\item Upper layer protocols like TCP and UDP break when the lower layer address
  becomes expired. To mitigate this, higher layer should use a ``preferred''
  address \tf{}
\item Describe the autoconfiguration-related variable
  \texttt{DupAddrDetectTransmits} and \texttt{ReransTimer}. What are their
  units? What type of ICMP message do they control? \ans{}
\item You can force an interface to re-form its link-local address by disabling
  and re-enabling the interface. \tf{}
\item A link-local address is always preferred and it's never timed out. \tf{}
\item Current implementations checks the duplication of link-local addresses and
  use that result to possibly skip the check for global addresses. This is not
  recommended. \tf{}
\item \texttt{Neighbor Solicitations} and \texttt{Advertisement} can be invalid. \tf{}
\item Which two of the multicast address the interface must join before sending
  a \texttt{Neighbor Solicitation}? \ans{}
\item \texttt{Router Advertisement (RA)} allows auto-configuration of the router
  of a host, but it can also be set manually. \tf{}
\item What's the length (in bits) of the advertised prefix in \texttt{RA}? \ans{} 
\end{enumerate}
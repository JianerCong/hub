
\section{systemd}
\label{sec:sysd-network}

\subsection{Description}

The main network must have the extension \texttt{.network}.

The \texttt{.network} files are read from

\begin{enumerate}
\item \cola{\texttt{/lib/systemd/network/}} : \colz{persistent}
\item \cola{\texttt{/usr/local/lib/systemd/network/}} : \colz{persistent}
\item \colb{\texttt{/run/systemd/network/}} : \colz{transient}
\item \colc{\texttt{/etc/systemd/network/}} : \colz{local admin}
\end{enumerate}

All config files are sorted (folders flattened) and read.
\colz{
  However, files with same names replace each other. The order is:
}

\begin{center}
  \ttfamily
  \colc{/etc/} > \colb{/run/} > \cola{/usr/} > \cola{/lib/}
  \rmfamily
  % \hfill
  \hspace{0.05\linewidth}
  (i.e. user's override system's)
\end{center}

\colz{An exception is that, if a file is empty or a symlink pointing to
  \texttt{/dev/null}, it will be disabled (masked).
  
  Along with the network file \texttt{aaa.network}, a ``drop-in'' directory
  \texttt{aaa.network.d/} may exist. All \texttt{.conf} files in it will be
  merged into \texttt{aaa.network}. Each of them must have appropriate section
  headers.
}

\subsection{Sections}

\newcommand{\secM}{\cola{\texttt{[Match]}}}
\newcommand{\secN}{\colb{\texttt{[Network]}}}

Each network file has two sections: \secM{} and \secN{}.

\secM{} tells to whom the network config applies.
\secN{} tells what to apply.

\colb{The first matching network file is applied, and the rest are ignored.} If
you just want to apply to all interface, do :
\begin{simplecf}
  [Match]
  Name=*
\end{simplecf}

Some basic settings are shown below:
\begin{simplecf}
  [MATCH]

# space-separated list of MAC address (Yes, forging the MAC is as simple as
# this)
MACAddress=01:23:45:67:89:ab 00-11-22-33-44-55 AABB.CCDD.EEFF

# space-separated list of device type, try: networkctl list
Type=ether wlan

[NETWORK]

Description=My wired network settings

DHCP=yes                        # or ipv4, ipv6, no

# For v6, usually, an RA will trigger the DHCPv6 client. But If you turn on
# DHCPv6 here explicitly, the client will be started even if there is no RA.

# systemd has built-in DHCPv4 server, but it is not enabled by default.
# DHCPServer=no 

# Enable link-local address autoconfiguration. Accepts yes, no, ipv4, and ipv6.
# Defaults to "no" when Bridge=yes, else "ipv6"
LinkLocalAddressing=ipv6
# LinkLocalAddressing=no

# Defaults to "stable-private" if IPv6StableSecret= is set, otherwise "eui64".
IPv6LinkLocalAddressGenerationMode=stable-privacy
# IPv6LinkLocalAddressGenerationMode=eui64
# IPv6LinkLocalAddressGenerationMode=none
# IPv6LinkLocalAddressGenerationMode=random

# Takes an IPv6 address. If set, the address is used as the stable secret for
# generating IPv6 link-local address. If
# IPv6LinkLocalAddressGenerationMode=stable while this is not set, then an
# address will be generated from the local machine ID and interface name.
IPv6StableSecretAddress=fe80::1

\end{simplecf}

\subsubsection{\texttt{IPv6Token}}

The \colb{\texttt{IPv6Token}} option specifies an optional address generation
mode for SLAAC. Supported modes are \texttt{prefixstable} and \texttt{static}.

\colz{

  When set to \cola{\texttt{static}}, an IPv6 address must be specified after a colon
  (\texttt{:}), and the \colZ{lower bits} of the supplied address are combined
  with the upper bits of a prefix received in a \cola{Router Advertisement}
  message to form a complete address.

  If no mode is specified, \texttt{static} is assumed. For example:
}
\begin{simplecf}
  IPv6Token=static:::1a:2b:3c:4d
  #IPv6Token=::1a:2b:3c:4d
\end{simplecf}
\colz{
  Note that if multiple prefixes are received in RA. Then addresses will be
  formed from each of them using the supplied address.

  This mode \colZ{implement SLAAC} but uses a static \cola{interface id} instead of one
  from EUI-64. But because this is static, if \cola{Duplicate Address Detection}
  detects that the address is already in use,this mode will fail.


  When the mode is set to \cola{\texttt{prefixstable}}, the RFC 7217[1] algorithm is
  used. This mode optionally takes an IPv6 address. In that case, only RA
  prefixes that matches the supplied address can be used to generate \cola{interface id}.
}

\begin{simplecf}
  IPv6Token=prefixstable
  # IPv6Token=prefixstable:2002:da8:1::
\end{simplecf}

\subsection{\verb|Address|}

A static IPv4 or IPv6 address and its prefix length, separated by a slash
(\texttt{/}). This is a short-hand for an \texttt{[Address]} section only
containing an Address key.

\section{systemd}
\label{sec:sysd-network}

\subsection{Description}

The main network must have the extension \texttt{.network}.

The \texttt{.network} files are read from

\begin{enumerate}
\item \cola{\texttt{/lib/systemd/network/}} : \colz{persistent}
\item \cola{\texttt{/usr/local/lib/systemd/network/}} : \colz{persistent}
\item \colb{\texttt{/run/systemd/network/}} : \colz{transient}
\item \colc{\texttt{/etc/systemd/network/}} : \colz{local admin}
\end{enumerate}

All config files are sorted (folders flattened) and read.
\colz{
  However, files with same names replace each other. The order is:
}

\begin{center}
  \ttfamily
  \colc{/etc/} > \colb{/run/} > \cola{/usr/} > \cola{/lib/}
  \rmfamily
  % \hfill
  \hspace{0.05\linewidth}
  (i.e. user's override system's)
\end{center}

\colz{An exception is that, if a file is empty or a symlink pointing to
  \texttt{/dev/null}, it will be disabled (masked).
  
  Along with the network file \texttt{aaa.network}, a ``drop-in'' directory
  \texttt{aaa.network.d/} may exist. All \texttt{.conf} files in it will be
  merged into \texttt{aaa.network}. Each of them must have appropriate section
  headers.
}

\subsection{Sections}

\newcommand{\secM}{\cola{\texttt{[Match]}}}
\newcommand{\secN}{\colb{\texttt{[Network]}}}

Each network file has two sections: \secM{} and \secN{}.

\secM{} tells to whom the network config applies.
\secN{} tells what to apply.

\colb{The first matching network file is applied, and the rest are ignored.} If
you just want to apply to all interface, do :
\begin{simplecf}
  [Match]
  Name=*
\end{simplecf}

\subsubsection{\secM{}}


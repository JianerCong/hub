
\newcommand{\dmq}{\texttt{dnsmasq}}
\section{\dmq{}}

\subsection{Intro}


\Cola{\dmq{}} is a lightweight DNS, TFTP, PXE, router advertisement and
DHCP server. It is intended to provide coupled DNS and DHCP service to a LAN.

\colz{
  \dmq{} \colZ{accepts DNS queries} and either answers them from a small, local, cache
  or forwards them to a real, recursive, DNS server. It loads the contents of
  \colZ{\texttt{/etc/hosts}} so that \colZ{local hostnames which do not appear in the global DNS can be resolved and also
    answers DNS queries for DHCP  configured hosts.}

  It can also act as the authoritative DNS server for one or more domains,
  allowing local names to appear in the global DNS. 

  \colZ{The \dmq{} DHCP server} supports static address assignments and multiple
  networks. It automatically sends a sensible default set of DHCP options
  (configurable).
}

\dSay{
  Oh.. I just realize that, for a local DNS name resolve, you just need to edit
  the \texttt{/etc/hosts}...
}

\subsection{Configure \dmq{}}

\colz{
  Configuring \dmq{} to act as an authoritative DNS server is complicated by the
  fact that it involves configuration of \colZ{external DNS servers} to provide
  delegation. 

  At startup, \dmq{} reads \colZ{\texttt{/etc/dnsmasq.conf}} (configurable
  through \texttt{--conf-file, --conf-dir}) The format of this file consists of
  one option per line, exactly as the long options detailed in the OPTIONS
  section but without the leading \texttt{--}. Line starting with \texttt{\#}
  are comments and ignored. (\emoji{parrot}: latest style)  For options which may only be
  specified once, \colZ{the configuration file overrides the commandline.}

  \colZ{Quoting} is allowed in a config file: between \texttt{"} quotes the
  special meanings of ``\colZ{\texttt{,:;\#}}''. The following escapes are
  allowed: ``\colZ{
    \texttt{
      \textbackslash{}\textbackslash{}
      \textbackslash{}"
      \textbackslash{}t
      \textbackslash{}e
      \textbackslash{}b
      \textbackslash{}r
      \textbackslash{}n
    }
  }'' }

\subsection{Notes for \texttt{SIGHUP}}
\colz{
  When it recieves a \colZ{\texttt{SIGHUP}}, \dmq{} \colZ{clears its cache and
    then re-loads
    \texttt{/etc/hosts, /etc/ethers}
    and any file given by \texttt{--dhcp-hostsfile, --dhcp--hostsdir,
      --dhcp-optsfile, --dhcp-optsdir, --addn-hosts, --hostsdir}.
  }  

  The DHCP lease change script is called for all existing DHCP leases. if
  \colZ{\texttt{--no-poll}} is set, SIGHUP also re-reads
  \texttt{\colZ{/etc/resolve.conf}}.
}

\texttt{SIGHUP} does \Cola{not} re-read the configuration file.

\dSay{
  So it's like a ``refresh''.
}

\cSay{Yeah.}

\subsection{dump log}

\colz{
  
  When it receives a \colZ{\texttt{SIGUSR1}}, \dmq{} writes stastics to the
  system log.

  When it receives a \colZ{\texttt{SIGUSR2}}, \dmq{} is logging directly to a
  file. (configured by \texttt{--log-facility}).
}

\subsection{some options}

\begin{itemize}
\item \texttt{-h,--no-hosts}: \colz{Don't read the hostnames in
    \texttt{/etc/hosts}.}
\item \texttt{-H,--addn-hosts=<file>}: \colz{Read the hostnames in
    \texttt{<file>} in addition to \texttt{/etc/hosts}.

    If \texttt{--no-hists} is given, read only \texttt{<file>}.
  }
\item \texttt{-d,--no-daemon}: \colz{Debug mode: \colZ{don't fork to the background}.
    don't write a pid file, don't change user id, generate a complete cache dump
    on receipt on \colZt{SIGUSR1}, log to stderr as well as syslog, don't fork new
    processes to handle TCP queries.

    \emoji{turtle} : Use \texttt{-k,--keep-in-foreground} in production.
  }
\item \texttt{-q,--log-queries}: \colz{Log the results of DNS queries handled by
    dnsmasq.}
\item \texttt{-8,--log-facility=<facility>}: \colz{
    If \texttt{<facility>} contains \texttt{/}, then it's considered a filename,
    and \dmq{} won't log to syslog, but will log to the specified file.
  }
\item \texttt{-p,--port=<port>}: \colz{Listen on \texttt{<port>} instead of the
    standard DNS port (53). Setting this to zero completely disables DNS,
    leaving only DHCP and/or TFTP.
  }
\end{itemize}

\subsection{How it does DNS}

\dmq{} is a  query forwarder: it is not capable of recursively answering
arbitrary queries starting from the root servers but forwards such queries to a
\cola{fully recursive DNS updtream server}
\colz{ which is typically provided by an \colZ{ISP}. By default, \dmq{} reads
  \colZ{\texttt{/etc/resolve.conf}} (or equivalent if \colZt{--resolv-file} is
  used) and re-reads it if it changes.

  This allows the DNS servers to be \colZ{set dynamically by PPP or DHCP}
}

\dSay{Oh, so it will watch a \texttt{resolve.conf} file.}

\cSay{Yeah..
  \texttt{-r,--resolv-filr = <file>}\newline
  Read the IP addresses of the \cola{upstream nameservers} from \texttt{<file>}.
  \colz{default to \texttt{/etc/resolv.conf}.}
}

\dSay{
  Is it possible to use commandline to specify it ?
}

\cSay{Yes. Use:

  \begin{enumerate}
  \item \texttt{-R,--no-resolv} Don't read \texttt{/etc/resolv.conf}. \colz{ Get
      upstream servers only from the command line or the \dmq{} configuration
      file.}
  \item \texttt{--rev-server=...} This specify the \cola{upstream server(s)} directly.
    \colz{For example, --rev-server=1.2.3.0/24,192.168.0.1}
  \end{enumerate}


}


\dSay{Wait, what's PPP?}

\cSay{ Point-to-Point protocol. \colz{ I think it's a
    protocol that also touch the ip address, but to the best of our knowledge,
    it has nothing to do with DHCP. } }

\colz{Absense of \texttt{/etc/resolve.conf} is not an error since it may not
  have been created before a PPP connection exists. }

\cSay{Oh, so PPP creates the \texttt{resolve.conf} ?}

\dSay{Seem so.}

\colz{
  \dmq{} simply keep checking in case this file is created at any time. \dmq{}
  can be told to parse more than one \texttt{resolve.conf} file. This is useful
  on a laptop, where both PPP and DHCP may be used.


  \colZ{Upstream servers may also be specified.} These server specifications
  optionally take a domain name which tells \dmq{} to use that server only to
  find names in that particular domain.

  In order to configure \dmq{} to act as chache on the host on which it is
  running, put \colZt{"name server 127.0.0.1"} in \colZt{/etc/resolve.conf} to
  force local processes to send queries to \dmq{}.

  Then either

  \begin{enumerate}
  \item specifying the upstream servers directly to \dmq{} using
    \colZt{--server} options, or
  \item putting their addresses real in another file, say
    \texttt{/home/me/my-resolve.conf} and use \texttt{--resolv-file
      /home/me/my-resolve.conf}.
  \end{enumerate}

  The second method allows for dynamic update of server addresses by PPP or DHCP.

  \colZ{Addresses in \texttt{/etc/hosts} will ``shadow'' different addresses for
    the same names in the upstream DNS, so ``\texttt{aaa.com 1.2.3.4}'' in
    \texttt{/etc/hosts} will ensure that queries for ``\texttt{aaa.com}'' always
    return \texttt{1.2.3.4} even if queries in the upstream DNS would say
    something else.}There's one exception to this: it's something to do with an
  upstream \texttt{CNAME} entry, but let's stop here.
}

\subsection{How it does DHCP}

\colz{
  The tag system works as follows:

  \begin{enumerate}
  \item For each DHCP request, \dmq{} collects a set of valid tags from active
    configuration lines which include \texttt{set:<tag>}, including
    \begin{enumerate}
    \item one from the \colZ{\texttt{--dhcp-range}} used to allocate the
      address,
    \item one from any matching \colZt{--dhcp-host}
    \item and ``known'' or ``known-ethernet'' if a \texttt{--dhcp-host} matches.
    \end{enumerate}
  \item The tag ``bootp'' is set for BOOTP requests, 
  \item a tag whose name is the name of the interface on which the request
    arrived is also set.
  \end{enumerate}

  \cSay{
    Here are some details, first,

    \texttt{-F,--dhcp-range=
        \cola{tag1:<tag1>,tag2:<tag2>,...}\colb{, set:<tag>}
        \colc{<start-addr>,[<end-addr>],}
        [<many options>]
      } Enable the DHCP server.

      Addresses will be given out from the range \colZt{<start-addr>} to
      \colZt{<end-addr>} and from statically defined addresses given in
      \texttt{--dhcp-host} options.

      \colz{

        A lease time can be given. By default, it's one hour for IPv4 and
        one day for IPv6.

        For IPv6, there's an prefix length which must be equal to or larger then
        the prefix length on the local interface. This defaults to 64. Unlike
        the IPv4 case, the prefix length is not automatically derived from the
        interface. confiugration. (minimum prefix length = 64.)

      }
  }

  \cSay{
    \colz{
      IPv6 supports another type of range, In this, the start address and
      optional end address \colZ{contain only the network part (ie
        \texttt{::1})} and they are followed by
      \texttt{constructor:<interface>}, for example,
      \colZt{--dhcp-range=::1,::400,constructor:eth0}, will look for addresses
      on \texttt{eth0} , and then create a range from \texttt{<network>::1} to
      \texttt{<network>::400}.  Note that not just any address on
      \texttt{eth0} will not do: it must not be an autoconfigured or privacy
      address, or be deprecated.

      If a \texttt{--dhcp-range} is only being used for stateless DHCP, then
      it can simply be:
      \colZt{--dhcp-range=::,constructor:eth0}
    }
  }

  \dSay{\colz{What about \texttt{--dhcp-host} ?}}

  \cSay{
{

  \ttfamily
  -G,--dhcp-host= \cola{[<hwaddr>]}
  \colz{
    [,id:<client\_id>|*]
    [,set:<tag>] [,tag:<tag>] \colb{[,<ipaddr>]} \cola{[,<hostname>]} \colZ{[,<lease\_time>]} [,ignore]
  }  
}
    

    Specify per host parameters for the DHCP server. This allows a machine with
    a particular hardware address to be always allocated the same
    \begin{enumerate}
    \item hostname,
    \item IP address and
    \item lease time.
    \end{enumerate}
  }

  \dSay{
    Oh, this one is pretty important for those servers that need a static IP.
  }

  \cSay{
    \colz{
      A hostname specified like this overrides any supplied by the \colZ{DHCP
        client on the host}.

      It is also allowable to omit the hardware address and include the
      hostname. In this case, the IP address and lease times will \colZ{applied
        to any machine claiming that name}. For example
      \colZt{--dhcp-host=00:11:22:33:44:55,myhost,infinite} tells \dmq{} to give
      the machine with hardware address \texttt{00:11:22:33:44:55} the name
      ``\texttt{myhost}'' and an infinite lease. For another example,
      \colZt{--dhcp-host=myhost2,10.0.0.2} tells \dmq{} to give the machine
      ``\texttt{myhost2}'' the IP address \texttt{10.0.0.2}
    }
  }

  \dSay{
    What ? Hosts can have names before they are assigned an IP address ?
  }

  \cSay{It just said so in the manual. Let's just believe it.}

  \dSay{
    Okay...
  }

  \cSay{
    \colz{
      Addresses allocated like this are not constrained to be in the range
      given in \texttt{--dhcp-range} options. For subnets which don't need a pool
      of dynamically allocated addresses, you can use a ``\texttt{static}''
      keyword in the \texttt{--dhcp-range} declaration.
    }
  }

  \dSay{ Oh, so DHCP can also be used to assign static IP addresses?}

  \cSay{Seem so.}

  \dSay{
    I remember for IPv6, there're different ways to do automatic address
    configuration...Right? What's said about this in the manual?
  }

  \cSay{
    For IPv6, the mode may be some combination of
    \begin{enumerate}
    \item \texttt{slaac} \colz{stateless address autoconfiguration},
    \item \texttt{ra-only} \colz{router advertisement only},
    \item \texttt{ra-names} \colz{router advertisement plus names from DNS}.
    \item \texttt{ra-stateless} \colz{address from SLAAC and others from DHCP}.
    \item \texttt{ra-advrouter} \colz{enables a mode where router address(es)rather
      than prefix(es) are included in the advertisements.}
  \item \texttt{off-link} \colz{
      tells \dmq{} to advertise the prefix without the on-link bit set.
    }
    \end{enumerate}
  }

  \dSay{What is \texttt{client\_id} ?}

  \cSay{It's a unique identifier for the client. It's called \texttt{DUID} in
    IPv6. It's an alternative to the hardware address.

    \colz{
      Thus:

      \colZt{--dhcp-host=id:01:02:03:04,...} refers to the host with client
      identifier \texttt{01:02:03:04}. It is also possible to use text as in
      \colZt{--dhcp-host=id:myhost,...}.
    }
  }

  \cSay{
    \colz{
      A single \texttt{--dhcp-host} option may contain
      \begin{enumerate}
      \item an IPv4 address, or 
      \item one or more IPv6 addresses, or
      \item both
      \end{enumerate}.

      IPv6 addresses must be bracketed by square brackets thus:

      \colZt{--dhcp-host=laptop,[2001:db8::1],[2001:db8::2]}
    }
  }

  

  Any configuration lines which include one or more \texttt{tag:<tag>}
  constructs will only be valid if all that tags are matched in the set derived
  above. Typically this is \colZt{--dhcp-option}, tagged version of
  \texttt{--dhcp-options} is prefered, provided that all the tags match
  somewhere in the set collected as described above. The prefix \texttt{!} on a
  tag means ``not'' so \colZt{--dhcp-option=tag:!purple,3,1.2.2.3.4} sends the
  option when the taf \texttt{purple} is not in the set of valid tags.
  (\emoji{turtle} : If using
  this in a command line rather than a config file, be sure to escape
  \texttt{!}, which is a shell metachar...)

  When selecting \colZt{--dhcp-options}, a tag from \colZt{--dhcp-range} is
  second class relative to other tags, to make it easy to override options for
  individual hosts, so
  \colZt{
    \newline
    --dhcp-range=set:interface1,....\newline
    --dhcp-option=tag:interface1,\cola{option:nis-domain,domain1} \newline
    --dhcp-option=tag:myhost,\cola{option:nis-domain,domain2 }
  }

  will set the NIS-domain to \colZt{domain1} for hosts in range, but override
  that to \colZt{domain2} for \colZt{myhost}.

  \colZ{
    Note that for \texttt{--dhcp-range} both \texttt{tag:<tag>} and
    \texttt{set:<tag>} are allowed, to both select the range in use based on
    (eg) \texttt{--dhcp-host},and to affect the options sent, based on the range
    selected.
  }
}

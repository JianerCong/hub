
\section{\dmq{}}

\subsection{Intro}


\Cola{\dmq{}} is a lightweight DNS, TFTP, PXE, router advertisement and
DHCP server. It is intended to provide coupled DNS and DHCP service to a LAN.

\colz{
  \dmq{} \colZ{accepts DNS queries} and either answers them from a small, local, cache
  or forwards them to a real, recursive, DNS server. It loads the contents of
  \colZ{\texttt{/etc/hosts}} so that \colZ{local hostnames which do not appear in the global DNS can be resolved and also
    answers DNS queries for DHCP  configured hosts.}

  It can also act as the authoritative DNS server for one or more domains,
  allowing local names to appear in the global DNS. 

  \colZ{The \dmq{} DHCP server} supports static address assignments and multiple
  networks. It automatically sends a sensible default set of DHCP options
  (configurable).
}

\dSay{
  Oh.. I just realize that, for a local DNS name resolve, you just need to edit
  the \texttt{/etc/hosts}...
}

\subsection{Configure \dmq{}}

\colz{
  Configuring \dmq{} to act as an authoritative DNS server is complicated by the
  fact that it involves configuration of \colZ{external DNS servers} to provide
  delegation. 

  At startup, \dmq{} reads \colZ{\texttt{/etc/dnsmasq.conf}} (configurable
  through \texttt{--conf-file, --conf-dir}) The format of this file consists of
  one option per line, exactly as the long options detailed in the OPTIONS
  section but without the leading \texttt{--}. Line starting with \texttt{\#}
  are comments and ignored. (\emoji{parrot}: latest style)  For options which may only be
  specified once, \colZ{the configuration file overrides the commandline.}

  \colZ{Quoting} is allowed in a config file: between \texttt{"} quotes the
  special meanings of ``\colZ{\texttt{,:;\#}}''. The following escapes are
  allowed: ``\colZ{
    \texttt{
      \textbackslash{}\textbackslash{}
      \textbackslash{}"
      \textbackslash{}t
      \textbackslash{}e
      \textbackslash{}b
      \textbackslash{}r
      \textbackslash{}n
    }
  }'' }

\subsection{Notes for \texttt{SIGHUP}}
\colz{
  When it recieves a \colZ{\texttt{SIGHUP}}, \dmq{} \colZ{clears its cache and
    then re-loads
    \texttt{/etc/hosts, /etc/ethers}
    and any file given by \texttt{--dhcp-hostsfile, --dhcp--hostsdir,
      --dhcp-optsfile, --dhcp-optsdir, --addn-hosts, --hostsdir}.
  }  

  The DHCP lease change script is called for all existing DHCP leases. if
  \colZ{\texttt{--no-poll}} is set, SIGHUP also re-reads
  \texttt{\colZ{/etc/resolve.conf}}.
}

\texttt{SIGHUP} does \Cola{not} re-read the configuration file.

\dSay{
  So it's like a ``refresh''.
}

\cSay{Yeah.}

\subsection{dump log}

\colz{
  
  When it receives a \colZ{\texttt{SIGUSR1}}, \dmq{} writes stastics to the
  system log.

  When it receives a \colZ{\texttt{SIGUSR2}}, \dmq{} is logging directly to a
  file. (configured by \texttt{--log-facility}).
}

\subsection{some options}

\begin{itemize}
\item \texttt{-h,--no-hosts}: \colz{Don't read the hostnames in
    \texttt{/etc/hosts}.}
\item \texttt{-H,--addn-hosts=<file>}: \colz{Read the hostnames in
    \texttt{<file>} in addition to \texttt{/etc/hosts}.

    If \texttt{--no-hists} is given, read only \texttt{<file>}.
  }
\item \texttt{-d,--no-daemon}: \colz{Debug mode: \colZ{don't fork to the background}.
    don't write a pid file, don't change user id, generate a complete cache dump
    on receipt on \colZt{SIGUSR1}, log to stderr as well as syslog, don't fork new
    processes to handle TCP queries.

    \emoji{turtle} : Use \texttt{-k,--keep-in-foreground} in production.
  }
\item \texttt{-q,--log-queries}: \colz{Log the results of DNS queries handled by
    dnsmasq.}
\item \texttt{-8,--log-facility=<facility>}: \colz{
    If \texttt{<facility>} contains \texttt{/}, then it's considered a filename,
    and \dmq{} won't log to syslog, but will log to the specified file.
  }
\item \texttt{-p,--port=<port>}: \colz{Listen on \texttt{<port>} instead of the
    standard DNS port (53). Setting this to zero completely disables DNS,
    leaving only DHCP and/or TFTP.
  }
\end{itemize}

\subsection{How it does DNS}

\dmq{} is a  query forwarder: it is not capable of recursively answering
arbitrary queries starting from the root servers but forwards such queries to a
\cola{fully recursive DNS updtream server}
\colz{ which is typically provided by an \colZ{ISP}. By default, \dmq{} reads
  \colZ{\texttt{/etc/resolve.conf}} (or equivalent if \colZt{--resolv-file} is
  used) and re-reads it if it changes.

  This allows the DNS servers to be \colZ{set dynamically by PPP or DHCP}
}

\dSay{Oh, so it will watch a \texttt{resolve.conf} file.}

\cSay{Yeah..
  \texttt{-r,--resolv-filr = <file>}\newline
  Read the IP addresses of the \cola{upstream nameservers} from \texttt{<file>}.
  \colz{default to \texttt{/etc/resolv.conf}.}
}

\dSay{
  Is it possible to use commandline to specify it ?
}

\cSay{Yes. Use:

  \begin{enumerate}
  \item \texttt{-R,--no-resolv} Don't read \texttt{/etc/resolv.conf}. \colz{ Get
      upstream servers only from the command line or the \dmq{} configuration
      file.}
  \item \texttt{--rev-server=...} This specify the \cola{upstream server(s)} directly.
    \colz{For example, --rev-server=1.2.3.0/24,192.168.0.1}
  \end{enumerate}


}


\dSay{Wait, what's PPP?}

\cSay{ Point-to-Point protocol. \colz{ I think it's a
    protocol that also touch the ip address, but to the best of our knowledge,
    it has nothing to do with DHCP. } }

\colz{Absense of \texttt{/etc/resolve.conf} is not an error since it may not
  have been created before a PPP connection exists. }

\cSay{Oh, so PPP creates the \texttt{resolve.conf} ?}

\dSay{Seem so.}

\colz{
  \dmq{} simply keep checking in case this file is created at any time. \dmq{}
  can be told to parse more than one \texttt{resolve.conf} file. This is useful
  on a laptop, where both PPP and DHCP may be used.


  \colZ{Upstream servers may also be specified.} These server specifications
  optionally take a domain name which tells \dmq{} to use that server only to
  find names in that particular domain.

  In order to configure \dmq{} to act as chache on the host on which it is
  running, put \colZt{"name server 127.0.0.1"} in \colZt{/etc/resolve.conf} to
  force local processes to send queries to \dmq{}.

  Then either

  \begin{enumerate}
  \item specifying the upstream servers directly to \dmq{} using
    \colZt{--server} options, or
  \item putting their addresses real in another file, say
    \texttt{/home/me/my-resolve.conf} and use \texttt{--resolv-file
      /home/me/my-resolve.conf}.
  \end{enumerate}

  The second method allows for dynamic update of server addresses by PPP or DHCP.

  \colZ{Addresses in \texttt{/etc/hosts} will ``shadow'' different addresses for
    the same names in the upstream DNS, so ``\texttt{aaa.com 1.2.3.4}'' in
    \texttt{/etc/hosts} will ensure that queries for ``\texttt{aaa.com}'' always
    return \texttt{1.2.3.4} even if queries in the upstream DNS would say
    something else.}There's one exception to this: it's something to do with an
  upstream \texttt{CNAME} entry, but let's stop here.
}



\section{Media Access Control (MAC) Address}

\subsection{What does it look like}

MAC address has format something like: \texttt{00-0c-29-3b-73-cb}

\cSay{
  In practice, these addresses are used to communicate between hosts in the
  same VLAN or subnet. If you look at a packet capture, at the start of a TCP
  conversation you'll see the sending host send a broadcast \Cola{ARP request} saying
  \begin{center}
    ``Who has IP address x.x.x.x ?''
  \end{center}
  The \Cola{ARP reply} from some host will be
  \begin{center}
    ``That's me, and my MAC address is aaaa.bbbb.cccc ?''
  \end{center}
}

\dSay{
  What if the target IP is on a different subnet ?
}

\cSay{
  Then the sender will \Cola{ARP for} the gateway for that subnet.

  Switch is much faster than router, because hosts can use MAC to communicate.
}

\dSay{So why router ?}

\cSay{
  Because switches only work in LAN.
}

\subsection{Organizationally Unique Identifier (OUI)}

\cSay{The leading (usually 3 ) bytes of a MAC address usually points to its
  manufacturer, like \texttt{OOtronics Ltd. Kibbutz Yizrael, IL} or \texttt{OOel Corporate Kedah,
    MY} see \url{https://standards-oui.ieee.org/oui/oui.txt}
}

\subsection{Address Resolution Protocol (ARP)}

\Cola{ARP} is used to find the MAC address of a network neighbour for a given IPv4 Address.

\cSay{
  When a host wants to \cola{talk} to another host \cola{in the same subnet} using its IP address, it will
  \begin{enumerate}
  \item check its \Cola{ARP cache} to see whether there's a MAC address that matches that IP.
  \item If there isn't, it will send an ARP request to the local broadcast address.
  \item When the corresponding MAC is found, it will \colb{establish port-to-port communications (Next)}.
  \end{enumerate}
}
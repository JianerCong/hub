\documentclass[dvipsnames]{article}
% \documentclass[dvipsnames]{ctexart}

\title{The WebAssembly System Interface (WASI)}
\usepackage{geometry}\geometry{
  a4paper,
  total={170mm,257mm},
  left=20mm,
  top=20mm,
}

\usepackage{graphicx}

\usepackage{svg}

\usepackage[skip=5pt plus1pt, indent=0pt]{parskip}
% Color
\newcommand{\mycola}{MidnightBlue}
\newcommand{\mycolb}{Mahogany}
\newcommand{\mycolc}{OliveGreen}

\newcommand{\cola}[1]{\textcolor{\mycola}{#1}}
\newcommand{\colb}[1]{\textcolor{\mycolb}{#1}}
\newcommand{\colc}[1]{\textcolor{\mycolc}{#1}}
\newcommand{\Cola}[1]{\textcolor{\mycola}{\textbf{#1}}}

% \let\emph\relax % there's no \RedeclareTextFontCommand
% \DeclareTextFontCommand{\emph}{\bfseries}
\renewcommand{\emph}[1]{\texbf{#1}}
\usepackage{amssymb}            %\mathbb

\usepackage{fontspec}

\setmonofont{Cascadia}[
Path=/usr/share/fonts/truetype/Cascadia_Code/,
Scale=0.85,
Extension = .ttf,
UprightFont=*Code,              %find CascadiaCode.ttf
BoldFont=*CodePL,               %find CascadiaCodePL.ttf ...
ItalicFont=*CodeItalic,
BoldItalicFont=*CodePLItalic
]
% --------------------------------------------------
% Windows
% \setmonofont{Cascadia}[
% Path=C:/Windows/Fonts/,
% Extension = .ttf,
% UprightFont=*Mono,              %find CascadiaMono.ttf
% BoldFont=*Code,               %find CascadiaCodePL.ttf ...
% ItalicFont=*Code,
% BoldItalicFont=*Code
% ]


\usepackage{minted}
\usepackage{tcolorbox}
\tcbuselibrary{skins}
\tcbuselibrary{minted}
\usepackage{tikz}
\usetikzlibrary{shapes} % ellipse node shape
\usetikzlibrary{shapes.multipart} % for line breaks in node text
\usetikzlibrary{arrows.meta}    %-o arrow head
\usetikzlibrary{arrows}
\usetikzlibrary{matrix}
\usetikzlibrary{snakes}

\usepackage{amsmath}
% ??? still xelatex?
% \usepackage{xeCJK}
\usepackage{emoji}
% \setemojifont{NotoColorEmoji.ttf}[Path=C:/Users/congj/repo/myFonts/]
% \setemojifont{TwitterColorEmoji-SVGinOT.ttf}[Path=C:/Users/congj/repo/myFonts/]
\usepackage{cleveref}

\date{\today}
\author{作者}

\newtcblisting{simplec}{
  listing engine=minted,
  minted language=c++,
  minted style=vs,
  minted options={fontsize=\small,autogobble,
    % framesep=1cm
  },
  tile,
  listing only,
  % bottom=0cm,
  % nobeforeafter, 
  boxsep=0mm,
  left=1mm,
  opacityback=0.5,
  colback=gray!20
}
\tcbuselibrary{breakable}
\newtcblisting{simplepy}{
  listing engine=minted,
  minted language=python,
  minted style=vs,
  minted options={fontsize=\small,autogobble,
    % framesep=1cm
  },
  tile,
  listing only,
  % bottom=0cm,
  % nobeforeafter,
  boxsep=0mm,
  left=1mm,
  opacityback=0.5,
  colback=gray!20,
  breakable
}
\newtcolorbox{blackbox}{tile,colback=black,colupper=white,nobeforeafter,halign=flush center}

\tikzstyle{myMatrix}=[matrix of nodes,below right,
nodes={above,text centered},                  %apply to all nodes
row sep=1cm,column sep=2cm]
\tikzstyle{every node}=[inner sep=0pt]

\newcommand\uptoleft[3][-o]{\draw[very thick,#1](#2.south) |- (#3.west);}
\newcommand\uptodown[3][-o]{\draw[very thick,#1](#2.south) to [out=270,in=90] (#3.north);}
\newcommand\downtoup[3][-latex]{\draw[very thick,#1](#2.north) to [out=90,in=270] (#3.south);}

\newcommand\lefttoright[3][-latex]{\draw[very thick,#1](#2.east) to[out=0,in=180] (#3.west);}
\newcommand\lefttodown[3][-latex]{\draw[very thick,#1](#2.east) to[out=0,in=90] (#3.north);}


\newtcolorbox{leftDialogBox}{
  tile, nobeforeafter, boxsep=0pt,
  % show bounding box,
  colback=\mycola!10,
  overlay={
    \begin{scope}
      % \fill[gray!10] (frame.east) circle (2pt);
      \fill[\mycola!10] (frame.east) --
      +(0,2mm) --
      +(3mm,0) --
      +(0,-2mm)
      ;
    \end{scope}
  }}


\newtcolorbox{rightDialogBox}{
  tile, nobeforeafter, boxsep=0pt,
  % show bounding box,
  colback=\mycola!10,
  overlay={
    \begin{scope}
      % \fill[gray!10] (frame.east) circle (2pt);
      \fill[\mycola!10] (frame.west) --
      +(0,2mm) --
      +(-3mm,0) --
      +(0,-2mm);
    \end{scope}
  }}

\newcommand{\mycolaa}{\mycola!20}


\usepackage{changepage}   % for the adjustwidth environment
\newenvironment{myIndent}[1][7mm]{\begin{adjustwidth}{#1}{}}{\end{adjustwidth}}

% --------------------------------------------------
\begin{document}
\maketitle{}

\section{Intro}
WASI is a modular system interface for WebAssembly. It's focused on security and portability.

\subsection{Why}
Developers are starting to push WebAssembly beyond the browser, because it
provides a fast, scalable, secure way to run the same code across all machines.

But we don't yet have a solid foundation to build upon. Code outside of a
browser needs a way to talk to the system -- \cola{a system interface}. And the
WebAssembly platform doesn't have that yet. (\emoji{parrot} Oh, like the
standard library \texttt{os} in python).

\subsection{What}

WebAssembly is an assembly language for a conceptual machine, not a physical
one. This is why it can be run across a variety of different machine
architectures.

So, WebAssembly needs a system interface for a \Cola{conceptual operating
  system}, not any \cola{single} operating system.

\section{What's a system interface}

\emoji{turtle} : Many people talk about languages like C giving you direct access to system
resources. But that's \cola{not true}.

These languages don't have direct access to do things like open or create files
on most systemns. (\emoji{parrot} Why not?)

\emoji{turtle} : Because these system resources such as

\begin{itemize}
\item files
\item memory
\item network connections
\end{itemize}

are too important for stability and security.

So we need a way to control which programs and users can access which resources.
People figured this out pretty early on, and came up with a way to provide this
control: \Cola{protection ring security}.

With protection ring security, the operating system basically puts a protective
barrier around the system's resources (\cref{fig:protection-ring}). This is the \cola{kernel}. Only kernel
can do things like creating new file.

The user's programs run outside of this kernel in something called \Cola{user
  mode}. (\emoji{parrot} : Oh, I heard that before.) If a program wants to do
anything like opening a file, it has to ask the kernel.
This is where the concept of the \Cola{system call} comes in.(\cref{fig:syscall}) 
\begin{figure}[ht]
  \centering
  \includegraphics[width=0.7\linewidth]{/home/me/Pictures/protection-ring-sec-1-500x298.png}
  \caption{Protection ring. Source: \texttt{https://hacks.mozilla.org/2019/03/standardizing-wasi-a-webassembly-system-interface/}}
  \label{fig:protection-ring}
\end{figure}


\begin{figure}[ht]
  \centering
  \includegraphics[width=0.7\linewidth]{/home/me/Pictures/syscall-1.png}
  \caption{System call. Source: \texttt{https://hacks.mozilla.org/2019/03/standardizing-wasi-a-webassembly-system-interface/}}
  \label{fig:syscall}
\end{figure}

For example. \texttt{printf} being compiled for Windows could use the Windows
API to interact with the machine. If it's being compiled for Mac or Linux, it
will use POSIX instead.


\begin{figure}[ht]
  \centering
  \includegraphics[width=0.7\linewidth]{/home/me/Pictures/implementations-1.png}
  \caption{Different implementations on different systems. Source:
    \texttt{https://hacks.mozilla.org/2019/03/standardizing-wasi-a-webassembly-system-interface/}}
  \label{fig:syscall}
\end{figure}

\section{What should this system interface look like}

We do have a proposal to start with:

\begin{itemize}
\item Create a modular set of standard interfaces
\item Start with a standardizing the most fundamental module, \texttt{wasi-core}
  (\cref{fig:wasi-1})
\end{itemize}

\begin{figure}[ht]
  \centering
  \includegraphics[width=0.7\linewidth]{/home/me/Pictures/wasi-1.png}
  \caption{The WASI modules. Source:
    \texttt{https://hacks.mozilla.org/2019/03/standardizing-wasi-a-webassembly-system-interface/}}
  \label{fig:wasi-1}
\end{figure}

\subsection{wasi-core}

\texttt{wasi-core} will contain the basics that all programs need. It will cover
much of the same ground as POSIX, including things such as
\begin{itemize}
\item files
\item network connections
\item clocks
\item random numbers
\end{itemize}

Languages like Rust will use \texttt{wasi-core} directly in their standard
libraries. For example, Rust's \texttt{open} is implemented by calling
\verb|__wasi_path_open| when it's compiled to WebAssembly. (\emoji{parrot} : Oh,
so WebAssembly compiled from Rust is already using \texttt{WASI}?.
\emoji{turtle}: Yeah.)

For C++, we've created a \texttt{wasi-sysroot} that implements \texttt{libc} in
terms of \texttt{wasi-core} functions(\cref{fig:libc}).(\emoji{parrot} : What's that?
\emoji{turtle} : It's a special \texttt{libc.h} to use when compiling C++ to
WebAssembly. \emoji{parrot} : What's the compiler ? \emoji{turtle}: For now, the
only compiler that can do is Clang>=10)

We expect compilers like Clang to be ready to interface with the WASI API, and
\colb{complete toolchains} like the \colb{Rust compiler and Emscripten} to use WASI as part of their system implementations.


\begin{figure}[ht]
  \centering
  \includegraphics[width=0.7\linewidth]{/home/me/Pictures/open-imps-1.png}
  \caption{The Compiler Options. Source:
    \texttt{https://hacks.mozilla.org/2019/03/standardizing-wasi-a-webassembly-system-interface/}}
  \label{fig:libc}
\end{figure}

\subsection{How does user's code call these WASI functions?}

The \cola{runtime} that is running the code passes the \texttt{wasi-core} functions as \Cola{imports} (\cref{fig:imports})
\begin{figure}[ht]
  \centering
  \includegraphics[width=0.7\linewidth]{/home/me/Pictures/imports-1.png}
  \caption{How \texttt{wasi} functions are called. Source:
  \texttt{https://hacks.mozilla.org/2019/03/standardizing-wasi-a-webassembly-system-interface/}}
  \label{fig:imports}
\end{figure}

% \begin{figure}[ht]
%   \centering
%   \includegraphics[width=0.7\linewidth]{/home/me/Pictures/wasi-1.png}
%   \caption{The WASI modules. Source:
%     \texttt{https://hacks.mozilla.org/2019/03/standardizing-wasi-a-webassembly-system-interface/}}
%   \label{fig:wasi-1}
% \end{figure}

\end{document}
% Local Variables:
% TeX-engine: luatex
% TeX-command-extra-options: "-shell-escape"
% TeX-master: "m.tex"
% End:
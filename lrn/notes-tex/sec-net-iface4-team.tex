
\section{team (deprecated in favor of bond)}
\label{sec:team}


Network teaming is a feature that combines or aggregates network interfaces to
provide a logical interface with higher throughput or redundancy.

\dSay{
  What's the difference between teaming and bonding?
}

\cSay{
  \colb{
    \emoji{warning}
    Network teaming is deprecated in Red Hat Enterprise Linux 9. Use bonding.
  }
}

Similar to bonding, you usually wanna plug teamed cables into a switch.

\subsection{\texttt{teamd} service, runners, and link-watcher}

\colz{ The team service, \colZt{teamd}, controls one instance of the \Cola{team
  driver}. This instance of the driver adds instances of a \Cola{hardware device driver}
  to form a team of network interfaces. The team driver presents a network
  interface, for example \colZt{team0}, to the kernel.

}

\dSay{
  Oh.. So each instance of \texttt{teamd} corresponds to a team
  , and each team corresponds to a network interface?
}

\cSay{
  Yes.
}

Bonding has \Cola{modes}, and teaming has \Cola{runners} as shown in \cref{tab:runners}

\begin{longtblr}[caption={Available runners}, label={tab:runners}]
  { colspec=XX[4], width = {0.9\linewidth} } \toprule
  Runner & Description \\
  \midrule
  \texttt{boardcast} & Transmits data over all ports.\\
  \texttt{roundrobin} & Transmits data over all ports in turn. \\
  \texttt{activebackup} & Transmits data over one port while the others are kept
  as a backup.\\
  \texttt{loadbalance} & Transmits data over all ports with active Tx load
  balancing and Berkeley Packet Filter (BPF)-based Tx port selectors.
  (\emoji{parrot} : ?? \emoji{see-no-evil-monkey} \emoji{hear-no-evil-monkey} \emoji{speak-no-evil-monkey}
  )\\
  \texttt{random} & Transmits data on a randomly selected port.\\
  \texttt{lacp} &  Implements the 802.3ad Link Aggregation Control Protocol (LACP).\\
  \bottomrule
\end{longtblr}

\colz{
  Administrators specify runners in JavaScript Object Notation (JSON) format,
  and the JSON code is compiled into an instance of \texttt{teamd} when the instance is
  created. Alternatively, when using NetworkManager, you can set the runner in
  the \texttt{team.runner} parameter, and NetworkManager auto-creates the corresponding
  JSON code.
}

\dSay{
  I like JSON, compared to config files....
}

\cSay{
  Yeah, me too.
}


Unlike bond, \texttt{teamd} uses a \Cola{link watcher} to monitor the state of
subordinate devices
\begin{longtblr}[caption={link watcher}, label={tab:link-watchers}]
  { colspec=XX[4], width = {0.9\linewidth} } \toprule
  link watcher & Description \\
  \midrule

  \texttt{ethtool} & The \texttt{libteam} library uses the \texttt{ethtool} utility to
  watch for link state changes. This is the \colb{default} link-watcher.\\

  \texttt{arp\_ping} & The \texttt{libteam} library uses the \texttt{arp\_ping}
  utility to monitor the presence of a far-end hardware address using Address
  Resolution Protocol (ARP).\\

  \texttt{nsna\_ping} & On IPv6 connections, the \texttt{libteam} library uses the
  \cola{Neighbor Advertisement} and \cola{Neighbor Solicitation} features from the IPv6
  \cola{Neighbor Discovery} protocol to monitor the presence of a neighbor’s interface.\\

  \bottomrule
\end{longtblr}

\colz{

  Each runner can use any link watcher, with the exception of \texttt{lacp}. This
  runner can only use the \texttt{ethtool} link watcher.

}

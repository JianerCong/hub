
\section{Networking}
\label{sec:network}

\colz{
  OpenStack-Ansible deploys \cola{Linux containers (LXC)} and uses Linux or Open
  vSwitch-based bridging between the container and the host interfaces to ensure
  that all traffic from containers flows over multiple host interface.

  All services in this deployment model use \cola{unique IP address}.
}

\dSay{
  Okay...So how to configure network for our environment ?
}

\cSay{
  \url{https://docs.openstack.org/openstack-ansible/2023.1/reference/inventory/openstack-user-config-reference.html#openstack-user-config-reference}
}

\dSay{
  What about Neutron ? How does it work?

}

\cSay{
  \url{https://docs.openstack.org/neutron/latest/admin/index.html}
}

\subsection{Physical host interfaces}
\label{sec:physical-host-interfaces}

In a typical production environment, physical network interfaces are combined in
bonded pairs for better rundundancy and performance.

\dSay{
  Wait...What do you mean by ``bonding'' ?
}

\cSay{
  bond is a Linux kernel driver that allows you to ``bind'' multiple physical or
  virtual interfaces together into a single aggregated interface. (Yes, two
  cables bond into one)
}


\dSay{
  Okay...But does it means the other ends of the cables should be bonded as well?
}

\cSay{
  Yes. But most of the time, the other ends are plugged into a switch.
}
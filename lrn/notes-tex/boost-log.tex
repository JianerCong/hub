% \documentclass{ctexart}
\documentclass[dvipsnames]{article}

\title{Boost log}

% better description list
\usepackage{enumitem}
\setlist[description]{leftmargin=\parindent,labelindent=\parindent}

\usepackage{geometry}\geometry{
  a4paper,
  total={170mm,257mm},
  left=20mm,
  top=20mm,
}

\usepackage[skip=10pt plus1pt, indent=0pt]{parskip}
% Color
\newcommand{\mycola}{MidnightBlue}
\newcommand{\mycolb}{Mahogany}
\newcommand{\mycolc}{OliveGreen}

\newcommand{\cola}[1]{\textcolor{\mycola}{#1}}
\newcommand{\colb}[1]{\textcolor{\mycolb}{#1}}
\newcommand{\colc}[1]{\textcolor{\mycolc}{#1}}
\newcommand{\colz}[1]{\textcolor{gray}{#1}}
\newcommand{\Cola}[1]{\textcolor{\mycola}{\textbf{#1}}}

% \let\emph\relax % there's no \RedeclareTextFontCommand
% \DeclareTextFontCommand{\emph}{\bfseries}
% \renewcommand{\emph}[1]{\texbf{#1}}

\usepackage{fontspec}
\setmonofont{Cascadia}[
Path=/usr/share/fonts/truetype/Cascadia_Code/,
Scale=0.85,
Extension = .ttf,
UprightFont=*Code,              %find CascadiaCode.ttf
BoldFont=*CodePL,               %find CascadiaCodePL.ttf ...
ItalicFont=*CodeItalic,
BoldItalicFont=*CodePLItalic
]
\usepackage{minted}
\usepackage{tcolorbox}
\tcbuselibrary{skins}
\tcbuselibrary{minted}
\usepackage{tikz}
\usetikzlibrary{shapes} % ellipse node shape
\usetikzlibrary{shapes.multipart} % for line breaks in node text
\usetikzlibrary{arrows.meta}    %-o arrow head
\usetikzlibrary{arrows}
\usetikzlibrary{matrix}


% Redefine em
% latex.sty just do: \DeclareTextFontCommand{\emph}{\em}

\let\emph\relax % there's no \RedeclareTextFontCommand
\DeclareTextFontCommand{\emph}{\bfseries\em}
\usepackage{amsmath}
\usepackage{cleveref}           %must be loaded after amsmath
% ??? still xelatex?
% \usepackage{xeCJK}
\usepackage{emoji}

\date{\today}
\author{me}

\newtcblisting{simplec}{
  listing engine=minted,
  minted language=c++,
  minted style=vs,
  minted options={fontsize=\small,autogobble,
    % framesep=1cm
  },
  tile,
  listing only,
  % bottom=0cm,
  % nobeforeafter, 
  boxsep=0mm,
  left=1mm,
  opacityback=0.5,
  colback=gray!20
}

\usepackage{comment}
\begin{document}
\maketitle

\begin{comment}
  % black and white {\fontspec{Symbola}\symbol{"1F343}} sudo apt -y install
  % fonts-symbola

  % The luatex emoji
  Usually you need to initialize a \Cola{sink} in order to use
  \texttt{boost.log}.

  \begin{tcolorbox}
    \emoji{parrot}  : But we still used trivial log all the time no? \\
    \emoji{turtle} : Yeah, that's because the library contains a `default` sink.
    (like the defualt constructor in C++). Once you defined a sink yourself, the
    default one is gone. Although you will still be able to use trivial logging
    macro.
  \end{tcolorbox}

  \section{File logging}

  Use the following to \Cola{initialize a logging sink} that stores log records
  into a text log.
  \begin{simplec}
    void init(){ logging::add_file_log("sample.log");

      logging::core::get()->set_filter( logging::trivial::severity >=
      logging::trivial::info ); }
  \end{simplec}

  \section{More sinks}

  You can register sink manually. Previously we created a file sink which is
  almost like
  \begin{simplec}
    void init() { // Construct the sink typedef sinks::synchronous_sink<
      sinks::text_ostream_backend > text_sink; boost::shared_ptr< text_sink > sink
      = boost::make_shared< text_sink >();

      // Add a stream to write log to sink->locked_backend()->add_stream(
      boost::make_shared< std::ofstream >("sample.log"));

      // Register the sink in the logging core
      logging::core::get()->add_sink(sink); }
  \end{simplec}

  \begin{tcolorbox}
    \emoji{parrot} : It looks like a sink has two classes: a \Cola{frontend} and a \Cola{ backend}.\\
    \emoji{turtle} : Yeah
  \end{tcolorbox}
  The frontend, (\texttt{synchronous\_sink}) is responsible for various common
  tasks for \emph{all sinks} such as
  \begin{itemize}
  \item thread synchronization model
  \item filtering %M-Ret to change
  \item formatting (for text-based sinks)
  \end{itemize}
  \begin{tcolorbox}
    \emoji{parrot} : What there are ``non-text-based'' sinks?\\
    \emoji{turtle} : Probably, but let's not dive in.
  \end{tcolorbox}

  The backend \verb|text_ostream_backend| implements everything specific to the
  sink, such as ``writting to a file''. The library provides a number of frontends
  and backends out of the box.

  The \verb|synchronous_sink| class template above means that the \emph{sink} is
  synchronous, that is, it allows for several threads to log simultaneously and
  will block in case of contention.

  This means that the backend \verb|text_ostream_backend| dosen't have to worry
  about multithreading at all.

  The \verb|text_ostream_backend| weites formatted log records into STL-compatible
  streams. We have used a \emph{file stream} above, but we could have used any
  type of stream. For example, adding output to console could look like as
  follows:

  \begin{simplec}
    #include <boost/core/null_deleter.hpp>

    // We have to provide an empty deleter to avoid destroying the global stream
    object boost::shared_ptr< std::ostream > stream(&std::clog,
    boost::null_deleter()); sink->locked_backend()->add_stream(stream);
  \end{simplec}

  The \verb|text_ostream_backend| supports adding several streams. In that case
  its output will be duplicated to all added streams.

  \begin{tcolorbox}
    \emoji{parrot} : So if we add a file stream and a console stream to a single
    backend, we got them both logged to?\\
    \emoji{turtle} : Yeah, and that is a use case suggested by the official,
    because all the formatting and filtering only happen once per record.\\

    \emoji{parrot} : Fine, but I realize that duplicating the same content is not
    that useful.\\
    \emoji{turtle} : That's when you need multiple sinks.
  \end{tcolorbox}

  \section{Creating loggers and writting logs}

  \subsection{Dedicated logger objects}
  Now that we have defined where and how the log is to be stored, it's time to see
  how we emit log records. In order to be able to make log records one has to
  create a \Cola{logging source} first. This would be a logger object:

  \begin{simplec}
    #include <boost/log/sources/severity_logger.hpp>
    boost::log::sources::severity_logger<logging::trivial::severity_level> slg;
  \end{simplec}
  \begin{tcolorbox}
    \emoji{parrot} : What's the logger for trivial logging?\\
    \emoji{turtle} : The library provide one and is used behind the scene.
  \end{tcolorbox}

  Unlike sinks, sources need \emph{not be} registered anywhere since they interact
  directly with the \emph{logging core}.

  Also note that there are two versions of loggers:
  \begin{itemize}
  \item thread-safe ones
  \item non-thread-safe ones.
  \end{itemize}

  For non-thread-safe ones, it is safe for different threads to write logs through
  different \Cola{instances} of loggers and thus there should be a seperate logger
  for each thread that writes logs.

  Thread-safe ones have \texttt{\_mt} suffix, and can potentially be very slow.

  All loggers are \emph{default} , \emph{copy-constructible} and support
  \emph{swapping}, so safe to be put into a class. (\emoji{turtle} : Actually it's
  recommended to do so.)

  The library provides a number of loggers with different features, such as
  \begin{itemize}
  \item severity
  \item channel support
  \end{itemize}

  \subsection{Gobal logger objects}

  In case you cannot put a logger into your class, the library provides a way of
  declaring global loggers:
  \begin{simplec}
    BOOST_LOG_INLINE_GLOBAL_LOGGER_DEFAULT(my_logger, src::logger_mt)
  \end{simplec}
  Here \verb|my_logger| is a \Cola{user-defined tag name } that will be used later
  to retrieve the logger instance and \verb|logger_mt| is the logger type. Any
  logger type
  \begin{itemize}
  \item provided by the library, or
  \item defined by the user
  \end{itemize}
  can participate in such declaration.
  \begin{tcolorbox}
    \emoji{turtle} : Use thread-safe logger as the global one (if you need a
    global logger).
  \end{tcolorbox}
  Later on you can acquire the logger like:
  \begin{simplec}
    src::logger_mt& lg = my_logger::get();
  \end{simplec}
  Then, \texttt{lg} will refer to the one and only one instance of the logger
  throughout the application, even if the application consists of multiple
  modules. The \texttt{get} function itself is thread-safe, so there is no need in
  additional synchronization around it.

  \subsection{Writing logs}

  Write a record to a logger \texttt{lg}:
  \begin{simplec}
    logging::record rec = lg.open_record(); // check if consumed by at least one
    sink [filtering applies here] if (rec){ logging::record_ostream strm(rec);
      strm << "Hello, World!"; strm.flush(); lg.push_record(boost::move(rec)); }
  \end{simplec}

  Of course, users are recommended to write their own macros instead of using the
  C++ logger interface directly like:
  \begin{simplec}
    BOOST_LOG(lg) << "Hello, World!"; //macro defined by library
  \end{simplec}

  \section{Add more information to \texttt{log::Attributes}}

  Each log record can have a number of \Cola{named attribute values} attached.
  Attributes can represent any essential information about the conditions in which
  log record occurred, such as
  \begin{itemize}
  \item position in code
  \item executable module name
  \item current date and time
  \item custom data
  \end{itemize}

  An attribute may behave as a \textbf{value generator}, in which case it would
  return a different value for each \textbf{log record} it's involved in. As soon
  as the attribute generates the value, the latter becomes independent from the
  creator and can be used by
  \begin{itemize}
  \item filters
  \item formatters
  \item sinks
  \end{itemize}

  But in order to use the attribute value one has to know its name and type, or at
  least a set of types it may have. There're a number of commonly used attributes
  implemented in the library.

  Aside from that, there're three \Cola{possible scopes} of attributes:
  \begin{itemize}
  \item source-specific
  \item thread-specific
  \item global
  \end{itemize}
  When a log record is made, attribute values from these three sets are joined
  into a \textbf{single} set and passed to \textbf{sinks}.\\
  \emoji{turtle} : So the origin of attributes makes no difference for sinks.

  \textbf{Any} attribute can be registered in \textbf{any} scope. When registered,
  an attribute is given a \textbf{unique name} in order to make it possible to
  search for it (Lower scope overwrites higer scope). Such behavior makes it
  possible to override global or thread-specific attributes with the ones
  registered in your local logger.

  \subsection{Commonly used attributes}

  There're attributes that are likely to be used in nearly any application.
  \textbf{Log record counter} and a \textbf{time stamp} for example. They can be
  added by:
  \begin{simplec}
    logging::add_common_attributes(); // add : LineID (record
    count),TimeStamp,ProcessID,ThreadID
  \end{simplec}

  \begin{tcolorbox}
    \emoji{turtle} : By default, when app starts, no attributes are registered.
    The app has to do it before writing any logs.\\
    \emoji{parrot} : Well then, how trivial-logging works?\\
    \emoji{turtle} : The answer is that \Cola{the default sink} dosen't really use
    any attribute values, except for the severity level, to compose its output.
    This is done to avoid need for any initialization for trivial-logging. Once
    you use \textbf{filters} or \textbf{formatters} and \textbf{non-default sinks}
    you will have to register the attributes you need.
  \end{tcolorbox}

  The \verb|add_common_attributes| function is one of the several convenience
  helpers.

  Some attributes are registered automatically on logger construction. For
  example, \verb|severity_logger| register a source-specific attribute
  ``Severity'' which can be used to add a level of emphasis for different log
  records:
  \begin{simplec}
    // We define our own severity levels enum severity_level {normal,
      notification, warning, error, critical};

    void logging_function(){ // The logger implicitly adds a source-specific
      attribute 'Severity' // of type 'severity_level' on construction
      src::severity_logger< severity_level > slg;

      BOOST_LOG_SEV(slg, normal) << "A regular message"; BOOST_LOG_SEV(slg,
      warning) << "Something bad is going on but I can handle it";
      BOOST_LOG_SEV(slg, critical) << "Everything crumbles, shoot me now!"; }
  \end{simplec}
  The \verb|BOOST_LOG_SEV| macro acts as if the following:
  \begin{simplec}
    void manual_logging(){ src::severity_logger< severity_level > slg;

      logging::record rec = slg.open_record(keywords::severity = normal); //
      ***********Provide named arguments if (rec){ logging::record_ostream
        strm(rec); strm << "A regular message"; strm.flush();
        slg.push_record(boost::move(rec)); } }
  \end{simplec}

  \subsection{More attributes}

  \verb|add_common_attributes| works somehow like
  \begin{simplec}
    void add_common_attributes(){ boost::shared_ptr< logging::core > core =
      logging::core::get(); core->add_global_attribute("LineID", attrs::counter<
      unsigned int >(1)); // ************** attribute classes
      core->add_global_attribute("TimeStamp", attrs::local_clock()); // ****
      derive from attribute

      // other attributes skipped for brevity }
  \end{simplec}
  The library provides a number of other \Cola{attribute classes}, including the
  \texttt{function} attribute that calls some function object on value
  acquisition. For example, we can register a \texttt{name\_scope} attribute:
  \begin{simplec}
    core->add_global_attribute("Scope", attrs::named_scope());
  \end{simplec}
  This will give us the ability to \textbf{store scope names in log} for every log
  record:
  \begin{simplec}
    void named_scope_logging(){ BOOST_LOG_NAMED_SCOPE("named_scope_logging");
      src::severity_logger< severity_level > slg; BOOST_LOG_SEV(slg, normal) <<
      "Hello from the function named_scope_logging!"; }
  \end{simplec}
  \begin{tcolorbox}
    \emoji{parrot} : What does \verb|BOOST_LOG_NAMED_SCOPE| mean ?\\
    \emoji{turtle} : ...I don't find it on the manual either...
  \end{tcolorbox}

  Logger-specific attributes (source-level) are also useful. \textbf{Severity
    levels} and \textbf{channel} names are the most obvious candidates to be
  implemented on this level.

  \begin{tcolorbox}
    \emoji{parrot} : So after all, how do I add my own attributes any way?\\
    \emoji{turtle} : Here:
  \end{tcolorbox}
  \begin{simplec}
    void tagged_logging(){ src::severity_logger< severity_level > slg;
      slg.add_attribute("Tag", attrs::constant< std::string >("My tag value"));

      BOOST_LOG_SEV(slg, normal) << "Here goes the tagged record"; }
  \end{simplec}
  \emoji{turtle} : And now log records made through \textbf{this logger} will be
  tagged with \texttt{"Tag"}. This attribute value may be used later in filtering
  and formatting.

  Another use of attributes is the ability to mark log records made by different
  application in order to hight \emph{activity related to a single process}. One
  can even implement a rough profiling tool to detect performance bottlenecks. For
  example:

  \begin{simplec}
    void timed_logging(){ BOOST_LOG_SCOPED_THREAD_ATTR("Timeline",
      attrs::timer()); src::severity_logger< severity_level > slg;
      BOOST_LOG_SEV(slg, normal) << "Starting to time nested functions";
      logging_function(); BOOST_LOG_SEV(slg, normal) << "Stopping to time nested
      functions"; }
  \end{simplec}
  Now every log record made from the \verb|logging_function()|, will contain the
  \texttt{"Timeline"} attribute with a \textbf{time duration passed since the
    attribute was registered.}.
  \begin{tcolorbox}
    \emoji{parrot} : This is pretty useful. At least I understand this. But how do
    I get (print) these attributes?\\
    \emoji{turtle} : Patience, Later..
  \end{tcolorbox}

  Based on these readings, one will be able to detect which parts of the code
  require more or less time to execute. The \texttt{"Timeline"} attribute will be
  unregistered upon leaving the scope of function \verb|timed_logging|.

  \subsection{Defining attribute placeholders}
  \label{sec:def_attr_placeholders}


  It is useful to define a \textbf{keyword describing a particular attribute} the
  app uses. This keyword will be able to participate in
  \begin{itemize}
  \item filtering, and
  \item formatting expression
  \end{itemize}
  like the \texttt{severity} placeholder we have used:
  \begin{simplec}
    BOOST_LOG_ATTRIBUTE_KEYWORD(line_id, "LineID", unsigned int)
    BOOST_LOG_ATTRIBUTE_KEYWORD(severity, "Severity", severity_level)
    BOOST_LOG_ATTRIBUTE_KEYWORD(tag_attr, "Tag", std::string)
    BOOST_LOG_ATTRIBUTE_KEYWORD(scope, "Scope", attrs::named_scope::value_type)
    BOOST_LOG_ATTRIBUTE_KEYWORD(timeline, "Timeline", attrs::timer::value_type)
  \end{simplec}

  \section{Log record formatting}
  By default, \Cola{formatter} is set, only the \textbf{log record messages} are
  written to the files. Even if you added attributes to the logging core or a
  logger, the attribute values \emph{will not} reach the output unless you specify
  a formatter that will use these values.

  Returning to one of the previous examples:

  \begin{simplec}
    void init(){ logging::add_file_log( /*...*/ keywords::format =
      "[%TimeStamp%]: %Message%"
      );

      logging::core::get()->set_filter( logging::trivial::severity >=
      logging::trivial::info ); }
  \end{simplec}

  \begin{tcolorbox}
    \emoji{parrot} : Hey, I see a \texttt{format} option.\\
    \emoji{turtle} : Yeah, if you prefer to \emph{set up sinks manually},
    \textbf{sink frontends} provide the \texttt{set\_formatter} member function
    for this purpose. \\
    \emoji{parrot} : I think I do like to set up sinks manually...Because I want
    to write to different log file No?
  \end{tcolorbox}

  The syntex for a format is
  \begin{center}
    \texttt{keywords::format = \cola{<format-spec>}}
  \end{center}
  And you can specify \texttt{<format-spec>} in a couple of ways.
  \subsection{$\lambda$-style formatter}
  \begin{simplec}
    void init() { logging::add_file_log (keywords::file_name = "sample_%N.log",
      // This makes the sink to write log records that look like this: // 1:
      <normal> A normal severity message // 2: <error> An error severity message
      keywords::format = ( expr::stream << expr::attr< unsigned int >("LineID") <<
      ": <" << logging::trivial::severity << "> " << expr::smessage )); }
  \end{simplec}
  Here the \texttt{stream} is a placeholder for the stream to format the record
  in.

  Other insertion arguments, such as \texttt{attr} and \texttt{message}, are
  manupulators that define what should be stored in the stream.

  \begin{tcolorbox}
    \emoji{parrot} : What? I saw \texttt{smessage} ?\\
    \emoji{turtle} : \texttt{smessage} means \texttt{string}, \texttt{wmessage}
    means \texttt{wstring}, \texttt{message} means one of the two.
  \end{tcolorbox}

  We have already seen the \texttt{severity} placeholder in \Cola{filtering
    expressions:}
  \begin{simplec}
    logging::core::get()->set_filter(logging::trivial::severity >=
    logging::trivial::info);
  \end{simplec}
  \begin{tcolorbox}
    \emoji{turtle} : The author finds it a nice unification to be able to use the
    same \textbf{placeholder} in both filters and formatter.
  \end{tcolorbox}

  The \texttt{attr} placeholder is similar to the \texttt{severity} placeholder as
  it represents the attribute value, too. The difference is that
  \begin{itemize}
  \item \texttt{severity} represents the particular attribute, with the name
    ``Severity'' and type \texttt{trivial::severity\_level}
  \item \texttt{attr} can be used to represent \textbf{any} attribute.
  \end{itemize}
  Otherwise the two are equivalent. So \texttt{logging::trivial::severity} is
  essentially
  \begin{simplec}
    expr::attr< logging::trivial::severity_level >("Severity")
  \end{simplec}

  \begin{tcolorbox}
    \emoji{parrot} : I prefer the explicit... It's simpler for me...\\
    \emoji{turtle} : But the author recommends defining these placeholder for
    \textbf{user-defined} attributes as in \cref{sec:def_attr_placeholders},
    because it's less error-prone.\\
    \emoji{parrot} : That makes sense, but we're not defining format a lot of time
    right?
  \end{tcolorbox}

  There are other \Cola{formatter manipulators} that provide advanced support for
  date, time and other types. Some manipulators accept additional arguments that
  customize their behavior. Most of these arguments are named and can be passedin
  \texttt{Boost.Parameter} style.

  \begin{tcolorbox}
    \emoji{parrot} : What is \texttt{Boost.Parameter} ?\\
    \emoji{turtle} : It's a magical library that allows you to specify
    ``named-parameter'' in C++. like \texttt{f(a=123,b=true)}.\\
    \emoji{parrot} : I am not gonna attempt that (for now)...
  \end{tcolorbox}

  For a change, here's how we manually initialize sinks:

  \begin{simplec}
    void init(){ typedef sinks::synchronous_sink< sinks::text_ostream_backend >
      text_sink; boost::shared_ptr< text_sink > sink = boost::make_shared<
      text_sink >();

      sink->locked_backend()->add_stream(boost::make_shared< std::ofstream
      >("sample.log"));

      sink->set_formatter( expr::stream // line id will be written in hex,
      8-digits, zero-filled << std::hex << std::setw(8) << std::setfill('0') <<
      expr::attr< unsigned int >("LineID") << ": <" << logging::trivial::severity
      << "> " << expr::smessage );

      logging::core::get()->add_sink(sink); }
  \end{simplec}

  \subsection{\texttt{Boost.Format}-style formatters}
  Familiar with \texttt{Boost.Format} ?

  \begin{simplec}
    void init(){ typedef sinks::synchronous_sink< sinks::text_ostream_backend >
      text_sink; boost::shared_ptr< text_sink > sink = boost::make_shared<
      text_sink >();

      sink->locked_backend()->add_stream( boost::make_shared< std::ofstream
      >("sample.log"));

      // This makes the sink to write log records that look like this: // 1:
      <normal> A normal severity message // 2: <error> An error severity message
      sink->set_formatter ( expr::format("%1%: <%2%> %3%")
      % expr::attr< unsigned int >("LineID") logging::trivial::severity
      % expr::smessage
      );

      logging::core::get()->add_sink(sink); }
  \end{simplec}

  The \texttt{format} placeholder accepts the format string with
  \textbf{positional specification} of all args being formatted. The same format
  specification can be used with the \verb|add_file_log| and similar functions.

  \begin{tcolorbox}
    \emoji{turtle} : Note that \emph{only positional format is currently
      supported}.
  \end{tcolorbox}

  \subsection{Specialized formatters}

  The library provides specialized formatters for a number of types, such as
  \begin{itemize}
  \item data
  \item time, and
  \item named scope
  \end{itemize}

  These formatters provide \emph{control over formatted values}. For example, here
  we use a format string compatible with \texttt{Boost.DateTime}.

  \begin{simplec}
    void init(){ logging::add_file_log( keywords::file_name = "sample_%N.log",
      // This makes the sink to write log records that look like this: //
      YYYY-MM-DD HH:MI:SS: <normal> A normal severity message // YYYY-MM-DD
      HH:MI:SS: <error> An error severity message keywords::format = (expr::stream
      << expr::format_date_time< boost::posix_time::ptime >("TimeStamp",
      "%Y-%m-%d %H:%M:%S")
      << ": <" << logging::trivial::severity << "> " << expr::smessage)); }
  \end{simplec}

  \subsection{String templates as formatters}
  \begin{simplec}
    void init(){ logging::add_file_log ( keywords::file_name = "sample_%N.log",
      keywords::format = "[%TimeStamp%]: %Message%"
      ); }
  \end{simplec}

  The template may contain a number of placeholders enclosed with percent signs
  (\verb|%|). Each placeholder must contain an attribute value name to insert
  instead of the placeholder. The \verb|%Message%| placeholder will be replaced
  with the logging record message.

  \begin{tcolorbox}
    \emoji{parrot} : Can \cola{textual format templates} be accepted by sink
    backends in
    the \verb|set_formatter| method? \\
    \emoji{turtle} : No. In order to parse textual template into a formatter
    function one has to call \verb|parse_formatter()|.
  \end{tcolorbox}

  \subsection{Custom formatting functions}

  You can add a custom formatter to a sink backend that supports formatting. The
  formatter is actually a function object that supports the following signature:
  \begin{simplec}
    void (logging::record_view const& rec, logging::basic_formatting_ostream<
    CharT >& strm);
  \end{simplec}
  Here \texttt{CharT} is the target character type. The \cola{formatter} will be
  invoked whenever a \cola{log record view \texttt{rec}} \textbf{passes filtering}
  and is to be stored in log.

  \begin{tcolorbox}
    \emoji{parrot} : I guess the \texttt{record\_view} is like a read-only
    \texttt{record} ? Like \verb|string_view| and \verb|string|...\\
    \emoji{turtle} : Yeah.
  \end{tcolorbox}

  \begin{simplec}
    void my_formatter(logging::record_view const& rec,
    logging::formatting_ostream& strm){ // Get the LineID attribute value and put
      it into the stream strm << logging::extract< unsigned int >("LineID", rec)
      << ": ";

      // The same for the severity level. // The simplified syntax is possible if
      attribute keywords are used. strm << "<" << rec[logging::trivial::severity]
      << "> ";

      // Finally, put the record message to the stream strm <<
      rec[expr::smessage]; }

    void init(){ // ... sink->set_formatter(&my_formatter);
      logging::core::get()->add_sink(sink); }
  \end{simplec}

  \section{Filtering revisited}
  Two sinks in action:
  \begin{simplec}
    BOOST_LOG_ATTRIBUTE_KEYWORD(line_id, "LineID", unsigned int)
    BOOST_LOG_ATTRIBUTE_KEYWORD(severity, "Severity", severity_level)
    BOOST_LOG_ATTRIBUTE_KEYWORD(tag_attr, "Tag", std::string)

    void init(){ // Setup the common formatter for all sinks logging::formatter
      fmt = expr::stream << ": <" << severity << ">\t" <<
      expr::if_(expr::has_attr(tag_attr)) [expr::stream << "[" << tag_attr << "]
      "] << expr::smessage;

      // Initialize sinks typedef sinks::synchronous_sink<
      sinks::text_ostream_backend > text_sink;

      boost::shared_ptr< text_sink > sink = boost::make_shared< text_sink >(); //
      sink 1 sink->locked_backend()->add_stream(boost::make_shared< std::ofstream
      >("full.log")); sink->set_formatter(fmt);
      logging::core::get()->add_sink(sink); // add sink 1

      sink = boost::make_shared< text_sink >(); // sink 2
      sink->locked_backend()->add_stream(boost::make_shared< std::ofstream
      >("important.log")); sink->set_formatter(fmt); sink->set_filter(severity >=
      warning || (expr::has_attr(tag_attr) && tag_attr == "IMPORTANT_MESSAGE"));
      // when the sink 2 is streamed to

      logging::core::get()->add_sink(sink); // add sink 2
      logging::add_common_attributes(); // Add attributes }
  \end{simplec}

  The formatter itself contains filter here. The \verb|has_attr| predicate checks
  whether the record contains the \cola{``Tag'' attribute value} and controls
  whether it is put into the file or not.

  \begin{tcolorbox}
    \emoji{parrot} : So in fact we can just make two sinks, each with a
    \cola{dedicated formatter} that will only print under certain conditions. In
    that case, we
    don't need a filter anymore right?\\
    \emoji{turtle} : Don't be too smart. Use filter.
  \end{tcolorbox}

  We used attribute keyword to specify the name and type of the attribute for the
  predicate, but it is also possible to specify them in the \verb|has_attr()|.

  \begin{tcolorbox}
    \emoji{turtle} : The filter syntex resembles C++ very much when attribute
    keywords are used. The author seems to be quite proud of it.\\
    \emoji{parrot} : So \cola{formatter} is a function. Is \colb{filter} so?\\
    \emoji{turtle} : Yes. Here:
  \end{tcolorbox}

  \begin{simplec}
    bool (logging::attribute_value_set const& attrs); // returns whether the
    message (record) will pass
  \end{simplec}

  \emoji{parrot} : There's an example using \texttt{Boost.Phoenix} (a $\lambda$
  library) to write filters, but I am gonna leave that.

  \section{Wide character logging}

  There are two ways:
  \begin{itemize}
  \item On \textbf{UNIX}, typically fome multibyte character encoding (e.g. UTF-8)
    is used to represent national characters. In this case \textbf{no additional
      setup is required}.
  \item On \textbf{Windows} the common practice is to use \cola{wide strings} to
    represent national characters. Also, most of the system API is \cola{wide
      character oriented}, which requires \textbf{Windows-specific sinks} to also
    support wide strings.
  \end{itemize}
  On the other hand, generic sinks, like the \cola{text file sink}, are
  byte-oriented (\emoji{turtle} : because, you write bytes to files.) This forces
  the library to perform \cola{character code conversion} when needed by the sink.

  To set up the library for this, one has to \cola{imbue the sink with a locale
    with appropriate \texttt{codecvt} facet}.
  \begin{tcolorbox}
    \emoji{parrot} : I've never heard of that...\Cola{codecvt}?\\
    \emoji{turtle} : \texttt{Boost.Locale} can help.\\
    \emoji{parrot} : What is a \textbf{local}?\\
    \emoji{turtle} : From what I am understanding, a local is like a \colb{region}
    plus an \colc{encoding}, such as \texttt{en\_US.UTF-8}, or
    \texttt{ru\_RU.UTF-8}. \\ Officially, \texttt{Boost.Locale} uses the POSIX
    naming convention for locales, i.e. a locale is defined as:
    \begin{center}
      \verb|language[_COUNTRY][.encoding][@variant]|
    \end{center}
    variant is additional options for specializing the locale, like \texttt{euro}
    or \texttt{calendar=hebrew}.

    \emoji{turtle} : Here's an example to set the global locale.
    \begin{simplec}
      #include <boost/locale.hpp> using namespace boost::locale; int main(){
        generator gen; // Create locale generator std::locale::global(gen("")); //
        "" - the system default locale, set // it globally }
    \end{simplec}

    \emoji{turtle} : Or a locale can be \textbf{imbued} into a stream such as \texttt{std::cout}.\\
    \emoji{parrot} : It seems like C++ has some standard locale:
    \begin{simplec}
      cout.imbue(std::locale("en_US.UTF-8"));
    \end{simplec}
    So why \texttt{Boost.Locale} ?\\
    \emoji{turtle} : Because Boost said there're some critical problem with
    \texttt{std::locale} such as
    \begin{itemize}
    \item {non-standard locale name}
    \item {problematic global locale side effects}
    \item {broken number formatting}....
    \end{itemize}
  \end{tcolorbox}

  \begin{simplec}
    // Declare attribute keywords BOOST_LOG_ATTRIBUTE_KEYWORD(severity,
    "Severity", severity_level) BOOST_LOG_ATTRIBUTE_KEYWORD(timestamp,
    "TimeStamp", boost::posix_time::ptime)

    void init_logging(){ boost::shared_ptr< sinks::synchronous_sink<
      sinks::text_file_backend > > sink = logging::add_file_log( "sample.log",
      keywords::format = expr::stream << expr::format_date_time(timestamp,
      "%Y-%m-%d, %H:%M:%S.%f")
      << " <" << severity.or_default(normal) << "> " << expr::message // note the
      message not smessage anymore );

      // The sink will perform character code conversion as needed, according to
      the locale set with imbue() std::locale loc =
      boost::locale::generator()("en_US.UTF-8"); sink->imbue(loc);

      // Let's add some commonly used attributes, like timestamp and record
      counter. logging::add_common_attributes(); }
  \end{simplec}

  First let's take a look at the formatter we pass in the \texttt{format}
  parameter. We initialize the sink with a \emph{narrow-character formatter}
  because the \cola{text file sink} processes bytes. It is also possible to use
  wide strings in the formatter, but not in format strings, like the one we used
  with \verb|format_date_time()|.

  Also note that we used \texttt{message} keyword to denote the \cola{log record
    messages}. This placeholder supports \textbf{both} narrow and wide character
  messages. \cola{As part of the \textbf{formatting} process, the library will
    convert wide character messages tp multibyte encoding using the \textbf{imbued
      locale}, which we set to UTF-8.}

  \begin{tcolorbox}
    \emoji{turtle} : \cola{Attribute values} can also contain wide strings.
  \end{tcolorbox}

  \section{Core facilities}

  All the information that the logging library processes is packed into a single
  object of type \texttt{record}. All attached data, including the message text,
  is represented as named attribute values that can be fetched and processed by
  the \cola{filters, formatters and sinks}.

  Particular \cola{attribute values} can be accessed in different ways, here are a
  few quick examples:

  \begin{itemize}
  \item Through \texttt{visit()} and \texttt{extract()}:
    \begin{simplec}
      enum A { ... }; std::ostream& operator<< (std::ostream& strm, A level);

      logging::record const& rec = ...; // Prints severity level through visit()
      API logging::visit< A >("my_attr", rec, [](A a){cout << a << endl;});

      // Prints severity level through extract() API logging::value_ref< A > level
      = logging::extract< A >("my_attr", rec); cout << level << endl;
    \end{simplec}
  \item By searching the \cola{set of attribute values} with the
    \texttt{attribute\_values()}
    \begin{simplec}
      log::attribute_value_set const& values = rec.attribute_values();
      log::attribute_value_set::const_iterator it = values.find("my_attr"); if (it
      != values.end()){ log::attribute_value const& value = it->second;

        // A single attribute value can also be visited or extracted std::cout <<
        value.extract< A >() << std::endl; }
    \end{simplec}

  \item By applying the \cola{subscript operator} with the \Cola{attribute
      keyword}. \\
    (\emoji{turtle} : This is actually a convinience wrapper around the value
    extraction API.)
    \begin{simplec}
      BOOST_LOG_ATTRIBUTE_KEYWORD(my_attr, "My Attribute", A) // Use the attribute
      keyword to communicate the name and type of the value logging::value_ref< A,
      tag::my_attr> a = rec[my_attr]; std::cout << a << std::endl;
    \end{simplec}
  \end{itemize}
  Log records cannot be copied, only moved. (\emoji{parrot} : like
  \verb|unique_ptr|). A record can be \cola{default-constructed} in which case it
  is in an empty state; \textbf{such records should not be passed to the library
    for processing}. (\emoji{turtle} : But you can practice with it yourself.).

  \textbf{Non-empty} log records can only be created by the \cola{logging core} as
  a \colb{result of successful filtering}. The non-empty record contains
  \cola{attribute values acquired from attributes.} More attribute values can be
  added \textbf{after filtering}. The added values will not affect filtering
  results but can still be used by \cola{formatters} and \cola{sinks}.
\end{comment}

\subsection{Record views}
\begin{simplec}
  #include <boost/log/core/record_view.hpp>
\end{simplec}
Record view is copyable and immutable. The library will automatically create a
recor view by calling the \texttt{lock()} method. The call will also make sure
the resulting view is not attached to the current thread if a sink is
asynchronous. The \texttt{lock()} call is a one time operation; \textbf{the
  record is left in the \cola{empty state} afterwards}.

All APIs for interacting with \cola{attribute values} described for
\texttt{record} are also applicable to \texttt{records\_view}s and can be used
in custom formatters and sinks.

\subsection{Logging core}
\begin{simplec}
  #include <boost/log/core/core.hpp>
\end{simplec}
The \Cola{logging core} is a central hub that provides the following facilities:
\begin{enumerate}
\item Maintains \textbf{global} and \textbf{thread-specific} attribute sets.
\item Perform \cola{global filtering}
\item Dispatches \texttt{record}s between \cola{sinks} by applying
  \Cola{sink-specific} filters.
\item \colz{Provides a global hook for exception handlers.}
\item Provides entry point for log \texttt{source}s to put \texttt{record}s to
\item Provides the \texttt{flush()} method that can be used to enforce the
  synchronized state for all \texttt{sink}s.
\end{enumerate}

There's only one core:
\begin{simplec}
  boost::shared_ptr< logging::core > core = logging::core::get();
\end{simplec}

\subsubsection{Attribute sets}

\begin{tcolorbox}
  \emoji{parrot} : So how to add \cola{global or thread-specific attribute}?\\
  \emoji{turtle} : There're 4 methods to use: \verb!(get|remove)_(global|thread)_attribute()!
\end{tcolorbox}
Attribute sets is like \texttt{std::map}, so the \verb|add_*| methods
\cola{accept an attribute name string (key) and a pointer to the
  attribute(mapped value)} and return \colb{a pair of iterator and
  \texttt{bool}}. The \verb|remove_*| methods accept an iterator to a previously
added attribute.
\begin{simplec}
  boost::shared_ptr< logging::core > core = logging::core::get();
  // Add a global attribute
  std::pair< logging::attribute_set::iterator, bool > res =
  core->add_global_attribute("LineID", attrs::counter< unsigned int >());
  // ...
  // Remove the added attribute, use the iterator to access
  core->remove_global_attribute(res.first);
\end{simplec}

\subsubsection{Global filtering}
We talk about filter more later, here it will suffice to say that the filter
accepts \cola{a set of attribute values} and returns a \texttt{bool} (passed or
not). 
\begin{simplec}
  // Set a global filter so that only error messages are logged
  enum A {v1,v2,v3};
  core->set_filter(expr::attr< A >("My Attr") >= v1);
  core->set_logging_enabled(false); //disable it
  // ...
  core->set_logging_enabled(true); // re-enable it
\end{simplec}

\subsubsection{Sink management}
\begin{simplec}
  core->add_sink(sink);
  // ...
  // Remove the sink
  core->remove_sink(sink);
\end{simplec}

\subsubsection{Exception handler}
Too advanced.

\subsubsection{Feeding log records}
One of the most important functions of \texttt{core} is providing an entry point
for all \texttt{sources} to feed log records into. So there're
\verb-(open|close)_record- methods.

The first method is used to initiate the \cola{record logging process}. It
accepts the \cola{source-specific set of attributes.} The method constructs a
common set of attribute values of three sets of attributes:
\begin{enumerate}
\item global
\item thread-specific
\item source-specific
\end{enumerate}
and applies \textbf{filtering}, which checks if \cola{there's at least one sink
  accepts a record with these attribute values}. 

If so, it returns a \Cola{non-empty record object}, which can be used to fill in the \textbf{log
  record message}. \\
Else, \Cola{empty record object} is returned.

When the \texttt{source} is ready to complete the logging procedure, it has to
call the \texttt{push\_record()} method. (\emoji{turtle} : Don't call it on
empty record). The record should be passed as as \texttt{rvalue} reference.
During the call the \verb-record_view- will be constructed and passed to the
\texttt{sink}s during filtering. This may involve \cola{formatting} and furthur
processing, like storing it into a file or sending it over the network. After
that the \texttt{record} is destroyed. 
\begin{simplec}
  logging::attribute_set const& attrs = ...;
  // Attempt to open a log record
  logging::record rec = core->open_record(attrs);
  if (rec){
    // Ok, the record is accepted. Compose the message now.
    logging::record_ostream strm(rec);
    strm << "Hello, World!";
    strm.flush();

    // Deliver the record to the sinks.
    core->push_record(boost::move(rec));
  }
\end{simplec}

\section{Logging sources}

The simplest logging source is \texttt{boost::log::sources::logger}.
\begin{tcolorbox}
  \emoji{parrot} : I included \texttt{<boost/log/sources/basic\_logger.hpp>}, but
  the logger is not there?\\
  \emoji{turtle} : Include \texttt{<boost/log/sources/logger.hpp>}
\end{tcolorbox}
Here an example:
\begin{simplec}
  class network_connection{
  src::logger lg;
  logging::attribute_set::iterator a;

public:
  void on_connected(std::string const& remote_addr){
    // Put the remote address into the logger to automatically attach it
    // to every log record written through the logger
    a = lg.add_attribute("RemoteAddress",
                                           attrs::constant< std::string >(remote_addr)).first;

    // The straightforward way of logging
    if (logging::record rec = lg.open_record()){
        rec.attribute_values().insert("Message",
                                      attrs::make_attribute_value(std::string("Connection established")));
        lg.push_record(boost::move(rec));
      }
  }
  void on_disconnected(){
    // The simpler way of logging: the above "if" condition is wrapped into a neat macro
    BOOST_LOG(lg) << "Connection shut down";

    // Remove the attribute with the remote address
    lg.remove_attribute(a);
  }
  void on_data_received(std::size_t size){
    // Put the size as an additional attribute
    // so it can be collected and accumulated later if needed.
    // The attribute will be attached only to this log record.
    BOOST_LOG(lg) << logging::add_value("ReceivedSize", size) << "Some data received";
  }
  void on_data_sent(std::size_t size){
    BOOST_LOG(lg) << logging::add_value("SentSize", size) << "Some data sent";
  }
};
\end{simplec}

\emoji{parrot} : ?? The I can't find \texttt{logging::add\_value()}...Fine...

\section{Loggers with channel support}

Sometimes it is important to associate log records with some application
component, such as the
\begin{itemize}
\item module 
\item or class name
\item the relation of the \cola{logged information} to some specific domain of
  appl functionality.
\item or some arbitraty tag that can be used later to route these records to a
  specific sink. This feature is fullfilled with loggers
  \verb|[w_]channel_logger[_mt]|
\end{itemize}
These loggers \textbf{automatically register an attribute named \cola{``Channel''}}
\begin{simplec}
  #include <boost/log/sources/channel_feature.hpp>
  #include <boost/log/sources/channel_logger.hpp>

  src::channel_logger< > m_net(keywords::channel = "net");
\end{simplec}

\section{Lambda expressions}
\begin{simplec}
  #include <boost/log/expressions/attr_fwd.hpp>
  #include <boost/log/expressions/attr.hpp>
\end{simplec}
The \texttt{attr} placeholder represents an attribute value in template
expressions. Given the record view or a set of attribute values, the placeholder
will attempt to extract the specified value from the argument upon invocation.
This can be roughly described with the following pseudo-code:

\begin{simplec}
  logging::value_ref< T, TagT > val = expr::attr< T, TagT >(name)(rec);
\end{simplec}

where \texttt{val} is the \cola{reference} to the \textbf{extracted value},
\texttt{name} and \texttt{T} are the \cola{attribute value name} and
\cola{type}, \texttt{TagT} is an optional tag and \texttt{rec} is the log
\texttt{record view} or \texttt{attribute value set}.

\subsection{Customizing fallback policy}

By default, when the \cola{requested attribute value} is not found,
\texttt{attr} will return an empty reference.
\begin{itemize}
\item In case of filters, this result in \texttt{false} in any ordering
  expressions.
\item in case of formatters the output from the placeholder will be empty. This
  behavior can be changed: 
\end{itemize}

\begin{enumerate}
\item to throw
  \begin{simplec}
    sink->set_filter(
      expr::attr< int >("Severity").or_throw() >= 5 &&
      expr::attr< std::string >("Channel").or_throw() == "net"
    );
  \end{simplec}
\item to use a default
  \begin{simplec}
    sink->set_filter(
      expr::attr< int >("Severity").or_default(0) >= 5 &&
      expr::attr< std::string >("Channel").or_default(std::string("general")) == "net"
    );
  \end{simplec}
\end{enumerate}
\begin{tcolorbox}
  \emoji{turtle} : Remember you always have \texttt{has\_attr}
\end{tcolorbox}
The default behavior is actually \texttt{or\_none}.

\subsection{Attribute tags and custom formatting operators}

\begin{tcolorbox}
  \emoji{parrot} : Although this section is a bit too advanced, there's in fact
  a new function I found handy: \verb|add_console_log()|
\end{tcolorbox}
The \texttt{TagT} type above defaults to \texttt{void}. \Cola{This is an
  attribute tag which can be used to customize the output formatters produce for
different attributes}. This tag is forwarded to the \verb|to_log()| manipulator
when the extracted attribute value is put to a stream:

\subsection{Defining attribute keywords}
\begin{simplec}
  #include <boost/log/expressions/keyword_fwd.hpp>
  #include <boost/log/expressions/keyword.hpp> 
\end{simplec}
Attribute keywords can be used as replacements for the \texttt{attr}
placeholders in filters and formatters. Use:
\begin{simplec}
  BOOST_LOG_ATTRIBUTE_KEYWORD(keyword, "Keyword", type)
\end{simplec}
\begin{tcolorbox}
  \emoji{parrot} : We have used this a lot.
\end{tcolorbox}

Here the macro declares a keyword \texttt{keyword} for an attribute named
``Keyword'' with the value type of \texttt{type}. Additionally, the macro
\cola{defines an attribute tag type \texttt{keyword} within the \texttt{tag}
  namespace}.

\begin{tcolorbox}
  \emoji{turtle} : So we can use \verb|logging::to_log_manip< A, tag::my_attr > const& manip|
  when defining the custom formatter.
\end{tcolorbox}

Attribute keywords behave the same way as the \texttt{attr} placeholders and can
be used both in \cola{filters and formatters}. The  \verb|or_throw| and
\verb|or_default| modifiers are also supported.

\texttt{rec[\cola{<keyword>}]} is also OK:
\begin{simplec}
  void print_severity(logging::record_view const& rec){
    logging::value_ref< severity_level, tag::severity > level = rec[severity];
    std::cout << level << std::endl;
}
\end{simplec}

\section{Predicate expressions}

\subsection{Attribute presence filter}
\begin{simplec}
  #include <boost/log/expressions/predicates/has_attr.hpp>
\end{simplec}
The filter \verb|has_attr| checks if an attribute value with the \cola{specified name}
and , optionally, \cola{type} is attached to a log record.

If no type specified to the filter: return \texttt{true} if any value with the
specified name is found.

This filter is usually used in conjection with \texttt{conditional formatters},
but it also can be used as a \cola{quick filter} based on the \cola{log record
  structure}. 

\subsection{Attribute names}
\begin{simplec}
  #include <boost/log/attributes/attribute_name.hpp>
\end{simplec}

\end{document}

% Local Variables:
% TeX-engine: luatex
% TeX-command-extra-options: "-shell-escape"
% End:

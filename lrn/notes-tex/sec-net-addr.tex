


\section{IPv6 Jargen}

\begin{itemize}
\item IP : \colz{Internet Protocol Version 6.}
\item ICMP : \colz{Internet Control Message Protocol for IPv6. (e.g. ping)}
\item node : \colz{a device that implement IP.}
\item router: \colz{a node that forwards IP packets not explicitly addressed to
    itself.}
\item host: \colz{a node that is not a router.}
\item upper layer: \colz{any layer above IP. Examples are TCP , UDP, control
    protocols (ICMP)}
\item link : \colz{a communication facility or medium over which nodes can
    communicate at the link layer, i.e. the layer immediately below IP. Examples
    are } Ethernet, 802.11 (Wifi).
\item interface : \colz{a node's attachment to a link}
\item neighbors : \colz{nodes attached to the same link.}
\item address : \colz{an IP-layer identifier for an interface or a set of
    interfaces.}
\item anycast address : \colz{an identifier for a set of interfaces (typically
    belonging to different nodes). \colZ{A packet sent to an anycast address is
      delivered to one of the interfaces identified by that address.
    }

    An anycast address is syntactically indistinguishable from a unicast
    address. So nodes sending packets to anycast addresses don't generally known
    that it is anycast.
  }
\item prefix : \colz{a bit string that consists of some number of intial bits
    of an address.}
\item link-layer address : \colz{a link-layer identifier for an interface. For
    example, IEEE 802 addresses (such as MAC addresses)}
\item on-link, off-link : \colz{an address is \colZ{on-link}, if it is assigned to an
    interface on a specified link.

    A node considers an address to be \colZ{on-link} if:

    \begin{enumerate}
    \item it is covered by one of the \colZ{link's prefixes},(e.g., as indicated
      by the on-link flag in the Prefix Information option) or
    \item a neighboring router specifies the address as the target of a \cola{Redirect
        message}, or
    \item a \cola{Neighbor Advertisement message} is received for the address, or
    \item any \cola{Neibor Discovery} message is received from the address.
    \end{enumerate}

    \colZ{If an address is not \cola{on-link}, it is \cola{off-link}.} It means
    that it is not assigned to any interfaces on the specified link.
  }
\end{itemize}

\dSay{
  This first one is a bit confusing. What's \cola{Prefix Information option} ?
}

\cSay{
  I am not sure either. Let's just move on.
}

\begin{itemize}
\item longest prefix match : \colz{  the process of determining which prefix (if any) in
  a set of prefixes covers a target address. }
\item reachablility : whether or not the one-way ``forward'' path to a neighbor
  is functioning properly. \colz{ It's kinda like ``whether IP is working for a
    node''.}
\item packet : \colz{an IP header plus payload.}
\item link MTU : \colz{the maximum transmission unit, i.e. the maximum packet
    size in octets that can be transmitted on a link.}
\item target : \colz{an address about which address resolution information is
    sought, or an address that is the new first hop when being redirected.
  }
\item proxy : a node that responds to Neighbor Discovery query messages on
  behalf of another node. 
\end{itemize}

\section{Addresses}

% 2.4 Address Type Representation

%    The specific type of an IPv6 address is indicated by the leading bits
%    in the address.  The variable-length field comprising these leading
%    bits is called the Format Prefix (FP).  The initial allocation of
%    these prefixes is as follows:

%     Allocation                            Prefix         Fraction of
%                                           (binary)       Address Space
%     -----------------------------------   --------       -------------
%     Reserved                              0000 0000      1/256
%     Unassigned                            0000 0001      1/256

%     Reserved for NSAP Allocation          0000 001       1/128
%     Reserved for IPX Allocation           0000 010       1/128

%     Unassigned                            0000 011       1/128
%     Unassigned                            0000 1         1/32
%     Unassigned                            0001           1/16

%     Aggregatable Global Unicast Addresses 001            1/8
%     Unassigned                            010            1/8
%     Unassigned                            011            1/8
%     Unassigned                            100            1/8
%     Unassigned                            101            1/8
%     Unassigned                            110            1/8

%     Unassigned                            1110           1/16
%     Unassigned                            1111 0         1/32
%     Unassigned                            1111 10        1/64
%     Unassigned                            1111 110       1/128
%     Unassigned                            1111 1110 0    1/512

%     Link-Local Unicast Addresses          1111 1110 10   1/1024
%     Site-Local Unicast Addresses          1111 1110 11   1/1024

%     Multicast Addresses                   1111 1111      1/256

\subsection{Address type representation}

The initial bits of an address form the \Cola{Format Prefix (FP)}. This tells us
the type of the address. The initial allocation of these prefixes is as follows
in \cref{tab:fp}:

\begin{table}[h]
  \centering
  \begin{tabularx}{1.0\linewidth}{XXX}
    \cola{Allocation} & \cola{Prefix} & \cola{Fraction of Address Space} \\
    \hline
    Reserved & \texttt{0000 0000 (00)} & 1/256\\
    Unassigned & \texttt{0000 0001 (01)} & 1/256\\
    Reserved for NSAP Allocation & \texttt{0000 001 (02)} & 1/128\\
    Reserved for IPX Allocation & \texttt{0000 010 (04)} & 1/128\\
    Link-Local Unicast Addresses & \texttt{1111 1110 10 (FE80)} & 1/1024\\
    Site-Local Unicast Addresses & \texttt{1111 1110 11 (FEC0)} & 1/1024\\
    Multicast Addresses & \texttt{1111 1111 (FF)} & 1/256\\
    Global-unicast Addresses & \texttt{001 (2)} & 1/8 \\
  \end{tabularx}
  \caption{The Format Prefix}
  \label{tab:fp}
\end{table}

In addition to the above, the prefixes \texttt{
  010, 011, 100, 101, 110, 1110, 1111 0, 1111 10, 1111 110, 1111 1110 0} are
\cola{unassigned} and reserved for future use.


\clearpage{}

\begin{itemize}
\item link-local address : \colz{a unicast address having link-only scope that
    can be used to reach neighbors.}
\item unspecified address : \colz{the address with all zero bits:
    \texttt{0:0:0:0:0:0:0:0},
    which kinda means ``I don't known my address''. It can be the source address.
  }
\item all-nodes multicast address : \colz{the link-local address to reach all
    nodes, \texttt{FF02::1}.}
\item all-routers multicast address : \colz{the link-local address to reach all
    routers, \texttt{FF02::2}.}
\end{itemize}

\subsection{special addresses}

There are two types of special addresses related to IPv4:
\begin{enumerate}
\item \cola{IPv4-compatible IPv6 addresses} : \texttt{<80-bits of 0s> + <16
    bits of 0s> + <32 bits of IPv4 address>}.
\item \cola{IPv4-mapped IPv6 addresses} : \texttt{<80-bits of 0s> + <16 bits of
    1s> + <32 bits of IPv4 address>}. (These represet nodes that do not support IPv6.)
\end{enumerate}

\colz{
  Addresses endded with trailing zeros are \colZ{anycast addresses}.
}

\colz{
  \cola{Multicast addresses} are identified by the leading bits \texttt{1111
    1111}.

  They have the following sctructure:

  \begin{center}
    \begin{tabularx}{0.8\linewidth}{X|X|X|X}
      \texttt{ff} & \texttt{flgs} & \texttt{scope} & \texttt{group ID} \\
      \hline
      8 bits & 4 bits & 4 bits & 112 bits \\
    \end{tabularx}
  \end{center}

  The \texttt{flgs} has the form \texttt{000T}. Where:
  \[
    \mathtt{T} = \begin{cases}
      1 & \text{transient} \\
                     0 & \text{``well-known'', the address is assigned by IEFT} \\
    \end{cases}
  \]
}

Some well-knwon multicast addresses are listed in \cref{tab:multicast}.

\begin{table}[h]
  \centering
  \begin{tabularx}{1.0\linewidth}{XX}
    Address & Description \\
    \hline
    \texttt{FF02::1} & All Nodes Address  \\
    \texttt{FF02::2} & All Routers Address  \\
    \texttt{FF02:0:0:0:0:1:FFXX:XXXX} & Solicited-Node Address \\
  \end{tabularx}
  \caption{Well-known multicast addresses}
  \label{tab:multicast}
\end{table}


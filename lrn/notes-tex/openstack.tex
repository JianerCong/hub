\documentclass[dvipsnames]{article}
% \documentclass[dvipsnames]{ctexart}

\title{Openstack}

\usepackage{geometry}\geometry{
a4paper,
total={170mm,257mm},
left=20mm,
top=20mm,
}


\usepackage[inkscapearea=page]{svg}
% 🐢 : I guess what we need is to just include the page... not the whole
% drawing... So set inkscapearea=page should work... If not, delete the
% .\svg-inkscape and try again. (this pkg does some magical caching..)
\usepackage[skip=5pt plus1pt, indent=0pt]{parskip}
% Color
\newcommand{\mycola}{MidnightBlue}
\newcommand{\mycolb}{Mahogany}
\newcommand{\mycolc}{OliveGreen}

\newcommand{\mycoli}{\mycolb}
\newcommand{\mycolii}{\mycola!30!\mycolb}
\newcommand{\mycoliii}{\mycola!70!\mycolb}
\newcommand{\mycoliv}{\mycola!90!\mycolb}
\newcommand{\mycolv}{\mycola}

\newcommand{\cola}[1]{\textcolor{\mycola}{#1}}
\newcommand{\colb}[1]{\textcolor{\mycolb}{#1}}
\newcommand{\colc}[1]{\textcolor{\mycolc}{#1}}
\newcommand{\Cola}[1]{\textcolor{\mycola}{\textbf{#1}}}

\newcommand{\coli}[1]{\textcolor{\mycoli}{#1}}
\newcommand{\colii}[1]{\textcolor{\mycolii}{#1}}
\newcommand{\coliii}[1]{\textcolor{\mycoliii}{#1}}
\newcommand{\coliv}[1]{\textcolor{\mycoliv}{#1}}
\newcommand{\colv}[1]{\textcolor{\mycolv}{#1}}

% 🦜 : \textcolor doesn't allow multiple paragraphs in it, so we used {\color{...}}
\newcommand{\colZ}[1]{
{\color{black}#1}
} %go back
\newcommand{\colz}[1]{
{\color{gray}#1}
}

% \let\emph\relax % there's no \RedeclareTextFontCommand
% \DeclareTextFontCommand{\emph}{\bfseries}
\renewcommand{\emph}[1]{\texbf{#1}}
\usepackage{amssymb}            %\mathbb

\usepackage{fontspec}

\setmonofont{Cascadia}[
Path=/usr/share/fonts/truetype/Cascadia_Code/,
Scale=0.85,
Extension = .ttf,
UprightFont=*Code,              %find CascadiaCode.ttf
BoldFont=*CodePL,               %find CascadiaCodePL.ttf ...
ItalicFont=*CodeItalic,
BoldItalicFont=*CodePLItalic
]
% --------------------------------------------------
% Windows
% \setmonofont{Cascadia}[
% Path=C:/Windows/Fonts/,
% Extension = .ttf,
% UprightFont=*Mono,              %find CascadiaMono.ttf
% BoldFont=*Code,               %find CascadiaCodePL.ttf ...
% ItalicFont=*Code,
% BoldItalicFont=*Code
% ]


\usepackage{minted}
\usepackage{tcolorbox}
\tcbuselibrary{skins}
\tcbuselibrary{minted}
\usepackage{tikz}
\usetikzlibrary{shapes} % ellipse node shape
\usetikzlibrary{shapes.multipart} % for line breaks in node text
\usetikzlibrary{arrows.meta}    %-o arrow head
\usetikzlibrary{arrows}
\usetikzlibrary{matrix}
\usetikzlibrary{snakes}

\usepackage{amsmath}
% ??? still xelatex?
% \usepackage{xeCJK}
\usepackage{emoji}
% \setemojifont{NotoColorEmoji.ttf}[Path=C:/Users/congj/repo/myFonts/]
% \setemojifont{TwitterColorEmoji-SVGinOT.ttf}[Path=C:/Users/congj/repo/myFonts/]


\newtcolorbox[auto counter]{myBox}[2][]{
fonttitle=\bfseries,title={共识~\thetcbcounter: #2},#1
}
\newtcolorbox[]{noteBox}[1][]{
tile,left=1mm,nobeforeafter,fontupper=\small,#1
}

\tikzstyle{myNode}=[inner sep=2pt,circle,text=white]
\date{\today}
\author{作者}

\newtcblisting{simplec}{
listing engine=minted,
minted language=c++,
minted style=vs,
minted options={fontsize=\small,autogobble,
% framesep=1cm
},
tile,
listing only,
% bottom=0cm,
% nobeforeafter, 
boxsep=0mm,
left=1mm,
opacityback=0.5,
colback=gray!20
}
\tcbuselibrary{breakable}
\newtcblisting{simplepy}{
listing engine=minted,
minted language=python,
minted style=vs,
minted options={fontsize=\small,autogobble,
% framesep=1cm
},
tile,
listing only,
% bottom=0cm,
% nobeforeafter,
boxsep=0mm,
left=1mm,
opacityback=0.5,
colback=gray!20,
breakable
}
\newtcolorbox{blackbox}{tile,colback=black,colupper=white,nobeforeafter,halign=flush center}

% \tikzstyle{myMatrix}=[matrix of nodes,below right,
% nodes={above,text centered},                  %apply to all nodes
% row sep=1cm,column sep=2cm]
% \tikzstyle{every node}=[inner sep=0pt]

\newcommand\uptoleft[3][-o]{\draw[very thick,#1](#2.south) |- (#3.west);}
\newcommand\uptodown[3][-o]{\draw[very thick,#1](#2.south) to [out=270,in=90] (#3.north);}
\newcommand\downtoup[3][-latex]{\draw[very thick,#1](#2.north) to [out=90,in=270] (#3.south);}

\newcommand\lefttoright[3][-latex]{\draw[very thick,#1](#2.east) to[out=0,in=180] (#3.west);}
\newcommand\lefttodown[3][-latex]{\draw[very thick,#1](#2.east) to[out=0,in=90] (#3.north);}

\newcommand{\topDialogBoxColored}[4][0.8\linewidth]{
  \node[above,text width=#1] at ([yshift=0.5cm]#2.north) {
    \begin{tcolorbox}[tile,
      nobeforeafter,
      boxsep=0pt,
      % show bounding box,
      colback=#4!10,
      overlay={
        \begin{scope}
          \fill[#4!10] (frame.south) --
          +(2mm,0) --
          +(0,-3mm) --
          +(-2mm,0)
          ;
        \end{scope}
      } ] #3
    \end{tcolorbox}
  };
}

\newcommand{\rightDialogBoxColored}[4][0.8\linewidth]{
  \node[right,text width=#1] at ([xshift=0.5cm]#2.east) {
    \begin{tcolorbox}[tile,
      nobeforeafter,
      boxsep=0pt,
      % show bounding box,
      colback=#4!10,
      overlay={
        \begin{scope}
          % \fill[gray!10] (frame.east) circle (2pt);
          \fill[#4!10] (frame.west) --
          +(0,2mm) --
          +(-3mm,0) --
          +(0,-2mm)
          ;
        \end{scope}
      } ] #3
    \end{tcolorbox}
  };
}

\newcommand{\rightDialogBox}[3][0.8\linewidth]{
% \node[right,text width=#1] at ([xshift=0.5cm]#2.east) {
% \begin{tcolorbox}[tile,
% nobeforeafter,
% boxsep=0pt,
% % show bounding box,
% colback=\mycola!10,
% overlay={
% \begin{scope}
% % \fill[gray!10] (frame.east) circle (2pt);
% \fill[\mycola!10] (frame.west) --
% +(0,2mm) --
% +(-3mm,0) --
% +(0,-2mm)
% ;
% \end{scope}
% } ] #3
% \end{tcolorbox}
% };
  \rightDialogBoxColored[#1]{#2}{#3}{\mycola}
}


\newcommand{\leftDialogBoxColored}[4][0.8\linewidth]{
  \node[left,text width=#1] at ([xshift=-0.5cm]#2.west) {
    \begin{tcolorbox}[tile,
      nobeforeafter,
      boxsep=0pt,
      % show bounding box,
      colback=#4!10,
      overlay={
        \begin{scope}
          % \fill[gray!10] (frame.east) circle (2pt);
          \fill[#4!10] (frame.east) --
          +(0,2mm) --
          +(3mm,0) --
          +(0,-2mm)
          ;
        \end{scope}
      } ] #3
    \end{tcolorbox}
  };
}

\newcommand{\leftDialogBox}[3][0.8\linewidth]{
% \node[left,text width=#1] at ([xshift=-0.5cm]#2.west) {
% \begin{tcolorbox}[tile,
% nobeforeafter,
% boxsep=0pt,
% % show bounding box,
% colback=gray!10,
% overlay={
% \begin{scope}
% % \fill[gray!10] (frame.east) circle (2pt);
% \fill[gray!10] (frame.east) --
% +(0,2mm) --
% +(3mm,0) --
% +(0,-2mm)
% ;
% \end{scope}
% } ] #3
% \end{tcolorbox}
% };
  \leftDialogBoxColored[#1]{#2}{#3}{gray}
}




\newcommand{\mycolaa}{\mycola!20}

\newcommand{\colZt}[1]{
  \colZ{\texttt{#1}}
}

\usepackage{changepage}   % for the adjustwidth environment
\newenvironment{myIndent}[1][7mm]{\begin{adjustwidth}{#1}{}}{\end{adjustwidth}}

\newcounter{myDefCounter}
\newcounter{myTheoCounter}

\tcbuselibrary{theorems}
\newtcbtheorem[use counter=myDefCounter,number within=section]{myDef}{Definition}%
{
% colback=green!5,colframe=green!35!black,
fonttitle=\bfseries}{def}

\newcommand{\myThmBeforeProof}{
\texttt{Proof: }\par
\begin{myIndent}
}

\newcommand{\myThmAfterProof}{
\end{myIndent}
\qed
}

\tcbuselibrary{skins}
\tcbsubskin{myThmSkin}{enhanced}{
parbox=false,
% colback=green!5,
colframe=\mycola,
fonttitle=\bfseries,
breakable,
before lower={\myThmBeforeProof{}},
after lower={\myThmAfterProof{}}
}

\newtcbtheorem[
use counter=myTheoCounter,number within=section
]{myTheo}{Theorem}%
{
skin=myThmSkin
}{thm}


\newtcbtheorem[use counter=myTheoCounter,number within=section]{myCor}{Corollary}%
{
% colback=green!5,
skin=myThmSkin
}{cor}

\newtcbtheorem[use counter=myTheoCounter,number within=section]{myLem}{Lemma}%
{% colback=green!5,
skin=myThmSkin
}{lem}

\newcommand{\refCorollary}[1]{Corollary~\ref{cor:#1}}
\newcommand{\refLemma}[1]{Lemma~\ref{lem:#1}}

\usepackage{hyperref}

\usepackage{cleveref}           %🦜 : must be loaded after hyperref
\crefname{myDefCounter}{definition}{definitions}
% ^^^ plural
\Crefname{myDefCounter}{Definition}{Definitions}
% ^^^^^^ type = counter name

\crefname{myTheoCounter}{theorem}{theorems}
\Crefname{myTheoCounter}{Theorem}{Theorems}

\usepackage{amsthm}             %for {proof}

\usepackage{tabularx}



\usepackage{enumitem}
\setlist[description]{leftmargin=0.1\linewidth,labelindent=0.1\linewidth}
\usepackage{placeins}           %for \FloatBarrier
% 🦜 : Remember to specify the mode, because without a \documentclass{}, emacs
% will consider this as a plain TeX document.

% --------------------------------------------------
\newcommand{\ans}{
  % \newline{}
  \par
  \hspace{0.1\linewidth}
  % \phantom{Ans: }
  % \colz{Ans: }
  \tikz[baseline=.5em]{\draw[thick,\mycola] (0,0) -- (0.8\linewidth,0pt);}
  % \rule[-1em]{0.8\linewidth}{1pt}
}
\newcommand{\tf}{\Cola{\hfill[T/F]}}

\usepackage{tabularray}
\UseTblrLibrary{booktabs,siunitx}
% Local Variables:
% mode: LaTeX
% TeX-engine: luatex
% TeX-command-extra-options: "-shell-escape"
% TeX-master: "m.tex"
% TeX-parse-self: t
% TeX-auto-save: t
% End:

% svgs
\newcommand{\svgOs}[2][0.3\linewidth]{\includesvg[width=#1]{../../mysvgs/#2.svg}}
% \newcommand{\svgNeutron}[1][1.5cm]{\includesvg[width=#1]{/home/me/Pictures/opstk/neutron.svg}}
\newcommand{\svgD}[1][8mm]{\includesvg[width=#1]{/home/me/Pictures/d1.svg}}
\newcommand{\svgC}[1][8mm]{\includesvg[width=#1]{/home/me/Pictures/c1.svg}}

% params:
% 1: width of dialog box
% 2: the speaker
% 3: the text
\newcommand{\leftSay}[3][0.7\linewidth]{
  \begin{tikzpicture}
    \node (N1) {
      #2
    };
    \leftDialogBox[#1]{N1}{
      #3
    }
  \end{tikzpicture}
}

\newcommand{\rightSay}[3][0.7\linewidth]{
  \begin{tikzpicture}
    \node (N1) {
      #2
    };
    \rightDialogBox[#1]{N1}{
      #3
    }
  \end{tikzpicture}
}

% 🦜 : Let d1 and c1 say and ask .
\newcommand{\dSay}[1]{
  \begin{tikzpicture}
    \node (D1) {
      \svgD
    };
    \rightDialogBox[0.7\linewidth]{D1}{
      #1
    }
  \end{tikzpicture}
}


\newcommand{\cSay}[1]{
  \begin{flushright}
    \begin{tikzpicture}
      \node (C1) {
        \svgC
      };
      \leftDialogBox[0.7\linewidth]{C1}{
        #1
      }
    \end{tikzpicture}
  \end{flushright}
}


% --------------------------------------------------
\begin{document}
\maketitle

\section{Consideration}

\subsection{The logical architecture}

\cSay{ OpenStack uses a microservice architecture. So it's composed of many many services. }

About this services:

\begin{itemize}
\item Most of them of them are written in \cola{Python}.
\item All of them provide a \cola{REST API}.
\item Each service may be implemented as different components. The components
  communicate with each other using a \cola{message queue}.
\end{itemize}

\subsubsection{Keystone - identity management}

\cSay{ Keystone is the identity management service. It's the simplest service in OpenStack.}

It provides a REST API for authentication and authorization. It also provides a
service catalog, which is a list of all the services.

\dSay{ Why is it keeping it ?}

\cSay{ I think probably because it needs to know which services are accessible for a user anyway.}

\subsubsection{Swift - object storage}


\cSay{ Swift is the object storage service (an object store). To store an object, it splits the object into chunks and stores them in different containers redundantly.}

Some of the benefits of this design are:

\begin{itemize}
\item It has no central brain, and indicates no \Cola{Single Point of Failure (SPOF)}.
\item It has \cola{auto-revovery}.
\item It has inexpensive hardware that can be used for redundant storage clusters.
\end{itemize}

\dSay{ Wow, \cola{auto-recovey} is cool.}

\subsubsection{Cinder - block storage}

\dSay{ What if I just want a block of bytes ? For example, for the volume of a VM.}

\cSay{ Then you can use Cinder. It provides raw volumes that can be used as hand disks in VMs.}

Some of the features of Cinder are:

\begin{itemize}
\item You can create snapshots of volumes.
\item You can create backups of volumes.
\end{itemize}

Also, like Keystone, different storage backends can be plugged into Cinder such
as from IBM, NetApp, Nexenta, and VMware.

\dSay{ What are these ?}

\cSay{ I don't know. I think they are companies that provide storage....}

\begin{tcolorbox}
\cSay{ Cinder replaces the old \Cola{Nova Volume service}. }
\end{tcolorbox}

\subsubsection{Manila - file share (since \cola{Juno})}

\cSay{ Manila is the file share service. It provides a remote file system. In
  operation, it resembles the \Cola{Network File System (NFS)} or \Cola{SAMBA}
  storage service that you can find in Linux.}


\dSay{Oh, I know SAMBA. I remember it's how linux can access windows file.}
\cSay{Yes, it's a file sharing protocol.}


In contrast to Cinder, it resembles the \Cola{Storage Area Network (SAN)}
service. In fact, NFS and SAMBA or the \Cola{Common Internet File System (CIFS)}
are supported as backend drivers. Manila provides the orchestration of shares on
the share services.

\dSay{What do you mean by \cola{orchestration of shares} ?}

\cSay{ I think it means that it will take care of the shares when they are read and written by different services.}

\dSay{Oh, so what's the difference between Swift, Cinder and Manila ?}

\cSay{ Let's see a table:}

\begin{table}[htbp]
  \centering
  \begin{tabularx}{0.8\textwidth}{XXXX}
    \textbf{Specification} & \textbf{Swift} & \textbf{Cinder} & \textbf{Manila} \\
    \hline
    \textbf{Access mode} & REST API & Block device & File system \\
    \textbf{Multi-access} & \emoji{check-mark-button} & \emoji{cross-mark} & \emoji{check-mark-button} \\
    \textbf{Persistent} & \emoji{check-mark-button} & \emoji{check-mark-button} & \emoji{check-mark-button} \\
    \textbf{Accessibility} & Anywhere & Within single VM & Within multiple VMs \\
  \end{tabularx}
\end{table}

\subsubsection{Nova - compute service}
\leftSay{\svgOs{nova}}{My name is \Cola{Nova}. I am the compute service, the most complex service in OpenStack.}

\dSay{Wait... I remember you said that Nova is replaced by Cinder ?}

\cSay{ Only the \Cola{volume service }is replaced by Cinder. }

\dSay{Oh, I see. So how it's complex ?}

\cSay{ First, it needs to interact with many other services. Second, its internal components are complex, because it needs to respond to user requests for running VMs.}

Let's decompose it into its components:

\begin{description}
\item[nova-api] : It accepts and responds to user requests of creating instances
  (VMs) via the OpenStack API or the EC2 API.
\item[nova-compute] : It creates and terminates VMs via the hypervisor's APIs
  (XenAPI for XenServer, libvirt for KVM and the VMware API for VMware).
\item[nova-network] : It provides network connectivity for VMs. It's deprecated
  in favor of Neutron \emoji{litter-in-bin-sign}.
\item[nova-scheduler] : It schedules VMs to run on different compute nodes.
  (\emoji{parrot} : So here's where OpenStack find the best node to run a VM ?,
  \emoji{turtle} : Yes).
\item[nova-conductor] : It provides database access for the other components.
  (\emoji{parrot} : Why bother adding access control ? \emoji{turtle} : This is
  for security. If some nodes are compromised, we can't just let them access the
  database however they want).
\end{description}

\subsubsection{Neutron - networking services}


\rightSay{\svgOs{neutron}}{
  My name is \Cola{Nova}. I am the compute service, the most complex service in OpenStack.
  My name is \Cola{Neutron}.I provide a \Cola{Network
    as Service (NaaS)} capability between \cola{interface devices (routers,
    switches, load balancers, VPNs, firewalls)} managed by other OpenStack
  services (such as Nova) or external networks.
}

\cSay{ \Cola{Neutron} has the following features:}

\begin{itemize}
\item It allows users to create their own networks and attach interfaces to them.
\item Its backend is extensible, so it can take advantage of commodity gear or
  vandor-supported equipment.
\end{itemize}

Neutron has three main components:

\begin{description}
\item[server] : It accepts API requests and routes them to the appropriate
  \cola{plugin} for action.
\item[plugins] : They perform the actual work for the orchestration of backend devices such as
  \begin{itemize}
  \item plugging in or unplugging ports
  \item creating networks and subnets
  \item IP addressing.
  \end{itemize}
\item[agents] : They run on compute and network nodes. They receive commands
  from the plugins on the server and bring the changes into effect on the
  individual nodes. (\emoji{parrot}: Oh, so agent is like a client ?
  \emoji{turtle}: Yes).

  There're different types of agents. They provide different services. For example:
  \begin{description}
  \item[L3 agents] : They provide routing services.
  \item[Open vSwitch agents] : They provide layer 2 connectivity, by plugging
    and unplugging ports onto \Cola{Open vSwitch(OVS)} bridges.
  \end{description}
  \emoji{parrot}: What is OVS ? \emoji{turtle}: It's a virtual switch.

\end{description}


\section{Networking requirements}

\cSay{
  Due to the use of containers to isolate the OpenStack services, special network
  configuration is needed. In particular \Cola{OpenStack Ansible} uses
  \Cola{linux bridges} to provide network connectivity to the LXC containers.
}

\dSay{What is a linux bridge ?}

\cSay{It's a virtual switch that can be used to connect multiple network
  interfaces together. You can see what are on your system by running the command
  \texttt{ip link show}.
}

\end{document}

% Local Variables:
% TeX-engine: luatex
% TeX-command-extra-options: "-shell-escape"
% TeX-master: "m.tex"
% End:
\section{VLAN tagging}

\colz{

  A \Cola{Virtual Local Area Network (VLAN)} is a logical network within a
  physical network. The \Cola{VLAN interface} \colb{tags packets} with the
  \colZt{VLAN ID} as they pass through the interface, and \colb{removes tags} of
  returning packets. \colZ{You create VLAN interfaces on top of another
    interface}, such as Ethernet, bond, team, or bridge devices. These
  interfaces are called the \Cola{parent interface}.

}

\subsection{Configuring VLAN tagging by using \texttt{nmcli}}

You can configure VLAN tagging by using \texttt{nmcli}.

\subsubsection{Check if the interface supports it}

First, check if the interface supports VLAN tagging:

\colz{
  \begin{enumerate}
  \item If you configure the VLAN on top of a bond interface, check that:
    \begin{enumerate}
    \item The ports of the bond are up.
    \item The bond is not configured with the \colZt{fail\_over\_mac=follow} option. A VLAN
      virtual device cannot change its MAC address to match the parent’s new MAC
      address. In such a case, the traffic would still be sent with the incorrect
      source MAC address.
    \item The bond is usually not expected to get IP addresses from a DHCP
      server or IPv6 auto-configuration. Ensure it by setting the
      \colZt{ipv4.method=disable} and \colZt{ipv6.method=ignore} options while
      creating the bond. Otherwise, if DHCP or IPv6 auto-configuration fails
      after some time, the interface might be brought down.
    \end{enumerate}
  \item \colZt{The switch}, the host is connected to, is configured to support VLAN
    tags. For details, see the documentation of your switch.
  \end{enumerate} 
}

\subsubsection{Do}

\begin{enumerate}
\item Display the network interfaces
  \begin{simplesh}
    sudo nmcli device status
    # DEVICE   TYPE      STATE         CONNECTION
    # enp1s0   ethernet  disconnected  enp1s0
    # bridge0  bridge    connected     bridge0
    # bond0    bond      connected     bond0
  \end{simplesh}
\item Create the VLAN interface. For example, to create a VLAN interface named
  \texttt{vlan10} that uses \texttt{enp1s0} as its parent interface and that
  tags packets with \texttt{VLAN ID} 10, enter:
\begin{simplesh}
n=vlan10
id=10   # Note that the VLAN must be within the range from 0 to 4094.
sudo nmcli connection add \
     type vlan \
     con-name $n \
     ifname $n \
     vlan.parent enp1s0 \
     vlan.id $id                
\end{simplesh}
\item \colz{ By default, the VLAN connection inherits the maximum transmission
    unit (MTU) from the parent interface. \cola{Optionally}, set a different MTU value:
  }
\begin{simplesh}
     sudo nmcli connection modify $n ethernet.mtu 2000
\end{simplesh}
\end{enumerate}


\subsection{Configure the VLAN device}

\subsubsection{IPv4}
For IPv4, we can either:
\begin{enumerate}
\item assign a static IP address:
  \begin{simplesh}
    sudo nmcli connection modify $n ipv4.addresses '192.0.2.1/24' \
                ipv4.gateway '192.0.2.254' ipv4.dns '192.0.2.253' \
                ipv4.method manual
  \end{simplesh}
\item use DHCP: then do nothing
\item make this device a port of another devices....
  \begin{simplesh}
    sudo nmcli connection modify $n ipv4.method disabled
  \end{simplesh}
\end{enumerate}

\subsubsection{IPv6}

Same as IPv4, we also have three options:
\begin{enumerate}
\item assign a static IP address to the bond interface:
  \begin{simplesh}
     sudo nmcli connection modify $n ipv6.addresses '2001:db8:1::1/32' \
          ipv6.gateway '2001:db8:1::fffe' \
          ipv6.dns '2001:db8:1::fffd' \
          ipv6.method manual
\end{simplesh}
\item use DHCP: then do nothing
\item make this device a port of another devices....
\begin{simplesh}
    sudo nmcli connection modify $n ipv6.method disabled
\end{simplesh}
\end{enumerate}

\subsection{Activate and verify the VLAN}

\begin{simplesh}
  sudo nmcli connection up $n
  ip -d addr show $n
  #4: vlan10@enp1s0: <BROADCAST,MULTICAST,UP,LOWER_UP> mtu 1500 qdisc noqueue state UP group default qlen 1000
  #    link/ether 52:54:00:72:2f:6e brd ff:ff:ff:ff:ff:ff promiscuity 0
  #    vlan protocol 802.1Q id 10 <REORDER_HDR> numtxqueues 1 numrxqueues 1 gso_max_size 65536 gso_max_segs 65535
  #    inet 192.0.2.1/24 brd 192.0.2.255 scope global noprefixroute vlan10
  #       valid_lft forever preferred_lft forever
  #    inet6 2001:db8:1::1/32 scope global noprefixroute
  #       valid_lft forever preferred_lft forever
  #    inet6 fe80::8dd7:9030:6f8e:89e6/64 scope link noprefixroute
  #       valid_lft forever preferred_lft forever
\end{simplesh}

\cSay{
  We see that \texttt{vlan10}'s parent is \texttt{enp1s0}.
}
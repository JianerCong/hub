\section{dig}
\label{sec:dig}

\newcommand{\dg}{\texttt{dig}}

Unless specified otherwise, \dg{} tries each of the servers listed in
\texttt{/etc/resolv.conf}. Finally, it resorts to \texttt{localhost}.

When no serve is specified, \dg{} performs the \texttt{NS} query for \texttt{.}.

It is possible to set per-user defaults in \texttt{\$HOME/.digrc}. Option
\texttt{-r} disables this feature.

\subsection{simple usage}

A typical usage of \dg{} looks like:

\begin{center}
  \ttfamily
  dig \cola{@server} \colc{name} \colb{type}
\end{center}

Where:

\begin{itemize}
\item \texttt{\cola{server}}: \colz{ is the name or IP address of the name
    server to query, can be either IPv4 or IPv6.

    If unspecified, \dg{} tries those in \texttt{/etc/resolv.conf}.
  }
\item \texttt{\colc{name}} \colz{the name of RR to be looked up}
\item \texttt{\colb{type}} \colz{the type of RR to be looked up, such as
    \begin{itemize}
    \item \texttt{A}: IPv4 address
    \item \texttt{AAAA}: IPv6 address
    \item \texttt{CNAME}: canonical name
      ...
    \end{itemize}
  }
\end{itemize}

\dSay{
  Why my \texttt{/etc/resolv.conf} point to \texttt{127.0.0.53} ?
}

\cSay{
  Because you are using \texttt{systemd-resolved} as your DNS resolver, and it
  listens on does some caching and altered the \texttt{/etc/resolv.conf} to
  accomplish that.
}

\subsection{options}
Some useful options are:

\begin{itemize}
\item \texttt{-4,-6}: \colz{ force \dg{} to use IPv4/6 only.}
\item \texttt{-b addr[\#port]} \colz{ set the source IP address of the query.
    \texttt{address must be a valid address on one of the host's interfaces, or
      \texttt{0.0.0.0} or \texttt{::}.
    }}
\item \texttt{-c class} \colz{ set the query class, default to \texttt{IN}. }
\item \texttt{-p port} \colz{ set the port number to query, default to 53.}
\end{itemize}

\subsection{quiz}

\begin{enumerate}
\item 
\end{enumerate}

\section{DHCP}
\label{sec:dhcp}

DHCP uses UDP. The client uses a link-local address (obtained from whatever
means) to send a DHCP message to a \cola{reserved multicast address}. Then the server will answer it.

\subsection{Terminology}
\label{sec:dhcp-term}

\begin{enumerate}
\item appropriate to the link: 
  \colz{
    An address is \cola{appropriate to the link} if
    the address is consistent with the DHCP's server's knowledge of the
    \begin{enumerate}
    \item network topology
    \item prefix assignment
    \item address assignment policies
    \end{enumerate}
  }
\item Identity association (IA) \colz{
    A collection of addresses assigned to a client.

    An interface can have one or more IAs, but each IA can only bind to one interface.

    Each IA has an associated \texttt{IAID}, which is chosen by the client.
    IAIDs should be unique within the client.
  }
\item IA for temporary addresses (IA-TA) \colz{
    A type of IA, self-explanatory.
  }
\item IA for non-temporary addresses (IA-NA) \colz{
    Another type of IA.
  }
\item \texttt{DUID} \colz{
    DHCP Unique Identifier. It's unique for both DHCP client and server.

\colZ{So it's not one per interface, but one per client.}
  }
\item binding \colz{

    A \cola{binding} is a set of some server data the server knows about the
    addresses in
    \begin{enumerate}
    \item an IA
    \item configuration explicitly assigned to the client
    \end{enumerate} .
    \colZ{
      So you can bind configuration to an IA, or explicitly to a client.
    }

    In the first case, binding is indexed by the tuple:
    \begin{center}
      \ttfamily
      (\cola{DUID}, \colb{IA\_Type}, \colc{IAID})
    \end{center}
    In the second case, binding is indexed just by \cola{\texttt{DUID}}.

    Some congifuration info do not need a binding. They are sent through a
    policy, such as info boardcast to all clients on the link.
  }
\item configuration parameter \colz{data to kept by server and to be
    sent to client.}
\item DHCP client \colz{a host using DHCP to obtain \colZ{configuration
      parameters}.}
\item DHCP server \colz{a node that responds to requests from DHCP clients.}
\item DHCP relay agent \colz{
    a node that acts as an intermediary to deliver DHCP messages between clients
    and servers. It's on the same link as the client, but not necessarily same
    as the server.
  }
\item DHCP domain \colz{
    a set of links managed by DHCP and operated by a single administrative entity.
  }
\item DHCP realm \colz{a name used to identify a DHCP domain.}
\item Reconfigure key \colz{
    A key supplied to a client by a server used to provide security for
    Reconfigure messages.
  }
\item transaction ID \colz{
    An opaque value used to match responses with requests.
  }
\end{enumerate}

\subsection{DHCP Constant}
\label{sec:dhcp-const}

Some constants used in DHCP are shown in \cref{tab:dhcp-const}.

\begin{table}[h]
  \centering
  \begin{tabularx}{\linewidth}{XsX}
    Name & Value & Description \\[1ex]
    \hline
    \verb|All_DHCP_Relay_Agents_and_Servers| & \verb|ff02::1:2| & A link-scoped
                                                                  multicast
                                                                  address. \\
    \verb|All_DHCP_Servers| & \verb|ff05::1:3| & A site-scoped multicast\\
    client UDP port & 546 & \\
    server UDP port & 547 & \\
  \end{tabularx}
  \caption{DHCP Constant}
  \label{tab:dhcp-const}
\end{table}

DHCP also defines some message types, shown in \cref{tab:dhcp-msg-type}.

\begin{longtblr}[caption = {DHCP Message Type.},
  label = {tab:dhcp-msg-type}
  ]%
  {colspec={ssX},
    rowhead = 1, % first row is shown on every page
  }
  \toprule
  % A & B & C \\
  \mbox{Name} & \mbox{Value} & \mbox{Description} \\
  \midrule
    \texttt{Solicit} & 1 & \colz{Sent by client to locate servers.} \\
    \texttt{Advertise} & 2 & \colz{Sent by server to indicate its presence.Usually as a
                           response to \texttt{Solicit}.} \\

    \texttt{Request} & 3 & \colz{Sent by client to request configuration parameters
                         (including addresses) from a server.} \\
    \texttt{Confirm} & 4 & \colz{Sent by client to say whether the once assigned
                         addresses are still \cola{appropriate to the link}.} \\
    \texttt{Renew} & 5 & \colz{
                       Sent by client to \colZ{the origin server} to extend the lifetimes of addresses and
                       to update other configuration parameters. 
                       }\\
    \texttt{Rebind} & 6 & \colz{
                        Similar to \texttt{Renew}, but sent to \colZ{any server}.
                        This is sent when no response to \texttt{Renew} is received.
                        } \\
    \texttt{Reply} & 7 & Sent by server to respond to \cola{all kinds of message} sent
                       by client.
                       \colz{
                       This more or less dumps the current configuration parameters that
                       server knows to the client.
                       } \\
    \texttt{Release} & 8 & \colz{
                         Sent by client to release some addresses assigned to it.
                         } \\
    \texttt{Decline} & 9 & \colz{
                         Sent by client to complain that the addresses assigned
                         to it are already in use.
                         } \\
    \texttt{Reconfigure} & 10 & \colz{
                              Sent by server to inform clients that the server
                              has new configuration parameters, so clients
                              should request them.(i.e. start a new
                              \texttt{Renew/Reply} transaction).
                            }\\

    \texttt{Information-Request} & 11 & \colz{
                                       Sent by client to request configuration
                                       parameters without assigning addresses.
                                     } \\
    \texttt{Relay-Forward} & 12 & \colz{ Sent by relay agent to forward messages
      between client, server and probably other relay agents.} \\
    \texttt{Relay-Reply} & 13 & \colz{ Sent by server to respond to relay agent.} \\
                              \bottomrule
\end{longtblr}

There're also some constants such as the timeout and max-try, but let's move on.

\subsection{DHCP Message Format}


All DHCP messages exchanged between server and client have a fixed format shown
in \cref{tab:dhcp-msg-format-server-client}.

\begin{longtblr}[caption = {server-client DHCP message format.}, label = {tab:dhcp-msg-format-server-client}]%
  {colspec={ssX},
    rowhead = 1, % first row is shown on every page
  }
  \toprule
  \mbox{Name} & \mbox{Size (bytes)} & \mbox{Description} \\
  \midrule
  \texttt{msg-type} & 1 & \colz{self-explanatory} \\
  \texttt{transaction-id} & 3 & \colz{self-explanatory } \\
  \texttt{options} & \colz{variable} & \colz{payloads} \\ 
  \bottomrule
\end{longtblr}

However, the message format between relay agent and server is different, shown
in \cref{tab:dhcp-msg-format-relay-server}.

\begin{longtblr}[caption = {server-relay DHCP Message Format.}, label = {tab:dhcp-msg-format-relay-client}]%
  {colspec={ssX},
    rowhead = 1, % first row is shown on every page
  }
  \toprule
  \mbox{Name} & \mbox{Size (bytes)} & \mbox{Description} \\
  \midrule
  \texttt{msg-type} & 1 & \colz{= \texttt{Relay-Forward | Relay-Reply}} \\
  \texttt{hop-count} & 1 & \colz{number of relay agents that have relayed this
                           message.} \\
                         \texttt{link-address} & 16 & \colz{
                           A global or site-local address that will be used by
                           the server to identify the \colZ{link on which the client is located.}
                         } \\
                         \texttt{peer-address} & 16 & \colz{
                           The address of the client or relay agent from which
                           the message was received.
                         } \\
                         \texttt{options} & \colz{variable} &
                         \cola{\texttt{<Relay Message option>} + }
                         \colz{other options} \\ 
  \bottomrule
\end{longtblr}

\subsection{DUID}

A DUID consists of a two-octet \cola{type field} and a variable-length
\cola{identifier field}. A DUID can be no more than 128 octets long. 
\section{DHCP}
\label{sec:dhcp}

DHCP uses UDP. The client uses a link-local address (obtained from whatever
means) to send a DHCP message to a \cola{reserved multicast address}. Then the server will answer it.

\subsection{Terminology}
\label{sec:dhcp-term}

\begin{enumerate}
\item appropriate to the link: 
  \colz{
    An address is \cola{appropriate to the link} if
    the address is consistent with the DHCP's server's knowledge of the
    \begin{enumerate}
    \item network topology
    \item prefix assignment
    \item address assignment policies
    \end{enumerate}
  }
\item Identity association (IA) \colz{
    A collection of addresses assigned to a client.

    Usually, it's one IA per interface.

    Each IA has an associated \texttt{IAID}, which is chosen by the client.
    IAIDs should be unique within the client.
  }
\item IA for temporary addresses (IA-TA) \colz{
    A type of IA, self-explanatory.
  }
\item IA for non-temporary addresses (IA-NA) \colz{
    Another type of IA.
  }
\item \texttt{DUID} \colz{
    DHCP Unique Identifier. It's unique for both DHCP client and server.

\colZ{So it's not one per interface, but one per client.}
  }
\item binding \colz{

    A \cola{binding} is a set of some server data the server knows about the
    addresses in
    \begin{enumerate}
    \item an IA
    \item configuration explicitly assigned to the client
    \end{enumerate} .
    \colZ{
      So you can bind configuration to an IA, or explicitly to a client.
    }

    In the first case, binding is indexed by the tuple:
    \begin{center}
      \ttfamily
      (\cola{DUID}, \colb{IA\_Type}, \colc{IAID})
    \end{center}
    In the second case, binding is indexed just by \cola{\texttt{DUID}}.

    Some congifuration info do not need a binding. They are sent through a
    policy, such as info boardcast to all clients on the link.
  }
\item configuration parameter \colz{data to kept by server and to be
    sent to client.}
\item DHCP client \colz{a host using DHCP to obtain \colZ{configuration
      parameters}.}
\item DHCP server \colz{a node that responds to requests from DHCP clients.}
\item DHCP relay agent \colz{
    a node that acts as an intermediary to deliver DHCP messages between clients
    and servers. It's on the same link as the client, but not necessarily same
    as the server.
  }
\item DHCP domain \colz{
    a set of links managed by DHCP and operated by a single administrative entity.
  }
\item DHCP realm \colz{a name used to identify a DHCP domain.}
\item Reconfigure key \colz{
    A key supplied to a client by a server used to provide security for
    Reconfigure messages.
  }
\item transaction ID \colz{
    An opaque value used to match responses with requests.
  }
\end{enumerate}


% We will want to consider what we need for special purpose devices such as
% iPhones, \Cola{Wireless Access Points (WAP)}, or even \Cola{Pre Execution
% Environment (PXE)} devices that can load their entire OS from DHCP
% information.



\section{Deploy it}
\label{sec:deploy}

\subsection{prepare the target host}
\label{sec:prepare-target-host}

\subsubsection{Install OS}

\begin{enumerate}
\item Install Ubuntu Server 22.04 LTS.
\item Configure at least one network iface to access the internet or suitable
  local repositories
\item Remove the \texttt{127.0.1.1} loopback in \texttt{/etc/hosts} if exists
\end{enumerate}

\subsubsection{Configure OS }
\label{sec:config-os}

\begin{simplesh}
    sudo apt update
    sudo apt dist-upgrade
    sudo apt install bridge-utils debootstrap openssh-server \
        tcpdump vlan python3
    sudo apt install linux-modules-extra-$(uname -r)
    # reboot$
\end{simplesh}

Set the password of \texttt{root} (Ubuntu disables login as root by default.).
This enables Ansible to login as \texttt{root}.
\begin{simplesh}
    sudo passwd root
\end{simplesh}

Add the SSH keys in \verb|/root/.ssh/authorized_keys| so that ansible can access.

OpenStack-Ansible deployments require the presence of a
\verb|/root/.ssh/id_rsa.pub| file on the \cola{deployment host}. The content of this
file is inserted into an \verb|authorized_keys| file for the containers, which is a
necessary step for the Ansible playbooks.

\dSay{
  What if I wanna override this file on the deployment host.
}

\cSay{
  set the variable \texttt{
    lxc\_container\_ssh\_key=/home/me/.ssh/id\_ed25519.pub
  }
  % 🦜 : It looks like verb|| doesn't work well in a group?
}

\subsubsection{Configure the storage}
\label{sec:conf-storage}

\dSay{
  Okay, let's talk about storage....
}

\cSay{
  \colz{
    The magic that os uses is \cola{Logical Volume Manager (LVM)}. The Block
    Storage(\cola{cinder}), and \cola{LXC containers} that optionally run the OpenStack
    infrastructure, can \colZ{optionally} use LVM for their data storage.
  }
}

\colz{
  OpenStack-Ansible \cola{automatically configures LVM} on the nodes, and \colZ{overrides} any
  existing LVM configuration. If you had a customized LVM
  configuration,(\emoji{parrot}: We don't.) edit the generated configuration
  file as needed.
}

\dSay{
  Do we have to configure this?
}

\cSay{
  \colz{
    It looks like we don't.

    It looks like it's only for the Block Storage service \cola{Cinder} and the \texttt{lxc}
    containers. Cinder is itself optional, and \texttt{lxc} can optionally be
    installed on a different filesystem than the main filesystem.
  }
}

\dSay{
  Okay. So how to enable Cinder ?
}

\cSay{
  \colz{
    Create an LVM \cola{volume group} named \colZ{\texttt{cinder-volumes}} on the storage
    host. Specify a \colZ{metadata size} of 2048 when creating the \cola{physical volumes}:
  }
}

\begin{simplesh}
 pvcreate --metadatasize 2048 /dev/sda1
 vgcreate cinder-volumes /dev/sda1
\end{simplesh}

\dSay{
  Wait, do we have to do this on every nodes that provides Cinder ? How do we
  know which node should provide the Cinder Service ?
}

\cSay{
  Em... It looks like the document didn't mention this....
}

\dSay{
  Okay...So how does it work with the \texttt{lxc} ?
}

\cSay{
  If you want to use LXC with LVM.
  \begin{enumerate}
  \item Set \texttt{lxc\_storage\_backing\_store: lvm} in
    \texttt{user\_variable.yml}.
  \item Create an LVM volume group named \texttt{lxc} on the storage host.
  \end{enumerate}

  If you set the variable but don't create the volume group, the containers are
  automatically installed in \texttt{/var/lib/lxc} by default.
}

\begin{simplesh}
pvcreate --metadatasize 2048 /dev/sda1
vgcreate lxc /dev/sda1
\end{simplesh}

\subsubsection{Configure the network}
\label{sec:conf-network}



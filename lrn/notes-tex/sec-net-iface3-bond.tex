
\section{bond}
\label{sec:bond}

Network bonding is a method to combine or aggregate network interfaces to
provide a logical interface with higher throughput or redundancy.

\subsection{bonding modes}

There are several bonding modes. The most common modes are:
\begin{longtblr}[caption={Common bonding modes}, label={tab:bonding-modes}]
  {
    colspec=XXX[4],
    width = {0.9\linewidth}
  }
  \toprule
  Mode number & Mode name & Description \\
  \midrule
  0 & \texttt{balance-rr} &
  Round-robin policy.
  \colz{
    Transmits packets in
    sequential order from the first available slave through the last. This
    mode provides load balancing and fault tolerance.
  }\\

  1 & \texttt{active-backup} & \colz{ Establishes that only one slave in the
    bond is active. A different slave becomes \colZ{active} if, and \colZ{only if}, the
    active slave fails. The bond's MAC address is externally visible on only
    one port (network adapter) to avoid confusing the switch. This mode
    provides fault tolerance. The primary option affects the behavior of this
    mode. } \\

  2 & \texttt{balance-xor} & \colz{ Transmits based on the selected transmit
    hash policy, which can be altered via the \texttt{xmit\_hash\_policy}
    option. This mode provides load balancing and fault tolerance.
  }\\

  3 & \texttt{broadcast} & \colz{ Transmits everything on all slave
    interfaces. This mode provides fault tolerance. }\\

  4 & \texttt{802.3ad} & \colz{ 
    IEEE 802.3ad Dynamic link aggregation policy. Creates aggregation groups

    that share the same speed and duplex settings. Utilizes all slaves in the
    active aggregator according to the 802.3ad specification.}\\

  5 & \texttt{balance-tlb} & \colz{
    Adaptive transmit load balancing. Establishes channel bonding that does
    \cola{not require any special switch support}.

    The outgoing traffic is distributed according to the current load
    (computed relative to the speed) on each slave. Incoming traffic is
    received by the current slave. If the receiving slave fails, another slave
    takes over the MAC address of the failed receiving slave. }\\
  6 & \texttt{balance-alb} &
  Adaptive transmit load balancing.

  \colz{ Establishes channel bonding that does \cola{not require any special switch
      support}. The outgoing traffic is distributed according to the current load
    (computed relative to the speed) on each slave. Incoming traffic is
    received by the current slave. If the receiving slave fails, another slave
    takes over the MAC address of the failed receiving slave.

  }\\
  
  \bottomrule
\end{longtblr}


\colz{ Some bonding mode are ``plug-and-play'' (no configuration on the switch
  is required.), for example, \texttt{balance-tlb (5)} and \texttt{active-backup
  (1)}.

  However, other bonding modes require configuring the switch to aggregate
  the links. For example, \cola{Cisco} switches requires \texttt{EtherChannel}
  for modes 0, 2, and 3, but for mode 4, the \Cola{Link Aggregation Control
    Protocol (LACP)} and \texttt{EtherChannel} are required. For further
  details, see the documentation of your switch.
}
\subsection{Create a bond}

Here's how to create a bond named \texttt{mybond} with \texttt{active-backup}.
\begin{simplesh}
  n=mybond
  sudo nmcli connection add type bond con-name $n ifname $n bond.options "mode=active-backup"
\end{simplesh}

\subsection{Bind the devices}

Now we display the interfaces:
\begin{simplesh}
  sudo nmcli device status
  # DEVICE   TYPE      STATE         CONNECTION
  # enp7s0   ethernet  disconnected  --
  # enp8s0   ethernet  disconnected  --
  # bridge0  bridge    connected     bridge0
  # bridge1  bridge    connected     bridge1
\end{simplesh}
These are the interfaces that we are going to bind together. In this example:

\begin{itemize}
\item \texttt{enp7s0} and \texttt{enp8s0} are not configured. To use these as
  ports(=slaves), we need to create a connection for each of them.
\item \texttt{bridge0} and \texttt{bridge1} are already configured. To use
  these as ports, we modify their profiles.
\end{itemize}

To assigns this to the bond:
\begin{simplesh}
  sudo nmcli connection add type ethernet slave-type bond con-name my-bond-port1 ifname enp7s0 master $n
  sudo nmcli connection add type ethernet slave-type bond con-name my-bond-port2 ifname enp8s0 master $n
\end{simplesh}
This create profiles for \texttt{enp7s0} and \texttt{enp8s0} and assign them to
the bond \texttt{mybond}.

Next, we bind the bridges:
\begin{simplesh}
  sudo nmcli connection modify bridge0 master $n #bind the bridge to the bond
  sudo nmcli connection modify bridge1 master $n
  sudo nmcli connection up bridge0 # reactive the connections
  sudo nmcli connection up bridge1
\end{simplesh}

\subsection{Configure the bond}

\subsubsection{IPv4}
For IPv4, we can either:
\begin{enumerate}
\item assign a static IP address to the bond interface:
  \begin{simplesh}
    sudo nmcli connection modify $n ipv4.addresses '192.0.2.1/24' \
                ipv4.gateway '192.0.2.254' ipv4.dns '192.0.2.253' \
                ipv4.dns-search 'example.com' ipv4.method manual
  \end{simplesh}
\item use DHCP: then do nothing
\item make this bond a port of another bond....
  \begin{simplesh}
    sudo nmcli connection modify $n ipv4.method disabled
  \end{simplesh}
\end{enumerate}

\subsubsection{IPv6}

Same as IPv4, we also have three options:
\begin{enumerate}
\item assign a static IP address to the bond interface:
  \begin{simplesh}
    sudo nmcli connection modify $n ipv6.addresses '2001:db8::1/64' \
            ipv6.gateway '2001:db8::fffe' ipv6.dns '2001:db8::fffd' \
            ipv6.dns-search 'example.com' ipv6.method manual
\end{simplesh}
\item use DHCP: then do nothing
\item make this bond a port of another bond....
\begin{simplesh}
    sudo nmcli connection modify $n ipv6.method disabled
\end{simplesh}
\end{enumerate}

\subsection{Activate and verify the bond}

\begin{simplesh}
  sudo nmcli connection up $n
  sudo nmcli connection
\end{simplesh}

\cSay{
  \colz{
    When you activate any port, NetworkManager also activates the bond.
  }
}

To show the status of the bond:

\begin{simplesh}
cat /proc/net/bonding/$n
\end{simplesh}
